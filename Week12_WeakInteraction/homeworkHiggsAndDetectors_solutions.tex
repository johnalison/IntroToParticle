\input{../latexSetup}

\lhead{\Large 33-444} % \hfill Introduction to Particle Physics \hfill Spring 2022}
\chead{\Large Introduction to Particle Physics} % \hfill Spring 2022}
\rhead{\Large Spring 2022} % \hfill Introduction to Particle Physics \hfill Spring 2022}
\begin{document}
\thispagestyle{fancy}


\begin{center}
{\huge \textbf{Homework Set \#9 }}
\large


\end{center}

{\large


\textbf{4)  $W$ boson decays to electrons. } \hfill \textit{(3 points)}\\
\begin{itemize}
\item[a)]{ The W can decay directly to an electron or to electron by decaying through a $\tau$. Draw the corresponding diagrams.
\bc
\includegraphics[width=0.3\textwidth]{./Wenu.pdf}
\includegraphics[width=0.3\textwidth]{./We3nu.pdf}
\ec
}
\item[b)]{ 
\be
Br(W\rightarrow \ell \nu)  = \frac{1}{\underbrace{3}_{\rmt{leptons}} + \underbrace{3}_{\rmt{color}} \times \underbrace{2}_{\rmt{2 quark generations}}} = \frac{1}{9} = 0.11
\ee
(Note the top-quark is heavier than the W, so the $W\rightarrow t, b $ decay is forbidden.)

Now,
\be
Br(\tau \rightarrow e \nu \bar{\nu})  = \frac{1}{\underbrace{2}_{\rmt{leptons}} + \underbrace{3}_{\rmt{color}} \times \underbrace{2}_{\rmt{1 quark generations}}} = \frac{1}{5} = 0.2
\ee
Here the $\tau$ can decay to two lepton generation (es and $\mu$ s), but only has enough mass to decay to one quark generation (u, d). 


So, 
\be
Br(W\rightarrow e + X) = \underbrace{\frac{1}{9}}_{W\rightarrow e\nu} + \underbrace{\frac{1}{9}}_{W\rightarrow \tau\nu} \times \underbrace{\frac{1}{5}}_{\tau\rightarrow e\nu}  = \frac{1.2}{9} = 0.13
\ee

}
\end{itemize}           

\textbf{4) $ H\rightarrow WW\rightarrow e \mu$ decaus. } \hfill \textit{(3 points)}\\

\be
Br(W\rightarrow \ell \nu) \sim \frac{1}{9} 
\ee

for e or $\mu$ can include the decays through $\tau$s as in problem 3 to get $\frac{1.2}{9}$

\be
Br(WW\rightarrow e\mu + X) = 2 \times \left(\frac{1.2}{9}\right)^2 \sim 0.036
\ee
Factor of two becuase you can get $e^+\mu^-$ or $e^-\mu^+$



\textbf{2) Tracking Detectors } \hfill \textit{(10 points)}\\
\begin{itemize}
\item[a)]{
\be
F = ma \Rightarrow m v^2/r_c = qvB  \Rightarrow r_c = p_T/qB
\ee
Now,
\be
r_c^2 = \left(\frac{L}{2}\right)^2 + (r_c -s )^2
\ee
Or (Ignoring terms 
\be
\frac{L^2}{4} = 2 r_c s - s^2
\ee
B/c $r_c >> s$, can safely drop $s^2$ relative to $r_c s$.  Thus
\be
s = \frac{L^2}{8r_c} = \frac{qBL^2}{8p_T}
\ee


}
\item[b)]{
\be
p_T = \frac{qBL^2}{8s}
\ee 
So,
\be
\Delta p_T = \frac{qBL^2}{8s^2} \Delta s 
\ee 

and
\be
\frac{\Delta p_T}{p_T} = \frac{\Delta s}{s} = \frac{8p_T}{qBL^2} \Delta s 
\ee 

}
\item[c)]{
For N = 50, $\epsilon$ = 100 $\mu m$, L = 1 m, and B = 1 T, $\Delta s \sim 50 \mu m = 50 10^{-6} m$

Now
$T = 2 \times 10^{-16} GeV^2$\\
$e = 0.3$
\be
\Delta p_T = \frac{8 (p_T[GeV])^2}{0.3 \times 2 \times 10^{-16} }  \frac{50 \times 10^{-6}}{5\times10 ^{15} GeV^{-1}} \sim 3\times10^{-3} (p_T[GeV])^2 GeV
\ee

At 1 GeV the uncertainty is $\sim 10^{-3}$ GeV,   At 100 GeV the uncertainty is 10 GeV.
}
\end{itemize}

\clearpage


\textbf{3) Limits of the Tracking System.} \hfill \textit{(5 points)}\\

\begin{itemize}
\item[a)]{
\be
r_c \sim 3 \frac{p_T[GeV] }{Q[e] B[T]}
\ee
Particles dont make it to the calorimeter when $r_{calo} \sim 2\times r_c$

or 
\be
p_T \sim \frac{q B r_{calo}}{6} = \frac{2 \times 1.1}{6} \sim 400 MeV
\ee

}

\item[b)]{
Estimate upper limit when $s\sim17 \mu m \sim 20 \times 10^{-6} m$

\be
p_T \sim \frac{0.3 \times 2\times 10^{-16} GeV^2}{8} \frac{0.5}{20 \times 10^{-6}} 0.5 \times 5 \times 10^{15} GeV^{-1}
p_T \sim 0.5\times 10^3 GeV
\ee

}
\item[c)]{
At the limit $\Delta s / s \sim 1  \Rightarrow \Delta p_T / p_T \sim 1 $, so $\Delta p_T \sim 500$ GeV


}
\end{itemize}

\vspace*{0.25in}

\textbf{4) Rapidity.} \hfill \textit{(15 points)}\\

\begin{itemize}
\item[a)]{
Under a boost along Z

\be
E \rightarrow E \gamma - \beta \gamma p_z
\ee
\be
p_z \rightarrow p_z \gamma - \beta \gamma E
\ee

So,

\begin{align*}
y \rightarrow \frac{1}{2} \log \frac{(E \gamma - \beta \gamma p_z) + (p_z \gamma - \beta \gamma E)}{(E \gamma - \beta \gamma p_z) -(p_z \gamma - \beta \gamma E)} \\ 
  = \frac{1}{2} \log \frac{\gamma - \beta \gamma}{\gamma + \beta \gamma}\frac{E+p_z}{E - p_z} = \frac{1}{2} \log \frac{E+p_z}{E - p_z} + \frac{1}{2} \log \frac{\gamma - \beta \gamma}{\gamma + \beta \gamma}\\
  = y + \frac{1}{2} \log \frac{\cosh \eta - \sinh \eta}{\cosh + \sinh \eta} = y + \frac{1}{2} \log \frac{e^{-\eta}} {e^{+\eta}}\\
  = y + \frac{1}{2} \log e^{-2\eta} = y - \eta
 \end{align*}

}
\item[b.]{
$y = \eta $ for mass-less particles
}
\item[d.]{
Green are electrons / Red are muons.
}
\item[e]{
I got:

eta1 = -1  /
phi1 = 70 /
pt1  = 30 

eta2 = 0  /
phi2 = 255 /
pt2  = 30 

eta3 = -0.2 / 
phi3 = 70  /
pt3  = 20

eta4 = 0.5 / 
phi4 = 200 /
pt4  = 25 

}
\item[f.]{
I get: (124.8, -14.2, 9.5, -26.3) GeV (E,vec{P})
}
\item[h.]{68. probably Zboson}
\item[i.]{43 probably off shell z}
\item[j.]{121 GeV probably a higgs}





\end{itemize}




}
\end{document}
