\documentclass[paper=letter,11pt]{scrartcl}

\KOMAoptions{headinclude=true, footinclude=false}
\KOMAoptions{DIV=14, BCOR=5mm}
\KOMAoptions{numbers=noendperiod}
\KOMAoptions{parskip=half}
\addtokomafont{disposition}{\rmfamily}
\addtokomafont{part}{\LARGE}
\addtokomafont{descriptionlabel}{\rmfamily}
%\setkomafont{pageheadfoot}{\normalsize\sffamily}
\setkomafont{pagehead}{\normalsize\rmfamily}
%\setkomafont{publishers}{\normalsize\rmfamily}
\setkomafont{caption}{\normalfont\small}
\setcapindent{0pt}
\deffootnote[1em]{1em}{1em}{\textsuperscript{\thefootnotemark}\ }


\usepackage{amsmath}
\usepackage[varg]{txfonts}
\usepackage[T1]{fontenc}
\usepackage{graphicx}
\usepackage{xcolor}
\usepackage[american]{babel}
% hyperref is needed in many places, so include it here
\usepackage{hyperref}

\usepackage{xspace}
\usepackage{multirow}
\usepackage{float}


\usepackage{braket}
\usepackage{bbm}
\usepackage{relsize}
\usepackage{tcolorbox}

\def\ketY{\ensuremath{\ket {\Psi}}}
\def\iGeV{\ensuremath{\textrm{GeV}^{-1}}}
%\def\mp{\ensuremath{m_{\textrm{proton}}}}
\def\rp{\ensuremath{r_{\textrm{proton}}}}
\def\me{\ensuremath{m_{\textrm{electron}}}}
\def\aG{\ensuremath{\alpha_G}}
\def\rAtom{\ensuremath{r_{\textrm{atom}}}}
\def\rNucl{\ensuremath{r_{\textrm{nucleus}}}}
\def\GN{\ensuremath{\textrm{G}_\textrm{N}}}
\def\ketX{\ensuremath{\ket{\vec{x}}}}
\def\ve{\ensuremath{\vec{\epsilon}}}


\def\ABCDMatrix{\ensuremath{\begin{pmatrix} A &  B  \\ C  & D \end{pmatrix}}}
\def\xyprime{\ensuremath{\begin{pmatrix} x' \\ y' \end{pmatrix}}}
\def\xyprimeT{\ensuremath{\begin{pmatrix} x' &  y' \end{pmatrix}}}
\def\xy{\ensuremath{\begin{pmatrix} x \\ y \end{pmatrix}}}
\def\xyT{\ensuremath{\begin{pmatrix} x & y \end{pmatrix}}}

\def\IMatrix{\ensuremath{\begin{pmatrix} 0 &  1  \\ -1  & 0 \end{pmatrix}}}
\def\IBoostMatrix{\ensuremath{\begin{pmatrix} 0 &  1  \\ 1  & 0 \end{pmatrix}}}
\def\JThree{\ensuremath{\begin{pmatrix}    0 & -i & 0  \\ i & 0  & 0 \\ 0 & 0 & 0 \end{pmatrix}}} 
\def\JTwo{\ensuremath{\begin{bmatrix}    0 & 0 & -i  \\ 0 & 0  & 0 \\ i & 0 & 0 \end{bmatrix}}}
\def\JOne{\ensuremath{\begin{bmatrix}    0 & 0 & 0  \\ 0 & 0  & -i \\ 0 & i & 0 \end{bmatrix}}}
\def\etamn{\ensuremath{\eta_{\mu\nu}}}
\def\Lmn{\ensuremath{\Lambda^\mu_\nu}}
\def\dmn{\ensuremath{\delta^\mu_\nu}}
\def\wmn{\ensuremath{\omega^\mu_\nu}}
\def\be{\begin{equation*}}
\def\ee{\end{equation*}}
\def\bea{\begin{eqnarray*}}
\def\eea{\end{eqnarray*}}
\def\bi{\begin{itemize}}
\def\ei{\end{itemize}}
\def\fmn{\ensuremath{F_{\mu\nu}}}
\def\fMN{\ensuremath{F^{\mu\nu}}}
\def\bc{\begin{center}}
\def\ec{\end{center}}
\def\nus{$\nu$s}

\def\adagger{\ensuremath{a_{p\sigma}^\dagger}}
\def\lineacross{\noindent\rule{\textwidth}{1pt}}

\newcommand{\multiline}[1] {
\begin{tabular} {|l}
#1
\end{tabular}
}

\newcommand{\multilineNoLine}[1] {
\begin{tabular} {l}
#1
\end{tabular}
}



\newcommand{\lineTwo}[2] {
\begin{tabular} {|l}
#1 \\
#2
\end{tabular}
}

\newcommand{\rmt}[1] {
\textrm{#1}
}


%
% Units
%
\def\m{\ensuremath{\rmt{m}}}
\def\GeV{\ensuremath{\rmt{GeV}}}
\def\pt{\ensuremath{p_\rmt{T}}}


\def\parity{\ensuremath{\mathcal{P}}}

\usepackage{cancel}
\usepackage{ mathrsfs }
\def\bigL{\ensuremath{\mathscr{L}}}

\usepackage{ dsfont }



\usepackage{fancyhdr}
\fancyhf{}

\usepackage{braket}

\def\ketY{\ensuremath{\ket {\Psi}}}
\def\iGeV{\ensuremath{\textrm{GeV}^{-1}}}
\def\mp{\ensuremath{m_{\textrm{proton}}}}
\def\rp{\ensuremath{r_{\textrm{proton}}}}
\def\me{\ensuremath{m_{\textrm{electron}}}}
\def\aG{\ensuremath{\alpha_G}}
\def\rAtom{\ensuremath{r_{\textrm{atom}}}}
\def\rNucl{\ensuremath{r_{\textrm{nucleus}}}}
\def\GN{\ensuremath{\textrm{G}_\textrm{N}}}

\def\be{\begin{equation*}}
\def\ee{\end{equation*}}


\usepackage{fancyhdr}
\usepackage{cancel}
\usepackage{ mathrsfs }





\fancyhf{}
\lhead{\Large 33-444} % \hfill Introduction to Particle Physics \hfill Spring 2019}
\chead{\Large Introduction to Particle Physics} % \hfill Spring 2019}
\rhead{\Large Spring 2019} % \hfill Introduction to Particle Physics \hfill Spring 2019}
\begin{document}
\thispagestyle{fancy}





%\begin{tabular}{c}
%{\large 33-444 \hfill Intro To Particle \hfill Spring 2019\\}
%\hline 
%\end{tabular}

\begin{center}
{\huge \textbf{Homework Set \#11 }}
\large

{\textbf{ Due Date:} Before class Friday April 26th  } 
\end{center}

{\large

\textbf{1) Spontaneous Symmetry Breaking } \hfill \textit{(10 points)}\\

\begin{itemize}
\item[a.]{ Let V($\phi$) be 
\be
V(\phi) = \frac{1}{2}\mu^2\phi^2 + \frac{1}{4}\lambda \phi^4
\ee
where $\phi$ is a real scalar feild and $\mu$ and $\lambda$ are constants. 
If $\mu^2 > 0$, where are the minima ?
}
\item[b.]{ If $\mu^2 < 0$, where are the minima?}
\item[c.]{ 
Now let $\phi$ be a complex field $\phi = \frac{1}{\sqrt{2}}(\phi_1 + i \phi_2)$.
And let,

\be
\mathcal{L} = (\partial_\mu \phi^*) (\partial^\mu \phi) - \mu^2\phi^*\phi - \lambda (\phi^*\phi)^2
\ee
with $\mu^2 < 0$ and $\lambda > 0$.
Expand the Lagrangian about the minimum as we did in class with $\phi(x) = \frac{1}{\sqrt{2}}(v+\eta(x)+i\epsilon(x))$. Write out all the terms. 
}
\item[d.]{ Do the same thing with the U(1) Lagrangian.  Expand about the minimum and write the Lagrangian including all interaction terms.}
\end{itemize}

\vspace*{0.25in}


\textbf{2) Higgs Self-Interaction } \hfill \textit{(10 points)}\\

\begin{itemize}
\item[a.]{ Let V($\phi$) be
\be
V(\phi) = \frac{1}{2}\mu^2\phi^2 + \frac{1}{4}\lambda \phi^4
\ee
with $\mu^2 < 0$ and $\lambda > 0$.
where $\phi$ is a real scalar field and $\mu$ and $\lambda$ are constants, with $\mu^2 < 0$ and $\lambda > 0$.
What is the coupling constant associated to the $h^3$ interaction in terms of $m_h$ and the position of the minimum ?
}
\item[b.]{ 
Assume the potential is: 
\be
V(\phi) = \frac{1}{2}\mu^2\phi^2 + \frac{1}{4}\lambda \phi^6
\ee
with $\mu^2 < 0$ and $\lambda > 0$.
What is the position of the minimum ?
What is the coupling constant associated to the $h^3$ interaction in terms of $m_h$ and the position of the minimum ?
}
\end{itemize}

\clearpage


\textbf{3) LEP: Z and the Higgs } \hfill \textit{(15 points)}\\

\begin{itemize}
\item[a]{Estimate the branching fraction of Z to $\nu$s ?}
\item[b]{Estimate the ratio of the  ee$\rightarrow$Z to ee$\rightarrow$H cross sections.}
\item[c]{LEP collected about 17 million Z bosons, for a discovery and study of the Higgs in direct e+e- collisions how many events would LEP have to collect ?}
\item[d]{If LEP collided electrons at 40MHz like the LHC, how long would it take to collect this data ?}
\item[e]{You can also produce a higgs by ``radiation it off of a virtual Z boson. Draw this diagram.}
\item[f]{Estimate the cross section for this process.}
\end{itemize}

}
\end{document}
