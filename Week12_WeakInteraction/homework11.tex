\input{latexSetup}
\usepackage{braket}

\def\ketY{\ensuremath{\ket {\Psi}}}
\def\iGeV{\ensuremath{\textrm{GeV}^{-1}}}
\def\mp{\ensuremath{m_{\textrm{proton}}}}
\def\rp{\ensuremath{r_{\textrm{proton}}}}
\def\me{\ensuremath{m_{\textrm{electron}}}}
\def\aG{\ensuremath{\alpha_G}}
\def\rAtom{\ensuremath{r_{\textrm{atom}}}}
\def\rNucl{\ensuremath{r_{\textrm{nucleus}}}}
\def\GN{\ensuremath{\textrm{G}_\textrm{N}}}

\def\be{\begin{equation*}}
\def\ee{\end{equation*}}


\usepackage{fancyhdr}
\usepackage{cancel}
\usepackage{ mathrsfs }





\fancyhf{}
\lhead{\Large 33-444} % \hfill Introduction to Particle Physics \hfill Spring 2020}
\chead{\Large Introduction to Particle Physics} % \hfill Spring 2020}
\rhead{\Large Spring 2020} % \hfill Introduction to Particle Physics \hfill Spring 2020}
\begin{document}
\thispagestyle{fancy}





%\begin{tabular}{c}
%{\large 33-444 \hfill Intro To Particle \hfill Spring 2020\\}
%\hline 
%\end{tabular}

\begin{center}
{\huge \textbf{Homework Set \#11 }}
\large

{\textbf{ Due Date:} Before class Friday April 26th  } 
\end{center}

{\large

\textbf{1) Spontaneous Symmetry Breaking } \hfill \textit{(10 points)}\\

\begin{itemize}
\item[a.]{ Let V($\phi$) be 
\be
V(\phi) = \frac{1}{2}\mu^2\phi^2 + \frac{1}{4}\lambda \phi^4
\ee
where $\phi$ is a real scalar feild and $\mu$ and $\lambda$ are constants. 
If $\mu^2 > 0$, where are the minima ?
}
\item[b.]{ If $\mu^2 < 0$, where are the minima?}
\item[c.]{ 
Now let $\phi$ be a complex field $\phi = \frac{1}{\sqrt{2}}(\phi_1 + i \phi_2)$.
And let,

\be
\mathcal{L} = (\partial_\mu \phi^*) (\partial^\mu \phi) - \mu^2\phi^*\phi - \lambda (\phi^*\phi)^2
\ee
with $\mu^2 < 0$ and $\lambda > 0$.
Expand the Lagrangian about the minimum as we did in class with $\phi(x) = \frac{1}{\sqrt{2}}(v+\eta(x)+i\epsilon(x))$. Write out all the terms. 
}
\item[d.]{ Do the same thing with the U(1) Lagrangian.  Expand about the minimum and write the Lagrangian including all interaction terms.}
\end{itemize}

\vspace*{0.25in}


\textbf{2) Higgs Self-Interaction } \hfill \textit{(10 points)}\\

\begin{itemize}
\item[a.]{ Let V($\phi$) be
\be
V(\phi) = \frac{1}{2}\mu^2\phi^2 + \frac{1}{4}\lambda \phi^4
\ee
with $\mu^2 < 0$ and $\lambda > 0$.
where $\phi$ is a real scalar field and $\mu$ and $\lambda$ are constants, with $\mu^2 < 0$ and $\lambda > 0$.
What is the coupling constant associated to the $h^3$ interaction in terms of $m_h$ and the position of the minimum ?
}
\item[b.]{ 
Assume the potential is: 
\be
V(\phi) = \frac{1}{2}\mu^2\phi^2 + \frac{1}{4}\lambda \phi^6
\ee
with $\mu^2 < 0$ and $\lambda > 0$.
What is the position of the minimum ?
What is the coupling constant associated to the $h^3$ interaction in terms of $m_h$ and the position of the minimum ?
}
\end{itemize}

\clearpage


\textbf{3) LEP: Z and the Higgs } \hfill \textit{(15 points)}\\

\begin{itemize}
\item[a]{Estimate the branching fraction of Z to $\nu$s ?}
\item[b]{Estimate the ratio of the  ee$\rightarrow$Z to ee$\rightarrow$H cross sections.}
\item[c]{LEP collected about 17 million Z bosons, for a discovery and study of the Higgs in direct e+e- collisions how many events would LEP have to collect ?}
\item[d]{If LEP collided electrons at 40MHz like the LHC, how long would it take to collect this data ?}
\item[e]{You can also produce a higgs by ``radiation it off of a virtual Z boson. Draw this diagram.}
\item[f]{Estimate the cross section for this process.}
\end{itemize}

}
\end{document}
