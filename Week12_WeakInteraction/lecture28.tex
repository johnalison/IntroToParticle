\input{../latexSetup}

\lhead{\Large 33-444} % \hfill Introduction to Particle Physics \hfill Spring 2022}
\chead{\Large Introduction to Particle Physics} % \hfill Spring 2022}
\rhead{\Large Spring 2022} % \hfill Introduction to Particle Physics \hfill Spring 2022}

\begin{document}
\thispagestyle{fancy}

\begin{center}
{\huge \textbf{Lecture 28}}
\end{center}

{\fontsize{14}{16}\selectfont

%\underline{\underline{Bonus}}  
%
%Clarification from last time ... 
%
%used ``obvious'' expansion about $\phi_1 = v$ and $\phi_2 = 0$
%
%\be
%\phi = (v + \eta + i \epsilon)
%\ee
%
%Put this into L and got 
%\be
%L' = \frac{1}{2}(\partial \epsilon)^2 + \frac{1}{2}(\partial \eta)^2 - v^2\lambda\eta^2 + \frac{1}{2}e^2v^2A^2 - evA_\mu\partial^\mu\epsilon - \frac{1}{4}F_{\mu\nu}F^{\mu\nu} + \rmt{(interactions)}
%\ee
%
%then we said that this looked like there was nonphysical DoF and we starter over with 
%\be
%\phi = (v+h) e^{i\theta/v}
%\ee
%
%A (potentially) cleaner way to see whats happening is to notice 
%
%\be
%\frac{1}{2} (\partial \epsilon)^2 + ev A_\mu \partial^\mu \epsilon + \frac{1}{2}e v^2 A^2 = \frac{1}{2}\left[ A_\mu + \frac{1}{ev} \partial_\mu \epsilon \right]^2
%\ee
%
%Now I can pick a gauge where 
%
%\be
%A_\mu \rightarrow A_\mu + \frac{1}{ev} \partial_\mu \epsilon
%\ee
%
%$\Rightarrow$ I only have the $\frac{1}{2}e^2v^2A^2$ term.
%
%(of course this is all equivalent as above)
%
%\clearpage

\textbf{\underline{\underline{Electroweak Unification}}}

Now that we have seen how we can use the Higgs Mechanism to generate masses for gauge bosons, we will see it works in the SM.

We will take the $L_{SM}$ with 3 mass-less spin-1 bosons and add a scalar (Higgs) w/the ``Mexican hat'' potential $\mu^2 < 0 $.

So we first need to talk about the initial $L_{SM}$ w/o the Higgs. 

This will lead us to talk about the Weak interaction group and ``Electro-weak'' unification. 
(Major step forward in physics) 

\textbf{\underline{\underline{QED}} (EM)}

Gauge invariance led us to the $L_{QED}$

\be
\psi_e(x) \rightarrow e^{iq\alpha(x)} \psi_e(x)
\ee
for arbitrary $\alpha(x)$ which is function of space-time.

We say that this is a ``local U(1) phase transformation'' (a number is a generator of U(1))

What will it take to make physics invariant under this transformation ?

Couple way to think about this.

Lets see what to the equations of motion (here the Dirac equation)

\be
(i\gamma^\mu \partial_\mu - m) \psi_e = 0 \rightarrow (i\gamma^\mu \partial_\mu - m) e^{iq\alpha(x)} \psi_e = (i\gamma^\mu \partial_\mu - q\gamma^\mu \partial_\mu \alpha - m) e^{iq\alpha(x)} \psi_e = 0
\ee

or 
\be
i\gamma^\mu \left[ (\partial_\mu + iq \partial_\mu \alpha) - m \right] \psi_e = 0
\ee
Which is not the Dirac Eq!

$\Rightarrow$ the free particle Dirac Equation does not posess local U(1) invariance. 
We have shown h that its not possible for a free theory (one with out interactions).

We need something to cancel the $iq\partial_\mu\alpha$ term.

So lets add a term

\be
iqA_\mu \rightarrow iqA_\mu - iq\partial_\mu\alpha
\ee

or 
\be
A_\mu \rightarrow A_\mu - \partial_\mu\alpha
\ee

So the new ``Dirac Eq'' (really the D.E. with an interaction to a gauge boson) is 
\be
i\gamma^\mu \left[ (\partial_\mu + iq A_\mu ) - m \right] \psi_e = 0
\ee
This is invariant under $\psi \rightarrow e^{iq\alpha(x)} \psi$. 
The derivative term brings down a piece that is exactly canceled by the $A_\mu$ transformation.

This of $(\partial_\mu + iq A_\mu )$ as a ``fancy'' derivative $D_\mu = (\partial_\mu + iq A_\mu )$ then get back the same for as the Dirac Eq with $\partial_\mu \rightarrow D_\mu$


\be
(i\gamma^\mu D_\mu  - m ) \psi = 0
\ee
Similarly the L that gives the Dirac equation is modified by $\partial_\mu \rightarrow D_\mu$.

This was all for a U(1) symmetry (phase symmetry).

This describes EM.  

In a world with out the weak interaction this would be quantum electrodynamics (QED).

\underline{Now to the weak interaction ...} (GSW)

Will not cover it in detail here, but suffice it to say that QED is fantastically successful, by far the quantitatively most successful theory in human history.

So would like to follow QED as closely as possible (also what else could we possible do?) 

So lets do the same thing, but pick a different group.

SU(2) is a group we looked at before

Reminder:
\bi
\item[-] Can use it to described rotations in 3D with complex spinors. 
\item[-] Generators of SU(2) are the Pauli matrices.
\ei
 
So now postulate

\be
\phi \rightarrow e^{i\rmt{(generators)}} \phi  \hspace*{0.3in} \rmt{for SU(2)} \hspace*{0.3in} \phi \rightarrow e^{ig\vec{\alpha}\cdot\vec{\sigma}} \phi 
\ee
where these are finite transformations.

Lets unpack this

\underline{Consider infinitesimal case first}

\bea
\phi \rightarrow e^{i\epsilon \vec{\alpha} \cdot \vec{\sigma}} \phi &=& \left(1 + \epsilon(\alpha_i \sigma_i)\right)\phi \\
&=& \left[1 + \epsilon \left( a_1(x) \begin{pmatrix} 0 & 1 \\ 1 & 0 \end{pmatrix} + a_2(x) \begin{pmatrix} 0 & i \\ -i & 0 \end{pmatrix} + a_3(x) \begin{pmatrix} 1 & 0 \\ 0 & -1 \end{pmatrix} \right) \right]\phi 
\eea
so $\phi$ is not a number, its a vector.

\be
\phi = \begin{pmatrix} \nu_e(x) \\ e(x)  \end{pmatrix} 
\ee

The Dirac Equation would be 

\be
(i\gamma^\mu \partial_\mu \mathds{1} - m \mathds{1}) \phi = 0
\ee
where $\mathds{1}$ is a 2$\times$2 is unit matrix.


Now when we do $\phi \rightarrow e^{ig\vec{\alpha}\cdot\vec{\sigma}} \phi $ the derivative brings down three factors:

\be
\sim \partial_\mu \alpha_1(x) \hspace*{0.5in} \sim \partial_\mu \alpha_2(x) \hspace*{0.5in} \sim \partial_\mu \alpha_3(x) 
\ee

under local SU(2) transformation.

\be
(i\gamma^\mu \partial_\mu \mathds{1} - m \mathds{1}) \phi = 0 \rightarrow  (i\gamma^\mu \partial_\mu  - m ) \underbrace{e^{ig\vec{\alpha}\cdot\vec{\sigma}}}_{e^{ig(\alpha_1 \sigma_1 + \alpha_2 \sigma_2 + \alpha_3 \sigma_3)}} \phi = 0
\ee

OR

\be
\left[i\gamma^\mu \left(\underbrace{\partial_\mu}_{\times\mathds{1}} + ig(\partial_\mu \alpha_1(x)) \underbrace{\sigma_1}_{2\times2} + ig(\partial_\mu \alpha_2(x))\underbrace{\sigma_2}_{2\times2} + ig(\partial_\mu \alpha_3(x))\underbrace{\sigma_3}_{2\times2} \right) - \underbrace{m}_{\times\mathds{1}} \right] \phi = 0
\ee

Clear that the only way to make this invariant under the local transformation is to add 3 new particles.


\bea
W_\mu^1 \rightarrow W_\mu^1 - \partial_\mu \alpha_1 \\
W_\mu^2 \rightarrow W_\mu^2 - \partial_\mu \alpha_2 \\
W_\mu^3 \rightarrow W_\mu^3 - \partial_\mu \alpha_3 
\eea

\be
\partial_\mu \rightarrow D_\mu = \partial_\mu \mathds{1} + i g \vec{W}_\mu \cdot \vec{\sigma}
\ee

The the ``D.E''  $(i\gamma_\mu D^\mu - m) \phi = 0$ is invariant.\\

{\Huge Questions ? ... ? }\\

B/c here is where it is going to get weird...

\bc
\fbox{\begin{minipage}{0.9\textwidth}
Want to stress the direct connection between the number of generators and the number of gauge bosons.
\end{minipage}
}
\ec


\textbf{The local gauge symmetry is dictating the particle content.}

}
\end{document}


