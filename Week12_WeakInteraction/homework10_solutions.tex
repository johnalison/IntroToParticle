\documentclass[paper=letter,11pt]{scrartcl}

\KOMAoptions{headinclude=true, footinclude=false}
\KOMAoptions{DIV=14, BCOR=5mm}
\KOMAoptions{numbers=noendperiod}
\KOMAoptions{parskip=half}
\addtokomafont{disposition}{\rmfamily}
\addtokomafont{part}{\LARGE}
\addtokomafont{descriptionlabel}{\rmfamily}
%\setkomafont{pageheadfoot}{\normalsize\sffamily}
\setkomafont{pagehead}{\normalsize\rmfamily}
%\setkomafont{publishers}{\normalsize\rmfamily}
\setkomafont{caption}{\normalfont\small}
\setcapindent{0pt}
\deffootnote[1em]{1em}{1em}{\textsuperscript{\thefootnotemark}\ }


\usepackage{amsmath}
\usepackage[varg]{txfonts}
\usepackage[T1]{fontenc}
\usepackage{graphicx}
\usepackage{xcolor}
\usepackage[american]{babel}
% hyperref is needed in many places, so include it here
\usepackage{hyperref}

\usepackage{xspace}
\usepackage{multirow}
\usepackage{float}


\usepackage{braket}
\usepackage{bbm}
\usepackage{relsize}
\usepackage{tcolorbox}

\def\ketY{\ensuremath{\ket {\Psi}}}
\def\iGeV{\ensuremath{\textrm{GeV}^{-1}}}
%\def\mp{\ensuremath{m_{\textrm{proton}}}}
\def\rp{\ensuremath{r_{\textrm{proton}}}}
\def\me{\ensuremath{m_{\textrm{electron}}}}
\def\aG{\ensuremath{\alpha_G}}
\def\rAtom{\ensuremath{r_{\textrm{atom}}}}
\def\rNucl{\ensuremath{r_{\textrm{nucleus}}}}
\def\GN{\ensuremath{\textrm{G}_\textrm{N}}}
\def\ketX{\ensuremath{\ket{\vec{x}}}}
\def\ve{\ensuremath{\vec{\epsilon}}}


\def\ABCDMatrix{\ensuremath{\begin{pmatrix} A &  B  \\ C  & D \end{pmatrix}}}
\def\xyprime{\ensuremath{\begin{pmatrix} x' \\ y' \end{pmatrix}}}
\def\xyprimeT{\ensuremath{\begin{pmatrix} x' &  y' \end{pmatrix}}}
\def\xy{\ensuremath{\begin{pmatrix} x \\ y \end{pmatrix}}}
\def\xyT{\ensuremath{\begin{pmatrix} x & y \end{pmatrix}}}

\def\IMatrix{\ensuremath{\begin{pmatrix} 0 &  1  \\ -1  & 0 \end{pmatrix}}}
\def\IBoostMatrix{\ensuremath{\begin{pmatrix} 0 &  1  \\ 1  & 0 \end{pmatrix}}}
\def\JThree{\ensuremath{\begin{pmatrix}    0 & -i & 0  \\ i & 0  & 0 \\ 0 & 0 & 0 \end{pmatrix}}} 
\def\JTwo{\ensuremath{\begin{bmatrix}    0 & 0 & -i  \\ 0 & 0  & 0 \\ i & 0 & 0 \end{bmatrix}}}
\def\JOne{\ensuremath{\begin{bmatrix}    0 & 0 & 0  \\ 0 & 0  & -i \\ 0 & i & 0 \end{bmatrix}}}
\def\etamn{\ensuremath{\eta_{\mu\nu}}}
\def\Lmn{\ensuremath{\Lambda^\mu_\nu}}
\def\dmn{\ensuremath{\delta^\mu_\nu}}
\def\wmn{\ensuremath{\omega^\mu_\nu}}
\def\be{\begin{equation*}}
\def\ee{\end{equation*}}
\def\bea{\begin{eqnarray*}}
\def\eea{\end{eqnarray*}}
\def\bi{\begin{itemize}}
\def\ei{\end{itemize}}
\def\fmn{\ensuremath{F_{\mu\nu}}}
\def\fMN{\ensuremath{F^{\mu\nu}}}
\def\bc{\begin{center}}
\def\ec{\end{center}}
\def\nus{$\nu$s}

\def\adagger{\ensuremath{a_{p\sigma}^\dagger}}
\def\lineacross{\noindent\rule{\textwidth}{1pt}}

\newcommand{\multiline}[1] {
\begin{tabular} {|l}
#1
\end{tabular}
}

\newcommand{\multilineNoLine}[1] {
\begin{tabular} {l}
#1
\end{tabular}
}



\newcommand{\lineTwo}[2] {
\begin{tabular} {|l}
#1 \\
#2
\end{tabular}
}

\newcommand{\rmt}[1] {
\textrm{#1}
}


%
% Units
%
\def\m{\ensuremath{\rmt{m}}}
\def\GeV{\ensuremath{\rmt{GeV}}}
\def\pt{\ensuremath{p_\rmt{T}}}


\def\parity{\ensuremath{\mathcal{P}}}

\usepackage{cancel}
\usepackage{ mathrsfs }
\def\bigL{\ensuremath{\mathscr{L}}}

\usepackage{ dsfont }



\usepackage{fancyhdr}
\fancyhf{}

\usepackage{braket}

\def\ketY{\ensuremath{\ket {\Psi}}}
\def\iGeV{\ensuremath{\textrm{GeV}^{-1}}}
\def\mp{\ensuremath{m_{\textrm{proton}}}}
\def\rp{\ensuremath{r_{\textrm{proton}}}}
\def\me{\ensuremath{m_{\textrm{electron}}}}
\def\aG{\ensuremath{\alpha_G}}
\def\rAtom{\ensuremath{r_{\textrm{atom}}}}
\def\rNucl{\ensuremath{r_{\textrm{nucleus}}}}
\def\GN{\ensuremath{\textrm{G}_\textrm{N}}}

\def\be{\begin{equation*}}
\def\ee{\end{equation*}}

\usepackage{graphicx} 


\DeclareGraphicsRule{*}{mps}{*}{} 


\usepackage{fancyhdr}
\usepackage{cancel}
\usepackage{ mathrsfs }

\usepackage{feynmp}



\fancyhf{}
\lhead{\Large 33-444} % \hfill Introduction to Particle Physics \hfill Spring 2020}
\chead{\Large Introduction to Particle Physics} % \hfill Spring 2020}
\rhead{\Large Spring 2020} % \hfill Introduction to Particle Physics \hfill Spring 2020}
\begin{document}
\thispagestyle{fancy}





%\begin{tabular}{c}
%{\large 33-444 \hfill Intro To Particle \hfill Spring 2020\\}
%\hline 
%\end{tabular}

\begin{center}
{\huge \textbf{Homework Set \#10 }}
\large

%{\textbf{ Due Date:} Before class {\textbf{Monday}} April 15th  } 
\end{center}

{\large

\textbf{1) Discrete Symmetries } \hfill \textit{(10 points)}\\

\begin{itemize}
\item[a.]{ Under Parity:
\be
 \vec{E} \rightarrow -\vec{E}, \vec{\triangledown} \rightarrow - \vec{\triangledown}, \rho \rightarrow \rho, \vec{B} \rightarrow\vec{B},\vec{J}\rightarrow-\vec{J}, \frac{\partial}{\partial t} \rightarrow \frac{\partial}{\partial t}
\ee

So 
\be
 \vec{\triangledown} \cdot \vec{E} = \rho,  \vec{\triangledown} \cdot \vec{B} = 0,   \vec{\triangledown} \times \vec{E} = -\frac{\partial \vec{B}}{\partial t} ,  \vec{\triangledown} \times \vec{B} = \vec{J} + \frac{\partial \vec{E}}{\partial t}
\ee
Are invariant
}
\item[b.]{ Under Charge Conjugation
\be
 \vec{E} \rightarrow -\vec{E}, \vec{\triangledown} \rightarrow \vec{\triangledown}, \rho \rightarrow -\rho, \vec{B} \rightarrow -\vec{B},\vec{J}\rightarrow-\vec{J}, \frac{\partial}{\partial t} \rightarrow \frac{\partial}{\partial t}
\ee
So ME are are invariant.
}
\item[c.]{ Under Time Reversal
\be
 \vec{E} \rightarrow \vec{E}, \vec{\triangledown} \rightarrow \vec{\triangledown}, \rho \rightarrow \rho, \vec{B} \rightarrow -\vec{B},\vec{J}\rightarrow-\vec{J}, \frac{\partial}{\partial t} \rightarrow -\frac{\partial}{\partial t}
\ee
So ME are are invariant.
}
\item[d.]{ 
We assume that the Hamiltonian is time independent.  So $\mathcal{T}H = H$

Then
\be
i\frac{\partial}{\partial t} \psi = H\psi   \rightarrow \mathcal{T}(i) \mathcal{T}(\frac{\partial}{\partial t})  \psi = \mathcal{T}(H)\psi
\ee
or
\be
 \mathcal{T}(i) -(\frac{\partial}{\partial t})  \psi = H\psi
\ee

Which is invariant if $\mathcal{T}(i) = -i$
}
\end{itemize}

\vspace*{0.25in}


\textbf{2) Kinematics of the IceCube Experiment. } \hfill \textit{(10 points)}\\

\begin{itemize}
\item[a)]{Read the Wikipedia article on Cherenkov radiation:  \href{https://en.wikipedia.org/wiki/Cherenkov_radiation}{https://en.wikipedia.org/wiki/Cherenkov\_radiation}}
\item[b)]{
$E = E_{\nu}$, $m = m_N$ 
\be
p_\mu = \frac{E m}{m+E(1-\cos\theta)}(1,0,sin\theta, cos\theta)
\ee

\be
p_n = \left( E + m - \frac{Em}{m+E(1-\cos\theta)} , 0, -\frac{Em\sin\theta}{m+E(1-\cos\theta)}, E - \frac{Em\cos\theta}{m+E(1-\cos\theta)} \right)
\ee


}
\item[c)]{

\be
|p_\mu| = \frac{E m}{m+E(1-\cos\theta)}
\ee
and so
\be
E = \frac{m |p_\mu|}{m-|p_\mu|(1-\cos\theta)}
\ee


\be
E_{smallest} = |p_\mu|
\ee

There is no largest E, the denominator vanishes when
\be
1-\cos\theta = \frac{m}{|p_\mu|}
\ee

For a given $|p_\mu|$ the

}
\item[d)]{

This result then tells us what the minimum and maximum angles between the anti-neutrino and anti-muon momenta could be. 
Of course, their momentum could be collinear, with $\theta_\mathrm{min} = 0$, which corresponds to the case when their energies are also equal. 
Even in the case in which the anti-neutrino has an infinite energy, the maximum scattering angle is

\be
1-\cos\theta = \frac{m}{|p_\mu|} \sim \frac{\theta_\mathrm{max}^2}{2}
\ee

So the maximum angle would be
\be
\theta_\mathrm{max} \sim \sqrt{2\times\frac{m}{|p_\mu|}} \sim \sqrt{2\times\frac{1}{2\times10^{6}}} \sim 0.001
\ee
}
\item[e)]{

\begin{eqnarray*}
\begin{fmffile}{simple_labels}
  \begin{fmfgraph*}(110,60)
    \fmflabel{$\bar{\nu}_\mu$}{i1}
    \fmflabel{p}{i2}
    \fmflabel{n}{o2}
    \fmflabel{$\mu^+$}{o1}
    \fmfleft{i1,i2}
    \fmfright{o1,o2}
    \fmf{plain}{i1,v1}
    \fmf{plain}{i2,v1}
    \fmf{plain}{v1,o2}
    \fmf{plain}{v1,o1}
    \fmfdot{v1}
    \fmflabel{  $\sim \frac{1}{m_W^2}$}{v1}
  \end{fmfgraph*}
\end{fmffile}
\end{eqnarray*}
}
\item[f)]{
\be
\sigma \propto |M|^2 \propto \frac{1}{m_W^4}
\ee
To get the right units need two powers of mass dimension
\be
\sigma \sim  \frac{m_p^2}{m_W^4}
\ee

}

\item[g)]{
The number of events per unit time that are observed is a product of the luminosity and the cross section:
\be
events/time = n_\nu n_N Vol c \sigma
\ee
where $n_\nu$ and $n_N$ is the number density of neutrinos and nucleons in IceCube, respectively, and Vol is the total volume of ice of IceCube. 
The number density of nucleons times the volume of IceCube is just the number of nucleons in IceCube, $N_N$ , so the event rate is also
\be
events/time = n_\nu N_N c \sigma 
\ee
To calculate the total number of events detected, we just multiply both sides of this expression by the total exposure time $t_{exp}$:
\be
events = n_\nu N_N c t_{exp} \sigma
\ee

Now, the total number of neutrinos $N_\nu$ is the number density times the total volume of neutrinos that passes through IceCube during the exposure time. 
Because neutrinos travel at the speed of light, the length of this volume $l = c t_{exp}$ and the cross sectional area is just the cross sectional area of IceCube, A.
Using these results, we can express the total number of observed events in the simple form of
\be
events = \frac{N_\nu N_N}{A} \sigma
\ee
or
\be
N_\nu = \frac{events\times A}{N_N \times \sigma}
\ee

Now, we just need to calculate each of these components.

To estimate the cross section from part f) 
\be
\sigma \sim   \frac{m_p^2}{m_W^4} \sim \frac{1^2}{(100)^4} \textrm{GeV}^{-2} \sim 10^{-8} \textrm{GeV}^{-2}
\ee
Now,
\be
GeV^{-1} \sim 0.2 fm
\ee

\be
\sigma \sim   10^{-8} \times 0.04 fm^{2} \sim  10^{-10} \times 10^{-30} m^2 \sim 10^{-40} m^2
\ee


The volume of IceCube is
\be
Vol = 1km^3 = 10^9 m^3 
\ee
The density of ice is about 1000 kg/m3, and so the total mass of the ice in IceCube is 
\be
m_{ice} =10^3 \times 10^9 kg = 10^{12} kg = 10^{12} \times 10^{27} GeV 
\ee
Which means $N_N \sim  10^{39}$.

The cross sectional area of this volume of neutrinos A is the cross sectional area of IceCube, which is approximately
\be
A = 1 km^2 = 10^6 m^2
\ee

So
\be
N_\nu = \frac{events\times A}{N_N \times \sigma} = \frac{3 \times 10^6 m^2}{10^{39} \times 10^{-40} m^2} \sim 3\times10^{5} \sim 3\times10^{7}
\ee

(Note: actually more appropriate to use $v\sim 250$ GeV instead of $m_W$ in the cross section estimate.  This gives $\sigma \sim 10^{-42} m^2$ and $N_\nu \sim 10^9$, or a few billion $\nu$s)
}
\end{itemize}



}
\end{document}
