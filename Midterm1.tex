
Yukawa’s Theory. In the 1930s, Hideki Yukawa predicted the existence of a new particle, now called the pion. It was theorized to be responsible for binding protons and neutrons together in atomic nuclei. Based on the size of an atomic nucleus, Yukawa was able to estimate the mass of the pion.1 Estimate the mass of the pion from the assumption that the relevant distance scale is the radius of an atomic nucleus of about 1 femtometer (10−15 meters). Express the mass in eV.




\textbf{3) Creating Black Holes } \hfill \textit{(5 points)}
\begin{itemize}
\item[(a)] With a large enough particle accelerator we can create black holes. 
(Assume a black hole is an 
 
\item[(b)] What are you estimated values in mks units ?
\item[(c)]Compare your estimates to actual values for Neutron Stars quoted online.
\item[(d)]Look up $m_{\textrm{neutron}}$. How does this compare with the assumption of $\mp \sim m_{\textrm{neutron}}$?
\end{itemize}

\vspace*{0.25in}
