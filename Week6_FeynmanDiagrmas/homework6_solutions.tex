\documentclass[paper=letter,11pt]{scrartcl}

\KOMAoptions{headinclude=true, footinclude=false}
\KOMAoptions{DIV=14, BCOR=5mm}
\KOMAoptions{numbers=noendperiod}
\KOMAoptions{parskip=half}
\addtokomafont{disposition}{\rmfamily}
\addtokomafont{part}{\LARGE}
\addtokomafont{descriptionlabel}{\rmfamily}
%\setkomafont{pageheadfoot}{\normalsize\sffamily}
\setkomafont{pagehead}{\normalsize\rmfamily}
%\setkomafont{publishers}{\normalsize\rmfamily}
\setkomafont{caption}{\normalfont\small}
\setcapindent{0pt}
\deffootnote[1em]{1em}{1em}{\textsuperscript{\thefootnotemark}\ }


\usepackage{amsmath}
\usepackage[varg]{txfonts}
\usepackage[T1]{fontenc}
\usepackage{graphicx}
\usepackage{xcolor}
\usepackage[american]{babel}
% hyperref is needed in many places, so include it here
\usepackage{hyperref}

\usepackage{xspace}
\usepackage{multirow}
\usepackage{float}


\usepackage{braket}
\usepackage{bbm}
\usepackage{relsize}
\usepackage{tcolorbox}

\def\ketY{\ensuremath{\ket {\Psi}}}
\def\iGeV{\ensuremath{\textrm{GeV}^{-1}}}
%\def\mp{\ensuremath{m_{\textrm{proton}}}}
\def\rp{\ensuremath{r_{\textrm{proton}}}}
\def\me{\ensuremath{m_{\textrm{electron}}}}
\def\aG{\ensuremath{\alpha_G}}
\def\rAtom{\ensuremath{r_{\textrm{atom}}}}
\def\rNucl{\ensuremath{r_{\textrm{nucleus}}}}
\def\GN{\ensuremath{\textrm{G}_\textrm{N}}}
\def\ketX{\ensuremath{\ket{\vec{x}}}}
\def\ve{\ensuremath{\vec{\epsilon}}}


\def\ABCDMatrix{\ensuremath{\begin{pmatrix} A &  B  \\ C  & D \end{pmatrix}}}
\def\xyprime{\ensuremath{\begin{pmatrix} x' \\ y' \end{pmatrix}}}
\def\xyprimeT{\ensuremath{\begin{pmatrix} x' &  y' \end{pmatrix}}}
\def\xy{\ensuremath{\begin{pmatrix} x \\ y \end{pmatrix}}}
\def\xyT{\ensuremath{\begin{pmatrix} x & y \end{pmatrix}}}

\def\IMatrix{\ensuremath{\begin{pmatrix} 0 &  1  \\ -1  & 0 \end{pmatrix}}}
\def\IBoostMatrix{\ensuremath{\begin{pmatrix} 0 &  1  \\ 1  & 0 \end{pmatrix}}}
\def\JThree{\ensuremath{\begin{pmatrix}    0 & -i & 0  \\ i & 0  & 0 \\ 0 & 0 & 0 \end{pmatrix}}} 
\def\JTwo{\ensuremath{\begin{bmatrix}    0 & 0 & -i  \\ 0 & 0  & 0 \\ i & 0 & 0 \end{bmatrix}}}
\def\JOne{\ensuremath{\begin{bmatrix}    0 & 0 & 0  \\ 0 & 0  & -i \\ 0 & i & 0 \end{bmatrix}}}
\def\etamn{\ensuremath{\eta_{\mu\nu}}}
\def\Lmn{\ensuremath{\Lambda^\mu_\nu}}
\def\dmn{\ensuremath{\delta^\mu_\nu}}
\def\wmn{\ensuremath{\omega^\mu_\nu}}
\def\be{\begin{equation*}}
\def\ee{\end{equation*}}
\def\bea{\begin{eqnarray*}}
\def\eea{\end{eqnarray*}}
\def\bi{\begin{itemize}}
\def\ei{\end{itemize}}
\def\fmn{\ensuremath{F_{\mu\nu}}}
\def\fMN{\ensuremath{F^{\mu\nu}}}
\def\bc{\begin{center}}
\def\ec{\end{center}}
\def\nus{$\nu$s}

\def\adagger{\ensuremath{a_{p\sigma}^\dagger}}
\def\lineacross{\noindent\rule{\textwidth}{1pt}}

\newcommand{\multiline}[1] {
\begin{tabular} {|l}
#1
\end{tabular}
}

\newcommand{\multilineNoLine}[1] {
\begin{tabular} {l}
#1
\end{tabular}
}



\newcommand{\lineTwo}[2] {
\begin{tabular} {|l}
#1 \\
#2
\end{tabular}
}

\newcommand{\rmt}[1] {
\textrm{#1}
}


%
% Units
%
\def\m{\ensuremath{\rmt{m}}}
\def\GeV{\ensuremath{\rmt{GeV}}}
\def\pt{\ensuremath{p_\rmt{T}}}


\def\parity{\ensuremath{\mathcal{P}}}

\usepackage{cancel}
\usepackage{ mathrsfs }
\def\bigL{\ensuremath{\mathscr{L}}}

\usepackage{ dsfont }



\usepackage{fancyhdr}
\fancyhf{}

\usepackage{braket}

\def\ketY{\ensuremath{\ket {\Psi}}}
\def\iGeV{\ensuremath{\textrm{GeV}^{-1}}}
%\def\mp{\ensuremath{m_{\textrm{proton}}}}
\def\rp{\ensuremath{r_{\textrm{proton}}}}
\def\me{\ensuremath{m_{\textrm{electron}}}}
\def\aG{\ensuremath{\alpha_G}}
\def\rAtom{\ensuremath{r_{\textrm{atom}}}}
\def\rNucl{\ensuremath{r_{\textrm{nucleus}}}}
\def\GN{\ensuremath{\textrm{G}_\textrm{N}}}

\def\be{\begin{equation*}}
\def\ee{\end{equation*}}


\usepackage{fancyhdr}
\usepackage{cancel}
\usepackage{ mathrsfs }





\fancyhf{}
\lhead{\Large 33-444} % \hfill Introduction to Particle Physics \hfill Spring 2019}
\chead{\Large Introduction to Particle Physics} % \hfill Spring 2019}
\rhead{\Large Spring 2019} % \hfill Introduction to Particle Physics \hfill Spring 2019}
\begin{document}
\thispagestyle{fancy}





%\begin{tabular}{c}
%{\large 33-444 \hfill Intro To Particle \hfill Spring 2019\\}
%\hline 
%\end{tabular}

\begin{center}
{\huge \textbf{Homework Set \#6}}
\large

{\textbf{ Solutions  } }
\end{center}

{\large

\textbf{1) $e^+ e^- \rightarrow \mu^+ \mu^-$ scattering   \hfill \textit{(10 points)} }

For high-energy scattering, the cross section scales as $\frac{1}{E_{CM}^2}$.
The constant of proportionality is given by the overlap of the spin wave functions.


\be
\braket{s_3 s_4 | s_1 s_2}
\ee
where $s_3$ and $s_4$ represent the final state muons spins and $s_1$ and $s_2$ represent the initial state electron spins.

\be
\braket{s_3 s_4 | s_1 s_2} = \sum_{\textrm{$\gamma$-spins}} \braket{s_3 s_4 | \epsilon_\gamma}  \braket{\epsilon_\gamma | s_1 s_2} 
\ee

In the circular polarization basis, there are four possible initial state spin combinations

\be
\ket{s_1 s_2} = \ket{LL} , \ket{RR},  \ket{LR}, or \ket{RL} 
\ee
However only the first two give combine to give spin one, which is needed to project on to the intermidate state photon. 
There are two photon polarization vectors $\epsilon_L (= \frac{1}{\sqrt{2}} \begin{pmatrix} 0, 1, -i, 0 \end{pmatrix})$ and  $\epsilon_R (= \frac{1}{\sqrt{2}} \begin{pmatrix} 0, 1, i, 0 \end{pmatrix})$.
By conservation of angular momentum, $\ket{s_1 s_2} = \ket{LL}$ will project onto $\epsilon_L$ and $\ket{s_1 s_2} = \ket{RR}$ will project onto $\epsilon_R$ .
This gives us the $\braket{\epsilon_\gamma | s_1 s_2}$ term in the sum over spins.

To calculate the other term we need to workout product of the photon polarizations with the muon spins.
As for the electrons, the combined muons spins must sum to one and be orthognal to the muon direction of motion. 
We can describe the muon scattering in terms of the scattering angle $\theta$:

\be
p_3 = E(1,0,sin(\theta),cos(\theta))
\ee

\be
p_4 = E(1,0,-sin(\theta),-cos(\theta))
\ee

The left and right-handed polarization vectors prepindicular to these four vectors is given by

\be
\begin{pmatrix} 1 & & & \\ & 1 & & \\ & & cos(\theta) & sin(\theta) \\ & & -sin(\theta) & cos(\theta) \end{pmatrix}  \frac{1}{\sqrt{2}} \begin{pmatrix} 0\\ 1 \\ \pm i \\ 0 \end{pmatrix} =  \frac{1}{\sqrt{2}} \begin{pmatrix} 0\\ 1 \\ \pm i cos(\theta) \\ \mp i sin(\theta) \end{pmatrix}
\ee

Where the upper sign corrisponds to the right-handed muon polarization ($\equiv \epsilon_R'$) and lower sign corrisponds to the left-handed muon polarization ($\equiv \epsilon_L'$).
There are four possible projections to cosider:

\be
\braket{LL|\epsilon_L} = {\epsilon_L'}^\mu {\epsilon_L}_\mu = - \frac{1}{2} (1 - cos(\theta))
\ee
\be
\braket{RR|\epsilon_L} = {\epsilon_R'}^\mu {\epsilon_L}_\mu = - \frac{1}{2} (1 + cos(\theta))
\ee
\be
\braket{LL|\epsilon_R} = {\epsilon_L'}^\mu {\epsilon_R}_\mu = - \frac{1}{2} (1 + cos(\theta))
\ee
\be
\braket{RR|\epsilon_R} = {\epsilon_R'}^\mu {\epsilon_R}_\mu = - \frac{1}{2} (1 - cos(\theta))
\ee

If we dont detect the muon spins and our initial beams are unpolarized we have to sum over initial and final state Matrix elements so in the end

\be
|\mathscr{M}_{tot}|^2 = 
\frac{1}{4}( 1 -2cos(\theta) + cos(\theta)^2) + 
\frac{1}{4}( 1 +2cos(\theta) + cos(\theta)^2) + 
\frac{1}{4}( 1 +2cos(\theta) + cos(\theta)^2) + 
\frac{1}{4}( 1 -2cos(\theta) + cos(\theta)^2) 
\ee
\be
= (1 + cos(\theta)^2) 
\ee
which agrees with what we found in class using a different basis.

\vspace*{0.25in}

\textbf{2)  $\gamma e \rightarrow \gamma e$ scattering \hfill \textit{(5 points)}}

(See next page...)




}





\end{document}
