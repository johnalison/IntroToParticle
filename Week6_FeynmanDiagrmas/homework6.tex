\input{latexSetup}
\usepackage{braket}

\def\ketY{\ensuremath{\ket {\Psi}}}
\def\iGeV{\ensuremath{\textrm{GeV}^{-1}}}
\def\mp{\ensuremath{m_{\textrm{proton}}}}
\def\rp{\ensuremath{r_{\textrm{proton}}}}
\def\me{\ensuremath{m_{\textrm{electron}}}}
\def\aG{\ensuremath{\alpha_G}}
\def\rAtom{\ensuremath{r_{\textrm{atom}}}}
\def\rNucl{\ensuremath{r_{\textrm{nucleus}}}}
\def\GN{\ensuremath{\textrm{G}_\textrm{N}}}

\def\be{\begin{equation*}}
\def\ee{\end{equation*}}


\usepackage{fancyhdr}
\usepackage{cancel}
\usepackage{ mathrsfs }





\fancyhf{}
\lhead{\Large 33-444} % \hfill Introduction to Particle Physics \hfill Spring 2020}
\chead{\Large Introduction to Particle Physics} % \hfill Spring 2020}
\rhead{\Large Spring 2020} % \hfill Introduction to Particle Physics \hfill Spring 2020}
\begin{document}
\thispagestyle{fancy}





%\begin{tabular}{c}
%{\large 33-444 \hfill Intro To Particle \hfill Spring 2020\\}
%\hline 
%\end{tabular}

\begin{center}
{\huge \textbf{Homework Set \#6}}
\large

{\textbf{ Due Date:} Before class Friday March 1st  } 
\end{center}

{\large

\textbf{1) $e^+ e^- \rightarrow \mu^+ \mu^-$ scattering   \hfill \textit{(10 points)} }

We calculated $\frac{d\sigma}{d\Omega}$ for $e^+ e^- \rightarrow \mu^+ \mu^-$ scattering in class using linear (x \& y) polarizations.
Repeat the calculation using circular polarization. 


\vspace*{0.25in}

\textbf{2)  $\gamma e \rightarrow \gamma e$ scattering \hfill \textit{(5 points)}}

\begin{itemize}

\item[a)]{ Draw all the leading order diagrams for $\gamma e \rightarrow \gamma e$.  
Label the momenta of the ingoing and outgoing particles and the internal propagators.
}
\item[b)]{What power of the coupling coupling constant is each graph proportional to?
}
\item[c)]{
What power of the coupling constant would the cross section be proportional to?
}
\item[d)]{
Second order diagrams contain loops of particles within them.  Draw the diagrams associated with the second order correction to $\gamma e \rightarrow \gamma e$.
Only consider $\gamma$ and $e$ internal lines. 
}
\item[e)]{
What power of the coupling constant is each of the second order diagrams proportional to ?
}
\end{itemize}


}





\end{document}
