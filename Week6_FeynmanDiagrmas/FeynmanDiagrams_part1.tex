\input{../latexSetup}

\lhead{\Large 33-444} % \hfill Introduction to Particle Physics \hfill Spring 2022}
\chead{\Large Introduction to Particle Physics} % \hfill Spring 2022}
\rhead{\Large Spring 2022} % \hfill Introduction to Particle Physics \hfill Spring 2022}

\begin{document}
\thispagestyle{fancy}

\begin{center}
{\huge \textbf{Lecture 13}}
\end{center}

{\fontsize{14}{16}\selectfont


\underline{\Large Cross Sections And Decay Rates}

20th century witnessed the development of collider physics. 
Effective means to determine which particles exist and their properties and interactions.

\begin{itemize}
\item[-] Rutherford discovery of the nucleus using $\alpha$ 1911 
\item[-] Andersen's discovery of anti-electrons 1932
\end{itemize}

These were made with ``Natural accelerators'' $\alpha$'s or cosmic rays

Around 1930 man made collisions started winning. 

eg: 1 MeV 

Now 13 TeV at the LHC.

Collisions map free fixed momenta initial states $\rightarrow$ final fixed momentum states. \\

QFT predicts \underline{probability} for projections to occur. 

Probabilities typically dependant on parameters (angles, momenta, etc) 

\begin{center}
$P(v_1, ... v_n)$ - differential probabilities
\end{center}

Given by 
\be
|\braket{\psi_{final}, +\infty|\psi_{initial}, -\infty}|^2
\ee


\be
\braket{f|S|i} \hspace{1in} \textrm{S-matrix}
\ee
QFT tell us how to calculate S given some Lagrangian (next week)

S-matrix elements are the primary object of interest for particle physics. 

S-matrix elements always calculated pertubatively

\be
S = \underbrace{1}_{\textrm{free theory}} +  \underbrace{iT}_{\textrm{perturbatively small}}
\ee



\underline{QFT + Lagrangian} gives a procedure (recipe) for calculating S\\
\hspace*{0.3in} Very nice interpretation in terms of picture ``Feynman diagrams''


\underline{\underline{QM Perturbation Theory}}

$H = H_0 + V$ 

Remember we are interested in how some free state at early times ($-\infty$) evolve to some (potentially) other free state at late times. 

At early times have a state with a given energy E, which is an eigenstate of $H_0$

\be
H_0\ket{\phi} = E\ket{\phi}
\ee

Including the interaction piece, will also be eigenstate of the full Hamiltonian with the same energy.

\be
H\ket{\psi} = E\ket{\psi}
\ee

Now we can formally write,
\be
\ket{\psi} = \ket{\phi} + \frac{1}{E-H_0}V\ket{\psi}
\ee
which can be verified by multiplying through by $(E-H_0)$.

Whats happening here is that the interaction at intermediate times is inducing transitions among the states $\ket{\phi}$, which are non-interacting at early (and late) times. 

So the full state $\ket{\psi}$ is given by the free state $\ket{\phi}$ plus a scattering term. 



Really want to express the full state $\ket{\psi}$ entirely in terms of $\ket{\phi}$.\\
\hspace*{0.3in} We do this by defining operator T:  $V\ket{\psi} = T\ket{\phi}$.

This gives us
\be
\ket{\psi} = \ket{\phi} + \frac{1}{E-H_0}T\ket{\phi}
\ee
or
\bea
V\ket{\psi} &=& V\ket{\phi} + V\frac{1}{E-H_0}T\ket{\phi} \\
&=& T\ket{\phi}
\eea

So we get a nice iterate equation for T
\be
T = V + V\frac{1}{E-H_0}T
\ee
which we can solve perturbatively in V.

eg
\be
T = V + V\frac{1}{E-H_0}V + V\frac{1}{E-H_0}V \frac{1}{E-H_0}V + ...
\ee

Of course, we are interested in inner products of these with the initial/final states

\be
\underbrace{\braket{\phi_f|T|\phi_i}}_{T_{fi}} = \underbrace{\braket{\phi_f|V|\phi_i}}_{V_{fi}} + \sum_j \frac{\braket{\phi_f|V|\phi_j}\braket{\phi_j|V|\phi_i}}{E-H_0} + ...
\ee
 
\underline{Example of scattering of 2 electrons}

\be
\ket{i} = \ket{e_1, e_2} \hspace*{1in}  \ket{f} = \ket{e_3, e_4}
\ee


\be
T_{fi} = V_{fi} + \sum_n V_{fn} \frac{1}{E_i-E_n} V_{ni} + ...
\ee

\bi
\item[-] $E_i$- is the initial (= final) energy
\item[-] $E_f$ - is the energy of the intermediate state
\ei

Now, the $\sum_n$ runs over energy thing in the Hilbert (Fock) space, however only certain states will be non-0.

In relativistic theory, the action-at-a-distance of EM is replaced by a process where 2 electrons interact with a $\gamma$ which travels at c. 
(This tells us there should be $\gamma$ in the intermediate state)


\be
V \sim e \int d^3x \psi \phi \psi  \hspace*{0.3in} \textrm{(Ignoring spin)}
\ee
this operator will have terms that go like ($\sim a_{e_3}^\dagger\ a_{\gamma}^\dagger\ a_{e_1}$)


However, here all terms involve $a_{\gamma}^\dagger$.\\
Because $\ket{i}$ and $\ket{f}$ do not contain a $\gamma$, $  V_{fi} = 0$

to get a non-zero term, we need $\ket{n}$ with a photon. 
\begin{figure}[h]
\centering
\includegraphics[width=0.99\textwidth]{./eeScattering.pdf}
\end{figure}



\be
T_{fi} = \frac{\braket{e_3 e_4|V|e_3 \gamma e_2}\braket{e_3 \gamma e_2|V|e_1 e_2 }}{(E_1+E_2) - (E_3 + E_4 + E_\gamma)} + \textrm{(2nd term)}
\ee
Note: $E_n \ne E_i$ which is allowed by uncertainty principle.


Look at 
\be
\braket{e_3 \gamma e_2|V|e_1 e_2} = \braket{\gamma e_3 |V|e_1}
\ee
(up to overall normalization from $\braket{e_2|e_2}$.




}
\end{document}

