\documentclass[margin,line]{res}

%\topmargin -.5in
\oddsidemargin -.5in
\evensidemargin -.5in
\textwidth=6.0in
\itemsep=0in
\parsep=0in

\newenvironment{list1}{
  \begin{list}{\ding{113}}{%
      \setlength{\itemsep}{0in}
      \setlength{\parsep}{0in} \setlength{\parskip}{0in}
      \setlength{\topsep}{0in} \setlength{\partopsep}{0in} 
      \setlength{\leftmargin}{0.17in}}}{\end{list}}
\newenvironment{list2}{
  \begin{list}{$\bullet$}{%
      \setlength{\itemsep}{0in}
      \setlength{\parsep}{0in} \setlength{\parskip}{0in}
      \setlength{\topsep}{0in} \setlength{\partopsep}{0in} 
      \setlength{\leftmargin}{0.2in}}}{\end{list}}

\usepackage{amssymb}
\renewcommand{\labelitemi}{$\bullet$}


\newcommand{\MYhref}[3][blue]{\href{#2}{\color{#1}{#3}}}%
\usepackage{xcolor}
\usepackage[colorlinks = true,
            linkcolor = blue,
            urlcolor  = blue,
            citecolor = blue,
            anchorcolor = blue]{hyperref}


\newcommand{\litem}{\item[\Large\textbf{-}]}

\usepackage{etoolbox}
\newtoggle{isCV}
\toggletrue{isCV}


\begin{document}

\name{\huge Introduction to Nuclear and Particle Physics (33444) \vspace*{.1in}}


\begin{resume}
{ \textit{\large``Not only God knows, I know, and by the end of the semester, you will know.''}\\ \hspace*{5in} -Sidney Coleman}

\section{Course Description:}
This course is an introduction to elementary particle physics, the description of Nature at the shortest distance scales. 
This class will emphasize the theoretical underpinnings of the Standard Model of particle physics and its experimental verification.
The first part of the course will focus on the implications of combining Quantum Mechanics and Special Relativity.
This union turns out to incredibly restrict the types of theories and particle interactions allowed.
In the second part of the course we will focus on the experimental methods for measuring elementary particle interactions and highlight the structure of the Standard Model.
With this introduction, we will survey various areas of active research including: the study of the Higgs boson, neutrino mixing, and the physics of flavor. 
Emphasis will be placed on what we do and do not know now and what we might know in 30 years.


\section{Professor:}
John Alison\\
Office: 7420 Wean\\
Email: johnalison@cmu.edu

\section{Prerequisites:} 
Intermediate Electricity and Magnetism I (33338)\\
Quantum Physics (33234)

\section{Lectures:}
Doherty Hall A325\\
 Monday, Wednesday,  Friday, 8:30-9:20 am 

\section{Office Hours:}
Tuesday 3-5 pm (Wean 7420)\\
Thursday, 3-5 pm (Wean 7420)


\section{Supplemental References:}
Mark Thomson, ``Modern Particle Physics'', Cambridge, 2013.
A modern, particle physics textbook for undergraduates.

Mark Thomson, ``Modern Particle Physics'', Cambridge, 2013.
A modern, particle physics textbook for undergraduates.

David Griffiths, ``Introduction to Elementary Particles'', Wiley, 2008.
An elementary introduction from a more historical perspective.

\section{Course Website:}  XXX

\section{Course Requirements:}
There will be three graded aspects of this course: $\sim$weekly homework problems and two in-class exams.
In calculating your final grade, your lowest homework score will be dropped.
Homework will develop methods discussed in lecture or flesh out sketches of arguements and unproven claims made in class.
Late homework can not be accepted.

\section{Grading:} 
The amounts to which the homework, final exam, and oral exam contributes to
your grade are:
\begin{center}
\begin{tabular}{lc}
Homework & 50\%\\
Mid-term Exam & 20\%\\
Final Exam & 30\%\\
\end{tabular}
\end{center}

\section{Preliminary Schedule:}

The following is a rough outline of topics we will discuss each week. 
This is a work in progress and will change as the semester goes on.

%\begin{tabular}

%\end{tabular}

\end{resume}
\end{document}




