\documentclass[margin,line]{res}

%\topmargin -.5in
\oddsidemargin -.5in
\evensidemargin -.5in
\textwidth=6.0in
\itemsep=0in
\parsep=0in

\newenvironment{list1}{
  \begin{list}{\ding{113}}{%
      \setlength{\itemsep}{0in}
      \setlength{\parsep}{0in} \setlength{\parskip}{0in}
      \setlength{\topsep}{0in} \setlength{\partopsep}{0in} 
      \setlength{\leftmargin}{0.17in}}}{\end{list}}
\newenvironment{list2}{
  \begin{list}{$\bullet$}{%
      \setlength{\itemsep}{0in}
      \setlength{\parsep}{0in} \setlength{\parskip}{0in}
      \setlength{\topsep}{0in} \setlength{\partopsep}{0in} 
      \setlength{\leftmargin}{0.2in}}}{\end{list}}

\usepackage{amssymb}
\renewcommand{\labelitemi}{$\bullet$}


\newcommand{\MYhref}[3][blue]{\href{#2}{\color{#1}{#3}}}%
\usepackage{xcolor}
\usepackage[colorlinks = true,
            linkcolor = blue,
            urlcolor  = blue,
            citecolor = blue,
            anchorcolor = blue]{hyperref}


\newcommand{\litem}{\item[\Large\textbf{-}]}

\usepackage{etoolbox}
\newtoggle{isCV}
\toggletrue{isCV}


\begin{document}

\name{\huge Introduction to Nuclear and Particle Physics (33444) \vspace*{.1in}}


\begin{resume}
{ \textit{\large``Not only God knows, I know, and by the end of the semester, you will know.''}\\ \hspace*{5in} -Sidney Coleman}

\section{Course Description:}
This course is an introduction to elementary particle physics, the description of Nature at the shortest distance scales. 
This class will emphasize the theoretical underpinnings of the Standard Model of particle physics and its experimental verification.
The first part of the course will focus on the implications of combining Quantum Mechanics and Special Relativity.
This union turns out to incredibly restrict the types of theories and particle interactions allowed.
In the second part of the course we will focus on the experimental methods for measuring elementary particle interactions and highlight the structure of the Standard Model.
With this introduction, we will survey various areas of active research including: the study of the Higgs boson, neutrino mixing, and the physics of flavor. 
Emphasis will be placed on what we do and do not know now and what we might know in 30 years.


\section{Professor:}
John Alison\\
Office: 7420 Wean\\
Email: johnalison@cmu.edu

\section{Prerequisites:} 
Intermediate Electricity and Magnetism I (33338)\\
Quantum Physics (33234)

\section{Lectures:}
Doherty Hall A325\\
 Monday, Wednesday,  Friday, 8:30-9:20 am 

\section{Office Hours:}
By appointment.



\section{Supplemental References:}
There are no required texts -- all reading material you need will be covered in the course and I will make my notes available on the course webpage. 
However it is often helpful in this subject to get other points of view.  
I have found the following books useful. 

Matthew Schwartz, ``Quantum Field Theory and the Standard Model'', Cambridge, 2014.\\
A modern QFT and particle physics textbook for undergraduates.

Brian Martin and Graham Shaw, ``Particle Physics'', 4th Edition , Wiley, 2017.\\
A modern particle physics textbook for undergraduates.

David Griffiths, ``Introduction to Elementary Particles'', Wiley, 2008.\\
An elementary introduction from a more historical perspective.

\section{Course Website:}  
\href{https://canvas.cmu.edu/courses/8695}{https://canvas.cmu.edu/courses/8695}

\section{Course Requirements:}
There will be three graded aspects of this course: $\sim$weekly homework problems and two in-class exams.
In calculating your final grade, your lowest homework score will be dropped.
Homework will develop methods discussed in lecture or flesh out sketches of arguements and unproven claims made in class.
Late homework can not be accepted.

\section{Grading:} 
The amounts to which the homework, final exam, and oral exam contributes to
your grade are:
\begin{center}
\begin{tabular}{lc}
Homework & 40\%\\
Mid-term Exam & 20\%\\
Final Exam & 40\%\\
\end{tabular}
\end{center}

\section{Homework Collaboration Policy:} 

Speak to each other!
You often learn just as effectively through discussion with your peers as you do from lecture. 
However, all homework you submit should be written individually and independently by you.
When it comes to discussing homework, any records of the discussion must be destroyed before you make any kind of write-up. 
If you have collaborated with anyone then you should declare who you worked with and the nature of your discussion. 
\textit{(Think citation for a publication)}
You will not be penalized for \underline{declared} homework collaboration. 
Undeclared collaboration is plagiarism and is considered cheating. 
All the work you submit should be your own work and reflect your understanding.


\section{Preliminary Schedule:}

The following is a rough outline of topics we will discuss each week. 
This is a work in progress and will change as the semester goes on.

\begin{tabular}{lll}
\textbf{Week 1}  & Big picture / Correct way to think about the world & Jan 14/16/18 \\
\textbf{Week 2}  & Relativity Refresher / Symmetries /  Lie Algebras & Jan -/23/25 \\
\textbf{Week 3}  & Quantum Refresher / QM + SR   & Jan 28/30/Feb 1 \\
%\textbf{Week 4}  &  & Feb 4/6/8\\
%\textbf{Week 5}  &  & Feb 11/13/15\\
%\textbf{Week 6}  &  & Feb 18/20/22\\
%\textbf{Week 7}  &  & Feb 25/27/Mar 1\\
%\textbf{Week 8}  &  & Mar 4/6/- \\
%\textbf{Week 9}  & Spring Break & Mar 11/13/15\\
%\textbf{Week 10} &  & Mar 18/20/22 \\
%\textbf{Week 11} &  & Mar 25/27/29\\
%\textbf{Week 12} &  & Apr 1/3/5\\
%\textbf{Week 13} &  & Apr 8/10/-\\
%\textbf{Week 14} &  & Apr 15/17/19\\
%\textbf{Week 15} &  & Apr 22/24/26\\
%\textbf{Week 16} &  & Apr 29/May 1/3\\
\end{tabular}

\end{resume}
\end{document}




