\input{latexSetup}
\usepackage{braket}

\def\ketY{\ensuremath{\ket {\Psi}}}
\def\iGeV{\ensuremath{\textrm{GeV}^{-1}}}
\def\mp{\ensuremath{m_{\textrm{proton}}}}
\def\rp{\ensuremath{r_{\textrm{proton}}}}
\def\me{\ensuremath{m_{\textrm{electron}}}}
\def\aG{\ensuremath{\alpha_G}}
\def\rAtom{\ensuremath{r_{\textrm{atom}}}}
\def\rNucl{\ensuremath{r_{\textrm{nucleus}}}}
\def\GN{\ensuremath{\textrm{G}_\textrm{N}}}


\usepackage{fancyhdr}

\fancyhf{}
\lhead{\Large 33-444} % \hfill Introduction to Particle Physics \hfill Spring 2019}
\chead{\Large Introduction to Particle Physics} % \hfill Spring 2019}
\rhead{\Large Spring 2019} % \hfill Introduction to Particle Physics \hfill Spring 2019}
\begin{document}
\thispagestyle{fancy}





%\begin{tabular}{c}
%{\large 33-444 \hfill Intro To Particle \hfill Spring 2019\\}
%\hline 
%\end{tabular}

\begin{center}
{\huge \textbf{Homework Set \#1}}
\large

{\textbf{ Due Date:} Before class Friday January 25th  }
\end{center}

{\large
\textbf{1) Show that SO(2) $\simeq$ U(1) } \hfill \textit{(2 points)}
%\begin{itemize}
%\item[(a)]What is your major/minor ? 
%\item[(b)]When do you plan on graduating?
%\item[(c)]What do you want to do after graduation ? (eg: grad school ? if so, what subject ? if not, what industry?)
%\item[(d)]What do you most want to get out of this course ? 
%\end{itemize}

\vspace*{0.25in}

\textbf{2) Algrebra of the Lorentz group} \hfill \textit{(5 points)}
%\begin{itemize}
%\item[(a)] Assume a solid is composed of closely packed atoms. What is the spacing between atoms?  Express your result in terms of  $\alpha$, \aG, \mp, and \me.
%\item[(b)] If you wanted to study the crystal structure of a solid material with $Z\sim10$ using light, what wavelength of photons would you need ?
%Express your result in terms of  $\alpha$, \aG, \mp, and \me.
%\item[(c)] Where in the spectrum of EM radiation do these photons lie?
%\end{itemize}

\vspace*{0.25in}


\textbf{3) Connection to $\beta$s and  $\gamma$s} \hfill \textit{(5 points)}

\vspace*{0.25in}

\textbf{4) Z->ee }

a) assume e massless, what are the energies and momenta of electrons. 

b) include the effects of the electron mass.
     what is the size of the correction ? 

c) bquarks are the heaviest thing that the z-boson can decay into. the bquark has a mass of ~5GeV. 
     what is the size of the finite b mass correction ?

\vspace*{0.25in}

\textbf{5)  GZK cutoff energy}


Rapidity. In experimental particle physics, it is often very useful to express the direction of motion of a particle in terms of its rapidity y. The rapidity is defined as
y= 1logE+pz , (2.159) 2 E − pz
for a particle with energy E and z-component of momentum pz. What makes rapidity so nice is its simple properties under Lorentz transformation. Perform a Lorentz boost of the energy and momentum along the zˆ axis with velocity β. How does the rapidity transform under this boost? You should be able to write the Lorentz-boosted rapidity as a simple function of the original rapidity.


\end{document}
