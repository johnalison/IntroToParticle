\documentclass[paper=letter,11pt]{scrartcl}

\KOMAoptions{headinclude=true, footinclude=false}
\KOMAoptions{DIV=14, BCOR=5mm}
\KOMAoptions{numbers=noendperiod}
\KOMAoptions{parskip=half}
\addtokomafont{disposition}{\rmfamily}
\addtokomafont{part}{\LARGE}
\addtokomafont{descriptionlabel}{\rmfamily}
%\setkomafont{pageheadfoot}{\normalsize\sffamily}
\setkomafont{pagehead}{\normalsize\rmfamily}
%\setkomafont{publishers}{\normalsize\rmfamily}
\setkomafont{caption}{\normalfont\small}
\setcapindent{0pt}
\deffootnote[1em]{1em}{1em}{\textsuperscript{\thefootnotemark}\ }


\usepackage{amsmath}
\usepackage[varg]{txfonts}
\usepackage[T1]{fontenc}
\usepackage{graphicx}
\usepackage{xcolor}
\usepackage[american]{babel}
% hyperref is needed in many places, so include it here
\usepackage{hyperref}

\usepackage{xspace}
\usepackage{multirow}
\usepackage{float}


\usepackage{braket}
\usepackage{bbm}
\usepackage{relsize}
\usepackage{tcolorbox}

\def\ketY{\ensuremath{\ket {\Psi}}}
\def\iGeV{\ensuremath{\textrm{GeV}^{-1}}}
%\def\mp{\ensuremath{m_{\textrm{proton}}}}
\def\rp{\ensuremath{r_{\textrm{proton}}}}
\def\me{\ensuremath{m_{\textrm{electron}}}}
\def\aG{\ensuremath{\alpha_G}}
\def\rAtom{\ensuremath{r_{\textrm{atom}}}}
\def\rNucl{\ensuremath{r_{\textrm{nucleus}}}}
\def\GN{\ensuremath{\textrm{G}_\textrm{N}}}
\def\ketX{\ensuremath{\ket{\vec{x}}}}
\def\ve{\ensuremath{\vec{\epsilon}}}


\def\ABCDMatrix{\ensuremath{\begin{pmatrix} A &  B  \\ C  & D \end{pmatrix}}}
\def\xyprime{\ensuremath{\begin{pmatrix} x' \\ y' \end{pmatrix}}}
\def\xyprimeT{\ensuremath{\begin{pmatrix} x' &  y' \end{pmatrix}}}
\def\xy{\ensuremath{\begin{pmatrix} x \\ y \end{pmatrix}}}
\def\xyT{\ensuremath{\begin{pmatrix} x & y \end{pmatrix}}}

\def\IMatrix{\ensuremath{\begin{pmatrix} 0 &  1  \\ -1  & 0 \end{pmatrix}}}
\def\IBoostMatrix{\ensuremath{\begin{pmatrix} 0 &  1  \\ 1  & 0 \end{pmatrix}}}
\def\JThree{\ensuremath{\begin{pmatrix}    0 & -i & 0  \\ i & 0  & 0 \\ 0 & 0 & 0 \end{pmatrix}}} 
\def\JTwo{\ensuremath{\begin{bmatrix}    0 & 0 & -i  \\ 0 & 0  & 0 \\ i & 0 & 0 \end{bmatrix}}}
\def\JOne{\ensuremath{\begin{bmatrix}    0 & 0 & 0  \\ 0 & 0  & -i \\ 0 & i & 0 \end{bmatrix}}}
\def\etamn{\ensuremath{\eta_{\mu\nu}}}
\def\Lmn{\ensuremath{\Lambda^\mu_\nu}}
\def\dmn{\ensuremath{\delta^\mu_\nu}}
\def\wmn{\ensuremath{\omega^\mu_\nu}}
\def\be{\begin{equation*}}
\def\ee{\end{equation*}}
\def\bea{\begin{eqnarray*}}
\def\eea{\end{eqnarray*}}
\def\bi{\begin{itemize}}
\def\ei{\end{itemize}}
\def\fmn{\ensuremath{F_{\mu\nu}}}
\def\fMN{\ensuremath{F^{\mu\nu}}}
\def\bc{\begin{center}}
\def\ec{\end{center}}
\def\nus{$\nu$s}

\def\adagger{\ensuremath{a_{p\sigma}^\dagger}}
\def\lineacross{\noindent\rule{\textwidth}{1pt}}

\newcommand{\multiline}[1] {
\begin{tabular} {|l}
#1
\end{tabular}
}

\newcommand{\multilineNoLine}[1] {
\begin{tabular} {l}
#1
\end{tabular}
}



\newcommand{\lineTwo}[2] {
\begin{tabular} {|l}
#1 \\
#2
\end{tabular}
}

\newcommand{\rmt}[1] {
\textrm{#1}
}


%
% Units
%
\def\m{\ensuremath{\rmt{m}}}
\def\GeV{\ensuremath{\rmt{GeV}}}
\def\pt{\ensuremath{p_\rmt{T}}}


\def\parity{\ensuremath{\mathcal{P}}}

\usepackage{cancel}
\usepackage{ mathrsfs }
\def\bigL{\ensuremath{\mathscr{L}}}

\usepackage{ dsfont }



\usepackage{fancyhdr}
\fancyhf{}

\usepackage{braket}

\def\ketY{\ensuremath{\ket {\Psi}}}
\def\iGeV{\ensuremath{\textrm{GeV}^{-1}}}
\def\mp{\ensuremath{m_{\textrm{proton}}}}
\def\rp{\ensuremath{r_{\textrm{proton}}}}
\def\me{\ensuremath{m_{\textrm{electron}}}}
\def\aG{\ensuremath{\alpha_G}}
\def\rAtom{\ensuremath{r_{\textrm{atom}}}}
\def\rNucl{\ensuremath{r_{\textrm{nucleus}}}}
\def\GN{\ensuremath{\textrm{G}_\textrm{N}}}

\def\be{\begin{equation*}}
\def\ee{\end{equation*}}

\usepackage{fancyhdr}

\fancyhf{}
\lhead{\Large 33-444} % \hfill Introduction to Particle Physics \hfill Spring 2020}
\chead{\Large Introduction to Particle Physics} % \hfill Spring 2020}
\rhead{\Large Spring 2020} % \hfill Introduction to Particle Physics \hfill Spring 2020}
\begin{document}
\thispagestyle{fancy}





%\begin{tabular}{c}
%{\large 33-444 \hfill Intro To Particle \hfill Spring 2020\\}
%\hline 
%\end{tabular}

\begin{center}
{\huge \textbf{Homework Set \#2}}
\large

{\textbf{ Solutions}   }
\end{center}

{\large
\textbf{1) Show that SO(2) $\simeq$ U(1) } \hfill \textit{(2 points)}
\begin{itemize}
\item[(a)] {
\be
zz^* = (x+iy)(x-iy) = x^2 + y^2
\ee
Given a vector in the complex plain specified by (x,y), $zz^*$ gives the length of the vector.
}
\item[(b)] {
\begin{align*}
M(\theta_1): z\rightarrow e^{i\theta_1}z\ \textrm{(+ complex conjugate)}\\
M(\theta_2): z\rightarrow e^{i\theta_2}z\ \textrm{(+ complex conjugate)}\\
M(\theta_1)M(\theta_2): z\rightarrow e^{i\theta_1}e^{i\theta_2}z = e^{i(\theta_1+\theta_2)}z = M(\theta_1 + \theta_2)
\end{align*}
}
\end{itemize}

\vspace*{0.25in}

\textbf{2) Work out the algebra of the generators of the Lorentz group} \hfill \textit{(5 points)}
Assuming: 

\be
J_1 = \begin{pmatrix} 0 & 0 & 0 & 0 \\ 0 & 0 & 0 & 0 \\ 0 & 0 & 0 & 1 \\ 0 & 0 & -1 & 0 \end{pmatrix} \hspace{0.5in}
J_2 = \begin{pmatrix} 0 & 0 & 0 & 0 \\ 0 & 0 & 0 & 1 \\ 0 & 0 & 0 & 0 \\ 0 & -1 & 0 & 0 \end{pmatrix} \hspace{0.5in}
J_3 = \begin{pmatrix} 0 & 0 & 0 & 0 \\ 0 & 0 & 1 & 0 \\ 0 & -1 & 0 & 0 \\ 0 & 0 & 0 & 0 \end{pmatrix} 
\ee


\begin{itemize}
\item[]{
\begin{align*}
[T_i,T_i] = 0 \\
[T_1,T_2] = J_3\\
[T_1,T_3] = J_2\\
[T_2,T_3] = J_1\\
\textrm{OR}\\
[T_i,T_j] = \epsilon_{ijk} J_k  \hspace{0.5in} J_2 \rightarrow -J_2
\end{align*}


\begin{align*}
[T_1,J_1] = 0  \\
[T_1,J_2] = -T_3 \\
[T_1,J_3] = T_2 \\
\\
[T_2,J_1] = T_3\\
[T_2,J_2] = 0 \\
[T_2,J_3] = -T_1 \\
\\
[T_3,J_1] = -T_2 \\
[T_3,J_2] = T_1 \\
[T_3,J_3] = 0 \\
\textrm{OR}\\
[T_i,J_j] = -\epsilon_{ijk} T_k  \hspace{0.5in} J_2 \rightarrow -J_2
\end{align*}

\begin{align*}
[J_i,J_j] = \epsilon_{ijk} J_k 
\end{align*}


}

\end{itemize}
%Estimate the size of life forms on earth in terms of $\alpha$, \aG, \mp, and \me.
%Assume life is a solid. 

\vspace*{0.25in}


\textbf{3) Connection to $\beta$s and  $\gamma$s} \hfill \textit{(5 points)}
\begin{itemize}
\item[(a)]{

\begin{equation*}
e^{I\eta} = 1 + I\eta + \frac{I^2\eta^2}{2!} + \frac{I^3\eta^3}{3!} + \frac{I^4\eta^4}{4!} + \dots
\end{equation*}

$I_B^2 = \begin{bmatrix} 1 & 0  \\ 0 & 1 \end{bmatrix} $

Show that $B(\eta) = e^{I_B\eta} = \cosh(\eta)+ I_B \sinh(\eta)$


\begin{equation*}
e^{I\eta} = I\left(\eta + \frac{I^2\eta^3}{3!} + \frac{I^4\eta^5}{5!} + \dots \right) + \left(1 + \frac{I^2\eta^2}{2!} + \frac{I^4\eta^4}{4!}  + \dots \right)
\end{equation*}

\begin{equation*}
e^{I\eta} = I\left(\eta + \frac{\eta^3}{3!} + \frac{\eta^5}{5!} + \dots \right) + \left(1 + \frac{\eta^2}{2!} + \frac{\eta^4}{4!}  + \dots \right) \\
= I \sinh(\eta) + \cosh(\eta)
\end{equation*}

}

\item[(b)]{
The origin of the primed frame is at $x'=0$ in the prime frame and at $x=vt$ in the unprimed frame (assuming the origins coincided at t=0)

\be
x = t'\sinh(\eta) \textrm{ and } t = t'\cosh(\eta)
\ee

\be
v=\frac{x}{t} = \tanh(\eta)  \textrm{ and } \cosh^{-2} = 1 - \tanh^2 
\ee

\be
\Rightarrow \cosh(\eta) = \frac{1}{\sqrt{1-v^2}} \equiv \gamma 
\ee

\be
\sinh(\eta) = \frac{v}{\sqrt{1-v^2}} = \beta\gamma 
\ee

}


\end{itemize}

\vspace*{0.25in}


\textbf{4) Z Boson decays}\hfill \textit{(5 points)}}
\begin{itemize}
\item[(a)]{
Same derivation as we did for the $\pi\rightarrow \gamma\gamma$ decay in class gives
\be
p_{e_1} = (m_Z/2,0,0,m_Z/2) \hspace{1in} p_{\gamma_2} = (m_Z/2,0,0,-m_Z/2)
\ee
}
\item[(b)]{
Including the mass term gives...

\be
p_{e_1} = (E_1,0,0,P_1) \hspace{1in} p_{\gamma_2} = (E_2,0,0,-P_2)
\ee
Momentum conservation implies $P_1 = -P_1 \equiv P$\\
Energy conservation implies $M_Z = E_1 + E2$ \\
So,

\be
p_{e_1} = (E,0,0,P) \hspace{1in} p_{\gamma_2} = (m_Z-E,0,0,-P)
\ee
Imposing ${P_1^e}^2 = m_e$ gives: $E = \sqrt{m_e^2 + P^2}$\\
Imposing ${P_2^e}^2 = m_e$ gives: $P^2 = (m_Z-E)^2 - m_e^2$\\
Combining implies, 
\be
E = \frac{m_Z}{2} \hspace{0.5in} \textrm{ and } \hspace{0.5in}P = \frac{m_Z}{2}\sqrt{1-4\frac{m_e^2}{m_Z^2}}
\ee

So no correction to the electron Energies.\\
The the electron momentum is $P \simeq \frac{m_Z}{2}\left(1-\frac{2m_e^2}{m_Z^2}\right)$, which gives a correction of order  $\left(\frac{m_e}{m_Z}\right)^2 \sim \left(\frac{10^{-3}\ \textrm{GeV}}{100\ \textrm{GeV}}\right)^2 \sim 10^{-10}$
\item[(b)]{
Including the mass term for the b-quark gives...\\
No correction to the energies. \\
The b-quark momentum has a correction of order  $\left(\frac{m_b}{m_Z}\right)^2 \sim \left(\frac{10\ \textrm{GeV}}{100\ \textrm{GeV}}\right)^2 \sim 10^{-2}$ about 1\%.
}


}
\end{itemize}

\vspace*{0.25in}

\textbf{5)  GZK cutoff energy} \hfill \textit{(5 points)}
\begin{itemize}
\item[(a)]{
$(P+\gamma) \rightarrow (p + \pi_0)$

The final momentum in the center of mass frame is:
\be
P_{F}^\mu = (m_p + m_\pi, 0, 0, 0 )
\ee

\be
P_{F}^2 = (m_p + m_\pi)^2 = m_p^2 + 2 m_p  m_\pi + m_\pi^2
\ee

The initial four vector is given by the sum of the proton and CMB photon four vectors.
\be
P_{I} = (P_p^\mu + P_\gamma^\mu)
\ee

\be
P_{I}^2 = \underbrace{P_p^2}_{= m_p^2} + 2 P_p \cdot P_\gamma + \underbrace{P_\gamma^2}_{= 0}
\ee
To evaluate $P_p \cdot P_\gamma$, can use reference frame where the photon and proton are colliding head on.

(Assume for the moment that we can neglect the proton mass compared to it momentum ...
\be
P_{p} = (p_p, 0 , 0, p_p)  \hspace{0.4in} P_{\gamma} = (E_{CMB}, 0 , 0, -E_{CMB})
\ee
Where $E_{CMB} = 3 \cdot 10^{-13} $GeV.

\be
P_p\cdot P_\gamma = 2 p_p E_{CMB} 
\ee

Imposing $P_I^2 = P_F^2$, allows us to solve for $p_p$.

\be
m_p^2 + 4 p_p E_{CMB}  = m_p^2 + 2 m_p  m_\pi + m_\pi^2
\ee

or

\be
p_p = \frac{ 2 m_p  m_\pi + m_\pi^2}{4 E_{CMB}} = \frac{2\cdot1\cdot0.14 + 0.14^2}{10\cdot10^{-13}} \textrm{GeV} \sim 3\times 10^{11} \textrm{GeV} \sim 10^{20} \textrm{eV}
\ee


}
\item[(b)]{
$\sim$ 30 J
}

\end{itemize}



\end{document}
