\documentclass[paper=letter,11pt]{scrartcl}

\KOMAoptions{headinclude=true, footinclude=false}
\KOMAoptions{DIV=14, BCOR=5mm}
\KOMAoptions{numbers=noendperiod}
\KOMAoptions{parskip=half}
\addtokomafont{disposition}{\rmfamily}
\addtokomafont{part}{\LARGE}
\addtokomafont{descriptionlabel}{\rmfamily}
%\setkomafont{pageheadfoot}{\normalsize\sffamily}
\setkomafont{pagehead}{\normalsize\rmfamily}
%\setkomafont{publishers}{\normalsize\rmfamily}
\setkomafont{caption}{\normalfont\small}
\setcapindent{0pt}
\deffootnote[1em]{1em}{1em}{\textsuperscript{\thefootnotemark}\ }


\usepackage{amsmath}
\usepackage[varg]{txfonts}
\usepackage[T1]{fontenc}
\usepackage{graphicx}
\usepackage{xcolor}
\usepackage[american]{babel}
% hyperref is needed in many places, so include it here
\usepackage{hyperref}

\usepackage{xspace}
\usepackage{multirow}
\usepackage{float}


\usepackage{braket}
\usepackage{bbm}
\usepackage{relsize}
\usepackage{tcolorbox}

\def\ketY{\ensuremath{\ket {\Psi}}}
\def\iGeV{\ensuremath{\textrm{GeV}^{-1}}}
%\def\mp{\ensuremath{m_{\textrm{proton}}}}
\def\rp{\ensuremath{r_{\textrm{proton}}}}
\def\me{\ensuremath{m_{\textrm{electron}}}}
\def\aG{\ensuremath{\alpha_G}}
\def\rAtom{\ensuremath{r_{\textrm{atom}}}}
\def\rNucl{\ensuremath{r_{\textrm{nucleus}}}}
\def\GN{\ensuremath{\textrm{G}_\textrm{N}}}
\def\ketX{\ensuremath{\ket{\vec{x}}}}
\def\ve{\ensuremath{\vec{\epsilon}}}


\def\ABCDMatrix{\ensuremath{\begin{pmatrix} A &  B  \\ C  & D \end{pmatrix}}}
\def\xyprime{\ensuremath{\begin{pmatrix} x' \\ y' \end{pmatrix}}}
\def\xyprimeT{\ensuremath{\begin{pmatrix} x' &  y' \end{pmatrix}}}
\def\xy{\ensuremath{\begin{pmatrix} x \\ y \end{pmatrix}}}
\def\xyT{\ensuremath{\begin{pmatrix} x & y \end{pmatrix}}}

\def\IMatrix{\ensuremath{\begin{pmatrix} 0 &  1  \\ -1  & 0 \end{pmatrix}}}
\def\IBoostMatrix{\ensuremath{\begin{pmatrix} 0 &  1  \\ 1  & 0 \end{pmatrix}}}
\def\JThree{\ensuremath{\begin{pmatrix}    0 & -i & 0  \\ i & 0  & 0 \\ 0 & 0 & 0 \end{pmatrix}}} 
\def\JTwo{\ensuremath{\begin{bmatrix}    0 & 0 & -i  \\ 0 & 0  & 0 \\ i & 0 & 0 \end{bmatrix}}}
\def\JOne{\ensuremath{\begin{bmatrix}    0 & 0 & 0  \\ 0 & 0  & -i \\ 0 & i & 0 \end{bmatrix}}}
\def\etamn{\ensuremath{\eta_{\mu\nu}}}
\def\Lmn{\ensuremath{\Lambda^\mu_\nu}}
\def\dmn{\ensuremath{\delta^\mu_\nu}}
\def\wmn{\ensuremath{\omega^\mu_\nu}}
\def\be{\begin{equation*}}
\def\ee{\end{equation*}}
\def\bea{\begin{eqnarray*}}
\def\eea{\end{eqnarray*}}
\def\bi{\begin{itemize}}
\def\ei{\end{itemize}}
\def\fmn{\ensuremath{F_{\mu\nu}}}
\def\fMN{\ensuremath{F^{\mu\nu}}}
\def\bc{\begin{center}}
\def\ec{\end{center}}
\def\nus{$\nu$s}

\def\adagger{\ensuremath{a_{p\sigma}^\dagger}}
\def\lineacross{\noindent\rule{\textwidth}{1pt}}

\newcommand{\multiline}[1] {
\begin{tabular} {|l}
#1
\end{tabular}
}

\newcommand{\multilineNoLine}[1] {
\begin{tabular} {l}
#1
\end{tabular}
}



\newcommand{\lineTwo}[2] {
\begin{tabular} {|l}
#1 \\
#2
\end{tabular}
}

\newcommand{\rmt}[1] {
\textrm{#1}
}


%
% Units
%
\def\m{\ensuremath{\rmt{m}}}
\def\GeV{\ensuremath{\rmt{GeV}}}
\def\pt{\ensuremath{p_\rmt{T}}}


\def\parity{\ensuremath{\mathcal{P}}}

\usepackage{cancel}
\usepackage{ mathrsfs }
\def\bigL{\ensuremath{\mathscr{L}}}

\usepackage{ dsfont }



\usepackage{fancyhdr}
\fancyhf{}

%\documentclass[margin,line]{res}
\usepackage{braket}
\usepackage{bbm}
\usepackage{relsize}

\def\ketY{\ensuremath{\ket {\Psi}}}
\def\iGeV{\ensuremath{\textrm{GeV}^{-1}}}


\def\ABCDMatrix{\ensuremath{\begin{pmatrix} A &  B  \\ C  & D \end{pmatrix}}}
\def\xyprime{\ensuremath{\begin{pmatrix} x' \\ y' \end{pmatrix}}}
\def\xyprimeT{\ensuremath{\begin{pmatrix} x' &  y' \end{pmatrix}}}
\def\xy{\ensuremath{\begin{pmatrix} x \\ y \end{pmatrix}}}
\def\xyT{\ensuremath{\begin{pmatrix} x & y \end{pmatrix}}}

\def\IMatrix{\ensuremath{\begin{pmatrix} 0 &  1  \\ -1  & 0 \end{pmatrix}}}
\def\IBoostMatrix{\ensuremath{\begin{pmatrix} 0 &  1  \\ 1  & 0 \end{pmatrix}}}
\def\JThree{\ensuremath{\begin{pmatrix}    0 & -i & 0  \\ i & 0  & 0 \\ 0 & 0 & 0 \end{pmatrix}}} 
\def\JTwo{\ensuremath{\begin{bmatrix}    0 & 0 & -i  \\ 0 & 0  & 0 \\ i & 0 & 0 \end{bmatrix}}}
\def\JOne{\ensuremath{\begin{bmatrix}    0 & 0 & 0  \\ 0 & 0  & -i \\ 0 & i & 0 \end{bmatrix}}}
\def\etamn{\ensuremath{\eta_{\mu\nu}}}
\def\Lmn{\ensuremath{\Lambda^\mu_\nu}}
\def\dmn{\ensuremath{\delta^\mu_\nu}}
\def\wmn{\ensuremath{\omega^\mu_\nu}}
\def\be{\begin{equation*}}
\def\ee{\end{equation*}}

%\def\xMu{\ensuremath{x^\mu}

\usepackage{fancyhdr}

\fancyhf{}
\lhead{\Large 33-444} % \hfill Introduction to Particle Physics \hfill Spring 2020}
\chead{\Large Introduction to Particle Physics} % \hfill Spring 2020}
\rhead{\Large Spring 2020} % \hfill Introduction to Particle Physics \hfill Spring 2020}

\begin{document}
\thispagestyle{fancy}

\begin{center}
{\huge \textbf{Lecture 5}}
\end{center}

{\fontsize{14}{16}\selectfont

\textbf{\underline{Review Kinematics}} Should be all old hat...needed for next homework so I thought it would be good to go through explicitly together.

\underline{3-vectors}

\be
\vec{v} = \begin{pmatrix} v_x \\ v_y \\ v_z \end{pmatrix}  
\ee

\begin{align*}
\vec{v}\cdot\vec{w} = v_x w_x + v_y w_y + v_z w_z\\
 = \underbrace{|\vec{v}|}_{|\vec{v}| = \sqrt{\vec{v}\cdot\vec{v}}}|\vec{w}|\cos \Delta \theta(\vec{v},\vec{w})
\end{align*}

\underline{4-vectors}

\be
v^\mu = \begin{pmatrix} v_0 \\ v_1 \\ v_2 \\ v_3 \end{pmatrix}   = \begin{pmatrix} v_0 \\ \vec{v} \end{pmatrix}
\ee


\begin{align*}
v^\mu w_\mu = v_0 w_0 - (v_x w_x + v_y w_y + v_z w_z) \\
= v_0 w_0 - (\vec{v} \cdot \vec{w})
\end{align*}

Roman vs Greek indices: 
$\mu\nu\rho\sigma$ etc usually means 0,1,2,3 (index full space-time) 
$i j k$ etc usually means 1,2,3 (eg: just the spatial components)

\underline{Relativity Energy Momentum 4-vectors}  ``Particle'' 4-vectors

\be
p^\mu =  \begin{pmatrix} E \\ \vec{p} \end{pmatrix}  
\ee
with the constraint, $E = \sqrt{|\vec{p}|^2 + M^2}$


\begin{align*}
p^\mu p_\mu =  E^2 - \vec{p}\cdot\vec{p} \\
= |\vec{p}|^2 + M^2 - |\vec{p}|^2 = M^2
\end{align*}

\underline{Massive Particles have a CoM frame:}

\be
p^\mu =  \begin{pmatrix} M \\ \vec{0} \end{pmatrix}  
\ee

\underline{Massless Particles do not !}
Best you can do is:

\be
p^\mu =  \begin{pmatrix} E \\ \begin{pmatrix} E \\ 0 \\0  \end{pmatrix}    \end{pmatrix}  
\ee
Can move (rotate) E to any component.

Check 
\begin{align*}
p^\mu p_\mu =  E^2 - E^2 = 0
\end{align*}

\noindent\rule{\textwidth}{1pt}
A common problem that we want to analyze in particle physics is the decay of an unstable particle. 
The vast majority of particles decay to two or more particles. 
The only particles for which there is no evidence that they decay are electrons, photons, and protons. 

Example: Pion decay
The pion is a natural that decays to two photons. 
Interesting that natural particle decays to photons, 
Now known to be composed of up and down quarks.


The pion has a mass of about 135 MeV and in its decay to photons energy and momentum are conserved. 
(We will ignore effects of angular momentum conservation for now.)
 How can we understand this decay using four-vectors?



Four-vectors depend on the frame in which they are evaluated, so we need to pick a frame to analyze this decay. 
The pion is massive so we can always boost to the frame where the pion is at rest. 
In this frame, its momentum four-vector, or four-momentum, is:

\be
p_\pi = \begin{pmatrix} m_\pi \\ \vec{0} \end{pmatrix}  
\ee

That is, when the pion is at rest, its energy is just set by the pion mass, $m_\pi$. 
Now, we want to determine the four-momentum of the two photons in the pion decay. 
We can again perform a rotation (= Lorentz transformation) to have the photons travel along the z-axis. 
By momentum conservation, the photons must be traveling back-to-back from the pion decay:

gamma <--- pi ---> gamma


In this frame, we can express the four-vectors of the two photons as:\\
\be
p_{\gamma_1} = (E_{\gamma_1},0,0,p^z_{\gamma_1}) \hspace{1in} p_{\gamma_2} = (E_{\gamma_2},0,0,p^z_{\gamma_2})
\ee

By energy and momentum conservation the sum of the four-vectors of the photons must add up to the pion’s four-momentum:

\be
p_\pi = p_{\gamma_1} + p_{\gamma_2}
\ee

This is actually a system of four equations, two of which are just 0 = 0. 

The non- trivial equations are conservation of energy and conservation of the z-component of momentum:

\be
m_\pi =E_{\gamma_1} +E_{\gamma_2} \hspace{1in} 0=p^z_{\gamma_1} +p^z_{\gamma_2} .
\ee

2nd equation gives: $p^z_{\gamma_1} = -p^z_{\gamma_2} \equiv p^z$

\be
p_{\gamma_1} = (E_{\gamma_1},0,0,p^z) \hspace{1in} p_{\gamma_2} = (E_{\gamma_2},0,0,-p^z)
\ee

1st equation gives: $E \equiv E_{\gamma_1}$

\be
p_{\gamma_1} = (E,0,0,p^z) \hspace{1in} p_{\gamma_2} = (E-m_\pi,0,0,-p^z)
\ee

Photons are massless: $\Rightarrow p_{\gamma_{1,2}}^2 = 0$



\be
p_{\gamma_1} = (m_\pi/2,0,0,m_\pi/2) \hspace{1in} p_{\gamma_2} = (m_\pi/2,0,0,-m_\pi/2)
\ee


Now Review Quatum Mechanics ... 

%p2π =(pγ1 +pγ2)2 =p2γ1 +p2γ2 +2pγ1 ·pγ2 =2pγ1 ·pγ2 .

}
\end{document}


