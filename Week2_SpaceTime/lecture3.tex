\documentclass[paper=letter,11pt]{scrartcl}

\KOMAoptions{headinclude=true, footinclude=false}
\KOMAoptions{DIV=14, BCOR=5mm}
\KOMAoptions{numbers=noendperiod}
\KOMAoptions{parskip=half}
\addtokomafont{disposition}{\rmfamily}
\addtokomafont{part}{\LARGE}
\addtokomafont{descriptionlabel}{\rmfamily}
%\setkomafont{pageheadfoot}{\normalsize\sffamily}
\setkomafont{pagehead}{\normalsize\rmfamily}
%\setkomafont{publishers}{\normalsize\rmfamily}
\setkomafont{caption}{\normalfont\small}
\setcapindent{0pt}
\deffootnote[1em]{1em}{1em}{\textsuperscript{\thefootnotemark}\ }


\usepackage{amsmath}
\usepackage[varg]{txfonts}
\usepackage[T1]{fontenc}
\usepackage{graphicx}
\usepackage{xcolor}
\usepackage[american]{babel}
% hyperref is needed in many places, so include it here
\usepackage{hyperref}

\usepackage{xspace}
\usepackage{multirow}
\usepackage{float}


\usepackage{braket}
\usepackage{bbm}
\usepackage{relsize}
\usepackage{tcolorbox}

\def\ketY{\ensuremath{\ket {\Psi}}}
\def\iGeV{\ensuremath{\textrm{GeV}^{-1}}}
%\def\mp{\ensuremath{m_{\textrm{proton}}}}
\def\rp{\ensuremath{r_{\textrm{proton}}}}
\def\me{\ensuremath{m_{\textrm{electron}}}}
\def\aG{\ensuremath{\alpha_G}}
\def\rAtom{\ensuremath{r_{\textrm{atom}}}}
\def\rNucl{\ensuremath{r_{\textrm{nucleus}}}}
\def\GN{\ensuremath{\textrm{G}_\textrm{N}}}
\def\ketX{\ensuremath{\ket{\vec{x}}}}
\def\ve{\ensuremath{\vec{\epsilon}}}


\def\ABCDMatrix{\ensuremath{\begin{pmatrix} A &  B  \\ C  & D \end{pmatrix}}}
\def\xyprime{\ensuremath{\begin{pmatrix} x' \\ y' \end{pmatrix}}}
\def\xyprimeT{\ensuremath{\begin{pmatrix} x' &  y' \end{pmatrix}}}
\def\xy{\ensuremath{\begin{pmatrix} x \\ y \end{pmatrix}}}
\def\xyT{\ensuremath{\begin{pmatrix} x & y \end{pmatrix}}}

\def\IMatrix{\ensuremath{\begin{pmatrix} 0 &  1  \\ -1  & 0 \end{pmatrix}}}
\def\IBoostMatrix{\ensuremath{\begin{pmatrix} 0 &  1  \\ 1  & 0 \end{pmatrix}}}
\def\JThree{\ensuremath{\begin{pmatrix}    0 & -i & 0  \\ i & 0  & 0 \\ 0 & 0 & 0 \end{pmatrix}}} 
\def\JTwo{\ensuremath{\begin{bmatrix}    0 & 0 & -i  \\ 0 & 0  & 0 \\ i & 0 & 0 \end{bmatrix}}}
\def\JOne{\ensuremath{\begin{bmatrix}    0 & 0 & 0  \\ 0 & 0  & -i \\ 0 & i & 0 \end{bmatrix}}}
\def\etamn{\ensuremath{\eta_{\mu\nu}}}
\def\Lmn{\ensuremath{\Lambda^\mu_\nu}}
\def\dmn{\ensuremath{\delta^\mu_\nu}}
\def\wmn{\ensuremath{\omega^\mu_\nu}}
\def\be{\begin{equation*}}
\def\ee{\end{equation*}}
\def\bea{\begin{eqnarray*}}
\def\eea{\end{eqnarray*}}
\def\bi{\begin{itemize}}
\def\ei{\end{itemize}}
\def\fmn{\ensuremath{F_{\mu\nu}}}
\def\fMN{\ensuremath{F^{\mu\nu}}}
\def\bc{\begin{center}}
\def\ec{\end{center}}
\def\nus{$\nu$s}

\def\adagger{\ensuremath{a_{p\sigma}^\dagger}}
\def\lineacross{\noindent\rule{\textwidth}{1pt}}

\newcommand{\multiline}[1] {
\begin{tabular} {|l}
#1
\end{tabular}
}

\newcommand{\multilineNoLine}[1] {
\begin{tabular} {l}
#1
\end{tabular}
}



\newcommand{\lineTwo}[2] {
\begin{tabular} {|l}
#1 \\
#2
\end{tabular}
}

\newcommand{\rmt}[1] {
\textrm{#1}
}


%
% Units
%
\def\m{\ensuremath{\rmt{m}}}
\def\GeV{\ensuremath{\rmt{GeV}}}
\def\pt{\ensuremath{p_\rmt{T}}}


\def\parity{\ensuremath{\mathcal{P}}}

\usepackage{cancel}
\usepackage{ mathrsfs }
\def\bigL{\ensuremath{\mathscr{L}}}

\usepackage{ dsfont }



\usepackage{fancyhdr}
\fancyhf{}


\lhead{\Large 33-444} % \hfill Introduction to Particle Physics \hfill Spring 2020}
\chead{\Large Introduction to Particle Physics} % \hfill Spring 2020}
\rhead{\Large Spring 2020} % \hfill Introduction to Particle Physics \hfill Spring 2020}

\begin{document}
\thispagestyle{fancy}

\begin{center}
{\huge \textbf{Lecture 3}}
\end{center}

{\fontsize{14}{16}\selectfont


\section*{Special Relativity}

Talking about relativity means talking about Lorentz invariance.

If there is a point in space time: \\
%xrightarrow[\text{world}]
\begin{center}
$(t,x) \xrightarrow[\text{observer}]{\text{another moving}} (t', x')$
\end{center}

Invariant notion of distance:
 
\begin{equation*}
t^2 - x^2 = t'^2 - x'^2
\end{equation*}

(This should all be familiar to you.)

We will re-cap this in an adult way...

\section*{\underline{Start with Rotations}}

Have an invariant notion of length of $\vec{r}$

\begin{eqnarray*}
x^2 + y^2 = x'^2 + y'^2  \hspace{1in} & x' = x\ cos(\theta) - y\ sin(\theta) \\
                          &  y' = y\ cos(\theta) + x\ sin(\theta)
\end{eqnarray*}

Look at this another way ... 

\begin{equation*}
\xyprime = \ABCDMatrix \xy  
\end{equation*}
We are after the set of all of matrices such that  $x^2 + y^2 = x'^2 + y'^2$

Start with the infinitesimal case
\begin{equation*}
\ABCDMatrix = \begin{pmatrix} 1 &  0  \\ 0 & 1 \end{pmatrix} + \epsilon \begin{pmatrix} a & b  \\ c & d \end{pmatrix}
\end{equation*}

We require:
$\xyprimeT \xyprime = \xyT \xy $ 

Multiplying through:
\begin{eqnarray*}
x' = x + \epsilon a x + \epsilon b y \\
y' = y + \epsilon  c x + \epsilon d y
\end{eqnarray*}


Keeping terms linear in $\epsilon$.

\begin{eqnarray*}
\textstyle
x'^2 + y'^2 = x^2 + y^2  + 2\epsilon\underbrace{(ax^2 + bxy + cxy + dy^2)}_{ =0 \hspace*{0.2in} \forall x \& y}
\end{eqnarray*}

so 

\begin{eqnarray*}
\textstyle
ax^2 + bxy + cxy + dy^2 = 0
\end{eqnarray*}

\begin{eqnarray*}
\textstyle
\Rightarrow a = d = 0  \hspace*{0.2in} \&   \hspace*{0.2in}  \underbrace{b=-c}_{\text{Can re-scale $\epsilon$ such that b=1}}
\end{eqnarray*}

So 

\begin{eqnarray*}
\xyprime = \left[ \mathlarger{\mathlarger{\mathbbm{1}}} + \epsilon   \IMatrix \right]\xy
\end{eqnarray*}

\section*{\underline{More Sophisticated Way:}}

$x_i$ where $i = 1,2$    $x_1 = x$ and $x_2 = y$

The rotation can now be written as:\\
\begin{equation*}
x'_i = R_{i1}x_1 + R_{i2}x_2 = \sum\limits_{j=1}^2 R_{ij} x_j \equiv R_{ij}x_j
\end{equation*}
In the last expression, the sum is implied by the repeated indices (known as Einstein notation).

\begin{equation*}
x'_i = R_{ij}x_j
\end{equation*}

$x'_i x'_i = x_i x_i$ is what it takes for $R$ to be a rotation. 
Need to find the special $R$s such that this is satisfied.

The identity matrix is written as $\delta_{ij}$, where $\delta_{ij} = 1$ if $i=j$, 0 otherwise.

If no rotation at all: $R_{ij} = \delta_{ij}$,  of an infinitesimal rotation:  
\begin{equation*}
R_{ij} = \delta_{ij} + \epsilon w_{ij}
\end{equation*}

\begin{equation*}
x'_{i} = x_i + \epsilon w_{ij} x_j
\end{equation*}


\begin{equation*}
x'_{i}x'_{i} = x_i x_{i} + 2 \epsilon \underbrace{w_{ij} x_j x_i}_{ =0 \hspace*{0.05in} \forall x }  + \mathcal{O}(\epsilon^2)
\end{equation*}

$\Rightarrow w_{ij}$ has to be anti-symmetric $w_{ij} = - w_{ji}$.

\noindent\rule{\textwidth}{1pt}

Back to previous example, find \underline{finite} rotations (without any mention of geometry etc.)

Take the infinitesimal rotation and do it n-times.

Define $\theta = N\epsilon$

\begin{eqnarray*}
\xyprime = \left[ \mathlarger{\mathlarger{\mathbbm{1}}} + \epsilon   \IMatrix \right]\xy \equiv \left[ \mathlarger{\mathlarger{\mathbbm{1}}} + \epsilon   \textrm{I} \right]\xy
\end{eqnarray*}

So the finite rotation ($R(\theta)$) given by,
\begin{eqnarray*}
R(\theta) = (1 + \epsilon I)(1 + \epsilon I) \dots (1 + \epsilon I) \dots = (1+\epsilon I)^N = \left(1+\frac{\theta}{N}I\right)^N
\end{eqnarray*}

Now let $N\rightarrow \infty$,  $R(\theta) = e^{I\theta}$.

Built up finite rotation from the infinitesimal rotations.
 
$x'(\theta) = R(\theta)x = e^{I\theta}$

The meaning of $e^X$ when $X$ is a matrix is simply the expansion.
\begin{eqnarray*}
e^X = 1 + X + \frac{X^2}{2!} + \frac{X^3}{3!} + \dots
\end{eqnarray*}


$I^2 =  \IMatrix\IMatrix = {\begin{pmatrix} -1 &  0  \\ 0  & -1 \end{pmatrix}} = \mathlarger{\mathlarger{\mathbbm{-1}}} ! $
We have just discovered i following our nose.

$R(\theta) = cos(\theta) + Isin(\theta) = {\begin{pmatrix} cos(\theta) &  sin(\theta)  \\ -sin(\theta)  & cos(\theta) \end{pmatrix}}$  (You will show in homework...)

\textbf{\underline{Strategy:}} First understand the action of the symmetry infinitesimally, then the big symmetry action is obtained by iterating the infinitesimal.
Always $e^X$ where $X$ is the generator. 
This is a great strategy for any kind of symmetry.

Will now do 3D rotations... Something new happens.

\section*{3D Rotations}

3-parameters associated with a 3D rotation.

Already saw, any rotation is of the form 

\begin{equation*}
x'_{i} = x_i + \epsilon w_{ij} x_j  \hspace{0.5in} \textrm{with } \hspace{0.5in} w_{ij} =- w_{ji}
\end{equation*}


Most general $3\time3$ anti-symmetric matrix: ${\begin{pmatrix} 0 & a & b  \\ -a & 0 & c \\ -b & -c & 0 \end{pmatrix}}$,  Now have 3 generators corresponding to the rotations in 3D.

\begin{equation*}
\epsilon w_{ij} = \epsilon_{12}\underbrace{{\begin{pmatrix} 0 & 1 & 0  \\ -1 & 0 & 0 \\ 0 & 0 & 0 \end{pmatrix}}}_{\text{generator we just saw}} + 
                  \epsilon_{13}{\begin{pmatrix} 0 & 0 & 1  \\ 0 & 0 & 0 \\ -1 & 0 & 0 \end{pmatrix}} +
                  \epsilon_{23}{\begin{pmatrix} 0 & 0 & 0  \\ 0 & 0 & -1 \\ 0 & -1 & 0 \end{pmatrix}}
\end{equation*}

How to get the finite version ? 
Easy, just exponentiate. 

Something new happens in 3D:
\begin{itemize}
\item[-]2D rotations commute        
\item[-]3D rotations do not
\end{itemize}

Define $J_3 = \JThree$, $J_2 = \JTwo$, and $J_1 = \JOne$

Rotations form a group.
Any three rotations give something that is also a rotation: 
\begin{equation*}
e^{i\theta_3 J_3}e^{i\theta_2 J_2}e^{i\theta_1 J_1} = e^{\phi_3 J_3 + \phi_2 J_2 + \phi_1 J_1}
\end{equation*}

This can only be possible if
\begin{equation*}
[J_1, J_2] = i J_3  \textrm{+ cyclic}
\end{equation*}


Can step back and think about this more abstractly. 
The matrices  we found form a group, but this group exists abstractly independent of these $3\times3$ matrices.
Fully determined by the commutation relations.
(Just like vectors and components)

In general, many matrices that satisfy the algebra (the commutation relations). 
These give different representations.

\underline{Deep:} rotations can act on more than just 3D vectors.

\begin{equation*}
\begin{bmatrix} \left(J_i\right) &  0  \\ 0  & \left(J_j\right) \end{bmatrix}
\end{equation*}
this is officially a representation.  
Its called a ``Reducible'' Representation.

\underline{More Generally} 
\begin{equation*}
[J_a, J_b] = i f_{abc} J_c \hspace{0.3in} \textrm{where $J_a$, $ a = 1, 2,$ ... dim. of the group}
\end{equation*}
Lie found all the possible symmetries when J is hermitian (there are not many).

\section*{One final example with rotations}
Lets look at traceless $2\times2$ hermitian matrices. 
Any $2\times2$, traceless, hermitian matrix can be written as: 
\begin{equation*}
M = \begin{pmatrix} z &  x+iy  \\ x-iy  & -z \end{pmatrix} \equiv \vec{\sigma} \cdot \vec{x}
\end{equation*}
where $\vec{\sigma} = (\sigma_x, \sigma_y, \sigma_z)$ and $\sigma_i$ are the Pauli matrices.

Note that $\textrm{det}(M) = - \vec{r}\cdot\vec{x} = - |\vec{x}|$.

Consider $M'=U^{\dagger}MU$, where U is unitary.
Any unitary matrix can be written as a phase $e^{i\theta}$ times a $2\times2$ hermitian matrix with det = 1.  (``Special Unitary Matrix'')
Because the phase cancels in $M'$ we will only consider U as unitary and det = 1. 
$U \in SU(2)$

If M is hermitian and traceless, then M' is still hermitian and traceless.
\begin{equation*}
M' = \sigma \cdot \vec{x'}_u
\end{equation*}
$\vec{x'}_{u} $ depends on U.  det(M') = det(M) $\Rightarrow \vec{x'}_u^2 = {\vec{x}}^2 $%= \vec{x}^2 $ length of \x
direct correspondence between $2\times2$ hermitian matrices \& rotations.

This is a 2D action of rotations.

Now easy to generalize all of this to the Lorentz group...

}
\end{document}


