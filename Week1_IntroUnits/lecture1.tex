\input{../latexSetup}

\lhead{\Large 33-444} % \hfill Introduction to Particle Physics \hfill Spring 2021}
\chead{\Large Introduction to Particle Physics} % \hfill Spring 2021}
\rhead{\Large Spring 2021} % \hfill Introduction to Particle Physics \hfill Spring 2021}

\begin{document}
\thispagestyle{fancy}

\begin{center}
{\huge \textbf{Lecture 1}}
\end{center}

{\fontsize{14}{16}\selectfont

\section{What this course is really about:}
The course title is: ``Introduction to Particle Physics''.
Technically correct, but misses the general philosophy. 
We are interested in how the world works. 
Turns out, one major lesson, probably the most important interesting scientific fact (which BTW didn't have to be this way.) things get easier/simpler at smaller scales. 
At these small scales things that naively seem to be totally unrelated are seen as aspects of the same thing. 

Can all be explained by the interaction of particles. 
This is why ``particles'' is in the title of the class, we are forced to talk about interactions of particles when we look at nature at a scale where the laws of physics are their simplest. 

\section{Warnings:}
This is a cutting edge field. 
\begin{itemize}
\item[-] Material presented only recently settled down. 
\item[-] Many theorems or equations/assumptions I wont be able to prove to you. 
\item[-] Most require advanced math, many lack formal proof.  (downside of studying a cutting edge field)
\item[-] Things can change rapidly. 
\item[-] Scope is incredibly broad. 
\end{itemize}

This field is extremely far-reaching. 
It is a mathematically dense and abstract topic. 
I will do my best to ignore the details and focus on underlying concepts.

Textbooks for undergraduates on particle physics are often organized historically, which can add confusion. 

All of the machinery, formalism, insight, and tools that you have gained as a physics student is essential for studying particle physics.


\section{20th Century revolutions:}
Special Relativity (SR) and Quantum Mechanics (QM).
Radical departures from what came before them.

QM much more radical. 
SR took one guy (Einstein) a few years to figure out. 
QM took an entire generation $\sim$15-20 years to straighten out. 

SR told us how to think clearly about ideas we were already using: space and time. 
QM threw everything out and invented a whole new framework/language with which physics had to be understood.
Much more abstract: Hilbert spaces/state vectors/Hermitian operators/ ect.

Major gains in explanatory power and unification.

Concepts thought different, faces of same thing:

\textbf{Relativity:}
\begin{itemize}
\item[-] Space and time
\item[-] Energy and Mass (also momentum)
\item[-] Electricity and Magnetism (also light)
\item[-] (Gravity shown to be result of warping of space time)
\end{itemize}

\textbf{Quantum Mechanics:}
\begin{itemize}
\item[-] Waves and Particles
\item[-] Chemistry and Physics
\end{itemize}

\section{Mission Barely Possible: QM + SR}
First 25 years of the 20th century were about the two revolutions (QM and SR).
85 years since then, were all about putting these together.

Combining Relativity and Quantum Mechanics naively seems impossible. 
SR and QM use different basic languages. 

\textbf{QM:} Time has special (fundamental) role. Specify \ketY\ at one time, prescription for how to evolve to later times.

\textbf{Relativity:} Time is not special! (can mix space and time by moving)

Turns out (just barely) possible using something called: \textbf{``Quantum Field Theory'' }
Basic framework for how the world works.
Dramatically restricts what a theory can possibly look like.
Quantum field theory is the framework in which the fundamental forces of Nature are formulated.

\textbf{Consequences of Union}

Anti-particles must exist:
\begin{itemize}
\item[-] Shocking / Unexpected
\item[-] Doubled everything in universe 
\item[-] Makes the vacuum interesting
\end{itemize}

Key role of Spin:
\begin{itemize}
\item[-] Relation between spin and particle type
\item[-] Dramatically limits types of particles can have
\end{itemize}

Major constraints on types of interactions allowed:
\begin{itemize}
\item[-] Only certain interaction will ever be important
\item[-] Always be a finite number of parameters that matter
\end{itemize}

\section{Anti-Particles}
\textbf{Causality}

What happens next can only depend of what happened before
(Does not depend on something that hasn't happened yet !)
If someone dies from a gun shot, the gun must be shot first.
Causality basic prerequisite to science !

\underline{Causality in Relativity}

Can't send signals faster than maximum speed.
\begin{figure}[h]
\centering
\includegraphics[width=0.6\textwidth]{./CausalityTimelike.pdf}
\end{figure}

All moving observers agree that A happens before B. => Can say safely say: ``A causes B''.


However, If you could go faster than c, things go wrong...
\begin{figure}[h]
\centering
\includegraphics[width=0.6\textwidth]{./CausalitySpacelike.pdf}
\end{figure}

Depending on how you move, disagree about what comes first. Causality is violated. Bullet hits B before A pulls trigger.

\clearpage
\underline{Causality in Relativistic QM}

w/QM always some non-zero probability of getting out of the light cone.
Can't know both $\Delta$ P and $\Delta$ X to arbitrary precision because of the uncertainty principle.
Problem, if send electron from A to B looks like current is going backwards in time from A to B.
Way out, if interpret this as B sending something to A.
But B has to send something with opposite charge. (know A lost charge).

For the theory to have a prayer of being both quantum mechanical and relativity there must exist particles that are identical to the electron (say) except with opposite charge.
Anti-particles !
Can play the same game for another type of particle => every particle must have an anti-particle partner.

\section{Vacuum}
Anti-mater in turn has major an impact on nothing, i.e. the vacuum.

What does it take to study empty space (“the vacuum”) ?
Nothing special...until try to check small regions

\textbf{Before QM:}
Build tiny robots. (Get tiny robots to build tinier robots, who build tiny robots etc. etc. etc.)

\textbf{With QM:}
At small distances, the uncertainty principle kicks in.
Need large $\Delta p$ (or equivalently large $\Delta E$)
Smaller and smaller distances, need higher and higher energies.

Empty Space becomes interesting.

When eventually get to small enough distances to need $\Delta E \sim$  2$m_e c^2$
Nothing prevents creation of particle - anti-particle pair.
Everything is conserved (energy/charge/momentum) so there is some non-zero probability for this to happen.

Completely changes our picture of the vacuum.
Simple act of looking creates something.
No sense in which the vacuum is empty!

People often say that accelerators are worlds most powerful microscopes:
What it is looking at is the vacuum.

\section{Other Implications Combining SR \& QM}

\underline{Spin}

\textbf{QM:}
Could accommodate spin. 
Any 1/2 integer value allowed.

\textbf{ QM + SR:}
Forced to talk spin (Something special w/mass-less particles).
Integer spin = Bosons / Half-integer = Fermions.
Can only have spin = : 0 | 1/2 | 1 | 3/2 | 2.

\underline{Interactions}

\textbf{QM:} Any conceivable interaction possible
\textbf{QM + SR:} Charge is conserved. 
Interactions are Local (no more action at a distance).
Only finite number of specific interactions allowed :
\begin{figure}[h]
\centering
\includegraphics[width=0.6\textwidth]{./AllowedInteractions.pdf}
\end{figure}

In this course we will outline this basic framework for any possible theory.
And then talk about the particular version that we actually observe.


\section{Our world}

Stuff in the world made of atoms:

\underline{Matter:}
Electrons: Negatively charged. 
Responsible for volume of atom.
Thought to be fundamental.

Nucleus: Positively charged.
Responsible for the mass of an atom.
Made of, protons and neutrons, which are made of quarks.
Quarks also thought to be fundamental.

Matter particles (electrons/quarks) are fermions.
Large collections of them behave like classical particles.

\underline{Forces}

\textbf{Gravity:}
Known since antiquity. 
Inverse square law.
Always attractive. 
Irrelevant for atomic/sub-atomic interactions.

\textbf{Electromagnetism:}
Known since antiquity.
Also inverse square law.
Can be Attractive or repulsive.
Holds electrons within atoms.

\textbf{Strong:}
Discovered early 1900s.
Only relevant at short distances.
No simple relationship (a la inverse square).
Responsible for holding together the nucleus.

\textbf{Weak:}
Discovered just before turn of 20th century.
Looks nothing like others.
Responsible for radioactive decay. 
Heats the sun/earth.

Our world both Relativistic and Quantum Mechanical => described in terms of a Quantum Field Theory (QFT).
The particular version of QFT that was found to describe our universe developed in the 1960-70s.
Called the ``Standard Model''.

All forces as important.
At large scales they Look very different from one another.

\begin{figure}[h]
\centering
\includegraphics[width=0.9\textwidth]{./Forces.pdf}
\end{figure}

Now is the first time that we see that all forces described in same basic way.
That the forces look very different to us... is a long distance illusion!

This is the reason we build colliders! Unity at small scales.

\section{Standard Model}


\begin{figure}[h]
\centering
\includegraphics[width=0.9\textwidth]{./SM.pdf}
\end{figure}

Naively you might only expect one generation, we see three. 
There is no good understanding for why we see three. 
Pretty sure that there are not more (will discuss how we know this later in the course).

The Standard Model took on modern form in 60s - 70s.
Makes very precise predictions, shown to be highly accurate.
Consistent theory of electromagnetic, weak and strong forces ...
... provided mass-less Matter and Force Carriers


Serious problem as matter and W, Z known to be massive !
We will come back to this problem later in the semester.

}

\end{document}


