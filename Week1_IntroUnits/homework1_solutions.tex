\documentclass[paper=letter,11pt]{scrartcl}

\KOMAoptions{headinclude=true, footinclude=false}
\KOMAoptions{DIV=14, BCOR=5mm}
\KOMAoptions{numbers=noendperiod}
\KOMAoptions{parskip=half}
\addtokomafont{disposition}{\rmfamily}
\addtokomafont{part}{\LARGE}
\addtokomafont{descriptionlabel}{\rmfamily}
%\setkomafont{pageheadfoot}{\normalsize\sffamily}
\setkomafont{pagehead}{\normalsize\rmfamily}
%\setkomafont{publishers}{\normalsize\rmfamily}
\setkomafont{caption}{\normalfont\small}
\setcapindent{0pt}
\deffootnote[1em]{1em}{1em}{\textsuperscript{\thefootnotemark}\ }


\usepackage{amsmath}
\usepackage[varg]{txfonts}
\usepackage[T1]{fontenc}
\usepackage{graphicx}
\usepackage{xcolor}
\usepackage[american]{babel}
% hyperref is needed in many places, so include it here
\usepackage{hyperref}

\usepackage{xspace}
\usepackage{multirow}
\usepackage{float}


\usepackage{braket}
\usepackage{bbm}
\usepackage{relsize}
\usepackage{tcolorbox}

\def\ketY{\ensuremath{\ket {\Psi}}}
\def\iGeV{\ensuremath{\textrm{GeV}^{-1}}}
%\def\mp{\ensuremath{m_{\textrm{proton}}}}
\def\rp{\ensuremath{r_{\textrm{proton}}}}
\def\me{\ensuremath{m_{\textrm{electron}}}}
\def\aG{\ensuremath{\alpha_G}}
\def\rAtom{\ensuremath{r_{\textrm{atom}}}}
\def\rNucl{\ensuremath{r_{\textrm{nucleus}}}}
\def\GN{\ensuremath{\textrm{G}_\textrm{N}}}
\def\ketX{\ensuremath{\ket{\vec{x}}}}
\def\ve{\ensuremath{\vec{\epsilon}}}


\def\ABCDMatrix{\ensuremath{\begin{pmatrix} A &  B  \\ C  & D \end{pmatrix}}}
\def\xyprime{\ensuremath{\begin{pmatrix} x' \\ y' \end{pmatrix}}}
\def\xyprimeT{\ensuremath{\begin{pmatrix} x' &  y' \end{pmatrix}}}
\def\xy{\ensuremath{\begin{pmatrix} x \\ y \end{pmatrix}}}
\def\xyT{\ensuremath{\begin{pmatrix} x & y \end{pmatrix}}}

\def\IMatrix{\ensuremath{\begin{pmatrix} 0 &  1  \\ -1  & 0 \end{pmatrix}}}
\def\IBoostMatrix{\ensuremath{\begin{pmatrix} 0 &  1  \\ 1  & 0 \end{pmatrix}}}
\def\JThree{\ensuremath{\begin{pmatrix}    0 & -i & 0  \\ i & 0  & 0 \\ 0 & 0 & 0 \end{pmatrix}}} 
\def\JTwo{\ensuremath{\begin{bmatrix}    0 & 0 & -i  \\ 0 & 0  & 0 \\ i & 0 & 0 \end{bmatrix}}}
\def\JOne{\ensuremath{\begin{bmatrix}    0 & 0 & 0  \\ 0 & 0  & -i \\ 0 & i & 0 \end{bmatrix}}}
\def\etamn{\ensuremath{\eta_{\mu\nu}}}
\def\Lmn{\ensuremath{\Lambda^\mu_\nu}}
\def\dmn{\ensuremath{\delta^\mu_\nu}}
\def\wmn{\ensuremath{\omega^\mu_\nu}}
\def\be{\begin{equation*}}
\def\ee{\end{equation*}}
\def\bea{\begin{eqnarray*}}
\def\eea{\end{eqnarray*}}
\def\bi{\begin{itemize}}
\def\ei{\end{itemize}}
\def\fmn{\ensuremath{F_{\mu\nu}}}
\def\fMN{\ensuremath{F^{\mu\nu}}}
\def\bc{\begin{center}}
\def\ec{\end{center}}
\def\nus{$\nu$s}

\def\adagger{\ensuremath{a_{p\sigma}^\dagger}}
\def\lineacross{\noindent\rule{\textwidth}{1pt}}

\newcommand{\multiline}[1] {
\begin{tabular} {|l}
#1
\end{tabular}
}

\newcommand{\multilineNoLine}[1] {
\begin{tabular} {l}
#1
\end{tabular}
}



\newcommand{\lineTwo}[2] {
\begin{tabular} {|l}
#1 \\
#2
\end{tabular}
}

\newcommand{\rmt}[1] {
\textrm{#1}
}


%
% Units
%
\def\m{\ensuremath{\rmt{m}}}
\def\GeV{\ensuremath{\rmt{GeV}}}
\def\pt{\ensuremath{p_\rmt{T}}}


\def\parity{\ensuremath{\mathcal{P}}}

\usepackage{cancel}
\usepackage{ mathrsfs }
\def\bigL{\ensuremath{\mathscr{L}}}

\usepackage{ dsfont }



\usepackage{fancyhdr}
\fancyhf{}


\def\ketY{\ensuremath{\ket {\Psi}}}
\def\iGeV{\ensuremath{\textrm{GeV}^{-1}}}
\def\mp{\ensuremath{m_{\textrm{proton}}}}
\def\rp{\ensuremath{r_{\textrm{proton}}}}
\def\me{\ensuremath{m_{\textrm{electron}}}}
\def\aG{\ensuremath{\alpha_G}}
\def\rAtom{\ensuremath{r_{\textrm{atom}}}}
\def\rNucl{\ensuremath{r_{\textrm{nucleus}}}}
\def\GN{\ensuremath{\textrm{G}_\textrm{N}}}


\lhead{\Large 33-444} % \hfill Introduction to Particle Physics \hfill Spring 2020}
\chead{\Large Introduction to Particle Physics} % \hfill Spring 2020}
\rhead{\Large Spring 2020} % \hfill Introduction to Particle Physics \hfill Spring 2020}
\begin{document}
\thispagestyle{fancy}





%\begin{tabular}{c}
%{\large 33-444 \hfill Intro To Particle \hfill Spring 2020\\}
%\hline 
%\end{tabular}

\begin{center}
{\huge \textbf{Homework Set \#1}}
\large

{\textbf{ Solutions}   }
\end{center}

{\large
\textbf{2) Solid State Physics} \hfill \textit{(5 points)}
\begin{itemize}
\item[(a)] {
We worked out in class $\rAtom \sim \frac{1}{Z\alpha m_e}$. (using $E\sim - \frac{Z\alpha}{r} + \frac{p^2}{m_e}$ and $p\times r \sim 1$)
So a solid has a spacing of \rAtom.
}
\item[(b)] {
To probe distances of order \rAtom\ need to photons with Energy $\sim \frac{1}{\rAtom} \sim Z \alpha \me$.
Assuming $Z\sim 10$, \\Energy $\sim 10 \cdot 10^{-2} \cdot 10^{-3}\ \GeV \sim 10^{-4} \GeV \sim 10^2\ \textrm{keV}$
}
\item[(c)] $10^2\ \textrm{keV}$ photons are x-rays.
\end{itemize}

\vspace*{0.25in}

\textbf{3) Strength of Gravity on Earth} \hfill \textit{(5 points)}
\begin{itemize}
\item[(a)]{
From class, $R_{\textrm{Planet}} \sim  \sqrt{\frac{\alpha}{\aG}} \times \rAtom$,   $\rho_{\textrm{solid}} \sim \frac{Z\mp}{\rAtom^3}$, and $M_{\textrm{Planet}} \sim \rho_{\textrm{Solid}} \times R_{\textrm{Planet}}^3$

\begin{equation*}
g_{\textrm{local}} \sim G_N \frac{M_{\textrm{Planet}}}{R_{Planet}^2} \sim \frac{G_N Z \mp R_{\textrm{Planet}}}{\rAtom^3}\sim \sqrt{\alpha_G \alpha} \frac{Z}{\mp \rAtom^2}
\end{equation*}
}
\item[(b)]{
\begin{align*}
g_{\textrm{local}}  \sim & (\alpha_G \alpha)^{1/2} \cdot Z \cdot \rAtom^{-2} \cdot \mp^{-1}\\
                    \sim & (10^{-39}10^{-2})^{1/2} \cdot 10 \cdot 10^{-8}\ \GeV^{-2} \cdot \GeV\\
                    \sim & 10^{-27.5} \GeV
\end{align*}

Need to convert \GeV\ to $\frac{m}{s^2}$ which has units of [distance]$\times$[time]$^{-2}$.
c has units of [distance]$\times$[time]$^{-1}$, h has units of [energy]$\times$[time].
So, can convert from [energy] to [distance]$\times$[time]$^{-2}$ by multiplying by c/h.
c = $10^8$ m/s, h = $10^{-15}$ eV$\cdot$s = $10^{-24}$ \GeV $\cdot$ s. 
So, $c/h = 10^8 \cdot 10^{24} = 10^{32} \frac{m}{\GeV s^2}$  
\begin{align*}
g_{\textrm{local}}  \sim & 10^{-27.5} \GeV  \times (1)\\
  \sim & 10^{-27.5} \GeV  \times \frac{c}{h} \\
  \sim & 10^{-27.5} \GeV  \times 10^{32}  \frac{m}{\GeV s^2} \\
  \sim & 10^{4} \frac{m}{s^2}
\end{align*}


}
\item[(c)]{
Not so close to $10 \frac{m}{s^2}$, If we has used $\rAtom \sim10^{-10} m$ instead of $10^{-12}$ which accounts for the screening of multple electrons in the atom. 
Would have calculated factor of $10^{-4}$ smaller or $g_\textrm{local}$ ~ 1, which is pretty close.
}

\end{itemize}
%Estimate the size of life forms on earth in terms of $\alpha$, \aG, \mp, and \me.
%Assume life is a solid. 

\vspace*{0.25in}

\textbf{4) Neutron Stars } \hfill \textit{(5 points)}
\begin{itemize}
\item[(a)]{
Neutron star will be stable when $P_{Neutron} \sim P_{Grav}$.
$P_{Neutron} \sim \frac{E_{Neutron}}{R_{Neutron}^3} \sim \mp^4$
\begin{equation*}
P_\textrm{Grav} \sim \frac{G_N \frac{M_{NS}^2}{R_{NS}}}{R_{NS}^3} \sim \frac{G_N M_{NS}^2}{R_{NS}^4}
\end{equation*}

\begin{equation*}
M_\textrm{NS} \sim \rho_{Neutron}\times R_{NS}^3
\end{equation*}

\begin{equation*}
\rho_{Neutron} \sim \mp^4
\end{equation*}


\begin{equation*}
P_\textrm{Grav} \sim P_{Neutron} \Rightarrow  G_N \mp^8 R_{NS}^2 \sim \mp^4
\end{equation*}

\begin{equation*}
R_{NS} \sim \sqrt{\frac{1}{\aG}} \frac{1}{\mp} \sim \sqrt{\frac{1}{\aG}} \rp
\end{equation*}

\begin{equation*}
\Rightarrow M_{NS} \sim \left(\frac{1}{\aG}\right)^{3/2} \mp 
\end{equation*}

Speed of sound $v_{NS} \sim \sqrt{\frac{P_{NS}}{\rho_{NS}}} \sim 1$

}


\item[(b)]{
$R_{NS} \sim 10^{19} 10^{-15} \text{m} \sim 10^4 \text{m} \sim 10\ \text{km} $  \\
$M_{NS} \sim 10^{58} 10^{-27} \text{kg} \sim 10^{31} \text{kg} \sim 10\ M_\odot $  \\
$v_{NS} \sim c$
}
\item[(c)]{
``Neutron stars have a radius of the order of 10 kilometres and a mass lower than a 2.16 solar masses.'' -wikipedia
}

\item[(d)]{
$\mp = 938.27$ MeV\\
$m_{\text{neutron}} = 939.56$ MeV $ \sim 1.001 \mp $

}
\end{itemize}

\vspace*{0.25in}

\newpage

\textbf{5) 2D Rotations } \hfill \textit{(3 points)}
\begin{itemize}
\item[(a)]{

\begin{equation*}
e^{I\theta} = 1 + I\theta + \frac{I^2\theta^2}{2!} + \frac{I^3\theta^3}{3!} + \frac{I^4\theta^4}{4!} + \dots
\end{equation*}

$I^2 = \begin{bmatrix} -1 & 0  \\ 0 & -1 \end{bmatrix} $

Show that $R(\Theta) = e^{I\Theta} = cos(\Theta)+ I sin(\Theta)$


\begin{equation*}
e^{I\theta} = I\left(\theta + \frac{I^2\theta^3}{3!} + \frac{I^4\theta^5}{5!} + \dots \right) + \left(1 + \frac{I^2\theta^2}{2!} + \frac{I^4\theta^4}{4!}  + \dots \right)
\end{equation*}

\begin{equation*}
e^{I\theta} = I\left(\theta - \frac{\theta^3}{3!} + \frac{\theta^5}{5!} + \dots \right) + \left(1 - \frac{\theta^2}{2!} + \frac{\theta^4}{4!}  + \dots \right) \\
= I sin(\theta) + cos(\theta)
\end{equation*}

  
}
\item[(b)]{

\begin{equation*}
\begin{pmatrix}  cos(\theta_1) & sin(\theta_1) \\ -sin(\theta_1) & cos(\theta_1) \end{pmatrix} 
\begin{pmatrix}  cos(\theta_2) & sin(\theta_2) \\ -sin(\theta_2) & cos(\theta_2) \end{pmatrix}  = \\
\end{equation*}

\begin{equation*}
\begin{pmatrix}  
cos(\theta_1)cos(\theta_2) - sin(\theta_1)sin(\theta_2) & cos(\theta_1)sin(\theta_2) + sin(\theta_1)cos(\theta_2) \\ 
-sin(\theta_1)cos(\theta_2) - cos(\theta_1)sin(\theta_2) & -sin(\theta_1)sin(\theta_2) + cos(\theta_1)cos(\theta_2) 
\end{pmatrix}  = 
\end{equation*}

\begin{equation*}
\begin{pmatrix}  
cos(\theta_1 + \theta_2)  & sin(\theta_1 + \theta_2) \\ 
-sin(\theta_1 + \theta_2) & cos(\theta_1 + \theta_2)
\end{pmatrix}  
\end{equation*}
which is clearly symmetric $\theta_1 \leftrightarrow \theta_2$

}
\end{itemize}

\vspace*{0.25in}

\textbf{6) 3D Rotations } \hfill \textit{(5 points)}
\begin{itemize}
\item[(a)]{

\begin{equation*}
J_1 J_2 - J_2 J_1 = \begin{bmatrix}    0 & 0 & 0  \\ 1 & 0  & 0 \\ 0 & 0 & 0 \end{bmatrix} - \begin{bmatrix}    0 & 1 & 0  \\ 0 & 0  & 0 \\ 0 & 0 & 0 \end{bmatrix}  = \begin{bmatrix}    0 & -1 & 0  \\ 1 & 0  & 0 \\ 0 & 0 & 0 \end{bmatrix} = -i J_3
\end{equation*}
etc.

}
\item[(b)]{
%Show that $M' = $ is also traceless and hermitian and that is has the same determinant as M.

$tr(M') = tr(U^{\dagger}MU) = tr(UU^{\dagger}M) = tr(M)$\\

$(M')^\dagger = (U^{\dagger}MU)^\dagger = (U M^{\dagger} U^{\dagger}) = (U^\dagger M U) = M'$\\


$det(M') = det(U^{\dagger})det(M)det(U) = 1 \times det(M) \times 1$
}

\end{itemize}



\end{document}
