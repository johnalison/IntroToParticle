\documentclass[paper=letter,11pt]{scrartcl}

\KOMAoptions{headinclude=true, footinclude=false}
\KOMAoptions{DIV=14, BCOR=5mm}
\KOMAoptions{numbers=noendperiod}
\KOMAoptions{parskip=half}
\addtokomafont{disposition}{\rmfamily}
\addtokomafont{part}{\LARGE}
\addtokomafont{descriptionlabel}{\rmfamily}
%\setkomafont{pageheadfoot}{\normalsize\sffamily}
\setkomafont{pagehead}{\normalsize\rmfamily}
%\setkomafont{publishers}{\normalsize\rmfamily}
\setkomafont{caption}{\normalfont\small}
\setcapindent{0pt}
\deffootnote[1em]{1em}{1em}{\textsuperscript{\thefootnotemark}\ }


\usepackage{amsmath}
\usepackage[varg]{txfonts}
\usepackage[T1]{fontenc}
\usepackage{graphicx}
\usepackage{xcolor}
\usepackage[american]{babel}
% hyperref is needed in many places, so include it here
\usepackage{hyperref}

\usepackage{xspace}
\usepackage{multirow}
\usepackage{float}


\usepackage{braket}
\usepackage{bbm}
\usepackage{relsize}
\usepackage{tcolorbox}

\def\ketY{\ensuremath{\ket {\Psi}}}
\def\iGeV{\ensuremath{\textrm{GeV}^{-1}}}
%\def\mp{\ensuremath{m_{\textrm{proton}}}}
\def\rp{\ensuremath{r_{\textrm{proton}}}}
\def\me{\ensuremath{m_{\textrm{electron}}}}
\def\aG{\ensuremath{\alpha_G}}
\def\rAtom{\ensuremath{r_{\textrm{atom}}}}
\def\rNucl{\ensuremath{r_{\textrm{nucleus}}}}
\def\GN{\ensuremath{\textrm{G}_\textrm{N}}}
\def\ketX{\ensuremath{\ket{\vec{x}}}}
\def\ve{\ensuremath{\vec{\epsilon}}}


\def\ABCDMatrix{\ensuremath{\begin{pmatrix} A &  B  \\ C  & D \end{pmatrix}}}
\def\xyprime{\ensuremath{\begin{pmatrix} x' \\ y' \end{pmatrix}}}
\def\xyprimeT{\ensuremath{\begin{pmatrix} x' &  y' \end{pmatrix}}}
\def\xy{\ensuremath{\begin{pmatrix} x \\ y \end{pmatrix}}}
\def\xyT{\ensuremath{\begin{pmatrix} x & y \end{pmatrix}}}

\def\IMatrix{\ensuremath{\begin{pmatrix} 0 &  1  \\ -1  & 0 \end{pmatrix}}}
\def\IBoostMatrix{\ensuremath{\begin{pmatrix} 0 &  1  \\ 1  & 0 \end{pmatrix}}}
\def\JThree{\ensuremath{\begin{pmatrix}    0 & -i & 0  \\ i & 0  & 0 \\ 0 & 0 & 0 \end{pmatrix}}} 
\def\JTwo{\ensuremath{\begin{bmatrix}    0 & 0 & -i  \\ 0 & 0  & 0 \\ i & 0 & 0 \end{bmatrix}}}
\def\JOne{\ensuremath{\begin{bmatrix}    0 & 0 & 0  \\ 0 & 0  & -i \\ 0 & i & 0 \end{bmatrix}}}
\def\etamn{\ensuremath{\eta_{\mu\nu}}}
\def\Lmn{\ensuremath{\Lambda^\mu_\nu}}
\def\dmn{\ensuremath{\delta^\mu_\nu}}
\def\wmn{\ensuremath{\omega^\mu_\nu}}
\def\be{\begin{equation*}}
\def\ee{\end{equation*}}
\def\bea{\begin{eqnarray*}}
\def\eea{\end{eqnarray*}}
\def\bi{\begin{itemize}}
\def\ei{\end{itemize}}
\def\fmn{\ensuremath{F_{\mu\nu}}}
\def\fMN{\ensuremath{F^{\mu\nu}}}
\def\bc{\begin{center}}
\def\ec{\end{center}}
\def\nus{$\nu$s}

\def\adagger{\ensuremath{a_{p\sigma}^\dagger}}
\def\lineacross{\noindent\rule{\textwidth}{1pt}}

\newcommand{\multiline}[1] {
\begin{tabular} {|l}
#1
\end{tabular}
}

\newcommand{\multilineNoLine}[1] {
\begin{tabular} {l}
#1
\end{tabular}
}



\newcommand{\lineTwo}[2] {
\begin{tabular} {|l}
#1 \\
#2
\end{tabular}
}

\newcommand{\rmt}[1] {
\textrm{#1}
}


%
% Units
%
\def\m{\ensuremath{\rmt{m}}}
\def\GeV{\ensuremath{\rmt{GeV}}}
\def\pt{\ensuremath{p_\rmt{T}}}


\def\parity{\ensuremath{\mathcal{P}}}

\usepackage{cancel}
\usepackage{ mathrsfs }
\def\bigL{\ensuremath{\mathscr{L}}}

\usepackage{ dsfont }



\usepackage{fancyhdr}
\fancyhf{}


\def\ketY{\ensuremath{\ket {\Psi}}}
\def\iGeV{\ensuremath{\textrm{GeV}^{-1}}}
\def\mp{\ensuremath{m_{\textrm{proton}}}}
\def\rp{\ensuremath{r_{\textrm{proton}}}}
\def\me{\ensuremath{m_{\textrm{electron}}}}
\def\aG{\ensuremath{\alpha_G}}
\def\rAtom{\ensuremath{r_{\textrm{atom}}}}
\def\rNucl{\ensuremath{r_{\textrm{nucleus}}}}
\def\GN{\ensuremath{\textrm{G}_\textrm{N}}}


\lhead{\Large 33-444} % \hfill Introduction to Particle Physics \hfill Spring 2022}
\chead{\Large Introduction to Particle Physics} % \hfill Spring 2022}
\rhead{\Large Spring 2022} % \hfill Introduction to Particle Physics \hfill Spring 2022}
\begin{document}
\thispagestyle{fancy}





%\begin{tabular}{c}
%{\large 33-444 \hfill Intro To Particle \hfill Spring 2022\\}
%\hline 
%\end{tabular}

\begin{center}
{\huge \textbf{Homework Set \#1}}
\large

{\textbf{ Solutions}   }
\end{center}

{\large
\textbf{2) Radius of Planets } \hfill \textit{(5 points)}
\begin{itemize}
\item[(a)] {
A planet is an object whose internal pressure coming from it being a solid made up of atoms is balanced by the gravitational pressure it feels.

Lets work out the internal pressure of a solid first:

In class we worked out that $\rAtom \sim \frac{1}{Z\alpha m_e}$. (using $E\sim - \frac{Z\alpha}{r} + \frac{p^2}{m_e}$ and $p\times r \sim 1$)

It follows from this that $E_{atom} \sim Z^2\alpha^2 m_e$ and $V_{atom} \sim \rAtom^3$. 

The atomic pressure (Units F/area or (better!) E/volume) is then $P_{solid} \sim E_{atom} / V_{atom} \sim \frac{Z^2\alpha^2 m_e}{(Z\alpha m_e)^{-3}} \sim \sim Z^5\alpha^5 m_e^{4} \sim Z\alpha \rAtom^{-4} $


Now lets do $P_{Grav}$. Here we need $E_{Grav}$ and $V_{Grav}$, the energy of the planet from the gravitational force and the volume over which it acts.

$V_{Grav}$ is the volume of the planet $\sim R_{Planet}^3$

The gravitational energy of a sphere is given by $E_{Grav} \sim G_N \frac{M_{Planet}^2}{R_{Planet}}$

So, $P_{Grav} \sim G_N \frac{M_{Planet}^2}{R_{Planet}^4}$. 

Assuming the planet is a solid made of atoms, we can write $M_{Planet} \sim \rho_{solid} R_{Planet}^3$, where $\rho_{solid} \sim \frac{Z m_p}{\rAtom^3}$ 

Then, $P_{Grav} \sim G_N \frac{Z^2 m_p^2 R_{Planet}^6}{\rAtom^6 R_{Planet}^4} \sim (G_N m_p^2) Z^2 \left(\frac{R_{Planet}}{\rAtom}\right)^2 \frac{1}{\rAtom^4} \sim \alpha_G Z^2 \left(\frac{R_{Planet}}{\rAtom}\right)^2 \rAtom^{-4} $. 

Setting $P_{Grav} \sim P_{Solid}$ gives

\be
\alpha_G Z^2 \left(\frac{R_{Planet}}{\rAtom}\right)^2 \rAtom^{-4} \sim Z\alpha \rAtom^{-4}
\ee

or

\be
\left(\frac{R_{Planet}}{\rAtom}\right)^2 \sim Z^{-1} \frac{\alpha}{\alpha_G}  \Rightarrow R_{Planet} \sim  \sqrt{\frac{\alpha}{Z\alpha_G}} \frac{1}{Z\alpha m_e}
\ee


}
\item[(b)] {
\be
R_{Planet} \sim   \sqrt{\frac{\alpha}{Z\alpha_G}} \rAtom
\ee
}
\item[(c)] {
Lets take $\rAtom \sim 10^{-10}$ m, $Z\sim10^2$, $\alpha \sim 10^{-2} $, and $\alpha_G \sim 10^{-38}$ 

\be
R_{Planet} \sim   \sqrt{\frac{10^{-2}}{10^2 10^{-38}}} 10^{-10} m \sim 10^{17} 10^{-10} m \sim 10^7 m 
\ee
vs actual $6.37 \times 10^6 m$ (Pretty Good!)
}
\end{itemize}

\vspace*{0.25in}

{\large
\textbf{3) Solid State Physics} \hfill \textit{(5 points)}
\begin{itemize}
\item[(a)] {
We worked out in class $\rAtom \sim \frac{1}{Z\alpha m_e}$. (using $E\sim - \frac{Z\alpha}{r} + \frac{p^2}{m_e}$ and $p\times r \sim 1$)
So a solid has a spacing of \rAtom.
}
\item[(b)] {
To probe distances of order \rAtom\ need to photons with Energy $\sim \frac{1}{\rAtom} \sim Z \alpha \me$.
Assuming $Z\sim 10$, \\Energy $\sim 10 \cdot 10^{-2} \cdot 10^{-3}\ \GeV \sim 10^{-4} \GeV \sim 10^2\ \textrm{keV}$
}
\item[(c)] $10^2\ \textrm{keV}$ photons are x-rays.
\end{itemize}

\vspace*{0.25in}

\textbf{4) 2D Rotations } \hfill \textit{(5 points)}
\begin{itemize}
\item[(a)]{

\begin{equation*}
e^{I\theta} = 1 + I\theta + \frac{I^2\theta^2}{2!} + \frac{I^3\theta^3}{3!} + \frac{I^4\theta^4}{4!} + \dots
\end{equation*}

$I^2 = \begin{bmatrix} -1 & 0  \\ 0 & -1 \end{bmatrix} $

Show that $R(\Theta) = e^{I\Theta} = cos(\Theta)+ I sin(\Theta)$


\begin{equation*}
e^{I\theta} = I\left(\theta + \frac{I^2\theta^3}{3!} + \frac{I^4\theta^5}{5!} + \dots \right) + \left(1 + \frac{I^2\theta^2}{2!} + \frac{I^4\theta^4}{4!}  + \dots \right)
\end{equation*}

\begin{equation*}
e^{I\theta} = I\left(\theta - \frac{\theta^3}{3!} + \frac{\theta^5}{5!} + \dots \right) + \left(1 - \frac{\theta^2}{2!} + \frac{\theta^4}{4!}  + \dots \right) \\
= I sin(\theta) + cos(\theta)
\end{equation*}

  
}
\item[(b)]{

\begin{equation*}
\begin{pmatrix}  cos(\theta_1) & sin(\theta_1) \\ -sin(\theta_1) & cos(\theta_1) \end{pmatrix} 
\begin{pmatrix}  cos(\theta_2) & sin(\theta_2) \\ -sin(\theta_2) & cos(\theta_2) \end{pmatrix}  = \\
\end{equation*}

\begin{equation*}
\begin{pmatrix}  
cos(\theta_1)cos(\theta_2) - sin(\theta_1)sin(\theta_2) & cos(\theta_1)sin(\theta_2) + sin(\theta_1)cos(\theta_2) \\ 
-sin(\theta_1)cos(\theta_2) - cos(\theta_1)sin(\theta_2) & -sin(\theta_1)sin(\theta_2) + cos(\theta_1)cos(\theta_2) 
\end{pmatrix}  = 
\end{equation*}

\begin{equation*}
\begin{pmatrix}  
cos(\theta_1 + \theta_2)  & sin(\theta_1 + \theta_2) \\ 
-sin(\theta_1 + \theta_2) & cos(\theta_1 + \theta_2)
\end{pmatrix}  
\end{equation*}
which is clearly symmetric $\theta_1 \leftrightarrow \theta_2$

}

\item[(c)]{
\be
zz^* = (x+iy)(x-iy) = x^2 + y^2
\ee
So, given a vector in the complex plain specified by (x,y), $zz^*$ gives the length of the vector.
}

Under the action of: $z\rightarrow e^{i\theta}z$,$z^*\rightarrow e^{-i\theta}z^*$
\begin{align*}
M(\theta_1): z\rightarrow e^{i\theta_1}z\ \textrm{(+ complex conjugate)}\\
M(\theta_2): z\rightarrow e^{i\theta_2}z\ \textrm{(+ complex conjugate)}\\
M(\theta_1)M(\theta_2): z\rightarrow e^{i\theta_1}e^{i\theta_2}z = e^{i(\theta_1+\theta_2)}z = M(\theta_1 + \theta_2)
\end{align*}

And $M(\theta_1)M(\theta_2) = M(\theta_1 + \theta_2) = M(\theta_2 + \theta_1) = M(\theta_2)M(\theta_1)$, because addition commutes.

\end{itemize}

\vspace*{0.25in}

\textbf{5) 3D Rotations } \hfill \textit{(5 points)}
\begin{itemize}
\item[(a)]{

${ [{\boldsymbol {J}}_{23},{\boldsymbol {J}}_{13}]={\boldsymbol {J}}_{12},\quad [{\boldsymbol {J}}_{12},{\boldsymbol {J}}_{23}]={\boldsymbol {J}}_{13},\quad [{\boldsymbol {J}}_{13},{\boldsymbol {J}}_{12}]={\boldsymbol {J}}_{23}}$


}
\item[(b)]{
%Show that $M' = $ is also traceless and hermitian and that is has the same determinant as M.

The most general 2x2 trace-less and hermitian matrix can be written as
\begin{equation*}
M = \begin{pmatrix} z &  x+iy  \\ x-iy  & -z \end{pmatrix} \equiv \vec{\sigma} \cdot \vec{r}
\end{equation*}

The determinant is given by $-z^2 - x^2 -y^2 = -r^2$. 

Consider $M'=U^{\dagger}MU$, where U is unitary.

Now, $tr(M') = tr(U^{\dagger}MU) = tr(UU^{\dagger}M) = tr(M)$\\

And $(M')^\dagger = (U^{\dagger}MU)^\dagger = (U M^{\dagger} U^{\dagger}) = (U^\dagger M U) = M'$\\

So $M'$ is also a 2x2 trace-less and hermitian matrix which can be written as $M = \vec{\sigma} \cdot \vec{r'}$\\

The determinant of M' is given by $det(M') = det(U^{\dagger})det(M)det(U) = 1 \times det(M) \times 1 = -r^2 $ . 

This implies that the Unitary transform performs a transformation that preserves the length of the vector $r$ defined by $\vec{\sigma} \cdot \vec{r}$.
The action of U preforms a rotation on the vector defined by M.

}

\end{itemize}

\vspace*{0.25in}

\textbf{6) Lorentz Transformations } \hfill \textit{(5 points)}
\begin{itemize}
\item[(a)]{


\begin{equation*}
e^{I\eta} = 1 + I\eta + \frac{I^2\eta^2}{2!} + \frac{I^3\eta^3}{3!} + \frac{I^4\eta^4}{4!} + \dots
\end{equation*}

$I_B^2 = \begin{bmatrix} 1 & 0  \\ 0 & 1 \end{bmatrix} $

\begin{equation*}
e^{I\eta} = I\left(\eta + \frac{I^2\eta^3}{3!} + \frac{I^4\eta^5}{5!} + \dots \right) + \left(1 + \frac{I^2\eta^2}{2!} + \frac{I^4\eta^4}{4!}  + \dots \right)
\end{equation*}

\begin{equation*}
e^{I\eta} = I\left(\eta + \frac{\eta^3}{3!} + \frac{\eta^5}{5!} + \dots \right) + \left(1 + \frac{\eta^2}{2!} + \frac{\eta^4}{4!}  + \dots \right) \\
= I \sinh(\eta) + \cosh(\eta)
\end{equation*}


}
\item[(b)]{

The origin of the primed frame is at $x'=0$ in the prime frame and at $x=vt$ in the unprimed frame (assuming the origins coincided at t=0)

\be
x = t'\sinh(\eta) \textrm{ and } t = t'\cosh(\eta)
\ee

\be
v=\frac{x}{t} = \tanh(\eta)  \textrm{ and } \cosh^{-2} = 1 - \tanh^2 
\ee

\be
\Rightarrow \cosh(\eta) = \frac{1}{\sqrt{1-v^2}} \equiv \gamma 
\ee

\be
\sinh(\eta) = \frac{v}{\sqrt{1-v^2}} = \beta\gamma 
\ee


}

\end{itemize}




\end{document}
