\documentclass[paper=letter,11pt]{scrartcl}

\KOMAoptions{headinclude=true, footinclude=false}
\KOMAoptions{DIV=14, BCOR=5mm}
\KOMAoptions{numbers=noendperiod}
\KOMAoptions{parskip=half}
\addtokomafont{disposition}{\rmfamily}
\addtokomafont{part}{\LARGE}
\addtokomafont{descriptionlabel}{\rmfamily}
%\setkomafont{pageheadfoot}{\normalsize\sffamily}
\setkomafont{pagehead}{\normalsize\rmfamily}
%\setkomafont{publishers}{\normalsize\rmfamily}
\setkomafont{caption}{\normalfont\small}
\setcapindent{0pt}
\deffootnote[1em]{1em}{1em}{\textsuperscript{\thefootnotemark}\ }


\usepackage{amsmath}
\usepackage[varg]{txfonts}
\usepackage[T1]{fontenc}
\usepackage{graphicx}
\usepackage{xcolor}
\usepackage[american]{babel}
% hyperref is needed in many places, so include it here
\usepackage{hyperref}

\usepackage{xspace}
\usepackage{multirow}
\usepackage{float}


\usepackage{braket}
\usepackage{bbm}
\usepackage{relsize}
\usepackage{tcolorbox}

\def\ketY{\ensuremath{\ket {\Psi}}}
\def\iGeV{\ensuremath{\textrm{GeV}^{-1}}}
%\def\mp{\ensuremath{m_{\textrm{proton}}}}
\def\rp{\ensuremath{r_{\textrm{proton}}}}
\def\me{\ensuremath{m_{\textrm{electron}}}}
\def\aG{\ensuremath{\alpha_G}}
\def\rAtom{\ensuremath{r_{\textrm{atom}}}}
\def\rNucl{\ensuremath{r_{\textrm{nucleus}}}}
\def\GN{\ensuremath{\textrm{G}_\textrm{N}}}
\def\ketX{\ensuremath{\ket{\vec{x}}}}
\def\ve{\ensuremath{\vec{\epsilon}}}


\def\ABCDMatrix{\ensuremath{\begin{pmatrix} A &  B  \\ C  & D \end{pmatrix}}}
\def\xyprime{\ensuremath{\begin{pmatrix} x' \\ y' \end{pmatrix}}}
\def\xyprimeT{\ensuremath{\begin{pmatrix} x' &  y' \end{pmatrix}}}
\def\xy{\ensuremath{\begin{pmatrix} x \\ y \end{pmatrix}}}
\def\xyT{\ensuremath{\begin{pmatrix} x & y \end{pmatrix}}}

\def\IMatrix{\ensuremath{\begin{pmatrix} 0 &  1  \\ -1  & 0 \end{pmatrix}}}
\def\IBoostMatrix{\ensuremath{\begin{pmatrix} 0 &  1  \\ 1  & 0 \end{pmatrix}}}
\def\JThree{\ensuremath{\begin{pmatrix}    0 & -i & 0  \\ i & 0  & 0 \\ 0 & 0 & 0 \end{pmatrix}}} 
\def\JTwo{\ensuremath{\begin{bmatrix}    0 & 0 & -i  \\ 0 & 0  & 0 \\ i & 0 & 0 \end{bmatrix}}}
\def\JOne{\ensuremath{\begin{bmatrix}    0 & 0 & 0  \\ 0 & 0  & -i \\ 0 & i & 0 \end{bmatrix}}}
\def\etamn{\ensuremath{\eta_{\mu\nu}}}
\def\Lmn{\ensuremath{\Lambda^\mu_\nu}}
\def\dmn{\ensuremath{\delta^\mu_\nu}}
\def\wmn{\ensuremath{\omega^\mu_\nu}}
\def\be{\begin{equation*}}
\def\ee{\end{equation*}}
\def\bea{\begin{eqnarray*}}
\def\eea{\end{eqnarray*}}
\def\bi{\begin{itemize}}
\def\ei{\end{itemize}}
\def\fmn{\ensuremath{F_{\mu\nu}}}
\def\fMN{\ensuremath{F^{\mu\nu}}}
\def\bc{\begin{center}}
\def\ec{\end{center}}
\def\nus{$\nu$s}

\def\adagger{\ensuremath{a_{p\sigma}^\dagger}}
\def\lineacross{\noindent\rule{\textwidth}{1pt}}

\newcommand{\multiline}[1] {
\begin{tabular} {|l}
#1
\end{tabular}
}

\newcommand{\multilineNoLine}[1] {
\begin{tabular} {l}
#1
\end{tabular}
}



\newcommand{\lineTwo}[2] {
\begin{tabular} {|l}
#1 \\
#2
\end{tabular}
}

\newcommand{\rmt}[1] {
\textrm{#1}
}


%
% Units
%
\def\m{\ensuremath{\rmt{m}}}
\def\GeV{\ensuremath{\rmt{GeV}}}
\def\pt{\ensuremath{p_\rmt{T}}}


\def\parity{\ensuremath{\mathcal{P}}}

\usepackage{cancel}
\usepackage{ mathrsfs }
\def\bigL{\ensuremath{\mathscr{L}}}

\usepackage{ dsfont }



\usepackage{fancyhdr}
\fancyhf{}


\def\ketY{\ensuremath{\ket {\Psi}}}
\def\iGeV{\ensuremath{\textrm{GeV}^{-1}}}
\def\mp{\ensuremath{m_{\textrm{proton}}}}
\def\rp{\ensuremath{r_{\textrm{proton}}}}
\def\me{\ensuremath{m_{\textrm{electron}}}}
\def\aG{\ensuremath{\alpha_G}}
\def\rAtom{\ensuremath{r_{\textrm{atom}}}}
\def\rNucl{\ensuremath{r_{\textrm{nucleus}}}}
\def\GN{\ensuremath{\textrm{G}_\textrm{N}}}


\lhead{\Large 33-444} % \hfill Introduction to Particle Physics \hfill Spring 2022}
\chead{\Large Introduction to Particle Physics} % \hfill Spring 2022}
\rhead{\Large Spring 2022} % \hfill Introduction to Particle Physics \hfill Spring 2022}
\begin{document}
\thispagestyle{fancy}





%\begin{tabular}{c}
%{\large 33-444 \hfill Intro To Particle \hfill Spring 2022\\}
%\hline 
%\end{tabular}

\begin{center}
{\huge \textbf{Homework Set \#1}}
\large

{\textbf{ Due Date:} Before class Friday February 4th  }
\end{center}

{\large
\textbf{1) You} \hfill \textit{(2 points)}
\begin{itemize}
\item[(a)]What is your major/minor ? 
\item[(b)]When do you plan on graduating?
\item[(c)]What do you want to do after graduation ? (eg: grad school ? if so, what subject ? if not, what industry?)
\item[(d)]What do you most want to get out of this course ? 
\end{itemize}

\vspace*{0.25in}

\textbf{2) Radius of Planets} \hfill \textit{(5 points)}
\begin{itemize}
\item[(a)] Calculate $r_{\textrm{planet}}$ in terms of $\alpha$, \aG, \mp, and \me. 
\item[(b)] Express your answer in terms of $r_{\textrm{atom}}$.
\item[(c)] How does this estimate compare with the radius of the earth ? 
\end{itemize}

\vspace*{0.25in}

%\textbf{3) Strength of Gravity on Earth} \hfill \textit{(5 points)}
%\begin{itemize}
%\item[(a)]Calculate the local strength of gravity $g_{\textrm{local}}$ in terms of $\alpha$, \aG, \mp, and \me.
%\item[(b)] Express your answer in terms of $r_{\textrm{atom}}$.
%\item[(b)]What is your estimated value in mks units Use $r_{\textrm{atom}} ~ 10^{-10} m$ ?
%\item[(c)]How does this compare with the well-known value of 9.8 m/s$^2$ ?
%\end{itemize}
%
%\vspace*{0.25in}

\textbf{3) Solid State Physics} \hfill \textit{(5 points)}
\begin{itemize}
\item[(a)] Assume a solid is composed of closely packed atoms. What is the spacing between atoms?  Express your result in terms of  $\alpha$, \aG, \mp, and \me.
\item[(b)] If you wanted to study the crystal structure of a solid material with $Z\sim10$ using light, what wavelength of photons would you need ?
Express your result in terms of  $\alpha$, \aG, \mp, and \me.
\item[(c)] Where in the spectrum of EM radiation do these photons lie?
\end{itemize}

\vspace*{0.25in}


%\textbf{4) Neutron Stars } \hfill \textit{(5 points)}
%\begin{itemize}
%\item[(a)]Estimate the radius, mass, and speed of sound for neutron stars in terms of $\alpha$, \aG, \mp, and \me.
%Assume that a neutron star is a solid made of neutrons and $\mp \sim m_{\textrm{neutron}}$. \\
%(Hint: the speed of sound is given by the square-root of the pressure over the mass density)
%\item[(b)]What are you estimated values in mks units ?
%\item[(c)]Compare your estimates to actual values for Neutron Stars quoted online.
%\item[(d)]Look up $m_{\textrm{neutron}}$. How does this compare with the assumption of $\mp \sim m_{\textrm{neutron}}$?
%\end{itemize}

\textbf{4) 2D Rotations } \hfill \textit{(5 points)}
\begin{itemize}
\item[(a)]Show that $R(\Theta) = e^{I\Theta} = cos(\Theta)+ I sin(\Theta)$
where, $I =  \begin{bmatrix}
    0 & 1  \\
    -1 & 0
  \end{bmatrix} $

\item[(b)]Show that 2D rotations commute and that the multiplciation law is given by $R(\Theta_{1})R(\Theta_{2}) = R(\Theta_{1} + \Theta_{2})$
\item[(c)]\textbf{Show that SO(2) $\simeq$ U(1)}. Consider the complex plane and independent variables $z$ and $z^*$ where $z = x+iy$ and $z^*$ is the complex conjugate.
What is $zz^*$ in terms of $x$ and $y$ ?
Consider the action of the operation: $z\rightarrow e^{i\theta}z$,$z^*\rightarrow e^{-i\theta}z^*$
Show that this action commutes and satisfies same multiplication law as we found for SO(2).
The group of transformations $e^{i\theta}$ is referred to as $U(1)$,  for Unitary and 1 dimensional.

\end{itemize}
}

\vspace*{0.25in}

\textbf{5) 3D Rotations } \hfill \textit{(5 points)}
\begin{itemize}
\item[(a)]Work out the ``algebra'' of the generators of 3D rotations $J_i$. \\
Where $
J_{12} =  \begin{bmatrix}    0 & -1 & 0  \\ 1 & 0  & 0 \\ 0 & 0 & 0 \end{bmatrix}, 
\hfill
 J_{13} =  \begin{bmatrix}    0 & 0 & 1  \\ 0 & 0  & 0 \\ -1 & 0 & 0 \end{bmatrix} 
\hfill
 J_{23} =  \begin{bmatrix}    0 & 0 & 0  \\ 0 & 0  & -1 \\ 0 & 1 & 0 \end{bmatrix} 
$\\
Working out the algebra means calculating the commutation relations $[J_i,J_j]$.
\item[(b)]\textbf{Show that SO(3) $\simeq$ SU(2)}
Let M be a traceless $2\times2$ hermitian matrix and U be a $2\times2$  unitary matrix.  
Write down the most general form of this matrix.
Show that it can be written in the form $\vec{\sigma}\cdot\vec{r}$, where $\vec{\sigma}$ is a vector of the 2x2 pauli matrices ie $(\sigma_x,\sigma_y,\sigma_z)$.
What is the determinant of M ?
Now consider performing a unitary transformation $M' = U^{\dagger}MU$, where U is a unitrary 2x2 matrix.
Show that $M'$ is also traceless and hermitian and therefore can be also written as $\vec{\sigma}\cdot\vec{r'}$
What is the determinant of M'?
Comment.
\end{itemize}

\vspace*{0.25in}

\textbf{6) Lorentz Transformations} \hfill \textit{(5 points)}

In class we showed the generator for boosts (in 1D is given by) $I_\mathrm{B} =  \begin{bmatrix}    0 & 1  \\    1 & 0  \end{bmatrix} $.
\begin{itemize}
\item[(a)]{Show that $B(\eta) = e^{\eta I_{\mathrm{B}}} = \cosh(\eta)+ I_\mathrm{B} \sinh(\eta)$}
\item[(b)]{
This implies
\be
\begin{pmatrix} x'^0 \\ x'^1\end{pmatrix} = \begin{pmatrix} \cosh\eta & \sinh\eta \\ \sinh\eta & \cosh\eta  \end{pmatrix} \begin{pmatrix} x^0 \\ x^1\end{pmatrix}
\ee
Derive the relationship  between $\cosh(\eta)$ and $\sinh(\eta)$ to $\beta = v$ and $\gamma = \frac{1}{\sqrt{1-v^2}} $\\
(Hint: consider the primed reference frame moving at velocity v with respect to the unprimed reference frame)
}
\end{itemize}


\end{document}
