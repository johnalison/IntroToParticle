\input{latexSetup}
%\documentclass[margin,line]{res}
\usepackage{braket}

\def\ketY{\ensuremath{\ket {\Psi}}}
\def\iGeV{\ensuremath{\textrm{GeV}^{-1}}}
\def\mp{\ensuremath{m_{\textrm{proton}}}}
\def\rp{\ensuremath{r_{\textrm{proton}}}}
\def\me{\ensuremath{m_{\textrm{electron}}}}
\def\aG{\ensuremath{\alpha_G}}
\def\rAtom{\ensuremath{r_{\textrm{atom}}}}
\def\rNucl{\ensuremath{r_{\textrm{nucleus}}}}
\def\GN{\ensuremath{\textrm{G}_\textrm{N}}}

\usepackage{fancyhdr}

\fancyhf{}
\lhead{\Large 33-444} % \hfill Introduction to Particle Physics \hfill Spring 2019}
\chead{\Large Introduction to Particle Physics} % \hfill Spring 2019}
\rhead{\Large Spring 2019} % \hfill Introduction to Particle Physics \hfill Spring 2019}

\begin{document}
\thispagestyle{fancy}

\begin{center}
{\huge \textbf{Lecture 2}}
\end{center}

{\fontsize{14}{16}\selectfont

\section{The right way to think about the world}

Talked last time about what our world consists of Matter/Forces ect.
This class we will start by talking about the ``right way'' to view all of this. 

Something that is not often taught (even in gradschool).
However it is easy and extremely powerful....

Compton lectures that I gave. 
Got into trouble. 
Diagrams like: 
Feature of Life -> Evolution -> DNA -> molecules -> chemistry  -> Atoms ->  Electrons-> Quantum Mechanics -> Standard Model
``Newton's Dream''

Go through a few examples of this kind of reasoning:
Teeth behind these statements

Idea that can describe world around us using a few basic physical parameters.
Powerful (Fun!) way of estimating $\sim$anything to order of magnitude.

\textbf{Dimensional Analysis and ``$\sim$''}

Put in the right physics to get answers to within ``geometric factors'':
\begin{itemize}
\item[-] Wont worry about factors of 2 or $\pi$ etc
\item[-] Use ``$\sim$'' not ``=''
\end{itemize}

\underline{Examples}

(Volume of something) $\sim$ (size$)^3$
Cube = $R^{3}$ $\sim$ $R^3$\\
Sphere = 4/3$\pi R^3$ = 4.2 R3 $\sim$ $R^3$\\
Sphere = 1/6$\pi D^3$  $\sim$ $D^3$\\
Cylinder = R$\times\pi R^{2}$  $\sim$ $R^3$  (if two scales use $r^2R$) 

Kinematic energy = 1/2 m$v^2$ $\sim$ m$v^2$\\
Ive been doing this already: ``$\Delta$p$\Delta X \geq $ h''
(...it is really $\Delta$p$\Delta X \geq $ h/(4$\pi$) )

\textbf{Natural Units}

Units...

I hate units! All numbers are really unit-less.
Always comparing some quantity relative to some standard. 
We will work in “Natural Units”.

Very easy. (Much easier than Metric/British/cgm/mks ...) \\
- Standard is set by basic physical principles.\\
- Numbers have direct physical interpretations.

c $\equiv$ 1: [Distance]/[Time] $\equiv$ 1\\
- Time and distance have same units\\
- Already familiar with this: ``Its about an hour from here''\\
- E = m\\

h $\equiv$ 1: [Energy]$\times$[Time] = 1 and [Energy]$\times$[Distance] = 1\\
- Energy (or Mass) is inversely related to distance or time.

Write everything in terms of [Energy]: Will often use 1 GeV $\sim$ \mp\ as basic unit.

Will now do everything in terms of GeV. 
Can use conversions to get back to human units

\textbf{Conversions:}\\
GeV = $10^-27 $ kg\\
\iGeV = $10^{-16}$ m\\
\iGeV = $6 \times 10^{-25}$ s\\

\underline{Examples:}\\
Proton Weight:  GeV \\
Proton Size:    \iGeV \\
My height: 1m $\sim$ 10$^{16}$ \iGeV \textit{(I am as tall as $10^{16}$ protons stacked on top of each other)}\\
My weight: 100kg $\sim$ 10$^{29}$ GeV \textit{(I am made of $10^{29}$ protons)}


\begin{figure}[h]
\centering
\includegraphics[width=0.9\textwidth]{./EMandGravity.pdf}
\end{figure}

\textbf{The world with 4 numbers.}\\
\underline{Claim:} $\sim$everything in world combination of these numbers\\
\begin{itemize}
\item[-]\mp = 1 GeV 
\item[-]$\alpha$ = 1/137
\item[-]\me = 10$^{-3}$ \GeV
\item[-]\aG $\equiv$ $G_N \mp^2$ = $10^{-39}$
\end{itemize}

Will work through some quick examples.

\underline{\textbf{Atoms:}}\\
$E\sim - \frac{Z\alpha}{r} + \frac{p^2}{m_e}$\\
$p\times r \sim 1 \Rightarrow$ $E\sim - \frac{Z\alpha}{r} + \frac{1}{m_e r^2}$ solving... $\rAtom \sim \frac{1}{Z\alpha m_e}$

Nucleus is protons + neutrons pacted in 3D volume. 
$\rNucl \sim Z^{1/3}\rp \sim \frac{Z^{1/3}}{\mp}$ 

Relative scale: 
$\frac{\rNucl}{\rAtom} \sim \frac{\alpha \me}{Z^{2/3}\mp} \sim 10^{-5}$  (Justifies claim that most of volume due to electron)

Back to the electron...
$p_e \sim \frac{1}{\rAtom} \sim \me (Z\alpha)  \Rightarrow  v_{\textrm{electron}} \sim (Z\alpha)$\\
Interesting physics content: 
v cannot be bigger than one ($\equiv$ c).  (when $\sim$1 electron relativistic, can get unstable to pair produciton, eg: excite electrons-positions from vacuum.)\\
$\Rightarrow$ there cannot be more than $\sim \frac{1}{\alpha}$ stable elements!

Also note $v_{\textrm{electron}} \sim (Z\alpha) << 1$ for low Z.
This is why:
\begin{itemize}
\item[-] we could do QM first with out relativity!
\item[-] electricity tends to be stronger than magnetism in everyday applications/conditions.
\end{itemize}


$E_{\textrm{atom}} \sim \frac{Z\alpha}{\rAtom} \sim Z^2\alpha^2\me$
(For hydrogen = $10^{-4} \times$ 0.5 MeV $\sim$ 50 eV  / Actually 13.6 eV)

For atoms,  electron mass is king... dominates.

\underline{\textbf{Solids:}}

(To within our $\sim$) Solids just atoms stacked next to each other

Mass Density: Mass/Volume
$\rho_{\textrm{solid}} \sim \frac{Z\mp}{\rAtom^3}$

Pressure of Solid: Force/Area or Energy/Volume

$P_{\textrm{solid}} \sim \frac{Z^2\alpha^2\me}{\rAtom^3} \sim Z^5\alpha^5\me^4$


(Ratio of two gqive the speed of sound)\\
$v_{\textrm{sound}}\sim \sqrt{\frac{P_{\textrm{solid}}} {\rho_{\textrm{solid}}}} \sim \sqrt{\alpha} {\mp \rAtom}$
\begin{tabular}{l}
Predict $\sim$25,000 m/s \\
Beryllium 12,890 m/s     \\
Diamond 12,000 m/s       \\
Steel 6000 m/s           \\
\end{tabular}


\underline{\textbf{Planets:}}

Solids where gravitational pressure balanced by solid pressure

$E_{\textrm{Gravity}} \sim \frac{\GN M_{\textrm{Planet}}^2}{R_{\textrm{Planet}}}$

$P_{\textrm{Gravity}} \sim \frac{E_{\textrm{Gravity}}}{V_{\textrm{Planet}}} \sim \frac{\GN M_{\textrm{Planet}}^2}{R_{\textrm{Planet}}^4}$

$M_{\textrm{Planet}} \sim \rho_{\textrm{solid}} \times R_{\textrm{Planet}}^3 \sim \frac{Z\mp R_{\textrm{Planet}}^3}{\rAtom^3}$

$P_{\textrm{Gravity}} \sim \frac{\GN Z^2 \mp^2 R_{\textrm{Planet}}^2}{\rAtom^6} $

Have a planet when $P_{\textrm{Gravity}} \sim P_{\textrm{Solid}}$  or $\frac{\GN Z^2 \mp^2 R_{\textrm{Planet}}^2}{\rAtom^6} \sim \frac{Z\alpha}{\rAtom^4}$

$\Rightarrow R_{\textrm{Planet}} \sim \sqrt{\frac{1}{\GN \mp^2 Z^3 \alpha \me^2}}  \sim \sqrt{\frac{\alpha}{\aG}} \times \rAtom $

Planets/atoms relative size direct result or realtive EM vs Gravity strength.

This is why things are big, despite being governed by microscopic laws.

\begin{tabular}{lcc}
  & $R_{\textrm{Earth}}$ & $M_{\textrm{Earth}}$  \\
\hline
Prediction:  & $10^7$ m & $10^{25}$ kg\\
Actual:      & $6.4 \times 10^6$ m & $5.9 \times 10^{24}$ kg\\
\end{tabular}

\underline{\textbf{Life etc.:}}

 
}

\end{document}


