\documentclass[paper=letter,11pt]{scrartcl}

\KOMAoptions{headinclude=true, footinclude=false}
\KOMAoptions{DIV=14, BCOR=5mm}
\KOMAoptions{numbers=noendperiod}
\KOMAoptions{parskip=half}
\addtokomafont{disposition}{\rmfamily}
\addtokomafont{part}{\LARGE}
\addtokomafont{descriptionlabel}{\rmfamily}
%\setkomafont{pageheadfoot}{\normalsize\sffamily}
\setkomafont{pagehead}{\normalsize\rmfamily}
%\setkomafont{publishers}{\normalsize\rmfamily}
\setkomafont{caption}{\normalfont\small}
\setcapindent{0pt}
\deffootnote[1em]{1em}{1em}{\textsuperscript{\thefootnotemark}\ }


\usepackage{amsmath}
\usepackage[varg]{txfonts}
\usepackage[T1]{fontenc}
\usepackage{graphicx}
\usepackage{xcolor}
\usepackage[american]{babel}
% hyperref is needed in many places, so include it here
\usepackage{hyperref}

\usepackage{xspace}
\usepackage{multirow}
\usepackage{float}


\usepackage{braket}
\usepackage{bbm}
\usepackage{relsize}
\usepackage{tcolorbox}

\def\ketY{\ensuremath{\ket {\Psi}}}
\def\iGeV{\ensuremath{\textrm{GeV}^{-1}}}
%\def\mp{\ensuremath{m_{\textrm{proton}}}}
\def\rp{\ensuremath{r_{\textrm{proton}}}}
\def\me{\ensuremath{m_{\textrm{electron}}}}
\def\aG{\ensuremath{\alpha_G}}
\def\rAtom{\ensuremath{r_{\textrm{atom}}}}
\def\rNucl{\ensuremath{r_{\textrm{nucleus}}}}
\def\GN{\ensuremath{\textrm{G}_\textrm{N}}}
\def\ketX{\ensuremath{\ket{\vec{x}}}}
\def\ve{\ensuremath{\vec{\epsilon}}}


\def\ABCDMatrix{\ensuremath{\begin{pmatrix} A &  B  \\ C  & D \end{pmatrix}}}
\def\xyprime{\ensuremath{\begin{pmatrix} x' \\ y' \end{pmatrix}}}
\def\xyprimeT{\ensuremath{\begin{pmatrix} x' &  y' \end{pmatrix}}}
\def\xy{\ensuremath{\begin{pmatrix} x \\ y \end{pmatrix}}}
\def\xyT{\ensuremath{\begin{pmatrix} x & y \end{pmatrix}}}

\def\IMatrix{\ensuremath{\begin{pmatrix} 0 &  1  \\ -1  & 0 \end{pmatrix}}}
\def\IBoostMatrix{\ensuremath{\begin{pmatrix} 0 &  1  \\ 1  & 0 \end{pmatrix}}}
\def\JThree{\ensuremath{\begin{pmatrix}    0 & -i & 0  \\ i & 0  & 0 \\ 0 & 0 & 0 \end{pmatrix}}} 
\def\JTwo{\ensuremath{\begin{bmatrix}    0 & 0 & -i  \\ 0 & 0  & 0 \\ i & 0 & 0 \end{bmatrix}}}
\def\JOne{\ensuremath{\begin{bmatrix}    0 & 0 & 0  \\ 0 & 0  & -i \\ 0 & i & 0 \end{bmatrix}}}
\def\etamn{\ensuremath{\eta_{\mu\nu}}}
\def\Lmn{\ensuremath{\Lambda^\mu_\nu}}
\def\dmn{\ensuremath{\delta^\mu_\nu}}
\def\wmn{\ensuremath{\omega^\mu_\nu}}
\def\be{\begin{equation*}}
\def\ee{\end{equation*}}
\def\bea{\begin{eqnarray*}}
\def\eea{\end{eqnarray*}}
\def\bi{\begin{itemize}}
\def\ei{\end{itemize}}
\def\fmn{\ensuremath{F_{\mu\nu}}}
\def\fMN{\ensuremath{F^{\mu\nu}}}
\def\bc{\begin{center}}
\def\ec{\end{center}}
\def\nus{$\nu$s}

\def\adagger{\ensuremath{a_{p\sigma}^\dagger}}
\def\lineacross{\noindent\rule{\textwidth}{1pt}}

\newcommand{\multiline}[1] {
\begin{tabular} {|l}
#1
\end{tabular}
}

\newcommand{\multilineNoLine}[1] {
\begin{tabular} {l}
#1
\end{tabular}
}



\newcommand{\lineTwo}[2] {
\begin{tabular} {|l}
#1 \\
#2
\end{tabular}
}

\newcommand{\rmt}[1] {
\textrm{#1}
}


%
% Units
%
\def\m{\ensuremath{\rmt{m}}}
\def\GeV{\ensuremath{\rmt{GeV}}}
\def\pt{\ensuremath{p_\rmt{T}}}


\def\parity{\ensuremath{\mathcal{P}}}

\usepackage{cancel}
\usepackage{ mathrsfs }
\def\bigL{\ensuremath{\mathscr{L}}}

\usepackage{ dsfont }



\usepackage{fancyhdr}
\fancyhf{}

%\documentclass[margin,line]{res}
\usepackage{braket}
\usepackage{bbm}
\usepackage{relsize}

\def\ketY{\ensuremath{\ket {\Psi}}}
\def\iGeV{\ensuremath{\textrm{GeV}^{-1}}}


\def\ABCDMatrix{\ensuremath{\begin{pmatrix} A &  B  \\ C  & D \end{pmatrix}}}
\def\xyprime{\ensuremath{\begin{pmatrix} x' \\ y' \end{pmatrix}}}
\def\xyprimeT{\ensuremath{\begin{pmatrix} x' &  y' \end{pmatrix}}}
\def\xy{\ensuremath{\begin{pmatrix} x \\ y \end{pmatrix}}}
\def\xyT{\ensuremath{\begin{pmatrix} x & y \end{pmatrix}}}

\def\IMatrix{\ensuremath{\begin{pmatrix} 0 &  1  \\ -1  & 0 \end{pmatrix}}}
\def\IBoostMatrix{\ensuremath{\begin{pmatrix} 0 &  1  \\ 1  & 0 \end{pmatrix}}}
\def\JThree{\ensuremath{\begin{pmatrix}    0 & -i & 0  \\ i & 0  & 0 \\ 0 & 0 & 0 \end{pmatrix}}} 
\def\JTwo{\ensuremath{\begin{bmatrix}    0 & 0 & -i  \\ 0 & 0  & 0 \\ i & 0 & 0 \end{bmatrix}}}
\def\JOne{\ensuremath{\begin{bmatrix}    0 & 0 & 0  \\ 0 & 0  & -i \\ 0 & i & 0 \end{bmatrix}}}
\def\etamn{\ensuremath{\eta_{\mu\nu}}}
\def\Lmn{\ensuremath{\Lambda^\mu_\nu}}
\def\dmn{\ensuremath{\delta^\mu_\nu}}
\def\wmn{\ensuremath{\omega^\mu_\nu}}
\def\be{\begin{equation*}}
\def\ee{\end{equation*}}

%\def\xMu{\ensuremath{x^\mu}

\usepackage{fancyhdr}

\fancyhf{}
\lhead{\Large 33-444} % \hfill Introduction to Particle Physics \hfill Spring 2019}
\chead{\Large Introduction to Particle Physics} % \hfill Spring 2019}
\rhead{\Large Spring 2019} % \hfill Introduction to Particle Physics \hfill Spring 2019}

\begin{document}
\thispagestyle{fancy}

\begin{center}
{\huge \textbf{Lecture 4}}
\end{center}

{\fontsize{14}{16}\selectfont


\section*{Special Relativity}

\underline{\textbf{Recap from Last Time}}

Talking about relativity means talking about Lorentz invariance.

If there is a point in space time: \\
%xrightarrow[\text{world}]
\begin{center}
$(t,x) \xrightarrow[\text{observer}]{\text{another moving}} (t', x')$
\end{center}

Invariant notion of distance:
 
\be
t^2 - x^2 = t'^2 - x'^2
\ee

\underline{\textbf{Started with Rotations}}
Where we have an invariant notion of length of $\vec{p}$
\be
x^2 + y^2 = x'^2 + y'^2
\ee

Started in 2D, looking at the infinitesimal rotation
This lead to what is called ``generator'' \\

\IMatrix\\

\underline{Form a group:}
A group is a set of actions (in our case rotations) that multiply (or compose) with operation denoted by $\cdot$.

Four criteria for group:
\begin{itemize}
\item[-] Have Identity element:  (no rotation or by 360)
\item[-] Every element of the group has an inverse (for us, rotate by $360-\theta$)
\item[-] The group is closed:  for any elements thier product is also in the group.
\item[-] The multiplication is associative.
\end{itemize}

If commute: $a\cdot b= b\cdot a$ say that the group is ``Abelian''.

Last lecture studied 3D rotations and thier group ``SO(2)'' refers to matriecies studied  (S = Special (det =1) / O = Orthogonal (preseves length) / 2 = 2x2)\\

$\begin{pmatrix} \cos(\theta) &  \sin(\theta)  \\ -\sin(\theta)  & \cos(\theta) \end{pmatrix}$\\

We also looked at 3D rotations, theier group was ``SO(3)'', what do you think this stands for ???\\
We saw at the end of the lecture that  $SO(3) \simeq SU(2)$   
SU(2) = (Special / Unitary / 2x2 matricies.

\textit{(You will show in for homework that this is equavilant to another group U(1).)}


\noindent\rule{\textwidth}{1pt}

\textbf{All the machinery in place to look at Lorentz Transformations...}

Well almost all the machinery...  Remember invariant need to preserve is $t^2 - |\vec{x}|^2$

Lets deal with this minus sign...

4-vectors:  $x^\mu = (t,\vec{x})$  

Encode the minus sign in matrix:

\be
\eta_{\mu\nu} = \left\{ \begin{array}{rl} 1 & \mbox{$\mu = \nu=0$}  \\ -1 & \mbox{$\mu=\nu=1,2,3\ (i)$}  \\ 0 & \mbox{otherwise} \end{array} \right.
\ee

Can now write: $x_\mu = \etamn x^\nu = (t,-\vec{x})$  

And finally: $x_\mu x^\mu = t^2 - |\vec{x}|^2$

(Comment on $\eta^{\mu\nu} = \eta_{\mu\nu}$ and $\eta^\mu_\nu = \dmn$)

Now we after all the transformations that leave  $x_\mu x^\mu$ (Lorentz transformations)

\be
x'^\mu = \underbrace{\Lambda^\mu_\nu}_{\mbox{4x4 matrix}} x^\nu 
\ee

\underline{Same thing as before} Start with the infintesimal transofmations.

\be
\Lmn  = \dmn + \epsilon \wmn
\ee


\be
x'^\mu  = x^\mu + \epsilon \wmn x^\nu
\ee

Require $x'^\mu x'_\mu = x^\mu x_\mu$ or 

\begin{align*}
\etamn x'^\mu x'^\nu = \etamn x^\mu x^\nu \\
= \etamn x^\mu x^\nu + 2 \epsilon \etamn x^\nu \omega^\mu_\rho x^\rho
\end{align*}


So, $\underbrace{\etamn \omega^\mu_\rho}_{\equiv \omega_{\mu\rho} } x^\nu  x^\rho = 0$ 

After all 4x4 anti-symmetric matricies  $\omega_{\mu\rho} = -\omega_{\rho\mu}$

Lets just be damn explicit

\be
\omega_{\mu\nu} = \begin{bmatrix} 0 & a & b & c \\ -a & 0 & A & B \\ -b & -A & 0 & C \\ -c & -B & -C & 0 \end{bmatrix}
\ee

However, the think that enters the transformation is $\omega^\mu_\nu$.\\

Now raising a 0-compentent is free,  $\omega_{0i} = \omega^0_i$\\
raising a i-compentent costs a -1,  $\omega_{i0} = -\omega^i_0$\\
But also know $\omega_{0i} = -\omega_{i0}$
$\Rightarrow {\omega^0}_i = + {\omega^i}_0$

Same argument shows: ${\omega^i}_j = - {\omega^j}_i$

\be
\omega^\mu_\nu = \begin{bmatrix} 0 & a & b & c \\ a & 0 & A & B \\ b & -A & 0 & C \\ c & -B & -C & 0 \end{bmatrix}
\ee

Generators of the Lorentz Group. We have 6 ``generators'' abcABC.

3-rotations (guess which)  / 3-boosts

Crucial sign differnece between boosts/rotations


${\omega^1}_2 = \begin{pmatrix} 0 & 0 & 0 & 0 \\ 0 & 0 & 1 & 0 \\ 0 & -1 & 0 & 0 \\ 0 & 0 & 0 & 0 \end{pmatrix}  \rightarrow e^{\theta \IMatrix} = \cos\theta + \IMatrix \sin\theta$\\

${\omega^0}_1 = \begin{pmatrix} 0 & 1 & 0 & 0 \\ 1 & 0 & 0 & 0 \\ 0 & 0 & 0 & 0 \\ 0 & 0 & 0 & 0 \end{pmatrix}  \rightarrow e^{\eta \IBoostMatrix} = \cosh\eta + \IBoostMatrix \sinh\eta$

$\IBoostMatrix^2 = 1$ whereas $\IMatrix^2 = -1$

\be
\begin{pmatrix} x'^0 \\ x'^1\end{pmatrix} =
\ee

Can (but I wont) work out the Lie Algrabra....Homework.


%%%%%%%%
%
%$x'_i x'_i = x_i x_i$ is what it takes for $R$ to be a rotation. 
%Need to find the special $R$s such that this is satisfied.
%
%The identity matrix is written as $\delta_{ij}$, where $\delta_{ij} = 1$ if $i=j$, 0 otherwise.
%
%If no rotation at all: $R_{ij} = \delta_{ij}$,  of an infinitesimal rotation:  
%\begin{equation*}
%R_{ij} = \delta_{ij} + \epsilon w_{ij}
%\end{equation*}
%
%\begin{equation*}
%x'_{i} = x_i + \epsilon w_{ij} x_j
%\end{equation*}
%
%
%\begin{equation*}
%x'_{i}x'_{i} = x_i x_{i} + 2 \epsilon \underbrace{w_{ij} x_j x_i}_{ =0 \hspace*{0.05in} \forall x }  + \mathcal{O}(\epsilon^2)
%\end{equation*}
%
%$\Rightarrow w_{ij}$ has to be anti-symmetric $w_{ij} = - w_{ji}$.
%
%\noindent\rule{\textwidth}{1pt}
%
%Back to previous example, find \underline{finite} rotations (without any mention of geometry etc.)
%
%Take the infinitesimal rotation and do it n-times.
%
%Define $\theta = N\epsilon$
%
%\begin{eqnarray*}
%\xyprime = \left[ \mathlarger{\mathlarger{\mathbbm{1}}} + \epsilon   \IMatrix \right]\xy \equiv \left[ \mathlarger{\mathlarger{\mathbbm{1}}} + \epsilon   \textrm{I} \right]\xy
%\end{eqnarray*}
%
%So the finite rotation ($R(\theta)$) given by,
%\begin{eqnarray*}
%R(\theta) = (1 + \epsilon I)(1 + \epsilon I) \dots (1 + \epsilon I) \dots = (1+\epsilon I)^N = \left(1+\frac{\theta}{N}I\right)^N
%\end{eqnarray*}
%
%Now let $N\rightarrow \infty$,  $R(\theta) = e^{I\theta}$.
%
%Built up finite rotation from the infinitesimal rotations.
% 
%$x'(\theta) = R(\theta)x = e^{I\theta}$
%
%The meaning of $e^X$ when $X$ is a matrix is simply the expansion.
%\begin{eqnarray*}
%e^X = 1 + X + \frac{X^2}{2!} + \frac{X^3}{3!} + \dots
%\end{eqnarray*}
%
%
%$I^2 =  \IMatrix\IMatrix = {\begin{pmatrix} -1 &  0  \\ 0  & -1 \end{pmatrix}} = \mathlarger{\mathlarger{\mathbbm{-1}}} ! $
%We have just discovered i following our nose.
%
%
%
%\textbf{\underline{Strategy:}} First understand the action of the symmetry infinitesimally, then the big symmetry action is obtained by iterating the infinitesimal.
%Always $e^X$ where $X$ is the generator. 
%This is a great strategy for any kind of symmetry.
%
%Will now do 3D rotations... Something new happens.
%
%\section*{3D Rotations}
%
%3-parameters associated with a 3D rotation.
%
%Already saw, any rotation is of the form 
%
%\begin{equation*}
%x'_{i} = x_i + \epsilon w_{ij} x_j  \hspace{0.5in} \textrm{with } \hspace{0.5in} w_{ij} =- w_{ji}
%\end{equation*}
%
%
%Most general $3\time3$ anti-symmetric matrix: ${\begin{pmatrix} 0 & a & b  \\ -a & 0 & c \\ -b & -c & 0 \end{pmatrix}}$,  Now have 3 generators corresponding to the rotations in 3D.
%
%\begin{equation*}
%\epsilon w_{ij} = \epsilon_{12}\underbrace{{\begin{pmatrix} 0 & 1 & 0  \\ -1 & 0 & 0 \\ 0 & 0 & 0 \end{pmatrix}}}_{\text{generator we just saw}} + 
%                  \epsilon_{13}{\begin{pmatrix} 0 & 0 & 1  \\ 0 & 0 & 0 \\ -1 & 0 & 0 \end{pmatrix}} +
%                  \epsilon_{23}{\begin{pmatrix} 0 & 0 & 0  \\ 0 & 0 & -1 \\ 0 & -1 & 0 \end{pmatrix}}
%\end{equation*}
%
%How to get the finite version ? 
%Easy, just exponentiate. 
%
%Something new happens in 3D:
%\begin{itemize}
%\item[-]2D rotations commute        
%\item[-]3D rotations do not
%\end{itemize}
%
%Define $J_3 = \JThree$, $J_2 = \JTwo$, and $J_1 = \JOne$
%
%Rotations form a group.
%Any three rotations give something that is also a rotation: 
%\begin{equation*}
%e^{i\theta_3 J_3}e^{i\theta_2 J_2}e^{i\theta_1 J_1} = e^{\phi_3 J_3 + \phi_2 J_2 + \phi_1 J_1}
%\end{equation*}
%
%This can only be possible if
%\begin{equation*}
%[J_1, J_2] = i J_3  \textrm{+ cyclic}
%\end{equation*}
%
%
%Can step back and think about this more abstractly. 
%The matrices  we found form a group, but this group exists abstractly independent of these $3\times3$ matrices.
%Fully determined by the commutation relations.
%(Just like vectors and components)
%
%In general, many matrices that satisfy the algebra (the commutation relations). 
%These give different representations.
%
%\underline{Deep:} rotations can act on more than just 3D vectors.
%
%\begin{equation*}
%\begin{bmatrix} \left(J_i\right) &  0  \\ 0  & \left(J_j\right) \end{bmatrix}
%\end{equation*}
%this is officially a representation.  
%Its called a ``Reducible'' Representation.
%
%\underline{More Generally} 
%\begin{equation*}
%[J_a, J_b] = i f_{abc} J_c \hspace{0.3in} \textrm{where $J_a$, $ a = 1, 2,$ ... dim. of the group}
%\end{equation*}
%Lie found all the possible symmetries when J is hermitian (there are not many).
%
%\section*{One final example with rotations}
%Lets look at traceless $2\times2$ hermitian matrices. 
%Any $2\times2$, traceless, hermitian matrix can be written as: 
%\begin{equation*}
%M = \begin{pmatrix} z &  x+iy  \\ x-iy  & -z \end{pmatrix} \equiv \vec{\sigma} \cdot \vec{x}
%\end{equation*}
%where $\vec{\sigma} = (\sigma_x, \sigma_y, \sigma_z)$ and $\sigma_i$ are the Pauli matrices.
%
%Note that $\textrm{det}(M) = - \vec{x}\cdot\vec{x} = - |\vec{x}|$.
%
%Consider $M'=U^{\dagger}MU$, where U is unitary.
%Any unitary matrix can be written as a phase $e^{i\theta}$ times a $2\times2$ hermitian matrix with det = 1.  (``Special Unitary Matrix'')
%Because the phase cancels in $M'$ we will only consider U as unitary and det = 1. 
%$U \in SU(2)$
%
%If M is hermitian and traceless, then M' is still hermitian and traceless.
%\begin{equation*}
%M' = \sigma \cdot \vec{x'}_u
%\end{equation*}
%$\vec{x'}_{u} $ depends on U.  det(M') = det(M) $\Rightarrow \vec{x'}_u^2 = {\vec{x}}^2 $%= \vec{x}^2 $ length of \x
%direct correspondence between $2\times2$ hermitian matrices \& rotations.
%
%This is a 2D action of rotations.
%
%Not easy to generalize all of this to the Lorentz group...

}
\end{document}


