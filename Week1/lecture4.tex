\documentclass[paper=letter,11pt]{scrartcl}

\KOMAoptions{headinclude=true, footinclude=false}
\KOMAoptions{DIV=14, BCOR=5mm}
\KOMAoptions{numbers=noendperiod}
\KOMAoptions{parskip=half}
\addtokomafont{disposition}{\rmfamily}
\addtokomafont{part}{\LARGE}
\addtokomafont{descriptionlabel}{\rmfamily}
%\setkomafont{pageheadfoot}{\normalsize\sffamily}
\setkomafont{pagehead}{\normalsize\rmfamily}
%\setkomafont{publishers}{\normalsize\rmfamily}
\setkomafont{caption}{\normalfont\small}
\setcapindent{0pt}
\deffootnote[1em]{1em}{1em}{\textsuperscript{\thefootnotemark}\ }


\usepackage{amsmath}
\usepackage[varg]{txfonts}
\usepackage[T1]{fontenc}
\usepackage{graphicx}
\usepackage{xcolor}
\usepackage[american]{babel}
% hyperref is needed in many places, so include it here
\usepackage{hyperref}

\usepackage{xspace}
\usepackage{multirow}
\usepackage{float}


\usepackage{braket}
\usepackage{bbm}
\usepackage{relsize}
\usepackage{tcolorbox}

\def\ketY{\ensuremath{\ket {\Psi}}}
\def\iGeV{\ensuremath{\textrm{GeV}^{-1}}}
%\def\mp{\ensuremath{m_{\textrm{proton}}}}
\def\rp{\ensuremath{r_{\textrm{proton}}}}
\def\me{\ensuremath{m_{\textrm{electron}}}}
\def\aG{\ensuremath{\alpha_G}}
\def\rAtom{\ensuremath{r_{\textrm{atom}}}}
\def\rNucl{\ensuremath{r_{\textrm{nucleus}}}}
\def\GN{\ensuremath{\textrm{G}_\textrm{N}}}
\def\ketX{\ensuremath{\ket{\vec{x}}}}
\def\ve{\ensuremath{\vec{\epsilon}}}


\def\ABCDMatrix{\ensuremath{\begin{pmatrix} A &  B  \\ C  & D \end{pmatrix}}}
\def\xyprime{\ensuremath{\begin{pmatrix} x' \\ y' \end{pmatrix}}}
\def\xyprimeT{\ensuremath{\begin{pmatrix} x' &  y' \end{pmatrix}}}
\def\xy{\ensuremath{\begin{pmatrix} x \\ y \end{pmatrix}}}
\def\xyT{\ensuremath{\begin{pmatrix} x & y \end{pmatrix}}}

\def\IMatrix{\ensuremath{\begin{pmatrix} 0 &  1  \\ -1  & 0 \end{pmatrix}}}
\def\IBoostMatrix{\ensuremath{\begin{pmatrix} 0 &  1  \\ 1  & 0 \end{pmatrix}}}
\def\JThree{\ensuremath{\begin{pmatrix}    0 & -i & 0  \\ i & 0  & 0 \\ 0 & 0 & 0 \end{pmatrix}}} 
\def\JTwo{\ensuremath{\begin{bmatrix}    0 & 0 & -i  \\ 0 & 0  & 0 \\ i & 0 & 0 \end{bmatrix}}}
\def\JOne{\ensuremath{\begin{bmatrix}    0 & 0 & 0  \\ 0 & 0  & -i \\ 0 & i & 0 \end{bmatrix}}}
\def\etamn{\ensuremath{\eta_{\mu\nu}}}
\def\Lmn{\ensuremath{\Lambda^\mu_\nu}}
\def\dmn{\ensuremath{\delta^\mu_\nu}}
\def\wmn{\ensuremath{\omega^\mu_\nu}}
\def\be{\begin{equation*}}
\def\ee{\end{equation*}}
\def\bea{\begin{eqnarray*}}
\def\eea{\end{eqnarray*}}
\def\bi{\begin{itemize}}
\def\ei{\end{itemize}}
\def\fmn{\ensuremath{F_{\mu\nu}}}
\def\fMN{\ensuremath{F^{\mu\nu}}}
\def\bc{\begin{center}}
\def\ec{\end{center}}
\def\nus{$\nu$s}

\def\adagger{\ensuremath{a_{p\sigma}^\dagger}}
\def\lineacross{\noindent\rule{\textwidth}{1pt}}

\newcommand{\multiline}[1] {
\begin{tabular} {|l}
#1
\end{tabular}
}

\newcommand{\multilineNoLine}[1] {
\begin{tabular} {l}
#1
\end{tabular}
}



\newcommand{\lineTwo}[2] {
\begin{tabular} {|l}
#1 \\
#2
\end{tabular}
}

\newcommand{\rmt}[1] {
\textrm{#1}
}


%
% Units
%
\def\m{\ensuremath{\rmt{m}}}
\def\GeV{\ensuremath{\rmt{GeV}}}
\def\pt{\ensuremath{p_\rmt{T}}}


\def\parity{\ensuremath{\mathcal{P}}}

\usepackage{cancel}
\usepackage{ mathrsfs }
\def\bigL{\ensuremath{\mathscr{L}}}

\usepackage{ dsfont }



\usepackage{fancyhdr}
\fancyhf{}

%\documentclass[margin,line]{res}
\usepackage{braket}
\usepackage{bbm}
\usepackage{relsize}

\def\ketY{\ensuremath{\ket {\Psi}}}
\def\iGeV{\ensuremath{\textrm{GeV}^{-1}}}


\def\ABCDMatrix{\ensuremath{\begin{pmatrix} A &  B  \\ C  & D \end{pmatrix}}}
\def\xyprime{\ensuremath{\begin{pmatrix} x' \\ y' \end{pmatrix}}}
\def\xyprimeT{\ensuremath{\begin{pmatrix} x' &  y' \end{pmatrix}}}
\def\xy{\ensuremath{\begin{pmatrix} x \\ y \end{pmatrix}}}
\def\xyT{\ensuremath{\begin{pmatrix} x & y \end{pmatrix}}}

\def\IMatrix{\ensuremath{\begin{pmatrix} 0 &  1  \\ -1  & 0 \end{pmatrix}}}
\def\IBoostMatrix{\ensuremath{\begin{pmatrix} 0 &  1  \\ 1  & 0 \end{pmatrix}}}
\def\JThree{\ensuremath{\begin{pmatrix}    0 & -i & 0  \\ i & 0  & 0 \\ 0 & 0 & 0 \end{pmatrix}}} 
\def\JTwo{\ensuremath{\begin{bmatrix}    0 & 0 & -i  \\ 0 & 0  & 0 \\ i & 0 & 0 \end{bmatrix}}}
\def\JOne{\ensuremath{\begin{bmatrix}    0 & 0 & 0  \\ 0 & 0  & -i \\ 0 & i & 0 \end{bmatrix}}}
\def\etamn{\ensuremath{\eta_{\mu\nu}}}
\def\Lmn{\ensuremath{\Lambda^\mu_\nu}}
\def\dmn{\ensuremath{\delta^\mu_\nu}}
\def\wmn{\ensuremath{\omega^\mu_\nu}}
\def\be{\begin{equation*}}
\def\ee{\end{equation*}}

%\def\xMu{\ensuremath{x^\mu}

\usepackage{fancyhdr}

\fancyhf{}
\lhead{\Large 33-444} % \hfill Introduction to Particle Physics \hfill Spring 2019}
\chead{\Large Introduction to Particle Physics} % \hfill Spring 2019}
\rhead{\Large Spring 2019} % \hfill Introduction to Particle Physics \hfill Spring 2019}

\begin{document}
\thispagestyle{fancy}

\begin{center}
{\huge \textbf{Lecture 4}}
\end{center}

{\fontsize{14}{16}\selectfont


\section*{Special Relativity}

\underline{\textbf{Re-cap from Last Time}}

Talking about relativity means talking about Lorentz invariance.

If there is a point in space time: \\
%xrightarrow[\text{world}]
\begin{center}
$(t,x) \xrightarrow[\text{observer}]{\text{another moving}} (t', x')$
\end{center}

Invariant notion of distance:
 
\be
t^2 - x^2 = t'^2 - x'^2
\ee

\underline{\textbf{Started with Rotations}}
Where we have an invariant notion of length of $\vec{p}$
\be
x^2 + y^2 = x'^2 + y'^2
\ee

Started in 2D, looking at the infinitesimal rotation
This lead to what is called ``generator'' \\

\IMatrix\\

\underline{Form a group:}
A group is a set of actions (in our case rotations) that multiply (or compose) with operation denoted by $\cdot$.

Four criteria for group:
\begin{itemize}
\item[-] Have Identity element:  (no rotation or by 360)
\item[-] Every element of the group has an inverse (for us, rotate by $360-\theta$)
\item[-] The group is closed:  for any elements their product is also in the group.
\item[-] The multiplication is associative.
\end{itemize}

If commute: $a\cdot b= b\cdot a$ say that the group is ``Abelian''.
If they do not commute say that the group is ``non-Abelian''.
We have already seen examples of both of these, both types are also critical for the structure of the SM.

Last lecture studied 2D rotations and their group ``SO(2)'' refers to matrices studied  (S = Special (det =1) / O = Orthogonal (preserves length) / 2 = 2x2)\\

$\begin{pmatrix} \cos(\theta) &  \sin(\theta)  \\ -\sin(\theta)  & \cos(\theta) \end{pmatrix}$\\

We also looked at 3D rotations, their group was ``SO(3)'', what do you think this stands for ???\\
We saw at the end of the lecture that  $SO(3) \simeq SU(2)$   
SU(2) = (Special / Unitary / 2x2 matrices.

\textit{(You will show in for homework that this is equivalent to another group U(1).)}


\noindent\rule{\textwidth}{1pt}

\textbf{All the machinery in place to look at Lorentz Transformations...}

Well almost all the machinery...  Remember invariant need to preserve is $t^2 - |\vec{x}|^2$

Lets deal with this minus sign...

4-vectors:  $x^\mu = (t,\vec{x})$  

Encode the minus sign in matrix:

\be
\eta_{\mu\nu} = \left\{ \begin{array}{rl} 1 & \mbox{$\mu = \nu=0$}  \\ -1 & \mbox{$\mu=\nu=1,2,3\ (i)$}  \\ 0 & \mbox{otherwise} \end{array} \right.
\ee

Can now write: $x_\mu = \etamn x^\nu = (t,-\vec{x})$  

And finally: $x_\mu x^\mu = t^2 - |\vec{x}|^2$

(Comment on $\eta^{\mu\nu} = \eta_{\mu\nu}$ and $\eta^\mu_\nu = \dmn$)

Now we after all the transformations that leave  $x_\mu x^\mu$ (Lorentz transformations)

\be
x'^\mu = \underbrace{\Lambda^\mu_\nu}_{\mbox{4x4 matrix}} x^\nu 
\ee

\underline{Same thing as before} Start with the infinitesimal transformations.

\be
\Lmn  = \dmn + \epsilon \wmn
\ee


\be
x'^\mu  = x^\mu + \epsilon \wmn x^\nu
\ee

Require $x'^\mu x'_\mu = x^\mu x_\mu$ or 

\begin{align*}
\etamn x'^\mu x'^\nu = \etamn x^\mu x^\nu \\
= \etamn x^\mu x^\nu + 2 \epsilon \etamn x^\nu \omega^\mu_\rho x^\rho
\end{align*}


So, $\underbrace{\etamn \omega^\mu_\rho}_{\equiv \omega_{\mu\rho} } x^\nu  x^\rho = 0$ 

After all 4x4 anti-symmetric matrices  $\omega_{\mu\rho} = -\omega_{\rho\mu}$

Lets just be damn explicit

\be
\omega_{\mu\nu} = \begin{bmatrix} 0 & a & b & c \\ -a & 0 & A & B \\ -b & -A & 0 & C \\ -c & -B & -C & 0 \end{bmatrix}
\ee

However, the think that enters the transformation is $\omega^\mu_\nu$.\\

Now raising a 0-component is free,  $\omega_{0i} = \omega^0_i$\\
raising a i-component costs a -1,  $\omega_{i0} = -\omega^i_0$\\
But also know $\omega_{0i} = -\omega_{i0}$
$\Rightarrow {\omega^0}_i = + {\omega^i}_0$

Same argument shows: ${\omega^i}_j = - {\omega^j}_i$

\be
\omega^\mu_\nu = \begin{bmatrix} 0 & a & b & c \\ a & 0 & A & B \\ b & -A & 0 & C \\ c & -B & -C & 0 \end{bmatrix}
\ee

Generators of the Lorentz Group. We have 6 ``generators'' abcABC.

3-rotations (guess which)  / 3-boosts

Crucial sign difference between boosts/rotations


${\omega^1}_2 = \begin{pmatrix} 0 & 0 & 0 & 0 \\ 0 & 0 & 1 & 0 \\ 0 & -1 & 0 & 0 \\ 0 & 0 & 0 & 0 \end{pmatrix}  \rightarrow e^{\theta \IMatrix} = \cos\theta + \IMatrix \sin\theta$\\

${\omega^0}_1 = \begin{pmatrix} 0 & 1 & 0 & 0 \\ 1 & 0 & 0 & 0 \\ 0 & 0 & 0 & 0 \\ 0 & 0 & 0 & 0 \end{pmatrix}  \rightarrow e^{\eta \IBoostMatrix} = \cosh\eta + \IBoostMatrix \sinh\eta$

$\IBoostMatrix^2 = 1$ whereas $\IMatrix^2 = -1$

\be
\begin{pmatrix} x'^0 \\ x'^1\end{pmatrix} = \begin{pmatrix} \cosh\eta & \sinh\eta \\ \sinh\eta & \cosh\eta  \end{pmatrix} = \begin{pmatrix} x^0 \\ x^1\end{pmatrix}
\ee

Can (but I wont) work out the Lie Algrabra....Homework.

Group is called SO$^+$(1,3)

\noindent\rule{\textwidth}{1pt}

Lets look at general $2\times2$ hermitian matrices, (not restricted to traceless as before).
Any $2\times2$, hermitian matrix can be written as: 
\begin{equation*}
M = \begin{pmatrix} x^0 + x^3 &  x^1+ix^2  \\ x^1-ix^2  & x^0-x^3 \end{pmatrix} \equiv \sigma_\mu x^\mu
\end{equation*}
where $\sigma_\mu = (1,\sigma_i)$ and $\sigma_i$ are the Pauli matrices.

Note that $\textrm{det}(M) = x^\mu x_\mu$

Consider the action,
\be
M' = L^\dagger M L
\ee
of L -any 2x2 complex matrix w/det(L) = 1.
This set of matrices is called SL(2,c) = Special / Linear/ 2x2 / Complex.

M' is still Hermitian $\Rightarrow M' = \sigma_\mu x_{(L)}^\mu$

And crucially, det(M) = det(M') $\Rightarrow {x_{(L)}}_\mu x_{(L)}^\mu = x_\mu x^\mu$

So the L's give (``furnish'') a representation of the Lorentz group.

We see here
SO$^+$(1,3) $\simeq$ SL(2,c)

}
\end{document}


