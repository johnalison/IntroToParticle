\documentclass[paper=letter,11pt]{scrartcl}

\KOMAoptions{headinclude=true, footinclude=false}
\KOMAoptions{DIV=14, BCOR=5mm}
\KOMAoptions{numbers=noendperiod}
\KOMAoptions{parskip=half}
\addtokomafont{disposition}{\rmfamily}
\addtokomafont{part}{\LARGE}
\addtokomafont{descriptionlabel}{\rmfamily}
%\setkomafont{pageheadfoot}{\normalsize\sffamily}
\setkomafont{pagehead}{\normalsize\rmfamily}
%\setkomafont{publishers}{\normalsize\rmfamily}
\setkomafont{caption}{\normalfont\small}
\setcapindent{0pt}
\deffootnote[1em]{1em}{1em}{\textsuperscript{\thefootnotemark}\ }


\usepackage{amsmath}
\usepackage[varg]{txfonts}
\usepackage[T1]{fontenc}
\usepackage{graphicx}
\usepackage{xcolor}
\usepackage[american]{babel}
% hyperref is needed in many places, so include it here
\usepackage{hyperref}

\usepackage{xspace}
\usepackage{multirow}
\usepackage{float}


\usepackage{braket}
\usepackage{bbm}
\usepackage{relsize}
\usepackage{tcolorbox}

\def\ketY{\ensuremath{\ket {\Psi}}}
\def\iGeV{\ensuremath{\textrm{GeV}^{-1}}}
%\def\mp{\ensuremath{m_{\textrm{proton}}}}
\def\rp{\ensuremath{r_{\textrm{proton}}}}
\def\me{\ensuremath{m_{\textrm{electron}}}}
\def\aG{\ensuremath{\alpha_G}}
\def\rAtom{\ensuremath{r_{\textrm{atom}}}}
\def\rNucl{\ensuremath{r_{\textrm{nucleus}}}}
\def\GN{\ensuremath{\textrm{G}_\textrm{N}}}
\def\ketX{\ensuremath{\ket{\vec{x}}}}
\def\ve{\ensuremath{\vec{\epsilon}}}


\def\ABCDMatrix{\ensuremath{\begin{pmatrix} A &  B  \\ C  & D \end{pmatrix}}}
\def\xyprime{\ensuremath{\begin{pmatrix} x' \\ y' \end{pmatrix}}}
\def\xyprimeT{\ensuremath{\begin{pmatrix} x' &  y' \end{pmatrix}}}
\def\xy{\ensuremath{\begin{pmatrix} x \\ y \end{pmatrix}}}
\def\xyT{\ensuremath{\begin{pmatrix} x & y \end{pmatrix}}}

\def\IMatrix{\ensuremath{\begin{pmatrix} 0 &  1  \\ -1  & 0 \end{pmatrix}}}
\def\IBoostMatrix{\ensuremath{\begin{pmatrix} 0 &  1  \\ 1  & 0 \end{pmatrix}}}
\def\JThree{\ensuremath{\begin{pmatrix}    0 & -i & 0  \\ i & 0  & 0 \\ 0 & 0 & 0 \end{pmatrix}}} 
\def\JTwo{\ensuremath{\begin{bmatrix}    0 & 0 & -i  \\ 0 & 0  & 0 \\ i & 0 & 0 \end{bmatrix}}}
\def\JOne{\ensuremath{\begin{bmatrix}    0 & 0 & 0  \\ 0 & 0  & -i \\ 0 & i & 0 \end{bmatrix}}}
\def\etamn{\ensuremath{\eta_{\mu\nu}}}
\def\Lmn{\ensuremath{\Lambda^\mu_\nu}}
\def\dmn{\ensuremath{\delta^\mu_\nu}}
\def\wmn{\ensuremath{\omega^\mu_\nu}}
\def\be{\begin{equation*}}
\def\ee{\end{equation*}}
\def\bea{\begin{eqnarray*}}
\def\eea{\end{eqnarray*}}
\def\bi{\begin{itemize}}
\def\ei{\end{itemize}}
\def\fmn{\ensuremath{F_{\mu\nu}}}
\def\fMN{\ensuremath{F^{\mu\nu}}}
\def\bc{\begin{center}}
\def\ec{\end{center}}
\def\nus{$\nu$s}

\def\adagger{\ensuremath{a_{p\sigma}^\dagger}}
\def\lineacross{\noindent\rule{\textwidth}{1pt}}

\newcommand{\multiline}[1] {
\begin{tabular} {|l}
#1
\end{tabular}
}

\newcommand{\multilineNoLine}[1] {
\begin{tabular} {l}
#1
\end{tabular}
}



\newcommand{\lineTwo}[2] {
\begin{tabular} {|l}
#1 \\
#2
\end{tabular}
}

\newcommand{\rmt}[1] {
\textrm{#1}
}


%
% Units
%
\def\m{\ensuremath{\rmt{m}}}
\def\GeV{\ensuremath{\rmt{GeV}}}
\def\pt{\ensuremath{p_\rmt{T}}}


\def\parity{\ensuremath{\mathcal{P}}}

\usepackage{cancel}
\usepackage{ mathrsfs }
\def\bigL{\ensuremath{\mathscr{L}}}

\usepackage{ dsfont }



\usepackage{fancyhdr}
\fancyhf{}

\usepackage{braket}

\def\ketY{\ensuremath{\ket {\Psi}}}
\def\iGeV{\ensuremath{\textrm{GeV}^{-1}}}
\def\mp{\ensuremath{m_{\textrm{proton}}}}
\def\rp{\ensuremath{r_{\textrm{proton}}}}
\def\me{\ensuremath{m_{\textrm{electron}}}}
\def\aG{\ensuremath{\alpha_G}}
\def\rAtom{\ensuremath{r_{\textrm{atom}}}}
\def\rNucl{\ensuremath{r_{\textrm{nucleus}}}}
\def\GN{\ensuremath{\textrm{G}_\textrm{N}}}


\usepackage{fancyhdr}

\fancyhf{}
\lhead{\Large 33-444} % \hfill Introduction to Particle Physics \hfill Spring 2019}
\chead{\Large Introduction to Particle Physics} % \hfill Spring 2019}
\rhead{\Large Spring 2019} % \hfill Introduction to Particle Physics \hfill Spring 2019}
\begin{document}
\thispagestyle{fancy}





%\begin{tabular}{c}
%{\large 33-444 \hfill Intro To Particle \hfill Spring 2019\\}
%\hline 
%\end{tabular}

\begin{center}
{\huge \textbf{Homework Set \#1}}
\large

{\textbf{ Due Date:} Before class Friday January 25th  }
\end{center}

{\large
\textbf{1) You} \hfill \textit{(2 points)}
\begin{itemize}
\item[(a)]What is your major/minor ? 
\item[(b)]When do you plan on graduating?
\item[(c)]What do you want to do after graduation ? (eg: grad school ? if so, what subject ? if not, what industry?)
\item[(d)]What do you most want to get out of this course ? 
\end{itemize}

\vspace*{0.25in}

\textbf{2) Solid State Physics} \hfill \textit{(5 points)}
\begin{itemize}
\item[(a)] Assume a solid is composed of closely packed atoms. What is the spacing between atoms?  Express your result in terms of  $\alpha$, \aG, \mp, and \me.
\item[(b)] If you wanted to study the crystal structure of a solid material with $Z\sim10$ using light, what wavelength of photons would you need ?
Express your result in terms of  $\alpha$, \aG, \mp, and \me.
\item[(c)] Where in the spectrum of EM radiation do these photons lie?
\end{itemize}

\vspace*{0.25in}

\textbf{3) Strength of Gravity on Earth} \hfill \textit{(5 points)}
\begin{itemize}
\item[(a)]Calculate the local strength of gravity $g_{\textrm{local}}$ in terms of $\alpha$, \aG, \mp, and \me.
\item[(b)]What is your estimated value in mks units ?
\item[(c)]How does this compare with the well-known value of 9.8 m/s$^2$ ?
\end{itemize}
%Estimate the size of life forms on earth in terms of $\alpha$, \aG, \mp, and \me.
%Assume life is a solid. 

\vspace*{0.25in}

\textbf{4) Neutron Stars } \hfill \textit{(5 points)}
\begin{itemize}
\item[(a)]Estimate the radius, mass, and speed of sound for neutron stars in terms of $\alpha$, \aG, \mp, and \me.
(Assume: A neutron star is a solid made of neutrons and $\mp \sim m_{\textrm{neutron}}$)
\item[(b)]What are you estimated values in mks units ?
\item[(c)]Compare your estimates to actual values for Neutron Stars quoted online.
\item[(d)]Look up $m_{\textrm{neutron}}$. How does this compare with the assumption of $\mp \sim m_{\textrm{neutron}}$?
\end{itemize}

\newpage

\textbf{5) 2D Rotations } \hfill \textit{(3 points)}
\begin{itemize}
\item[(a)]Show that $R(\Theta) = e^{I\Theta} = cos(\Theta)+ I sin(\Theta)$
where, $I =  \begin{bmatrix}
    0 & 1  \\
    -1 & 0
  \end{bmatrix} $
\end{itemize}
}

\vspace*{0.25in}

\textbf{6) 3D Rotations } \hfill \textit{(5 points)}
\begin{itemize}
\item[(a)]Work out the ``algebra'' of the generators of 3D rotations $J_i$. \\
Where $
J_3 =  \begin{bmatrix}    0 & -i & 0  \\ i & 0  & 0 \\ 0 & 0 & 0 \end{bmatrix}, 
\hfill
 J_2 =  \begin{bmatrix}    0 & 0 & -i  \\ 0 & 0  & 0 \\ i & 0 & 0 \end{bmatrix} 
\hfill
 J_1 =  \begin{bmatrix}    0 & 0 & 0  \\ 0 & 0  & -i \\ 0 & i & 0 \end{bmatrix} 
$\\
(These generators different from the T's derived in class by a factor of $i$)\\
Working out the algebra means calculating the commutation relations $[J_i,J_j]$.
\item[(b)]Let M be a traceless $2\times2$ hermitian matrix and U be a $2\times2$  hermitian matrix with determinant = 1.  
Show that $M' = U^{\dagger}MU$ is also traceless and hermitian and that is has the same determinant as M.
\end{itemize}



\end{document}
