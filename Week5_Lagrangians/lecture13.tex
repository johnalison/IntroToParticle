\documentclass[paper=letter,11pt]{scrartcl}

\KOMAoptions{headinclude=true, footinclude=false}
\KOMAoptions{DIV=14, BCOR=5mm}
\KOMAoptions{numbers=noendperiod}
\KOMAoptions{parskip=half}
\addtokomafont{disposition}{\rmfamily}
\addtokomafont{part}{\LARGE}
\addtokomafont{descriptionlabel}{\rmfamily}
%\setkomafont{pageheadfoot}{\normalsize\sffamily}
\setkomafont{pagehead}{\normalsize\rmfamily}
%\setkomafont{publishers}{\normalsize\rmfamily}
\setkomafont{caption}{\normalfont\small}
\setcapindent{0pt}
\deffootnote[1em]{1em}{1em}{\textsuperscript{\thefootnotemark}\ }


\usepackage{amsmath}
\usepackage[varg]{txfonts}
\usepackage[T1]{fontenc}
\usepackage{graphicx}
\usepackage{xcolor}
\usepackage[american]{babel}
% hyperref is needed in many places, so include it here
\usepackage{hyperref}

\usepackage{xspace}
\usepackage{multirow}
\usepackage{float}


\usepackage{braket}
\usepackage{bbm}
\usepackage{relsize}
\usepackage{tcolorbox}

\def\ketY{\ensuremath{\ket {\Psi}}}
\def\iGeV{\ensuremath{\textrm{GeV}^{-1}}}
%\def\mp{\ensuremath{m_{\textrm{proton}}}}
\def\rp{\ensuremath{r_{\textrm{proton}}}}
\def\me{\ensuremath{m_{\textrm{electron}}}}
\def\aG{\ensuremath{\alpha_G}}
\def\rAtom{\ensuremath{r_{\textrm{atom}}}}
\def\rNucl{\ensuremath{r_{\textrm{nucleus}}}}
\def\GN{\ensuremath{\textrm{G}_\textrm{N}}}
\def\ketX{\ensuremath{\ket{\vec{x}}}}
\def\ve{\ensuremath{\vec{\epsilon}}}


\def\ABCDMatrix{\ensuremath{\begin{pmatrix} A &  B  \\ C  & D \end{pmatrix}}}
\def\xyprime{\ensuremath{\begin{pmatrix} x' \\ y' \end{pmatrix}}}
\def\xyprimeT{\ensuremath{\begin{pmatrix} x' &  y' \end{pmatrix}}}
\def\xy{\ensuremath{\begin{pmatrix} x \\ y \end{pmatrix}}}
\def\xyT{\ensuremath{\begin{pmatrix} x & y \end{pmatrix}}}

\def\IMatrix{\ensuremath{\begin{pmatrix} 0 &  1  \\ -1  & 0 \end{pmatrix}}}
\def\IBoostMatrix{\ensuremath{\begin{pmatrix} 0 &  1  \\ 1  & 0 \end{pmatrix}}}
\def\JThree{\ensuremath{\begin{pmatrix}    0 & -i & 0  \\ i & 0  & 0 \\ 0 & 0 & 0 \end{pmatrix}}} 
\def\JTwo{\ensuremath{\begin{bmatrix}    0 & 0 & -i  \\ 0 & 0  & 0 \\ i & 0 & 0 \end{bmatrix}}}
\def\JOne{\ensuremath{\begin{bmatrix}    0 & 0 & 0  \\ 0 & 0  & -i \\ 0 & i & 0 \end{bmatrix}}}
\def\etamn{\ensuremath{\eta_{\mu\nu}}}
\def\Lmn{\ensuremath{\Lambda^\mu_\nu}}
\def\dmn{\ensuremath{\delta^\mu_\nu}}
\def\wmn{\ensuremath{\omega^\mu_\nu}}
\def\be{\begin{equation*}}
\def\ee{\end{equation*}}
\def\bea{\begin{eqnarray*}}
\def\eea{\end{eqnarray*}}
\def\bi{\begin{itemize}}
\def\ei{\end{itemize}}
\def\fmn{\ensuremath{F_{\mu\nu}}}
\def\fMN{\ensuremath{F^{\mu\nu}}}
\def\bc{\begin{center}}
\def\ec{\end{center}}
\def\nus{$\nu$s}

\def\adagger{\ensuremath{a_{p\sigma}^\dagger}}
\def\lineacross{\noindent\rule{\textwidth}{1pt}}

\newcommand{\multiline}[1] {
\begin{tabular} {|l}
#1
\end{tabular}
}

\newcommand{\multilineNoLine}[1] {
\begin{tabular} {l}
#1
\end{tabular}
}



\newcommand{\lineTwo}[2] {
\begin{tabular} {|l}
#1 \\
#2
\end{tabular}
}

\newcommand{\rmt}[1] {
\textrm{#1}
}


%
% Units
%
\def\m{\ensuremath{\rmt{m}}}
\def\GeV{\ensuremath{\rmt{GeV}}}
\def\pt{\ensuremath{p_\rmt{T}}}


\def\parity{\ensuremath{\mathcal{P}}}

\usepackage{cancel}
\usepackage{ mathrsfs }
\def\bigL{\ensuremath{\mathscr{L}}}

\usepackage{ dsfont }



\usepackage{fancyhdr}
\fancyhf{}

%\documentclass[margin,line]{res}
\usepackage{braket}
\usepackage{bbm}
\usepackage{relsize}
\usepackage{tcolorbox}

\def\ketY{\ensuremath{\ket {\Psi}}}
\def\iGeV{\ensuremath{\textrm{GeV}^{-1}}}
\usepackage{cancel}

\def\ABCDMatrix{\ensuremath{\begin{pmatrix} A &  B  \\ C  & D \end{pmatrix}}}
\def\xyprime{\ensuremath{\begin{pmatrix} x' \\ y' \end{pmatrix}}}
\def\xyprimeT{\ensuremath{\begin{pmatrix} x' &  y' \end{pmatrix}}}
\def\xy{\ensuremath{\begin{pmatrix} x \\ y \end{pmatrix}}}
\def\xyT{\ensuremath{\begin{pmatrix} x & y \end{pmatrix}}}

\def\IMatrix{\ensuremath{\begin{pmatrix} 0 &  1  \\ -1  & 0 \end{pmatrix}}}
\def\IBoostMatrix{\ensuremath{\begin{pmatrix} 0 &  1  \\ 1  & 0 \end{pmatrix}}}
\def\JThree{\ensuremath{\begin{pmatrix}    0 & -i & 0  \\ i & 0  & 0 \\ 0 & 0 & 0 \end{pmatrix}}} 
\def\JTwo{\ensuremath{\begin{bmatrix}    0 & 0 & -i  \\ 0 & 0  & 0 \\ i & 0 & 0 \end{bmatrix}}}
\def\JOne{\ensuremath{\begin{bmatrix}    0 & 0 & 0  \\ 0 & 0  & -i \\ 0 & i & 0 \end{bmatrix}}}
\def\etamn{\ensuremath{\eta_{\mu\nu}}}
\def\Lmn{\ensuremath{\Lambda^\mu_\nu}}
\def\dmn{\ensuremath{\delta^\mu_\nu}}
\def\wmn{\ensuremath{\omega^\mu_\nu}}
\def\be{\begin{equation*}}
\def\ee{\end{equation*}}
\def\bea{\begin{eqnarray*}}
\def\eea{\end{eqnarray*}}
\def\fmn{\ensuremath{F_{\mu\nu}}}
\def\fMN{\ensuremath{F^{\mu\nu}}}

\def\adagger{\ensuremath{a_{p\sigma}^\dagger}}

%\def\xMu{\ensuremath{x^\mu}

\usepackage{fancyhdr}

\fancyhf{}
\lhead{\Large 33-444} % \hfill Introduction to Particle Physics \hfill Spring 2020}
\chead{\Large Introduction to Particle Physics} % \hfill Spring 2020}
\rhead{\Large Spring 2020} % \hfill Introduction to Particle Physics \hfill Spring 2020}

\begin{document}
\thispagestyle{fancy}

\begin{center}
{\huge \textbf{Lecture 13}}
\end{center}

{\fontsize{14}{16}\selectfont

\textbf{\underline{Noether's Theorm}} 

Lagrangian may be invariant under some type of transformation (variation) 

eg: $\phi \rightarrow \phi + \delta$

This transformation is a symmerty of the Lagrangian

Say $\phi$ is complex:  2 DoF  $\phi$ and $\phi^*$.

And you have a Lagrangian given by


\be
\mathcal{L} = (\partial_\mu \phi)(\partial^\mu \phi^*) - m^2 \phi \phi^*
\ee

\underline{symmetry}
\be
\phi \rightarrow e^{-i\alpha} \phi \hspace{1in} \phi^* \rightarrow e^{i\alpha} \phi^*
\ee


Whenever we have a continous symmetry (meaning there is a continious limit)

\bea
\frac{\delta \mathcal{L}}{\delta \alpha} = 0 &=& \sum_n \left[ \frac{\partial \mathcal{L}}{\partial \phi_n} \frac{\delta \phi_n}{\delta \alpha}  +  \frac{\partial \mathcal{L}}{\partial (\partial_\mu \phi_n)} \frac{\delta (\partial_\mu \phi_n)}{\delta \alpha} \right]  \\ 
&=& \sum_n \left[ \underbrace{\left[ \frac{\partial \mathcal{L}}{\partial \phi_n} - \partial_\mu \frac{\partial \mathcal{L}}{\partial (\partial_\mu \phi_n)} \right]}_{=0\ \textrm{Euler Lagrange}} \frac{\delta \phi_n}{\delta \alpha}  +  \partial_\mu \left[ \frac{\partial \mathcal{L}}{\partial (\partial_\mu \phi_n)} \frac{\delta \phi_n}{\delta \alpha} \right] \right]
\eea

\be
\phi_n = \{\phi, \phi^*\}
\ee

$\Rightarrow$

\be
\partial_\mu J^\mu = 0
\ee

with 

\be
J^\mu = \sum_n \left[ \frac{\partial \mathcal{L}}{\partial (\partial_\mu \phi_n)} \frac{\delta \phi_n}{\delta \alpha} \right]
\ee

$J^\mu$ is a conserved current.  ``Noethers Current''


\underline{total ``charge''}
\be
Q \equiv \int d^3x\ J_0 
\ee


\be
\partial_t Q = \int d^3x\ \partial_t J_0 = \underbrace{\int d^3x\ \vec{\nabla} \cdot \vec{J}}_{\textrm{Vanishes on the Boundary}} = 0
\ee
 
Q does not change with time! 

Very gerneral and important theorm \underline{``Noethers Theorem''}

If $\mathcal{L}$ has (``enjoys'') a continous symmetry, there exists an associated current that is conserved. 

\be
\phi(x) \rightarrow \phi(x + \epsilon) = \phi(x) + \epsilon^\mu \partial_\mu \phi(x)
\ee

This leaves $\mathcal{L}$ and $\mathcal{S}$ invariant and gives a four vector of noether currents. 

$\Rightarrow$ Noether's theorm tells us why energy and momentum are conserved. 






}
\end{document}



% LocalWords:  partilces
