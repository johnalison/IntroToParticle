\input{../latexSetup}

\lhead{\Large 33-444} % \hfill Introduction to Particle Physics \hfill Spring 2022}
\chead{\Large Introduction to Particle Physics} % \hfill Spring 2022}
\rhead{\Large Spring 2022} % \hfill Introduction to Particle Physics \hfill Spring 2022}

\begin{document}
\thispagestyle{fancy}

\begin{center}
{\huge \textbf{Lecture 13}}
\end{center}

{\fontsize{14}{16}\selectfont

\textbf{\underline{Noether's Theorem}} 

Lagrangian may be invariant under some type of transformation (variation) 

eg: $\phi \rightarrow \phi + \delta$

This transformation is a symmetry of the Lagrangian

Say $\phi$ is complex:  2 DoF  $\phi$ and $\phi^*$.

And you have a Lagrangian given by


\be
\mathcal{L} = (\partial_\mu \phi)(\partial^\mu \phi^*) - m^2 \phi \phi^*
\ee

\underline{symmetry}
\be
\phi \rightarrow e^{-i\alpha} \phi \hspace{1in} \phi^* \rightarrow e^{i\alpha} \phi^*
\ee


Whenever we have a continuous symmetry (meaning there is a continuous limit)

\bea
\frac{\delta \mathcal{L}}{\delta \alpha} = 0 &=& \sum_n \left[ \frac{\partial \mathcal{L}}{\partial \phi_n} \frac{\delta \phi_n}{\delta \alpha}  +  \frac{\partial \mathcal{L}}{\partial (\partial_\mu \phi_n)} \frac{\delta (\partial_\mu \phi_n)}{\delta \alpha} \right]  \\ 
&=& \sum_n \left[ \underbrace{\left[ \frac{\partial \mathcal{L}}{\partial \phi_n} - \partial_\mu \frac{\partial \mathcal{L}}{\partial (\partial_\mu \phi_n)} \right]}_{=0\ \textrm{Euler Lagrange}} \frac{\delta \phi_n}{\delta \alpha}  +  \partial_\mu \left[ \frac{\partial \mathcal{L}}{\partial (\partial_\mu \phi_n)} \frac{\delta \phi_n}{\delta \alpha} \right] \right]
\eea

\be
\phi_n = \{\phi, \phi^*\}
\ee

$\Rightarrow$

\be
\partial_\mu J^\mu = 0
\ee

with 

\be
J^\mu = \sum_n \left[ \frac{\partial \mathcal{L}}{\partial (\partial_\mu \phi_n)} \frac{\delta \phi_n}{\delta \alpha} \right]
\ee

$J^\mu$ is a conserved current.  ``Noether's Current''


\underline{total ``charge''}
\be
Q \equiv \int d^3x\ J_0 
\ee


\be
\partial_t Q = \int d^3x\ \partial_t J_0 = \underbrace{\int d^3x\ \vec{\nabla} \cdot \vec{J}}_{\textrm{Vanishes on the Boundary}} = 0
\ee
 
Q does not change with time! 

Very general and important theorem \underline{``Noether's Theorem''}

If $\mathcal{L}$ has (``enjoys'') a continuous symmetry, there exists an associated current that is conserved. 

\be
\phi(x) \rightarrow \phi(x + \epsilon) = \phi(x) + \epsilon^\mu \partial_\mu \phi(x)
\ee

This leaves $\mathcal{L}$ and $\mathcal{S}$ invariant and gives a four vector of noether currents. 

$\Rightarrow$ Noether's theorem tells us why energy and momentum are conserved. 


\noindent\rule{\textwidth}{1pt}

\clearpage

\underline{\Large Cross Sections And Decay Rates}

20th century witnessed the development of collider physics. 
Effective means to determine which particles exist and their properties and interactions.

\begin{itemize}
\item[-] Rutherford discovery of the nucleus using $\alpha$ 1911 
\item[-] Andersen's discovery of anti-electrons 1932
\end{itemize}

These were made with ``Natural accelerators'' $\alpha$'s or cosmic rays

Around 1930 man made collisions started winning. 

eg: 1 MeV 

Now 13 TeV at the LHC.

Collisions map free fixed momenta initial states $\rightarrow$ final fixed momentum states. \\

QFT predicts \underline{probability} for projections to occur. 

Probabilities typically dependant on parameters (angles, momenta, etc) 

\begin{center}
$P(v_1, ... v_n)$ - differential probabilities
\end{center}

Given by 
\be
|\braket{\psi_{final}, +\infty|\psi_{initial}, -\infty}|^2
\ee


\be
\braket{f|S|i} \hspace{1in} \textrm{S-matrix}
\ee
QFT tell us how to calculate S given some Lagrangian (next week)

S-matrix elements are the primary object of interest for particle physics. 

In this lecture we will relate S-matrix elements to scattering 
\lineTwo{cross sections}{decay rates}
which we can measure experimentally

\clearpage

\underline{\underline{Cross Sections}}

\fbox{\begin{minipage}{\textwidth}
\underline{Aside:}

Probabilities are dimensionless and [0-1] absolute $\Rightarrow$ extremely subtle

to calculate P need to know all possible outcomes a priori!

QM$\Rightarrow$ need complete basis.

Usually impossible in QCD 
\lineTwo{\# final states $\infty$}{final states not fully known }

(Particle decays is an exception)

\end{minipage}}

Natural quantity to measure. 

eg: Rutherford was interested in the size of the nucleus. 

By colliding $\alpha$-particles w/gold foil and measuring how many partilces are scattered, can determine $\sigma = \pi r^2$

\underline{Single nucleus}
\begin{figure}[h]
\centering
\includegraphics[width=0.4\textwidth]{./Scattering}
\end{figure}

\be
\sigma = \frac{\# \textrm{- scattered}}{\textrm{time} \times \textrm{(Number density in beam)} \times \textrm{velocity of beam}} = \frac{1}{T}\frac{1}{\Phi} N
\ee

\underline{Real Experiment}
other factors: \lineTwo{number density of nuclei in foil}{cross sectional area of the beam (if smaller than foil)}

This stuff and T and $\Phi$ depend on details of the experiment. 

In contrast, $\sigma$ is a property of particles being scattered. 

\underline{In QM} generalize notion of cross sectional area to ``cross section'' \multiline{-units of area \\ -abstract measure of\\ interaction strength}

eg: Classically the $\alpha$ will either scatter or not. 

Quantum Mechanically,  there is some probability for scattering.

\be
d\sigma = \frac{1}{T}\underbrace{\frac{1}{\Phi}}_{\substack{\textrm{Normalized}\\ \textrm{to one particle}}} \underbrace{dP}_{\textrm{QM probability of scattering}}
\ee

\multiline{d$\sigma$ \\ dP } - differential in kinematic variables $\theta$'s and Ps


\be
\underbrace{dN}_{\# \textrm{of scatters}} = \underbrace{L}_{\textrm{``integrated luminosity}} \times d\sigma 
\ee
(take eq as definition of L)

So number of observed events is direct measurement of cross section (See in presentation and papers)


\underline{Relate to S-matrix}

practically impossible to collide more than two particles at a time.

$\ket{i}$ will always be a 2 particle state

\be
P_1 + P_2 \rightarrow \{P_j\}
\ee

\underline{Rest frame} of one partilces $\Phi = \frac{|\vec{v}|}{V}$

\underline{Center of mass frame}  $\Phi = \frac{|\vec{v_1} - \vec{v_2}|}{V}$


So, 
\be
d\sigma = \frac{V}{T}\frac{1}{|\vec{v_1} - \vec{v_2}|} dP
\ee

\be
dP = \frac{|\braket{f|S|i}|^2}{\braket{f|f}\braket{i|i}} \underbrace{d\Pi}_{\substack{\textrm{Region of final state momenta} \\ \textrm{ we are considering}}}
\ee

On interval of size L, the momenta of available states are $P_n = \frac{2\pi n}{L}$ (from particle in a box).

$\Rightarrow$ throughout a volume V
\be
N = \int \frac{V}{(2\pi)^3} d^3p
\ee


\be
d\Pi = \prod_j \frac{V}{(2\pi)^3} d^3p_j
\ee
where j runs over final state partilces.

OK, lets deal with the normalization factors. 

Note, $\braket{f|f}$ and $\braket{i|i} \ne 1$ (The inner products are not Lorentz invariant...) 


\bea
\braket{p'|p} &=& (2\pi)^3\ 2E\ \delta^3(p'-p) \\
\braket{p|p}  &=& (2\pi)^3\ 2E_p\ \delta^3(0) \\
  &=& 2 E_p V
\eea

\bc
\fbox{\begin{minipage}{0.5\textwidth}
we say the $\delta^3(0)$ is ``regulated by V''.

\be
\delta^3(p) = \frac{1}{(2\pi)^3}\int d^3x\ e^{ipx}
\ee

\be
\delta^3(0) = \frac{1}{(2\pi)^3}\int d^3x = \frac{V}{(2\pi)^3}
\ee
\end{minipage}}
\ec

$\Rightarrow$
\be
\braket{i|i} = \braket{p_1 p_2 |p_1 p_2} = 2 E_1 V\ 2 E_2 V
\ee


\be
\braket{f|f} = \prod_j (2 E_j V)
\ee

Now have to deal with $\braket{f|S|i}$


S elements always calculated pertubatively

\be
S = \underbrace{1}_{\textrm{free theory}} +  \underbrace{iT}_{\textrm{perturbatively small}}
\ee


}
\end{document}

