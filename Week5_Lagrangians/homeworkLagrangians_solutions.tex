\documentclass[paper=letter,11pt]{scrartcl}

\KOMAoptions{headinclude=true, footinclude=false}
\KOMAoptions{DIV=14, BCOR=5mm}
\KOMAoptions{numbers=noendperiod}
\KOMAoptions{parskip=half}
\addtokomafont{disposition}{\rmfamily}
\addtokomafont{part}{\LARGE}
\addtokomafont{descriptionlabel}{\rmfamily}
%\setkomafont{pageheadfoot}{\normalsize\sffamily}
\setkomafont{pagehead}{\normalsize\rmfamily}
%\setkomafont{publishers}{\normalsize\rmfamily}
\setkomafont{caption}{\normalfont\small}
\setcapindent{0pt}
\deffootnote[1em]{1em}{1em}{\textsuperscript{\thefootnotemark}\ }


\usepackage{amsmath}
\usepackage[varg]{txfonts}
\usepackage[T1]{fontenc}
\usepackage{graphicx}
\usepackage{xcolor}
\usepackage[american]{babel}
% hyperref is needed in many places, so include it here
\usepackage{hyperref}

\usepackage{xspace}
\usepackage{multirow}
\usepackage{float}


\usepackage{braket}
\usepackage{bbm}
\usepackage{relsize}
\usepackage{tcolorbox}

\def\ketY{\ensuremath{\ket {\Psi}}}
\def\iGeV{\ensuremath{\textrm{GeV}^{-1}}}
%\def\mp{\ensuremath{m_{\textrm{proton}}}}
\def\rp{\ensuremath{r_{\textrm{proton}}}}
\def\me{\ensuremath{m_{\textrm{electron}}}}
\def\aG{\ensuremath{\alpha_G}}
\def\rAtom{\ensuremath{r_{\textrm{atom}}}}
\def\rNucl{\ensuremath{r_{\textrm{nucleus}}}}
\def\GN{\ensuremath{\textrm{G}_\textrm{N}}}
\def\ketX{\ensuremath{\ket{\vec{x}}}}
\def\ve{\ensuremath{\vec{\epsilon}}}


\def\ABCDMatrix{\ensuremath{\begin{pmatrix} A &  B  \\ C  & D \end{pmatrix}}}
\def\xyprime{\ensuremath{\begin{pmatrix} x' \\ y' \end{pmatrix}}}
\def\xyprimeT{\ensuremath{\begin{pmatrix} x' &  y' \end{pmatrix}}}
\def\xy{\ensuremath{\begin{pmatrix} x \\ y \end{pmatrix}}}
\def\xyT{\ensuremath{\begin{pmatrix} x & y \end{pmatrix}}}

\def\IMatrix{\ensuremath{\begin{pmatrix} 0 &  1  \\ -1  & 0 \end{pmatrix}}}
\def\IBoostMatrix{\ensuremath{\begin{pmatrix} 0 &  1  \\ 1  & 0 \end{pmatrix}}}
\def\JThree{\ensuremath{\begin{pmatrix}    0 & -i & 0  \\ i & 0  & 0 \\ 0 & 0 & 0 \end{pmatrix}}} 
\def\JTwo{\ensuremath{\begin{bmatrix}    0 & 0 & -i  \\ 0 & 0  & 0 \\ i & 0 & 0 \end{bmatrix}}}
\def\JOne{\ensuremath{\begin{bmatrix}    0 & 0 & 0  \\ 0 & 0  & -i \\ 0 & i & 0 \end{bmatrix}}}
\def\etamn{\ensuremath{\eta_{\mu\nu}}}
\def\Lmn{\ensuremath{\Lambda^\mu_\nu}}
\def\dmn{\ensuremath{\delta^\mu_\nu}}
\def\wmn{\ensuremath{\omega^\mu_\nu}}
\def\be{\begin{equation*}}
\def\ee{\end{equation*}}
\def\bea{\begin{eqnarray*}}
\def\eea{\end{eqnarray*}}
\def\bi{\begin{itemize}}
\def\ei{\end{itemize}}
\def\fmn{\ensuremath{F_{\mu\nu}}}
\def\fMN{\ensuremath{F^{\mu\nu}}}
\def\bc{\begin{center}}
\def\ec{\end{center}}
\def\nus{$\nu$s}

\def\adagger{\ensuremath{a_{p\sigma}^\dagger}}
\def\lineacross{\noindent\rule{\textwidth}{1pt}}

\newcommand{\multiline}[1] {
\begin{tabular} {|l}
#1
\end{tabular}
}

\newcommand{\multilineNoLine}[1] {
\begin{tabular} {l}
#1
\end{tabular}
}



\newcommand{\lineTwo}[2] {
\begin{tabular} {|l}
#1 \\
#2
\end{tabular}
}

\newcommand{\rmt}[1] {
\textrm{#1}
}


%
% Units
%
\def\m{\ensuremath{\rmt{m}}}
\def\GeV{\ensuremath{\rmt{GeV}}}
\def\pt{\ensuremath{p_\rmt{T}}}


\def\parity{\ensuremath{\mathcal{P}}}

\usepackage{cancel}
\usepackage{ mathrsfs }
\def\bigL{\ensuremath{\mathscr{L}}}

\usepackage{ dsfont }



\usepackage{fancyhdr}
\fancyhf{}


\lhead{\Large 33-444} % \hfill Introduction to Particle Physics \hfill Spring 2022}
\chead{\Large Introduction to Particle Physics} % \hfill Spring 2022}
\rhead{\Large Spring 2022} % \hfill Introduction to Particle Physics \hfill Spring 2022}
\begin{document}
\thispagestyle{fancy}





%\begin{tabular}{c}
%{\large 33-444 \hfill Intro To Particle \hfill Spring 2022\\}
%\hline 
%\end{tabular}

\begin{center}
{\huge \textbf{Homework Set Lagrangians}}
\large

{\textbf{ Due Date:} Before class Friday February 22nd  } 
\end{center}


{\large
\textbf{2) Use the Weyl basis of the $\gamma$ matricies to solve the Dirac equation for a spinor $\psi$ \hfill \textit{(5 points)} }

See Text Section 2.2



{\large
\textbf{3) Clifford Algebra \hfill \textit{(5 points)} }

In writing the Dirac equation, we chose a particular representation of the $\gamma$ matrices that satisfied $\{\gamma_\mu,\gamma_\nu\} = 2\eta_{\mu\nu}$, which is called the Clifford algebra. 
The choice we used in class is called the Weyl basis. In this problem, we will study the Clifford algebra and the Weyl basis.
\begin{itemize}
\item[a)]
Will calculate:

\be
\gamma_\mu \gamma_\nu + \gamma_\nu \gamma_\mu 
\ee
Three cases to consider:

Case 1) $\mu = \nu = 0$
\be
\gamma_0 \gamma_0   = \begin{pmatrix} 0 & 1 \\ 1 & 0 \end{pmatrix}  \begin{pmatrix} 0 & 1 \\ 1 & 0 \end{pmatrix}  = \begin{pmatrix} 1 & 0 \\ 0 & 1 \end{pmatrix} 
\ee
So, 
\be
\gamma_0 \gamma_0  + \gamma_0 \gamma_0 = \begin{pmatrix} 2 & 0 \\ 0 & 2 \end{pmatrix} 
\ee

Case 2) $\mu = 0, \nu = i$  (and reversed)

\be
\gamma_0 \gamma_i   = \begin{pmatrix} 0 & 1 \\ 1 & 0 \end{pmatrix}  \begin{pmatrix} 0 & \sigma_i \\ -\sigma_i & 0 \end{pmatrix}  = \begin{pmatrix} -\sigma_i & 0 \\ 0 & \sigma_i \end{pmatrix} 
\ee
\be
\gamma_i \gamma_0   = \begin{pmatrix} 0 & \sigma_i \\ -\sigma_i & 0 \end{pmatrix} \begin{pmatrix} 0 & 1 \\ 1 & 0 \end{pmatrix}   = \begin{pmatrix} \sigma_i & 0 \\ 0 & -\sigma_i \end{pmatrix} 
\ee

So,
\be
\gamma_0 \gamma_i  + \gamma_i \gamma_0 = \gamma_i \gamma_0  + \gamma_0 \gamma_i = 0
\ee

Case 3) $\mu = i, \nu = j$ 

\be
\gamma_i \gamma_j   = \begin{pmatrix} 0 & \sigma_i \\ -\sigma_i & 0 \end{pmatrix} \begin{pmatrix} 0 & \sigma_j \\ -\sigma_j & 0 \end{pmatrix}  = \begin{pmatrix} -\sigma_i \sigma_j & 0 \\ 0 & -\sigma_i \sigma_j \end{pmatrix} =  \begin{pmatrix} -\delta_{ij} & 0 \\ 0 & -\delta_{ij} \end{pmatrix}
\ee
So,
\be
\gamma_i \gamma_j  + \gamma_j \gamma_i = -2\delta_{ij}
\ee



\item[b)]

\begin{equation*}
\gamma_\mu = S\gamma_\mu^{\mathrm{Weyl}}S^\dagger, 
\end{equation*}
Multiply both sides on left by S.

\be
\gamma_\mu S = S \gamma_\mu^W
\ee

Let $S = \begin{pmatrix} a & b \\ c & d \end{pmatrix}$, where a, b, c and d are 2x2 matricies.

Consider $\mu =0$

\be
\begin{pmatrix} 1 & 0 \\ 0 & -1 \end{pmatrix} \begin{pmatrix} a & b \\ c & d \end{pmatrix} = \begin{pmatrix} a & b \\ c & d \end{pmatrix} \begin{pmatrix} 0 & 1 \\ 1 & 0 \end{pmatrix}
\ee
or
\be
\begin{pmatrix} a & b \\ -c & -d \end{pmatrix}  = \begin{pmatrix} b & a \\ d & c \end{pmatrix} 
\ee
so, $a=b, c=-d$


Now require $SS^\dagger = 1$

\be
\begin{pmatrix} a & a \\ c & -c \end{pmatrix} \begin{pmatrix} a^\dagger & c^\dagger \\ a^\dagger & -c^\dagger \end{pmatrix} = \begin{pmatrix} 2 a a^\dagger & ac^\dagger - ac^\dagger \\ ca^\dagger - ca^\dagger & 2cc^\dagger \end{pmatrix} = \begin{pmatrix} 1 & 0 \\ 0 & 1 \end{pmatrix} 
\ee

So a and c are hermitian. The only 2x2 matricies that are hermitian are $I$ or $\sigma_k$, so $\sqrt{2}a = \pm I$  or $\pm \sigma_a$ and same for c.


Now Consider $\mu = i$
\be
\begin{pmatrix} 0 & \sigma_i \\ -\sigma_i & 0 \end{pmatrix} \begin{pmatrix} a & a \\ c & -c \end{pmatrix} = \begin{pmatrix} a & a \\ c & -c \end{pmatrix} \begin{pmatrix} 0 & \sigma_i \\ -\sigma_i & 0 \end{pmatrix}
\ee

\be
\begin{pmatrix} \sigma_i c & - \sigma_i c \\ -\sigma_i a & -\sigma_i a \end{pmatrix} = \begin{pmatrix} -a \sigma_i & a \sigma_i \\ c \sigma_i & c \sigma_i \end{pmatrix}
\ee

Assume both a and c are pauli matricies,
then, $\sigma_i \sigma_c + \sigma_a \sigma_i = \delta_{ic} + \delta_{ai}  = 0$, which cannot hold when $i = a$ or $c$. 
So a and c cannot both be pauli matrices. 

Assume one (a) is a pauli matrices and c is $I$
then,  $\sigma_i  + \sigma_a \sigma_i = \sigma_i+ \delta_{ai}  = 0$, which also cannot is impossible. 

So both A and C are $\pm I$ and above implies that $a = -c$

So $S = \frac{1}{\sqrt{2}}\begin{pmatrix} I & I \\ -I & I \end{pmatrix}$
and $S^\dagger = \frac{1}{\sqrt{2}}\begin{pmatrix} I & -I \\ I & I \end{pmatrix}$


\item[c)]
 Applying the similarity transformation to the Dirac equation, gives 
\begin{equation*}
(i S \gamma_\mu S^\dagger \partial^{\mu} - m)\psi = S(i\gamma \cdot \partial - m)S^\dagger\psi = 0.
\end{equation*}

\begin{equation*}
S(i\gamma \cdot \partial - m)S^\dagger\psi = 0 \Rightarrow (i\gamma \cdot \partial - m)S^\dagger\psi = 0
\end{equation*}

Where here $\psi$ is the solution in the new basis, and $S^\dagger \psi$ is the solution in the old basis.


We know from class that $S^\dagger \psi = \psi_+ = \begin{pmatrix} C \\ C \end{pmatrix} e ^{-imt}$ or $\psi_- \begin{pmatrix} C \\ -C \end{pmatrix} e^{imt}$

So $\psi = S\psi_+ = \sqrt{2} \begin{pmatrix}  C \\ 0 \end{pmatrix} e ^{-imt}$ or $S\psi_- = -\sqrt{2} \begin{pmatrix} 0 \\  C \end{pmatrix} e ^{imt}$ which is either 


\end{itemize}

\vspace*{0.25in}

\textbf{4) Show that  $\mathscr{L}_{EM} = -\frac{1}{4}F_{\mu\nu}F^{\mu\nu} - J^{\mu}A_{\mu}$ is gauge invariant \hfill \textit{(5 points)}}

Gauge invariance imples that the $\mathscr{L}_{EM}$ is unchanged under $A_\mu \rightarrow A_\mu + \partial_\mu \lambda$


Look at,
\be
F_{\mu\nu} = \partial_\mu A_\nu - \partial_\nu A_\mu
\ee
Under a gauge transformation
\begin{eqnarray*}
\partial_\mu A_\nu - \partial_\nu A_\mu \rightarrow& \partial_\mu (A_\nu + \partial_\nu \lambda) - \partial_\nu (A_\mu + \partial_\mu \lambda)\\
                                        =& \partial_\mu A_\nu - \partial_\nu A_\mu  + (\partial_\mu \partial_\nu \lambda - \partial_\nu \partial_\mu \lambda)\\
                                        =& \partial_\mu A_\nu - \partial_\nu A_\mu  
\end{eqnarray*}
because the partial derivatives commute.

So the $F_\mu\nu$ term is gauge invariant, which means that the first term in $\mathscr{L}_{EM}$ is also gauge invariant.

The second term transforms as:
\begin{eqnarray*}
J^\mu A_\mu \rightarrow& J^\mu(A_\mu + \partial_\mu \lambda) = J^\mu A_\mu + J^\mu  \partial_\mu \lambda
\end{eqnarray*}

Because $\mathscr{L}$ lives in an integral over all space, we can integrate by parts to move the derivative form $\lambda$ to $J^\mu$

\begin{eqnarray*}
J^\mu A_\mu \rightarrow& J^\mu(A_\mu + \partial_\mu \lambda) = J^\mu A_\mu - (\partial_\mu J^\mu)\lambda
\end{eqnarray*}

Conservation of charge implies that $\partial_\mu J^\mu = 0$, which means that $J^\mu A_\mu$ is also gauge invariant.


\vspace*{0.25in}


%\begin{itemize}
%\item[] { }
%\end{itemize}


\textbf{5) Maxwell’s Equations.\hfill \textit{(5 points)}}
\begin{itemize}
\item[a)]In class,  I mentioned  that Gauss’s law follows from taking the 0 component of the equations of motion of the electromagnetic Lagrangian $\partial_\mu F^{\mu\nu} = J^\nu$. 

$J^\mu = (\rho, \vec{J})$

Lets look at $\nu = 0$,
\begin{align}
\partial_\mu F^{\mu0} = J^0 \\
\partial_0 \underbrace{F^{00}}_{0} 0 - \partial_i (-E_i) = \rho \\
\partial_i E_i = \rho \\
\vec{\triangledown} \cdot \vec{E} = \rho \\
\end{align}




\item[b)]Show that the three other of Maxwell’s equations follow from the equations of motion and the Bianchi identity ($\partial_\mu F_{\nu\rho} +\partial_\rho F_{\mu\nu} +\partial_\nu F_{\rho\mu} =0$). 
Lets look at $\nu = i$
\begin{align}
\partial_\mu F^{\mu i} = J^i \\
\partial_0 F^{0i} - \partial_j F^{ji} = J^i \\
\partial_t (-E_i) - \partial_j F^{ji} = J^i \\
\end{align}

Now,
\be
\partial_j F^{ji} = - \vec{\triangledown} \times \vec{B}
\ee

So, 
\begin{align}
\vec{\triangledown} \times \vec{B} = \frac{\partial \vec{E}}{\partial t}  + \vec{J}
\end{align}


now taking $\mu = 0, \nu = i$ and $\rho = j$ in the bianchi identity imples:
\begin{align*}
\partial_0 F_{ij} + \partial_j F_{0i} + \partial_j F_{j0} = 0\\
\partial_t \vec{B} + \vec{\triangledown} \times E = 0
\end{align*}

choosing i = j =k gives 
\be
\vec{\triangledown} \cdot B = 0
\ee


\end{itemize}




}





\end{document}
