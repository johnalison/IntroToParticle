\input{latexSetup}
%\documentclass[margin,line]{res}
\usepackage{braket}
\usepackage{bbm}
\usepackage{relsize}
\usepackage{tcolorbox}

\def\ketY{\ensuremath{\ket {\Psi}}}
\def\iGeV{\ensuremath{\textrm{GeV}^{-1}}}
\usepackage{cancel}

\def\ABCDMatrix{\ensuremath{\begin{pmatrix} A &  B  \\ C  & D \end{pmatrix}}}
\def\xyprime{\ensuremath{\begin{pmatrix} x' \\ y' \end{pmatrix}}}
\def\xyprimeT{\ensuremath{\begin{pmatrix} x' &  y' \end{pmatrix}}}
\def\xy{\ensuremath{\begin{pmatrix} x \\ y \end{pmatrix}}}
\def\xyT{\ensuremath{\begin{pmatrix} x & y \end{pmatrix}}}

\def\IMatrix{\ensuremath{\begin{pmatrix} 0 &  1  \\ -1  & 0 \end{pmatrix}}}
\def\IBoostMatrix{\ensuremath{\begin{pmatrix} 0 &  1  \\ 1  & 0 \end{pmatrix}}}
\def\JThree{\ensuremath{\begin{pmatrix}    0 & -i & 0  \\ i & 0  & 0 \\ 0 & 0 & 0 \end{pmatrix}}} 
\def\JTwo{\ensuremath{\begin{bmatrix}    0 & 0 & -i  \\ 0 & 0  & 0 \\ i & 0 & 0 \end{bmatrix}}}
\def\JOne{\ensuremath{\begin{bmatrix}    0 & 0 & 0  \\ 0 & 0  & -i \\ 0 & i & 0 \end{bmatrix}}}
\def\etamn{\ensuremath{\eta_{\mu\nu}}}
\def\Lmn{\ensuremath{\Lambda^\mu_\nu}}
\def\dmn{\ensuremath{\delta^\mu_\nu}}
\def\wmn{\ensuremath{\omega^\mu_\nu}}
\def\be{\begin{equation*}}
\def\ee{\end{equation*}}
\def\bea{\begin{eqnarray*}}
\def\eea{\end{eqnarray*}}
\def\fmn{\ensuremath{F_{\mu\nu}}}
\def\fMN{\ensuremath{F^{\mu\nu}}}

\def\adagger{\ensuremath{a_{p\sigma}^\dagger}}

%\def\xMu{\ensuremath{x^\mu}

\usepackage{fancyhdr}

\fancyhf{}
\lhead{\Large 33-444} % \hfill Introduction to Particle Physics \hfill Spring 2019}
\chead{\Large Introduction to Particle Physics} % \hfill Spring 2019}
\rhead{\Large Spring 2019} % \hfill Introduction to Particle Physics \hfill Spring 2019}

\begin{document}
\thispagestyle{fancy}

\begin{center}
{\huge \textbf{Lecture 12}}
\end{center}

{\fontsize{14}{16}\selectfont

\textbf{\underline{Relativistic Wave Equations continued}} 

Now lets do the Spin-1 case.

EM is a thoery of Spin-1 interactions.

EM was the first lorentz invariant theory.

Maxwells Equations
\bea
\vec{\nabla} \cdot \vec{E} = \rho \hspace{1in} \vec{\nabla} \times \vec{B} - \frac{\partial\vec{E}}{\partial t} = \vec{J} \\
\vec{\nabla} \cdot \vec{B} = 0 \hspace{1in}  \vec{\nabla} \times \vec{E} - \frac{\partial\vec{B}}{\partial t} = 0
\eea
Not manifestly Lorentz Invariant.

However can put them in a LI form. 

Define $J^\mu = (\rho,\vec{J})$

Now, $\vec{E}$ and $\vec{B}$ have 6-components total.

Same number as an anti-symmetric rank two tensor: $\fmn = - F_{\nu\mu}$ :  

16 - 4 (diagonal) - 6 (lower triangle)


\be
F_{\mu\nu} = \begin{pmatrix} 0  & E_x & E_y & E_z  \\ -E_x  & 0 & -B_z & B_y \\ -E_y  & B_z & 0 & -B_x \\  -E_z  & -B_y & B_x & 0 \end{pmatrix}
\ee

This transforms under a Lorentz transformation as

\be
F_{\mu\nu} \rightarrow {\Lambda_\mu}^\rho {\Lambda_\nu}^\sigma F_{\rho\sigma}
\ee

\be
\partial_\mu \fMN = J^\nu
\ee
Is the Lorentz Invariant ``equation of Motion'' gives 2 of Maxwells equations. 
The ``Bianchi Identity'' 
\be
\partial_\mu F_{\nu\rho} + \partial_\rho F_{\mu\nu} + \partial_\nu F_{\rho\mu}  = 0 
\ee
gives the others.

\noindent\rule{\textwidth}{1pt}

Interestingly, can also express this more simply using potentials.

eg: express E and B in terms of potentials

\be
\vec{E} = \vec{\nabla}V - \frac{\partial \vec{A}}{\partial t} \hspace{1in}  \vec{B} = \vec{\nabla} \times \vec{A}
\ee


Can use potentials defined up to ``gauge transformation''.\\
(Horrible name. Just means redifinition.) 
\be
V \rightarrow V + \frac{\partial \vec{\lambda}}{\partial t} \hspace{1in}  \vec{A} = \vec{A} + \vec{\nabla} \lambda
\ee
these give same physics for any choice of $\lambda$.

Note here there is an enourmour redudancy in our description of the physics. 

OK, now lets Introduce

\be
A_\mu = (V, \vec{A}) \hspace{1in} A_\mu \rightarrow A_\mu + \partial_\mu \lambda
\ee

which allows us to write 

\be
\fmn = \partial_\mu A_\nu - \partial_\nu A_\mu
\ee
(Note that \fmn is invariant under gauge transformations.)

Gauge transformation are \underline{\underline{NOT}} deep.  
More of redundancy, not really saying anything. 

Stress that $A_\mu$ and $A_\mu + \partial_\mu \lambda$ represent the same physics state. 

Can formulate EM from Lagrangian point if using 

\be
L_{EM} = -\frac{1}{4} \fmn\fMN - J^\mu A_\mu
\ee

If the Lagrangian is both Lorentz Invariant and Gauge Invariant, the physics that follows from it will also be. 
HW: Check the gauge invariance.

\noindent\rule{\textwidth}{1pt}

Spin-1 particles that this Langriang describes are the photons.  
Force carriers of the electro-magnetic interaction. 

\underline{Photon equation of motion.}
Almost always most useful to think in terms of $A_\mu$ directly. 

$A_\mu$ ``photon field'' (boson) directly describes photon DoF.

Assume no sources $J_\mu$ = 0\\
Photon equation of motion is then:
\be
\partial_\mu \fMN  = 0  \Rightarrow  \partial^2 A_\mu - \partial_\mu \partial \cdot A = (\eta_{\mu\nu} \partial^2 - \partial_\mu  \partial_\nu ) A^\nu = 0
\ee


\underline{Ansatz}:
\be
A_\mu = \epsilon_\mu(p) e^{-ip\cdot x}
\ee
where $\epsilon(p)$ is  polarization vector.

eq motion $\rightarrow$ 

\bea
(\eta_{\mu\nu} \underbrace{p^2}_{=0} - p_\mu  p_\nu ) \epsilon^\nu = 0\\
\eea
or $p\cdot \epsilon = 0$

OK, with out loss of generality, can assume
\be
P = (E,0,0,E)
\ee 

Most general form of $\epsilon(p) = (a, b, c, a)$. 


Convenient to express b \& c as

\be
\epsilon_R(p) = \frac{1}{\sqrt{2}}(0,1, i,0)  \hspace{1in} \epsilon_L(p) = \frac{1}{\sqrt{2}}(0,1, -i,0)
\ee


\be
\epsilon_R \cdot \epsilon_L = -1 \hspace{1in}  {\epsilon_R}^2 = {\epsilon_L}^2 = 0
\ee

\bea
\longrightarrow P \hspace{1in} \longrightarrow P\\
\underbrace{\rightarrow}_{Spin} \hspace{1in}  \underbrace{\leftarrow}_{Spin}
\eea


What about a ? 
(Note: only expect 2 DoF form Little group) 

Gauge invariance $\Rightarrow$ a is non-physical. 


$A_\mu = 0 $ is \underline{\underline{the same}} as $A_\mu = \partial_\mu \lambda$ for any $\lambda$
(this is in the directtion along $p_\mu$)

Thats all for spin-1. 

\noindent\rule{\textwidth}{1pt}

Ok lets back up and talk more generally now about Lagrangians. 

Unlike classically, usually dont talk about KE and PE. 

\underline{Kinetic Terms} (bilinear - exactly two feilds) \\
$\partial_\mu \phi \partial^\mu \phi$,  $\bar{\psi} \slash{\partial} \psi$, $\frac{1}{4} \fmn^2$,  ... 

\underline{Interaction Terms} (3 or more feilds)\\
$\lambda \phi^3$, $g\bar{\psi}\gamma_\mu A^\mu\psi$, $g\partial_\mu\phi A^\mu\phi$,  $gA_\mu^2A_\nu^2$, ...

\noindent\rule{\textwidth}{1pt}

Will start off doing a little dimensional analysis that wil lturn out to have shockingly large remifications. 

We will do something very simple that is true, and turns out will skip over 40 years of massive confusion in the field (30s - 60s). 

The infinities that emerge when you actually do calculations clouded these results, turns out to all be a red herring



}
\end{document}



% LocalWords:  partilces
