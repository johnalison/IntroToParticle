\input{../latexSetup}

\lhead{\Large 33-444} % \hfill Introduction to Particle Physics \hfill Spring 2022}
\chead{\Large Introduction to Particle Physics} % \hfill Spring 2022}
\rhead{\Large Spring 2022} % \hfill Introduction to Particle Physics \hfill Spring 2022}
\begin{document}
\thispagestyle{fancy}





%\begin{tabular}{c}
%{\large 33-444 \hfill Intro To Particle \hfill Spring 2022\\}
%\hline 
%\end{tabular}

\begin{center}
{\huge \textbf{Homework Set Relativistic QM}}
\large

{\textbf{ Due Date:} Friday February 25th   } 
\end{center}

{\large

\textbf{1) Read Section 2.2 }

\vspace*{0.5in}

\textbf{2) Use the Weyl basis of the $\gamma$ matricies to solve the Dirac equation for a spinor $\psi$ \hfill \textit{(5 points)} }

Assume a solution to the Dirac equation of the form
\be
\psi = u(p) e^{-i p\cdot x}
\ee
where $u(p)$ is a for competent spinor that only depends on p.
Note the $e^{-i p\cdot x}$ is the expected dependence implied by relativistic energy-momentum relation (via the Klien Gordon equation).
Derive a matrix equation for $u(p)$.
How many solutions are there ?
Show that the solution(s) can be written in terms concatenation of a two-component sponor.

\vspace*{0.5in}

\textbf{3) Clifford Algebra \hfill \textit{(10 points)} }

In writing the Dirac equation, we chose a particular representation of the $\gamma$ matrices that satisfied $\{\gamma_\mu,\gamma_\nu\} = 2\eta_{\mu\nu}$, which is called the Clifford algebra. 
The choice we used in class (and in the problem above) is called the Weyl basis.
In this problem, we will study the Clifford algebra and the Weyl basis.
\begin{itemize}
\item[a)]
By explicit multiplication, show that the $\gamma$ matrices in the Weyl basis satisfy the Clifford algebra.
\item[b)]
The Weyl basis isn’t the unique choice of matrices that satisfy the Clifford algebra. 
Another set of matrices that does is
\begin{equation*}
\gamma_0 = \begin{pmatrix} I &  0 \\ 0 & -I \end{pmatrix} \hspace{0.5in} \gamma_i = \begin{pmatrix} 0 & \sigma_i \\  -\sigma_i & 0 \end{pmatrix}
\end{equation*}
This choice of $\gamma$ matrices is related to the Weyl basis by a similarity transformation:
\begin{equation*}
\gamma_\mu = S\gamma_\mu^{\mathrm{Weyl}}S^\dagger, 
\end{equation*}
where S is a unitary matrix such that $SS^\dagger = I$.
A similarity transformation of the $\gamma$ matrices respects the Clifford algebra. 
Determine the matrix S that relates the Weyl basis to this new basis of $\gamma$ matrices.

\item[c)]
 Applying the similarity transformation to the Dirac equation, gives 
\begin{equation*}
(i S \gamma_\mu S^\dagger \partial^{\mu} - m)\psi = S(i\gamma \cdot \partial - m)S^\dagger\psi = 0.
\end{equation*}
Given the solution to the Dirac equation that you found in problem 2), use the similarity transformation S you found above to find the solutions to the Dirac equation using the $\gamma$ matrix in the new basis.
\end{itemize}

\vspace*{0.25in}

\textbf{4) Show that  $\mathscr{L}_{EM} = -\frac{1}{4}F_{\mu\nu}F^{\mu\nu} - J^{\mu}A_{\mu}$ is gauge invariant \hfill \textit{(3 points)}}


\vspace*{0.25in}


%\begin{itemize}
%\item[] { }
%\end{itemize}


\textbf{5) Maxwell’s Equations.\hfill \textit{(5 points)}}

In class,  I claimed that Maxwell’s equations follow from the equations of motion:  $\partial_\mu F^{\mu\nu} = J^\nu$ and the Bianchi identity: ($\partial_\mu F_{\nu\rho} +\partial_\rho F_{\mu\nu} +\partial_\nu F_{\rho\mu} =0$.
Show this explicitly.
\textit{Hint: take individual components and identify the corresponding Maxwell’s equation.
}


}





\end{document}
