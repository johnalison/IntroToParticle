\input{latexSetup}
\usepackage{braket}

\def\ketY{\ensuremath{\ket {\Psi}}}
\def\iGeV{\ensuremath{\textrm{GeV}^{-1}}}
\def\mp{\ensuremath{m_{\textrm{proton}}}}
\def\rp{\ensuremath{r_{\textrm{proton}}}}
\def\me{\ensuremath{m_{\textrm{electron}}}}
\def\aG{\ensuremath{\alpha_G}}
\def\rAtom{\ensuremath{r_{\textrm{atom}}}}
\def\rNucl{\ensuremath{r_{\textrm{nucleus}}}}
\def\GN{\ensuremath{\textrm{G}_\textrm{N}}}

\def\be{\begin{equation*}}
\def\ee{\end{equation*}}


\usepackage{fancyhdr}
\usepackage{cancel}




\fancyhf{}
\lhead{\Large 33-444} % \hfill Introduction to Particle Physics \hfill Spring 2019}
\chead{\Large Introduction to Particle Physics} % \hfill Spring 2019}
\rhead{\Large Spring 2019} % \hfill Introduction to Particle Physics \hfill Spring 2019}
\begin{document}
\thispagestyle{fancy}





%\begin{tabular}{c}
%{\large 33-444 \hfill Intro To Particle \hfill Spring 2019\\}
%\hline 
%\end{tabular}

\begin{center}
{\huge \textbf{Homework Set \#5}}
\large

{\textbf{ Due Date:} Before class Friday February 22nd  } 
\end{center}

{\large
\textbf{1: Clifford Algebra \hfill \textit{(5 points)} }
\vspace*{0.25in}

\textbf{2: Show that the $\mathscr{L}_{EM}$ is gauge invariant \hfill \textit{(5 points)}}


\vspace*{0.25in}


%\begin{itemize}
%\item[] { }
%\end{itemize}


3)
Maxwell’s Equations. In Eq. 2.125, we showed how Gauss’s law follows from taking the 0 component of the equations of motion of the electromagnetic Lagrangian. Show explicitly that the three other of Maxwell’s equations follow from Eqs. 2.124 and 2.126. To do this, take individual components and identify the corresponding Maxwell’s equation.



5) Lagrangians 
  - derive the Euler Lagrane Eq from setting the change in action to 0
  - What is the noethers current for the example we considered in class

6) Dark matter searches



}





\end{document}
