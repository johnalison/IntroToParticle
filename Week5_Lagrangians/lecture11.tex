\documentclass[paper=letter,11pt]{scrartcl}

\KOMAoptions{headinclude=true, footinclude=false}
\KOMAoptions{DIV=14, BCOR=5mm}
\KOMAoptions{numbers=noendperiod}
\KOMAoptions{parskip=half}
\addtokomafont{disposition}{\rmfamily}
\addtokomafont{part}{\LARGE}
\addtokomafont{descriptionlabel}{\rmfamily}
%\setkomafont{pageheadfoot}{\normalsize\sffamily}
\setkomafont{pagehead}{\normalsize\rmfamily}
%\setkomafont{publishers}{\normalsize\rmfamily}
\setkomafont{caption}{\normalfont\small}
\setcapindent{0pt}
\deffootnote[1em]{1em}{1em}{\textsuperscript{\thefootnotemark}\ }


\usepackage{amsmath}
\usepackage[varg]{txfonts}
\usepackage[T1]{fontenc}
\usepackage{graphicx}
\usepackage{xcolor}
\usepackage[american]{babel}
% hyperref is needed in many places, so include it here
\usepackage{hyperref}

\usepackage{xspace}
\usepackage{multirow}
\usepackage{float}


\usepackage{braket}
\usepackage{bbm}
\usepackage{relsize}
\usepackage{tcolorbox}

\def\ketY{\ensuremath{\ket {\Psi}}}
\def\iGeV{\ensuremath{\textrm{GeV}^{-1}}}
%\def\mp{\ensuremath{m_{\textrm{proton}}}}
\def\rp{\ensuremath{r_{\textrm{proton}}}}
\def\me{\ensuremath{m_{\textrm{electron}}}}
\def\aG{\ensuremath{\alpha_G}}
\def\rAtom{\ensuremath{r_{\textrm{atom}}}}
\def\rNucl{\ensuremath{r_{\textrm{nucleus}}}}
\def\GN{\ensuremath{\textrm{G}_\textrm{N}}}
\def\ketX{\ensuremath{\ket{\vec{x}}}}
\def\ve{\ensuremath{\vec{\epsilon}}}


\def\ABCDMatrix{\ensuremath{\begin{pmatrix} A &  B  \\ C  & D \end{pmatrix}}}
\def\xyprime{\ensuremath{\begin{pmatrix} x' \\ y' \end{pmatrix}}}
\def\xyprimeT{\ensuremath{\begin{pmatrix} x' &  y' \end{pmatrix}}}
\def\xy{\ensuremath{\begin{pmatrix} x \\ y \end{pmatrix}}}
\def\xyT{\ensuremath{\begin{pmatrix} x & y \end{pmatrix}}}

\def\IMatrix{\ensuremath{\begin{pmatrix} 0 &  1  \\ -1  & 0 \end{pmatrix}}}
\def\IBoostMatrix{\ensuremath{\begin{pmatrix} 0 &  1  \\ 1  & 0 \end{pmatrix}}}
\def\JThree{\ensuremath{\begin{pmatrix}    0 & -i & 0  \\ i & 0  & 0 \\ 0 & 0 & 0 \end{pmatrix}}} 
\def\JTwo{\ensuremath{\begin{bmatrix}    0 & 0 & -i  \\ 0 & 0  & 0 \\ i & 0 & 0 \end{bmatrix}}}
\def\JOne{\ensuremath{\begin{bmatrix}    0 & 0 & 0  \\ 0 & 0  & -i \\ 0 & i & 0 \end{bmatrix}}}
\def\etamn{\ensuremath{\eta_{\mu\nu}}}
\def\Lmn{\ensuremath{\Lambda^\mu_\nu}}
\def\dmn{\ensuremath{\delta^\mu_\nu}}
\def\wmn{\ensuremath{\omega^\mu_\nu}}
\def\be{\begin{equation*}}
\def\ee{\end{equation*}}
\def\bea{\begin{eqnarray*}}
\def\eea{\end{eqnarray*}}
\def\bi{\begin{itemize}}
\def\ei{\end{itemize}}
\def\fmn{\ensuremath{F_{\mu\nu}}}
\def\fMN{\ensuremath{F^{\mu\nu}}}
\def\bc{\begin{center}}
\def\ec{\end{center}}
\def\nus{$\nu$s}

\def\adagger{\ensuremath{a_{p\sigma}^\dagger}}
\def\lineacross{\noindent\rule{\textwidth}{1pt}}

\newcommand{\multiline}[1] {
\begin{tabular} {|l}
#1
\end{tabular}
}

\newcommand{\multilineNoLine}[1] {
\begin{tabular} {l}
#1
\end{tabular}
}



\newcommand{\lineTwo}[2] {
\begin{tabular} {|l}
#1 \\
#2
\end{tabular}
}

\newcommand{\rmt}[1] {
\textrm{#1}
}


%
% Units
%
\def\m{\ensuremath{\rmt{m}}}
\def\GeV{\ensuremath{\rmt{GeV}}}
\def\pt{\ensuremath{p_\rmt{T}}}


\def\parity{\ensuremath{\mathcal{P}}}

\usepackage{cancel}
\usepackage{ mathrsfs }
\def\bigL{\ensuremath{\mathscr{L}}}

\usepackage{ dsfont }



\usepackage{fancyhdr}
\fancyhf{}


\lhead{\Large 33-444} % \hfill Introduction to Particle Physics \hfill Spring 2020}
\chead{\Large Introduction to Particle Physics} % \hfill Spring 2020}
\rhead{\Large Spring 2020} % \hfill Introduction to Particle Physics \hfill Spring 2020}

\begin{document}
\thispagestyle{fancy}

\begin{center}
{\huge \textbf{Lecture 11}}
\end{center}

{\fontsize{14}{16}\selectfont

\textbf{\underline{Relativistic Wave Equations}} 

Lets look at the Schrodinger Equation
\be
i \frac{d}{dt}\psi = - \left( \frac{\nabla^2}{2m} + V \right) \psi
\ee

\underline{Problems} 
- Conservation of non-relativistic energy.


$E \leftrightarrow i \frac{d}{dt}$ 
and
$p \leftrightarrow -i \nabla$


$\Rightarrow$ Schrodinger Equation $E = \frac{p^2}{2m} + V$

- time/space not on equal footing.


\noindent\rule{\textwidth}{1pt}

Start w/relativistic E/p relation.
\be
E^2 - p^2 - m^2 = 0
\ee

Leads us to the \underline{Klein-Gordon} equation
\be
\left(-\frac{\partial^2}{\partial t^2} + \nabla^2 - m^2 \right)\phi(x,t) = 0
\ee


Solution, 

\be
\phi(x,t) = e^{ipx}
\ee
Called ``On shell-solutions''.

Everything is Lorentz Invariant. 

\underline{Note} Two solutions E>0 and E<0.

\noindent\rule{\textwidth}{1pt}

Alternative Standard Powerful formulation of Physics via Lagrangians.

KG eq
\be
\frac{\partial^2}{\partial t^2}\phi  =  \nabla^2\phi - m^2 \phi
\ee

where $\phi(x,t)$ permeates space and time

\underline{Looks like}

\be
\frac{d^2}{d t^2}x(t)  =  -\frac{\partial U}{\partial x}
\ee
Where U is a potential energy.

\begin{itemize}
\item[-] $\frac{\partial^2}{\partial t^2}\phi$ - looks like the ``acceleration'' of the field
\item[-] $- m^2 \phi$ - effective ``Force'' for the mass (harmonic oscillator)
\item[-] $\nabla^2\phi$ - shear force (shearing fabric takes force)
\end{itemize}

Now have some intuition about what the KG equation is telling us.

Integrate force to get the potential

\be
-\frac{\partial U}{\partial \phi} = \nabla^2\phi - m^2 \phi \Rightarrow U = \frac{1}{2}(\nabla \phi)(\nabla \phi) + \frac{m^2}{2}\phi^2
\ee

KE form generalization of $1/2 \dot{x}^2$ to fields.

\be
K = 1/2 (\frac{\partial}{\partial t}\phi)^2
\ee

so,

\be
K + U  = \frac{1}{2} (\frac{\partial}{\partial t}\phi)^2 + \frac{1}{2}(\nabla \phi)(\nabla \phi) + \frac{m^2}{2}\phi^2
\ee

Get total energy from integrating, 

\be
H(t) = \int d\vec{x} \left[  \frac{1}{2} (\frac{\partial}{\partial t}\phi)^2 + \frac{1}{2}(\nabla \phi)(\nabla \phi) + \frac{m^2}{2}\phi^2  \right]
\ee
with this you could go back to three lectures ago and replace $\frac{\nabla\phi^\dagger\nabla\phi}{2m}$ and do everything again (we wont bother here...)


However we can use this to formulate the field eq in a totally different, ultimately more useful way...


Think of classical mechanics of point partilces.
Can all be formulated with Lagrangians and the principle of least action. 


\be
L = \frac{1}{2}\dot{x}^2 - U(x)
\ee 
difference in KE and potential

The action
\be
S[x(t)] = \int dt L
\ee
you could take this as the starting point and derive Newton's 2nd law by minimizing the action.

We will do the same for relativistic fields.

\be
L = K - U = \int d^3x \left[  \frac{1}{2} (\frac{\partial}{\partial t}\phi)^2 - \frac{1}{2}(\nabla \phi)(\nabla \phi) - \frac{m^2}{2}\phi^2  \right]
\ee



\be
S[\phi(t)] = \int d^4x \left[ \textrm{(Above)} \right]
\ee
KG can be found from minimizing the action wrt $\phi$.


\be
\partial_\mu = (\frac{\partial}{\partial t}, -\vec{\nabla})
\ee

\be
S[\phi(t)] = \int d^4x \left[ \frac{1}{2} (\partial_\mu\phi)(\partial^\mu\phi) - \frac{m^2}{2}\phi^2      \right]
\ee
manifestly L.I


This tells us how to construct general LI descriptions of relativistic QM systems.
Need a LI Lagrangian and we are ``done'' (meaning $\sim$ all of physics follows from this) 

KG 2nd order in $x^\mu$.  
Contains no info about intrinsic angular momentum. (No free Lorentz indices)
$\phi$ from KG is Spin-0 field.

\begin{tcolorbox}
Two more relativistic systems important for SM
\begin{itemize}
\item[-] Spin-1 ``EM''
\item[-] Spin-1/2 ``Dirac Equation''
\end{itemize}
\end{tcolorbox}


\noindent\rule{\textwidth}{1pt}
Start with the the ``Dirac Equation''.

Can we find a 1st order relativistic eq of motion?

Assume there exists a wave eq. linear in x and t.

\be
\left(\alpha \frac{\partial}{\partial t} + \vec{\beta}\vec{\nabla}   \right)\psi = m\psi
\ee
for some $\alpha, \vec{\beta}$, and m. 

If relativistic invariant, must imply the KG. 

Assume that it is the square root

\be
\left(\alpha \frac{\partial}{\partial t} + \vec{\beta}\vec{\nabla}   \right) \left(\alpha \frac{\partial}{\partial t} + \vec{\beta}\vec{\nabla}   \right)\psi = m^2\psi
\ee

$\Rightarrow$

\be
\left[\alpha^2 \frac{\partial^2}{\partial t^2} + \alpha \vec{\beta} \vec{\nabla} \frac{\partial}{\partial t} + \vec{\beta}\vec{\nabla}\alpha \frac{\partial}{\partial t} + (\beta\cdot\nabla)^2 \right] \psi = m^2 \psi
\ee
to get $(-\frac{\partial^2}{\partial t^2} + \nabla^2)\psi = m^2 \psi$

$\Rightarrow$

\begin{itemize}        
\item[-] $\alpha^2 = -1$ - can be a complex number 
\item[-] $\beta_i\beta_j = \delta_{ij}$ - cannot be numbers
\item[-] $\alpha\beta_i + \beta_i\alpha = 0$
\end{itemize}        

Lets keep going...  

\underline{Define}


$\alpha = i \gamma_0 $ \hspace*{1in} $\beta_i = i \gamma_i $

$\gamma_0\gamma_0 = 1$ \hspace*{1in} $\gamma_i\gamma_j = - \delta_{ij}$ \hspace*{1in} $\gamma_0\gamma_i + \gamma_i\gamma_0 = 0$

\be
\underbrace{\gamma_\mu\gamma_\nu + \gamma_\nu\gamma_\mu}_{\{ \gamma_\nu, \gamma_\mu \}} = \begin{pmatrix} 2 & 0 & 0 & 0 \\ 0 & -2 & 1 & 0 \\ 0 & 0 & -2 & 0 \\ 0 & 0 & 0 & -2 \end{pmatrix} = 2 \eta_{\mu\nu}
\ee

\be
(i\gamma_\mu\partial^\mu -m )\psi = 0
\ee
``Dirac Equation''     
$\gamma$ matrices are 4x4.

Can use (``Weyl basis'') 
\be
\gamma_\mu = \begin{pmatrix} 0 & \sigma_\mu \\ \bar{\sigma}_\mu & 0 \end{pmatrix}
\ee
where 

$\sigma = (1,\vec{\sigma})$ \hspace*{1in} $\bar{\sigma} = (1,-\vec{\sigma})$

There are other choices you can work out...

Dirac equation describes spin 1/2 partilces (e)

B/c $\gamma_\mu$ are matrices, the $\psi$ solution will be a  4-component vector ``spinor''.


\underline{Solution to DE}
work out in homework ...

\noindent\rule{\textwidth}{1pt}

Solutions are four component objeccts spinors $\psi$ (complex) 

Use $\bar{\psi}$ to denote the complex conjugate of the spinor.

The dirac equation can be formulated from Lagrangian principle via
\be
S[\psi,\bar{\psi}] = \int d^4x\ \bar{\psi}[i\gamma_\mu\partial^\mu - m]\psi
\ee

Varrying the action with respect to $\bar{\psi}$ gives the dirac equation.

}
\end{document}



% LocalWords:  partilces
