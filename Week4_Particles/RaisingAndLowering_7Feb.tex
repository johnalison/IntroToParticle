\input{../latexSetup}


\lhead{\Large 33-444} % \hfill Introduction to Particle Physics \hfill Spring 2022}
\chead{\Large Introduction to Particle Physics} % \hfill Spring 2022}
\rhead{\Large Spring 2022} % \hfill Introduction to Particle Physics \hfill Spring 2022}

\begin{document}
\thispagestyle{fancy}

\begin{center}
{\huge \textbf{Lecture 9}}
\end{center}

{\fontsize{14}{16}\selectfont

\textbf{\underline{Particles Interactions}} 

``Quantum Field Theory in a week''

{\Large \underline{\textbf{Summary From Last Time}}}

\begin{tabular}{l}
\underline{Massive:}\\
\\
\hspace{0.5in}$\ket{P, \sigma}$ and $U[\Lambda] \ket{P, \sigma} = \sum_{\sigma'} R_{\sigma,\sigma'} \ket{\Lambda P, \sigma'}$, where the R is a rotation matrix.\\
\\
\underline{Mass-less:}\\
\\
\hspace{0.5in}$\ket{P, h}$ and $U[\Lambda] \ket{P, h} = e^{ih\Theta(W)}\ket{\Lambda P, h}$, where the coefficient is just a phase.\\
\end{tabular}\\

\noindent\rule{\textwidth}{1pt}
Now talk about all these particles states in a more convenient way.  

For every given momenta we have stacks of Hilbert space
\begin{itemize}
\item[-] infinitely many possibilities for Bosons $0\rightarrow N$
\item[-] fixed number (depending on the spin) for fermions
\end{itemize}

\vspace{0.5in}
Ultimately interested in the interactions between particles. 
\begin{itemize}
\item[-] Defined by some Hamiltonian        
\item[-] Could specify the Hamiltonian by describing how it acts on all states in the Hilbert space
\item[-] Instead \underline{for our convenience} introduce creation and annihilation operators.
\begin{itemize}
\item[] These keep track of the states in a simple way...
\end{itemize}
\end{itemize}

First define the vacuum state: $\ket{0}$.
Then define single particle states as:
\be
\underbrace{\ket{P \sigma}}_{\textrm{These are primary}} \equiv \adagger \ket{0}
\ee
This \underline{defines} \adagger.



\be
\ket{P_1 \sigma_1, P_2, \sigma_2 } = a_{p_1\sigma_1}^\dagger a_{p_2\sigma_2}^\dagger \ket{0}
\ee

etc. etc. etc.


Similarly,  can define annihilation operators


\begin{align*}
a_{P,\sigma} \ket{0} = 0\\
a_{P,\sigma} \ket{P,\sigma} = \ket{0}
\end{align*}
a - removes states. 


Can encode boson/fermion statistics in $a$ and $a^\dagger$.

\begin{center}
\begin{tabular}{c|c}
Bosons & Fermions \\
$[a_{P_1,\sigma_1}^\dagger, a_{P_2,\sigma_2}^\dagger] = 0 $  &  $\{a_{P_1,\sigma_1}^\dagger, a_{P_2,\sigma_2}^\dagger\} = 0 $\\  
$[a_{P_1,\sigma_1}, a_{P_2,\sigma_2}] = 0  $  &  $\{a_{P_1,\sigma_1}, a_{P_2,\sigma_2}\} = 0  $\\  
\end{tabular}
\end{center}


Normalization

\be
\braket{p',\sigma'|p,\sigma} = \delta_{\sigma',\sigma} \delta^3(\vec{p} - \vec{p'})
\ee 
will use compressed notation

\be
\braket{p',\sigma'|p,\sigma} = \delta_{\sigma',\sigma} \delta_{p,p'}
\ee 

This implies

For Bosons,
\be
[a_{p,\sigma}, a_{p',\sigma'}^\dagger] = \delta_{\sigma',\sigma} \delta_{p,p'}
\ee


For Fermions,
\be
\{a_{p,\sigma}, a_{p',\sigma'}^\dagger\} = \delta_{\sigma',\sigma} \delta_{p,p'}
\ee

\underline{Note}, these are operators that \underline{We} have \underline{defined} four \underline{our} convenience.
Makes it easier to talk about the states.  (Of course what really matters is the states)

\clearpage

Usually you see it presented as:
\begin{itemize}
\item[-] Start with fields
\item[-] Quantize them
\item[-] then find these commutation relation
\end{itemize}

But really the fields are secondary concepts, what comes first is the particles. \\
``Not one deep thing going on here''

\noindent\rule{\textwidth}{1pt}


OK, What would be the free Hamiltonian ?

\be
H = \sum\limits_{p,\sigma} E_p\ a^\dagger_{p,\sigma} a_{p,\sigma}
\ee

Were labeling states by 4-momenta, but the energy is constrained ($E^2 = p^2 + m^2$).

Can label by the 3-momentum

\be
\braket{\vec{p}',\sigma'|\vec{p},\sigma} = (2\pi)^3 2 E_p \delta^3(\vec{p}-\vec{p}')
\ee 


\underline{Note}  $d^3p$ is not Lorentz invariant, but $d^3p/2E_p$ is.  (HW).

To make my life easier by defining

\be
\int \cancel{d}^3p = \int \frac{d^3p}{2E}
\ee

So can write the free Hamiltonion as:

\bea
H & = \sum\limits_{p,\sigma} E_p a^\dagger_{p,\sigma} a_{p,\sigma} \\
  &=  \int \cancel{d}^3p E_p a^\dagger_{p,\sigma} a_{p,\sigma} 
\eea

Now lets imagine building interactions.  Add interaction hamiltonian.

eg: an interaction where two particles come in and two particles go out:
\begin{figure}[h]
\centering
\includegraphics[width=0.4\textwidth]{./Interaction.pdf}
\end{figure}

This would correspond to adding an interaction term to the Hamiltonian of the form:

\bea
& \int \cancel{d}^3p_1 \cancel{d}^3p_2 \cancel{d}^3p_3 \cancel{d}^3p_4\ \delta(\vec{p_1}..\vec{p_4})\ \delta(E_1 + ... E_4) \\
& \underbrace{a^\dagger_{p_4,\sigma_4} a^\dagger_{p_3,\sigma_3} a_{p_2,\sigma_2} a_{p_1,\sigma_1} }_{\textrm{Acts on the initial state and gives the final state}} V(p's, \sigma's)   + \textrm{Hermitian Conjugate}
\eea

Interactions are made out of strings of a's and $a^\dagger$'s.
Also easy in this picture to talk about the creation and destruction of particles (Not just scattering) 

Very convenient to use this to map between particle states.

So far we have not said the word ``Field''. 

\end{document}


