\input{latexSetup}
%\documentclass[margin,line]{res}
\usepackage{braket}
\usepackage{bbm}
\usepackage{relsize}

\def\ketY{\ensuremath{\ket {\Psi}}}
\def\iGeV{\ensuremath{\textrm{GeV}^{-1}}}


\def\ABCDMatrix{\ensuremath{\begin{pmatrix} A &  B  \\ C  & D \end{pmatrix}}}
\def\xyprime{\ensuremath{\begin{pmatrix} x' \\ y' \end{pmatrix}}}
\def\xyprimeT{\ensuremath{\begin{pmatrix} x' &  y' \end{pmatrix}}}
\def\xy{\ensuremath{\begin{pmatrix} x \\ y \end{pmatrix}}}
\def\xyT{\ensuremath{\begin{pmatrix} x & y \end{pmatrix}}}

\def\IMatrix{\ensuremath{\begin{pmatrix} 0 &  1  \\ -1  & 0 \end{pmatrix}}}
\def\IBoostMatrix{\ensuremath{\begin{pmatrix} 0 &  1  \\ 1  & 0 \end{pmatrix}}}
\def\JThree{\ensuremath{\begin{pmatrix}    0 & -i & 0  \\ i & 0  & 0 \\ 0 & 0 & 0 \end{pmatrix}}} 
\def\JTwo{\ensuremath{\begin{bmatrix}    0 & 0 & -i  \\ 0 & 0  & 0 \\ i & 0 & 0 \end{bmatrix}}}
\def\JOne{\ensuremath{\begin{bmatrix}    0 & 0 & 0  \\ 0 & 0  & -i \\ 0 & i & 0 \end{bmatrix}}}
\def\etamn{\ensuremath{\eta_{\mu\nu}}}
\def\Lmn{\ensuremath{\Lambda^\mu_\nu}}
\def\dmn{\ensuremath{\delta^\mu_\nu}}
\def\wmn{\ensuremath{\omega^\mu_\nu}}
\def\be{\begin{equation*}}
\def\ee{\end{equation*}}

\def\adagger{\ensuremath{a_{p\sigma}^\dagger}}

%\def\xMu{\ensuremath{x^\mu}

\usepackage{fancyhdr}

\fancyhf{}
\lhead{\Large 33-444} % \hfill Introduction to Particle Physics \hfill Spring 2019}
\chead{\Large Introduction to Particle Physics} % \hfill Spring 2019}
\rhead{\Large Spring 2019} % \hfill Introduction to Particle Physics \hfill Spring 2019}

\begin{document}
\thispagestyle{fancy}

\begin{center}
{\huge \textbf{Lecture 9}}
\end{center}

{\fontsize{14}{16}\selectfont

\textbf{\underline{Particles Interactions}} 

``Quantum Field Theory in a week''

{\Large \underline{\textbf{Summary From Last Time}}}

\begin{tabular}{l}
\underline{Massive:}\\
\\
\hspace{0.5in}$\ket{P, \sigma}$ and $U[\Lambda] \ket{P, \sigma} = \sum_{\sigma'} R_{\sigma,\sigma'} \ket{\Lambda P, \sigma'}$, where the R is a rotation matrix.\\
\\
\underline{Massless:}\\
\\
\hspace{0.5in}$\ket{P, h}$ and $U[\Lambda] \ket{P, h} = e^{ih\Theta(W)}\ket{\Lambda P, h}$, where the coefficient is just a phase.\\
\end{tabular}\\

\noindent\rule{\textwidth}{1pt}
Now talk about all these particles states in a more convient way.  

For every given momenta we have stacks of Hilbert space
\begin{itemize}
\item[-] infinitely many possibliities for Bosons $0\rightarrow N$
\item[-] fixed number (depending on the spin) for fermions
\end{itemize}

\vspace{0.5in}
Ultimately interested in the interactions between partilces. 
\begin{itemize}
\item[-] Defined by some Hamiltonian        
\item[-] Could specify the Hamiltonian by describing how it acts on all states in the Hilbert space
\item[-] Instead \underline{for our convience} introduce creation and anhilation operators
These keep track of teh states in a simple way...
\end{itemize}

First define the vacuum state: $\ket{0}$.
Then define single particle states as:

\be
\underbrace{\ket{P \sigma}}_{\textrm{These are primary}} \equiv \adagger \ket{0}
\ee
This \underline{defines} \adagger.



\be
\ket{P_1 \sigma_1, P_2, \sigma_2 } = a_{p_1\sigma_1}^\dagger a_{p_2\sigma_2}^\dagger \ket{0}
\ee

etc. etc. etc.


Similarly,  can define anhilation operators


\begin{align*}
a_{P,\sigma} \ket{0} = 0\\
a_{P,\sigma} \ket{P,\sigma} = \ket{0}
\end{align*}
a - removes states. 


Can encode boson/fermion statistics in $a$ and $a^\dagger$.

\begin{center}
\begin{tabular}{c|c}
Bosons & Fermions \\
$[a_{P_1,\sigma_1}^\dagger, a_{P_2,\sigma_2}^\dagger] = 0 $  &  $\{a_{P_1,\sigma_1}^\dagger, a_{P_2,\sigma_2}^\dagger\} = 0 $\\  
$[a_{P_1,\sigma_1}, a_{P_2,\sigma_2}] = 0  $  &  $\{a_{P_1,\sigma_1}, a_{P_2,\sigma_2}\} = 0  $\\  
\end{tabular}
\end{center}



}
\end{document}


