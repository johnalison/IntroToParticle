\documentclass[paper=letter,11pt]{scrartcl}

\KOMAoptions{headinclude=true, footinclude=false}
\KOMAoptions{DIV=14, BCOR=5mm}
\KOMAoptions{numbers=noendperiod}
\KOMAoptions{parskip=half}
\addtokomafont{disposition}{\rmfamily}
\addtokomafont{part}{\LARGE}
\addtokomafont{descriptionlabel}{\rmfamily}
%\setkomafont{pageheadfoot}{\normalsize\sffamily}
\setkomafont{pagehead}{\normalsize\rmfamily}
%\setkomafont{publishers}{\normalsize\rmfamily}
\setkomafont{caption}{\normalfont\small}
\setcapindent{0pt}
\deffootnote[1em]{1em}{1em}{\textsuperscript{\thefootnotemark}\ }


\usepackage{amsmath}
\usepackage[varg]{txfonts}
\usepackage[T1]{fontenc}
\usepackage{graphicx}
\usepackage{xcolor}
\usepackage[american]{babel}
% hyperref is needed in many places, so include it here
\usepackage{hyperref}

\usepackage{xspace}
\usepackage{multirow}
\usepackage{float}


\usepackage{braket}
\usepackage{bbm}
\usepackage{relsize}
\usepackage{tcolorbox}

\def\ketY{\ensuremath{\ket {\Psi}}}
\def\iGeV{\ensuremath{\textrm{GeV}^{-1}}}
%\def\mp{\ensuremath{m_{\textrm{proton}}}}
\def\rp{\ensuremath{r_{\textrm{proton}}}}
\def\me{\ensuremath{m_{\textrm{electron}}}}
\def\aG{\ensuremath{\alpha_G}}
\def\rAtom{\ensuremath{r_{\textrm{atom}}}}
\def\rNucl{\ensuremath{r_{\textrm{nucleus}}}}
\def\GN{\ensuremath{\textrm{G}_\textrm{N}}}
\def\ketX{\ensuremath{\ket{\vec{x}}}}
\def\ve{\ensuremath{\vec{\epsilon}}}


\def\ABCDMatrix{\ensuremath{\begin{pmatrix} A &  B  \\ C  & D \end{pmatrix}}}
\def\xyprime{\ensuremath{\begin{pmatrix} x' \\ y' \end{pmatrix}}}
\def\xyprimeT{\ensuremath{\begin{pmatrix} x' &  y' \end{pmatrix}}}
\def\xy{\ensuremath{\begin{pmatrix} x \\ y \end{pmatrix}}}
\def\xyT{\ensuremath{\begin{pmatrix} x & y \end{pmatrix}}}

\def\IMatrix{\ensuremath{\begin{pmatrix} 0 &  1  \\ -1  & 0 \end{pmatrix}}}
\def\IBoostMatrix{\ensuremath{\begin{pmatrix} 0 &  1  \\ 1  & 0 \end{pmatrix}}}
\def\JThree{\ensuremath{\begin{pmatrix}    0 & -i & 0  \\ i & 0  & 0 \\ 0 & 0 & 0 \end{pmatrix}}} 
\def\JTwo{\ensuremath{\begin{bmatrix}    0 & 0 & -i  \\ 0 & 0  & 0 \\ i & 0 & 0 \end{bmatrix}}}
\def\JOne{\ensuremath{\begin{bmatrix}    0 & 0 & 0  \\ 0 & 0  & -i \\ 0 & i & 0 \end{bmatrix}}}
\def\etamn{\ensuremath{\eta_{\mu\nu}}}
\def\Lmn{\ensuremath{\Lambda^\mu_\nu}}
\def\dmn{\ensuremath{\delta^\mu_\nu}}
\def\wmn{\ensuremath{\omega^\mu_\nu}}
\def\be{\begin{equation*}}
\def\ee{\end{equation*}}
\def\bea{\begin{eqnarray*}}
\def\eea{\end{eqnarray*}}
\def\bi{\begin{itemize}}
\def\ei{\end{itemize}}
\def\fmn{\ensuremath{F_{\mu\nu}}}
\def\fMN{\ensuremath{F^{\mu\nu}}}
\def\bc{\begin{center}}
\def\ec{\end{center}}
\def\nus{$\nu$s}

\def\adagger{\ensuremath{a_{p\sigma}^\dagger}}
\def\lineacross{\noindent\rule{\textwidth}{1pt}}

\newcommand{\multiline}[1] {
\begin{tabular} {|l}
#1
\end{tabular}
}

\newcommand{\multilineNoLine}[1] {
\begin{tabular} {l}
#1
\end{tabular}
}



\newcommand{\lineTwo}[2] {
\begin{tabular} {|l}
#1 \\
#2
\end{tabular}
}

\newcommand{\rmt}[1] {
\textrm{#1}
}


%
% Units
%
\def\m{\ensuremath{\rmt{m}}}
\def\GeV{\ensuremath{\rmt{GeV}}}
\def\pt{\ensuremath{p_\rmt{T}}}


\def\parity{\ensuremath{\mathcal{P}}}

\usepackage{cancel}
\usepackage{ mathrsfs }
\def\bigL{\ensuremath{\mathscr{L}}}

\usepackage{ dsfont }



\usepackage{fancyhdr}
\fancyhf{}

%\documentclass[margin,line]{res}
\usepackage{braket}
\usepackage{bbm}
\usepackage{relsize}

\def\ketY{\ensuremath{\ket {\Psi}}}
\def\iGeV{\ensuremath{\textrm{GeV}^{-1}}}


\def\ABCDMatrix{\ensuremath{\begin{pmatrix} A &  B  \\ C  & D \end{pmatrix}}}
\def\xyprime{\ensuremath{\begin{pmatrix} x' \\ y' \end{pmatrix}}}
\def\xyprimeT{\ensuremath{\begin{pmatrix} x' &  y' \end{pmatrix}}}
\def\xy{\ensuremath{\begin{pmatrix} x \\ y \end{pmatrix}}}
\def\xyT{\ensuremath{\begin{pmatrix} x & y \end{pmatrix}}}

\def\IMatrix{\ensuremath{\begin{pmatrix} 0 &  1  \\ -1  & 0 \end{pmatrix}}}
\def\IBoostMatrix{\ensuremath{\begin{pmatrix} 0 &  1  \\ 1  & 0 \end{pmatrix}}}
\def\JThree{\ensuremath{\begin{pmatrix}    0 & -i & 0  \\ i & 0  & 0 \\ 0 & 0 & 0 \end{pmatrix}}} 
\def\JTwo{\ensuremath{\begin{bmatrix}    0 & 0 & -i  \\ 0 & 0  & 0 \\ i & 0 & 0 \end{bmatrix}}}
\def\JOne{\ensuremath{\begin{bmatrix}    0 & 0 & 0  \\ 0 & 0  & -i \\ 0 & i & 0 \end{bmatrix}}}
\def\etamn{\ensuremath{\eta_{\mu\nu}}}
\def\Lmn{\ensuremath{\Lambda^\mu_\nu}}
\def\dmn{\ensuremath{\delta^\mu_\nu}}
\def\wmn{\ensuremath{\omega^\mu_\nu}}
\def\be{\begin{equation*}}
\def\ee{\end{equation*}}

%\def\xMu{\ensuremath{x^\mu}

\usepackage{fancyhdr}

\fancyhf{}
\lhead{\Large 33-444} % \hfill Introduction to Particle Physics \hfill Spring 2019}
\chead{\Large Introduction to Particle Physics} % \hfill Spring 2019}
\rhead{\Large Spring 2019} % \hfill Introduction to Particle Physics \hfill Spring 2019}

\begin{document}
\thispagestyle{fancy}

\begin{center}
{\huge \textbf{Lecture 8}}
\end{center}

{\fontsize{14}{16}\selectfont

\textbf{\underline{Particles}} 

Now, QM + SR.  (``Quantum Field Theory in a week'')

\begin{center}
\textbf{Particles! Particles! Particles! }
\end{center}


Fundamentally everything we are talking about is the interaction of particles.
No such thing as sometimes waves, sometimes particles, its all particles.
No particle-wave duality (old fashion).

\underline{QM Particles}
Sometimes macroscopic collections of QM particles (when they are bosons) have a nice interpretation as classical waves .
Macroscopic collections of fermions look like classical particles.

\fbox{\begin{minipage}{\textwidth}
(By the way do people know what I'm talking about when I say bosons vs fermions? 
QM, identical particles are fundamentally indistinguishable. 

\be
|\psi(p_1,p_2)|^2 = |\psi(p_2,p_1)|^2 \Rightarrow \psi(p_1,p_2) = \pm \psi(p_2,p_1)
\ee
Categorizes all particles in nature into two classes (NR QM effect)\\
$\psi(p_1,p_2) = + \psi(p_2,p_1)$ (for bosons)\\
$\psi(p_1,p_2) = + \psi(p_2,p_1)$ (for fermions)
)
\end{minipage}}

Fields are also a secondary notion.
They are a convienent way of talking about the interactions of particles.

What particles are:
single particle states are irreducible representations of the Poincare group. 
Poincare group = (translation + Lorentz transformations.)

We have symmetries, (handed down to us from the 1st half o the twenty century) its a good idea to talk about what they can act on.
The things they act on can be broken down into the irreducible representations.
Associated with every Poincare transformation, there should be some unitary operator that acts on the Hilbert space of the theory.


Why do particles have something to do with symmetries?
Momentum eigenstates behave nicely under translations. 
In a world that is translationally invariant, useful to talk about momentum eigenstates. 
We use the same word for particles moving in different directions even thought they are different states, because they are related under rotations. 

So, what are the labels you can have on states for which translations and rotations act nicely?
Non-relativistically, translations labeled by $\vec{P}$, Rotations by spin. 

We want to generalize this to relativity. 
Translations, Rotations and Boosts, the whole Lorentz group.

What are the possible labels?

The answer is going to end up being
\begin{itemize}
\item[-)] \underline{Massive:} Same thing we are used to $\vec{P_\mu}$ and spin
\item[-)] \underline{Massless:} Always $\vec{P_\mu}$, but now labeled by helicity, not spin
\end{itemize}

Different number of DoF than massive particles.  
\textbf{Basic, deep and striking feature of relativistic QM}
B/c I cant boost to a frame where the massless particle is at rest.

\section*{Sketch the argument more formally}

Start with translations
\be
\ket{P^\mu, \sigma}
\ee
Label the states with $P^\mu$ and whatever else they are (which we will come to shortly) call $\sigma$.

Under $x^\mu\rightarrow x^\mu + a^\mu$,
\be
U(T(a^\mu)) \ket{P^\mu, \sigma} = e^{iP_\mu a^\mu}\ket{P^\mu, \sigma}
\ee

Now have to talk about how Lorentz transformations act. 
where things get interesting...


\noindent\rule{\textwidth}{1pt}

For ${\Lambda^\mu}_\nu$ there is some unitary operator $U[\Lambda]$ that acts on the states.
Need to know, $U[\Lambda] \ket{P, \sigma}$.
What can this action possible be?

Most naive answer:  $U[\Lambda] \ket{P, \sigma} = \ket{\Lambda P, \sigma}$.

This would give action under ${\Lambda^\mu}_\nu$, but not the most general one.

Most general one,
\be
U[\Lambda] \ket{P, \sigma} = \sum_{\sigma'} D_{\sigma,\sigma'}(\Lambda) \ket{\Lambda P, \sigma'}
\ee
where $D_{\sigma,\sigma'}(\Lambda)$ is a matrix in $\sigma$ space.

What can the $D$'s possibly be?

Non-relativistically, if the particle had spin $D$ would be nothing other than the rotation acting on the spin. (But we dont know that yet...)

\noindent\rule{\textwidth}{1pt}

In order to figure out what the $D$'s are...

Pick some convenient reference momentum $k^\mu$,  eg for massive particles $k^\mu = (m,0,0,0)$.

Any other momentum $\Lambda k$

\be
P^\mu = L^\mu_\nu(p) k^\nu
\ee

Here even the L's are your choice in the sense that there are often multiple transformations that get you from $k \rightarrow P$.

Now \underline{define} what we mean by 
\be
\ket{P, \sigma} \equiv U(L(P)) \ket{k, \sigma}
\ee
Now defined every state in the theory.

Remember we are after $U[\Lambda] \ket{P, \sigma}$

\be
U[\Lambda] \ket{P, \sigma} = U[\Lambda] U(L(P)) \ket{k, \sigma}
\ee

(know the ultimate momentum is $\Lambda P$, again tempted to write as $U[L(\Lambda P)] \ket{k, \sigma}$


\be
 = \left(U[L(\Lambda P)] U[L^{-1}(\Lambda P)] \right) U[\Lambda] U(L(P)) \ket{k, \sigma}
\ee

$U$s for a groups, so $U(g_1)U(g_2) = U(g_1g_2)$

\be
 = U[L(\Lambda P)] \underbrace{U[L^{-1}(\Lambda P) \Lambda L(P)]}_{W(\Lambda, P) \mbox{ for Wigner}} \ket{k, \sigma}
\ee

What does $W(\Lambda,p)$ do to $k$ ?

Takes $k\rightarrow P \rightarrow \Lambda P \rightarrow k$.

So, $W^\mu_\nu(\Lambda P) k^\nu = k^\mu$

Or,

\be
U[W] \ket{k, \sigma} = \sum_{\sigma'} D_{\sigma,\sigma'}\ket{k, \sigma'}
\ee

Now have a simpler problem.

\be
U[\Lambda] \ket{P, \sigma} =  \sum_{\sigma'} D_{\sigma,\sigma'}(W(\Lambda,P)) \ket{\Lambda P, \sigma'}
\ee

Nice and interesting equation.  
Tell us about what acts on the $\sigma$s alone. 
The things that act on the them alone are not general Lorentz transformations. 
The matrices $D$ have to furnish the representations of a different group. 
The $W$s are Lorentz transformations that leave k invariant.

``Little Group''  $W\cdot k= k$

The indices $\sigma$ furnish representations of the little group.  
Much simpler problem. 

\textbf{\underline{For massive particles}}  a nice choice for k = (m,0,0,0). 
What is the little group ? 
Rotations: Leave k invariant. 
$\sigma$ indices have to furnish representations of rotation group. 
Massive particles are labeled by spin. 
Part of their label. 
Don't need to think of deeper origin. 
Particles are just the representations. 

Now lets do the more interesting massless case. 


\textbf{\underline{Massless Case}}  What is the little group here? Harder to visualize. 
One obvious, if the particle is moving along the z-direction and we rotate around the z-axis, leaves the momentum unchanged. 

There are two others,  there should be three...3 for the massive case, this number cannot change discontinuously. 

Lets find the generators,

\be
W^\mu_\nu  = \dmn + \epsilon \wmn
\ee

want $W\cdot k = k \Rightarrow \omega \cdot k = 0 $.

Most general form of $\omega$ we found before,

\be
\omega_{\mu\nu} = \begin{bmatrix} 0 & a & b & c \\ -a & 0 & A & B \\ -b & -A & 0 & C \\ -c & -B & -C & 0 \end{bmatrix}
\ee

we are interested in the special one that annihilate $k = (1,1,0,0)$.
Now,
\be
\omega^\mu_\nu k^\nu = \begin{pmatrix} a \\ a \\ b-A \\ c-B\end{pmatrix} = 0
\ee

Which implies 
\be
\omega_{\mu\nu} = \begin{bmatrix} 0 & 0 & A & B \\ 0 & 0 & A & B \\ A & -A & 0 & C \\ B & -B & -C & 0 \end{bmatrix}
\ee
is the most general $\omega$ for massless particles.

Once again there are 3 generators. 
\begin{itemize}
\item[-] C is rotations about the 1 direction
\item[-] A Boost in y followed by rotation to bring k back.  
\item[-] same for B
\end{itemize}

Define:


\be
J_{23} = \begin{bmatrix} 0 & 0 & 0 & 0 \\ 0 & 0 & 0 & 0 \\ 0 & 0 & 0 & 1 \\ 0 & 0 & -1 & 0 \end{bmatrix} \hspace{0.5in} 
T_{2}  = \begin{bmatrix} 0 & 0 & 1 & 0 \\ 0 & 0 & 1 & 0 \\ 1 & -1 & 0 & 0 \\ 0 & 0 & 0 & 0 \end{bmatrix} \hspace{0.5in} 
T_{3}  = \begin{bmatrix} 0 & 0 & 0 & 1 \\ 0 & 0 & 0 & 1 \\ 0 & 0 & 0 & 0 \\ 1 & -1 & 0 & 0 \end{bmatrix}
\ee

Exercise, work out the commutation relations between these guys. 

What you'll find: 

\be
[T_2,T_3] = 0 \hspace{0.5in} [T_2,J_{23}] = T_3 \hspace{0.5in} [T_3,J_{23}] = -T_2 \hspace{0.5in} 
\ee
This is exactly what you expect for translations + rotations in a plane. This group has a name,  $E(2)$.

How does this group act on particles ?
Need to come up with representations.  $T_2$ and $T_3$ commute so we could label the states by their eigenvalues $\ket{t_2,t_3}$.

Small problem, if $\ket{t_2,t_3}$ is a state and I act on it with J:

\be
e^{i\Theta J_23} \ket{t_2,t_3} = \ket{t_2',t_3'} 
\ee
I get another state, for each value of $\Theta$ I chosen.

In other words, if one state in this rep has non-zero $t_2'$ or $t_3'$, then there are a continuous infinity of other states!
We dont see a continuous infinite number of massless particles. 

We we must simply declare that we are interested in states with $t_1, t_2 = 0$

So, no what do I label these states by?
Only thing left is eigenvalue $J_{23}$.

Massless particles only labeled by the spin in the direction of motion. 
This is referred to as helicity. 

{\Large \underline{\textbf{Summary}}}

\begin{tabular}{l}
\underline{Massive:}\\
\\
\hspace{0.5in}$\ket{P, \sigma}$ and $U[\Lambda] \ket{P, \sigma} = \sum_{\sigma'} R_{\sigma,\sigma'} \ket{\Lambda P, \sigma'}$, where the R is a rotation matrix.\\
\\
\underline{Massless:}\\
\\
\hspace{0.5in}$\ket{P, h}$ and $U[\Lambda] \ket{P, h} = e^{ih\Theta(W)}\ket{\Lambda P, h}$, where the coefficient is just a phase.\\
\end{tabular}

}
\end{document}


