\input{latexSetup}
\usepackage{braket}

\def\ketY{\ensuremath{\ket {\Psi}}}
\def\iGeV{\ensuremath{\textrm{GeV}^{-1}}}
\def\mp{\ensuremath{m_{\textrm{proton}}}}
\def\rp{\ensuremath{r_{\textrm{proton}}}}
\def\me{\ensuremath{m_{\textrm{electron}}}}
\def\aG{\ensuremath{\alpha_G}}
\def\rAtom{\ensuremath{r_{\textrm{atom}}}}
\def\rNucl{\ensuremath{r_{\textrm{nucleus}}}}
\def\GN{\ensuremath{\textrm{G}_\textrm{N}}}

\def\be{\begin{equation*}}
\def\ee{\end{equation*}}


\usepackage{fancyhdr}
\usepackage{cancel}




\fancyhf{}
\lhead{\Large 33-444} % \hfill Introduction to Particle Physics \hfill Spring 2020}
\chead{\Large Introduction to Particle Physics} % \hfill Spring 2020}
\rhead{\Large Spring 2020} % \hfill Introduction to Particle Physics \hfill Spring 2020}
\begin{document}
\thispagestyle{fancy}





%\begin{tabular}{c}
%{\large 33-444 \hfill Intro To Particle \hfill Spring 2020\\}
%\hline 
%\end{tabular}

\begin{center}
{\huge \textbf{Homework Set \#4}}
\large

{\textbf{ Due Date:} Before class Friday February 15th  } 
\end{center}

{\large
\textbf{1) Find the generators of the ``Little Group'' for Massive particles \hfill \textit{(5 points)}}


\vspace*{0.25in}

\textbf{2) Heisenberg Equation of Motion \hfill \textit{(5 points)}}
\begin{itemize}
\item[a)] { Show that $\frac{dA(t)_H}{dt} = -i [A(t)_H, H]$.  Where $A(t)_H$ is an operator in the Heisenberg representation and H is the Hamiltonian. This equation is referred to as the Heisenberg equation of motion.}
\item[b)] { Show that $\phi_H(x,t) = e^{-iE_pt} \phi_S(x) $ satisfies the Heisenberg equation of motion. Where $\phi_S(x)= \int \cancel{dp} e^{i\vec{p}\cdot\vec{x}} a^\dagger $ is the operator in the Schrodinger representation and $H = \int d^3p E_p a^\dagger a$ }
\end{itemize}

\vspace*{0.25in}

\textbf{3) Show that $\int \cancel{d^3p} \equiv \int \frac{d^3\vec{p}}{2E_p}$ is Lorentz invariant. \hfill \textit{(2 points)}}

\textit{ (Hint: $\int d^4p\ \delta(E^2 - (|\vec{p}|^2 + m^2))$ is clearly Lorentz invariant.)}

\vspace*{0.25in}

\textbf{4) Anti-Particles  \hfill \textit{(5 points)}}
\begin{itemize}
\item[a)] {Expand ${\Phi^\dagger}^2 \Phi^2$  in terms of $a$, $a^\dagger$, $b$, and $b^\dagger$  (Ignore the exponentials and integrals)}
\item[b)] {Sketch diagrams of the processes that each term corresponds to.}
\item[c)] {Let the charge ($Q$) of particle $a$ be $q_a$ and the charge of particle $b$ be $q_b$.  Calculate $\Delta Q$ for each process. }
\item[d)] {What happens if you take $q_a = -q_b$ ?}
\end{itemize}

%Sketch diagrams of the partile content of |psi(-inf)> to |psi(+inf)> for each term. like we did in class for a^\dagger a^\dagger a a = 

%LEt the charge of a be +q and the charge of b be -q, what is Delta(Q) in each of the interactions in b) ?



}

\end{document}
