\input{latexSetup}
\usepackage{braket}

\def\ketY{\ensuremath{\ket {\Psi}}}
\def\iGeV{\ensuremath{\textrm{GeV}^{-1}}}
\def\mp{\ensuremath{m_{\textrm{proton}}}}
\def\rp{\ensuremath{r_{\textrm{proton}}}}
\def\me{\ensuremath{m_{\textrm{electron}}}}
\def\aG{\ensuremath{\alpha_G}}
\def\rAtom{\ensuremath{r_{\textrm{atom}}}}
\def\rNucl{\ensuremath{r_{\textrm{nucleus}}}}
\def\GN{\ensuremath{\textrm{G}_\textrm{N}}}

\def\be{\begin{equation*}}
\def\ee{\end{equation*}}


\usepackage{fancyhdr}
\usepackage{cancel}
\usepackage{ mathrsfs }





\fancyhf{}
\lhead{\Large 33-444} % \hfill Introduction to Particle Physics \hfill Spring 2020}
\chead{\Large Introduction to Particle Physics} % \hfill Spring 2020}
\rhead{\Large Spring 2020} % \hfill Introduction to Particle Physics \hfill Spring 2020}
\begin{document}
\thispagestyle{fancy}





%\begin{tabular}{c}
%{\large 33-444 \hfill Intro To Particle \hfill Spring 2020\\}
%\hline 
%\end{tabular}

\begin{center}
{\huge \textbf{Homework Set \#8 }}
\large

{\textbf{ Due Date:} Before class Friday March 29th  } 
\end{center}

{\large

\textbf{1) Muon decays: } \hfill \textit{(10 points)}\\
\begin{itemize}
  \item[a)]{ The muon decays via the weak interaction,  At low energy ($E << m_W$), this can be approximated as a point-like interaction. 
  Draw the diagram describing muon decay to an electron assuming a point-like weak interaction.
}
  \item[b)]{ What are the dimensions of the coupling constant, associated to this diagram  ?
  }
  \item[c)] How does the decay rate $\Gamma$ (decays/unit time)  depend on the muon mass ? 
  \item[d)]{ The muon has a mass of $\sim$0.1 GeV and a lifetime of $\sim 1 \mu s$. The tau lepton has a mass of {$\sim$1 GeV}. Estimate the lifetime of the tau lepton in $\mu s$.
}
  \item[e)] {Suppose that the photon could couple at the same vertex to the muon and the electron. Then the muon could decay as $\mu\rightarrow e \gamma$. 
  Estimate the ratio of the $\mu$ lifetime in this world to that in our world without this interaction.
  }
\end{itemize}



\vspace*{0.25in}

\textbf{2) A new force. } \hfill \textit{(5 points)}\\
Assume there is another force of nature felt by electrons associated with the exchange of a new X boson of mass 1 TeV (= 1000 GeV).
\begin{itemize}
\item[a)]{ Estimate an upper limit on the range of this new force in meters.
}
%\item[b)]{ Assume muons also interact with this new boson. How could you determine the spin of X ?}
\item[b)]{ Assume that this new X boson could also decay to spin-1/2 dark matter particles $\psi_{DM}$. At low energies (<< 1 TeV) the X interaction can be described by a point-like interaction. Estimate the coupling constant associated to dark matter scattering $e \psi_{DM} \rightarrow e \psi_{DM}$.   (Assume the X coupling at high energies is the same as for EM)  }
\item[c)]{ Assume there was a direct $X\rightarrow e\mu$ interaction. This would allow the muon to decay via $\mu^- \rightarrow e^-e^+e^-$. Draw the corresponding diagram and estimate the impact of the muon lifetime from this process.  How does it compare to the lifetime in the standard model?     }
\end{itemize}

\vspace*{0.25in}

\textbf{3)  $W$ boson decays to electrons. } \hfill \textit{(3 points)}\\
\begin{itemize}
\item[a)]{ The W can decay directly to an electron or to electron by decaying through a $\tau$. Draw the corresponding diagrams.}
\item[a)]{ Estimate how often the W decays to an electron. }
\end{itemize}           

\vspace*{0.25in}

\textbf{4) Higgs Boson Decays. } \hfill \textit{(3 points)}\\
The Higgs boson decays at a high rate to pairs of W-bosons.
A clean way to observe this signal is in the $e\mu$ + Missing transverse energy final state. How often do $H\rightarrow WW$ events lead to a $e\mu$ final state?

\vspace*{0.25in}

\textbf{5) Galactic Collisions.} \hfill \textit{(10 points)}\\
The Milky Way and Andromeda (M31) galaxies will collide in about 4 billion years. 
Both galaxies are spiral-type and contain about 100 billion stars each, which is comparable to the number of protons in a bunch at the LHC. 
\begin{itemize}
\item[a)] {The two galaxies are similar in size and disk-shaped, with a diameter of about $10^5$ light-years and a thickness of about $10^3$ light-years. 
If the relative velocity of the Milky Way and Andromeda galaxies is about $10^5$ m/s, estimate the flux factor (luminosity) at collision. 
How does this compare to the luminosity of proton collisions at the LHC?
}
\item[b)] {What is the average distance between two stars in these galaxies? 
Assuming that an average star is about the size of our Sun (radius $7 \times 10^8$ m), what is the ratio of the average distance to star radius? 
Correspondingly, what is the ratio between the average distance between protons in a bunch at the LHC to the proton radius? 
}
\item[c)] {If the process of collision of the Milky Way and Andromeda galaxies takes about a billion years, approximately how many individual stars will collide? 
Is this similar to the number of proton collisions per bunch crossing at the LHC?}
\end{itemize}


}





\end{document}
