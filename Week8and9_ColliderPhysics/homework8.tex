\documentclass[paper=letter,11pt]{scrartcl}

\KOMAoptions{headinclude=true, footinclude=false}
\KOMAoptions{DIV=14, BCOR=5mm}
\KOMAoptions{numbers=noendperiod}
\KOMAoptions{parskip=half}
\addtokomafont{disposition}{\rmfamily}
\addtokomafont{part}{\LARGE}
\addtokomafont{descriptionlabel}{\rmfamily}
%\setkomafont{pageheadfoot}{\normalsize\sffamily}
\setkomafont{pagehead}{\normalsize\rmfamily}
%\setkomafont{publishers}{\normalsize\rmfamily}
\setkomafont{caption}{\normalfont\small}
\setcapindent{0pt}
\deffootnote[1em]{1em}{1em}{\textsuperscript{\thefootnotemark}\ }


\usepackage{amsmath}
\usepackage[varg]{txfonts}
\usepackage[T1]{fontenc}
\usepackage{graphicx}
\usepackage{xcolor}
\usepackage[american]{babel}
% hyperref is needed in many places, so include it here
\usepackage{hyperref}

\usepackage{xspace}
\usepackage{multirow}
\usepackage{float}


\usepackage{braket}
\usepackage{bbm}
\usepackage{relsize}
\usepackage{tcolorbox}

\def\ketY{\ensuremath{\ket {\Psi}}}
\def\iGeV{\ensuremath{\textrm{GeV}^{-1}}}
%\def\mp{\ensuremath{m_{\textrm{proton}}}}
\def\rp{\ensuremath{r_{\textrm{proton}}}}
\def\me{\ensuremath{m_{\textrm{electron}}}}
\def\aG{\ensuremath{\alpha_G}}
\def\rAtom{\ensuremath{r_{\textrm{atom}}}}
\def\rNucl{\ensuremath{r_{\textrm{nucleus}}}}
\def\GN{\ensuremath{\textrm{G}_\textrm{N}}}
\def\ketX{\ensuremath{\ket{\vec{x}}}}
\def\ve{\ensuremath{\vec{\epsilon}}}


\def\ABCDMatrix{\ensuremath{\begin{pmatrix} A &  B  \\ C  & D \end{pmatrix}}}
\def\xyprime{\ensuremath{\begin{pmatrix} x' \\ y' \end{pmatrix}}}
\def\xyprimeT{\ensuremath{\begin{pmatrix} x' &  y' \end{pmatrix}}}
\def\xy{\ensuremath{\begin{pmatrix} x \\ y \end{pmatrix}}}
\def\xyT{\ensuremath{\begin{pmatrix} x & y \end{pmatrix}}}

\def\IMatrix{\ensuremath{\begin{pmatrix} 0 &  1  \\ -1  & 0 \end{pmatrix}}}
\def\IBoostMatrix{\ensuremath{\begin{pmatrix} 0 &  1  \\ 1  & 0 \end{pmatrix}}}
\def\JThree{\ensuremath{\begin{pmatrix}    0 & -i & 0  \\ i & 0  & 0 \\ 0 & 0 & 0 \end{pmatrix}}} 
\def\JTwo{\ensuremath{\begin{bmatrix}    0 & 0 & -i  \\ 0 & 0  & 0 \\ i & 0 & 0 \end{bmatrix}}}
\def\JOne{\ensuremath{\begin{bmatrix}    0 & 0 & 0  \\ 0 & 0  & -i \\ 0 & i & 0 \end{bmatrix}}}
\def\etamn{\ensuremath{\eta_{\mu\nu}}}
\def\Lmn{\ensuremath{\Lambda^\mu_\nu}}
\def\dmn{\ensuremath{\delta^\mu_\nu}}
\def\wmn{\ensuremath{\omega^\mu_\nu}}
\def\be{\begin{equation*}}
\def\ee{\end{equation*}}
\def\bea{\begin{eqnarray*}}
\def\eea{\end{eqnarray*}}
\def\bi{\begin{itemize}}
\def\ei{\end{itemize}}
\def\fmn{\ensuremath{F_{\mu\nu}}}
\def\fMN{\ensuremath{F^{\mu\nu}}}
\def\bc{\begin{center}}
\def\ec{\end{center}}
\def\nus{$\nu$s}

\def\adagger{\ensuremath{a_{p\sigma}^\dagger}}
\def\lineacross{\noindent\rule{\textwidth}{1pt}}

\newcommand{\multiline}[1] {
\begin{tabular} {|l}
#1
\end{tabular}
}

\newcommand{\multilineNoLine}[1] {
\begin{tabular} {l}
#1
\end{tabular}
}



\newcommand{\lineTwo}[2] {
\begin{tabular} {|l}
#1 \\
#2
\end{tabular}
}

\newcommand{\rmt}[1] {
\textrm{#1}
}


%
% Units
%
\def\m{\ensuremath{\rmt{m}}}
\def\GeV{\ensuremath{\rmt{GeV}}}
\def\pt{\ensuremath{p_\rmt{T}}}


\def\parity{\ensuremath{\mathcal{P}}}

\usepackage{cancel}
\usepackage{ mathrsfs }
\def\bigL{\ensuremath{\mathscr{L}}}

\usepackage{ dsfont }



\usepackage{fancyhdr}
\fancyhf{}

\usepackage{braket}

\def\ketY{\ensuremath{\ket {\Psi}}}
\def\iGeV{\ensuremath{\textrm{GeV}^{-1}}}
\def\mp{\ensuremath{m_{\textrm{proton}}}}
\def\rp{\ensuremath{r_{\textrm{proton}}}}
\def\me{\ensuremath{m_{\textrm{electron}}}}
\def\aG{\ensuremath{\alpha_G}}
\def\rAtom{\ensuremath{r_{\textrm{atom}}}}
\def\rNucl{\ensuremath{r_{\textrm{nucleus}}}}
\def\GN{\ensuremath{\textrm{G}_\textrm{N}}}

\def\be{\begin{equation*}}
\def\ee{\end{equation*}}


\usepackage{fancyhdr}
\usepackage{cancel}
\usepackage{ mathrsfs }





\fancyhf{}
\lhead{\Large 33-444} % \hfill Introduction to Particle Physics \hfill Spring 2020}
\chead{\Large Introduction to Particle Physics} % \hfill Spring 2020}
\rhead{\Large Spring 2020} % \hfill Introduction to Particle Physics \hfill Spring 2020}
\begin{document}
\thispagestyle{fancy}





%\begin{tabular}{c}
%{\large 33-444 \hfill Intro To Particle \hfill Spring 2020\\}
%\hline 
%\end{tabular}

\begin{center}
{\huge \textbf{Homework Set \#8 }}
\large

{\textbf{ Due Date:} Friday March 27th  } 
\end{center}

{\large

\textbf{1) Z boson decays: } \hfill \textit{(5 points)}\\
In class we estimated the branching ratio of $Z\rightarrow ee$ decays and $Z\rightarrow bb$ decays.  
Repeat these calculations.
What approximations where made ?
Compare this rough estimate to the experimentally measured values using the Particle Data Group's (PDG) Review of Particle Physics. 
You can find it at http://pdg.lbl.gov.
All measured properties of all known particles are recorded here.\\




\textbf{2) Muon decays: } \hfill \textit{(10 points)}\\
\begin{itemize}
  \item[a)]{ The muon decays via the weak interaction,  At low energy ($E << m_W$), this can be approximated as a point-like interaction. 
  Draw the diagram describing muon decay to an electron assuming a point-like weak interaction.
}
  \item[b)]{ What are the dimensions of the coupling constant, associated to this diagram  ?
  }
  \item[c)] How does the decay rate $\Gamma$ (decays/unit time)  depend on the muon mass ? 
  \item[d)]{ The muon has a mass of $\sim$0.1 GeV and a lifetime of $\sim 1 \mu s$. The tau lepton has a mass of {$\sim$1 GeV}. Estimate the lifetime of the tau lepton in $\mu s$.
}
  \item[e)] {Suppose that the photon could couple at the same vertex to the muon and the electron. Then the muon could decay as $\mu\rightarrow e \gamma$. 
  Estimate the ratio of the $\mu$ lifetime in this world to that in our world without this interaction.
  }
\end{itemize}



\vspace*{0.25in}

\textbf{3) A new force. } \hfill \textit{(5 points)}\\
Assume there is another force of nature felt by electrons associated with the exchange of a new X boson of mass 1 TeV (= 1000 GeV).
\begin{itemize}
\item[a)]{ Estimate an upper limit on the range of this new force in meters.
}
%\item[b)]{ Assume muons also interact with this new boson. How could you determine the spin of X ?}
\item[b)]{ Assume that this new X boson could also decay to spin-1/2 dark matter particles $\psi_{DM}$. At low energies (<< 1 TeV) the X interaction can be described by a point-like interaction. Estimate the coupling constant associated to dark matter scattering $e \psi_{DM} \rightarrow e \psi_{DM}$.   (Assume the X coupling at high energies is the same as for EM)  }
\item[c)]{ Assume there was a direct $X\rightarrow e\mu$ interaction. This would allow the muon to decay via $\mu^- \rightarrow e^-e^+e^-$. Draw the corresponding diagram and estimate the impact of the muon lifetime from this process.  How does it compare to the lifetime in the standard model?     }
\end{itemize}

\clearpage

\textbf{4)  $W$ boson decays to electrons. } \hfill \textit{(3 points)}\\
\begin{itemize}
\item[a)]{ The W can decay directly to an electron or to electron by decaying through a $\tau$. Draw the corresponding diagrams.}
\item[b)]{ Estimate how often the W decays to an electron. }
\end{itemize}           

\vspace*{0.25in}

\textbf{5) Higgs Boson Decays. } \hfill \textit{(3 points)}\\
The Higgs boson decays at a high rate to pairs of W-bosons.
A clean way to observe this signal is in the $e\mu$ + Missing transverse energy final state. How often do $H\rightarrow WW$ events lead to a $e\mu$ final state?

\vspace*{0.25in}

\textbf{6) Galactic Collisions.} \hfill \textit{(10 points)}\\
The Milky Way and Andromeda (M31) galaxies will collide in about 4 billion years. 
Both galaxies are spiral-type and contain about 100 billion stars each, which is comparable to the number of protons in a bunch at the LHC. 
\begin{itemize}
\item[a)] {The two galaxies are similar in size and disk-shaped, with a diameter of about $10^5$ light-years and a thickness of about $10^3$ light-years. 
If the relative velocity of the Milky Way and Andromeda galaxies is about $10^5$ m/s, estimate the flux factor (luminosity) at collision. 
How does this compare to the luminosity of proton collisions at the LHC?
}
\item[b)] {What is the average distance between two stars in these galaxies? 
Assuming that an average star is about the size of our Sun (radius $7 \times 10^8$ m), what is the ratio of the average distance to star radius? 
Correspondingly, what is the ratio between the average distance between protons in a bunch at the LHC to the proton radius? 
}
\item[c)] {If the process of collision of the Milky Way and Andromeda galaxies takes about a billion years, approximately how many individual stars will collide? 
Is this similar to the number of proton collisions per bunch crossing at the LHC?}
\end{itemize}


}





\end{document}
