\input{latexSetup}
%\documentclass[margin,line]{res}
\usepackage{braket}
\usepackage{bbm}
\usepackage{relsize}

\def\ketY{\ensuremath{\ket {\Psi}}}
\def\iGeV{\ensuremath{\textrm{GeV}^{-1}}}
\def\mp{\ensuremath{m_{\textrm{proton}}}}
\def\rp{\ensuremath{r_{\textrm{proton}}}}
\def\me{\ensuremath{m_{\textrm{electron}}}}
\def\aG{\ensuremath{\alpha_G}}
\def\rAtom{\ensuremath{r_{\textrm{atom}}}}
\def\rNucl{\ensuremath{r_{\textrm{nucleus}}}}
\def\GN{\ensuremath{\textrm{G}_\textrm{N}}}

\def\ABCDMatrix{\ensuremath{\begin{pmatrix} A &  B  \\ C  & D \end{pmatrix}}}
\def\xyprime{\ensuremath{\begin{pmatrix} x' \\ y' \end{pmatrix}}}
\def\xyprimeT{\ensuremath{\begin{pmatrix} x' &  y' \end{pmatrix}}}
\def\xy{\ensuremath{\begin{pmatrix} x \\ y \end{pmatrix}}}
\def\xyT{\ensuremath{\begin{pmatrix} x & y \end{pmatrix}}}

\def\IMatrix{\ensuremath{\begin{pmatrix} 0 &  1  \\ -1  & 0 \end{pmatrix}}}
\def\JThree{\ensuremath{\begin{pmatrix}    0 & -i & 0  \\ i & 0  & 0 \\ 0 & 0 & 0 \end{pmatrix}}} 
\def\JTwo{\ensuremath{\begin{bmatrix}    0 & 0 & -i  \\ 0 & 0  & 0 \\ i & 0 & 0 \end{bmatrix}}}
\def\JOne{\ensuremath{\begin{bmatrix}    0 & 0 & 0  \\ 0 & 0  & -i \\ 0 & i & 0 \end{bmatrix}}}

\usepackage{fancyhdr}

\fancyhf{}
\lhead{\Large 33-444} % \hfill Introduction to Particle Physics \hfill Spring 2020}
\chead{\Large Introduction to Particle Physics} % \hfill Spring 2020}
\rhead{\Large Spring 2020} % \hfill Introduction to Particle Physics \hfill Spring 2020}

\begin{document}
\thispagestyle{fancy}

\begin{center}
{\huge \textbf{Lecture 21}}
\end{center}

{\fontsize{14}{16}\selectfont

\section*{Collider Physics}

We will start talking  about collider physics. 
Start with a cartoon of what/how we measure/detect particles.
This will motivate certian calculations.  
We will then return and fill in detail later.


\underline{To first order}  (we will come back and make this more precise next week)
Particles (either protons or electrons) collide along the z-axis and a whole bunch of other particles shoot out in all directions.   

We build detectors (which you can think of as large cameras) to take pictures (snapshots) of what came out. 
These pictures are called ``events''.
An example of a picture or event from the ATLAS detector at the LHC is shown in Figure~\ref{fig:EventDisplay}.


\begin{figure}[h]
\centering
\includegraphics[width=1.0\textwidth]{./EventDisplayNoLabel.pdf}
\caption{Event Display}\label{fig:EventDisplay}
\end{figure}

\clearpage

There are four basic types of images that the detectors capture. 
You can think of these as the basic outputs of the detectors. 
(Of course this is not the true output of the detectors which really just measure voltage or charge, but it is a conveient abstraction and its the level at which most experimental HEP physicists think)

This basic outputs are shown in Figure~\ref{fig:BasicInputs}.
Correlations among these four basic image types, tell us what kind of particles where produced in the collision.

\begin{figure}[h]
\centering
\includegraphics[width=1.0\textwidth]{./BasicInputs.pdf}
\caption{Basic Outputs}\label{fig:BasicInputs}
\end{figure}

\clearpage

Of course we are really interested in the particles,  not the different image types. 
But by correlation these images we can infer which particles were present. 

How this is done for the fermions is shown in Figure~\ref{fig:Fermions}.
The detector signature of the Bosons is shown in Figure~\ref{fig:Bosons}.
\begin{figure}[h]
\centering
\includegraphics[width=1.0\textwidth]{./Fermions.pdf}
\caption{Signature of Fermions}\label{fig:Fermions}
\end{figure}

\begin{figure}[h]
\centering
\includegraphics[width=1.0\textwidth]{./Bosons.pdf}
\caption{Signature of Bosons}\label{fig:Bosons}
\end{figure}

\begin{figure}[h]
\centering
\includegraphics[width=1.0\textwidth]{./HiggsBoson.pdf}
\caption{Signature of the Higgsv Boson}\label{fig:HiggsBosons}
\end{figure}

\begin{figure}[h]
\centering
\includegraphics[width=1.0\textwidth]{./EventDisplayLabeled.pdf}
\end{figure}

}
\end{document}


