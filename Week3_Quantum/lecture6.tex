\documentclass[paper=letter,11pt]{scrartcl}

\KOMAoptions{headinclude=true, footinclude=false}
\KOMAoptions{DIV=14, BCOR=5mm}
\KOMAoptions{numbers=noendperiod}
\KOMAoptions{parskip=half}
\addtokomafont{disposition}{\rmfamily}
\addtokomafont{part}{\LARGE}
\addtokomafont{descriptionlabel}{\rmfamily}
%\setkomafont{pageheadfoot}{\normalsize\sffamily}
\setkomafont{pagehead}{\normalsize\rmfamily}
%\setkomafont{publishers}{\normalsize\rmfamily}
\setkomafont{caption}{\normalfont\small}
\setcapindent{0pt}
\deffootnote[1em]{1em}{1em}{\textsuperscript{\thefootnotemark}\ }


\usepackage{amsmath}
\usepackage[varg]{txfonts}
\usepackage[T1]{fontenc}
\usepackage{graphicx}
\usepackage{xcolor}
\usepackage[american]{babel}
% hyperref is needed in many places, so include it here
\usepackage{hyperref}

\usepackage{xspace}
\usepackage{multirow}
\usepackage{float}


\usepackage{braket}
\usepackage{bbm}
\usepackage{relsize}
\usepackage{tcolorbox}

\def\ketY{\ensuremath{\ket {\Psi}}}
\def\iGeV{\ensuremath{\textrm{GeV}^{-1}}}
%\def\mp{\ensuremath{m_{\textrm{proton}}}}
\def\rp{\ensuremath{r_{\textrm{proton}}}}
\def\me{\ensuremath{m_{\textrm{electron}}}}
\def\aG{\ensuremath{\alpha_G}}
\def\rAtom{\ensuremath{r_{\textrm{atom}}}}
\def\rNucl{\ensuremath{r_{\textrm{nucleus}}}}
\def\GN{\ensuremath{\textrm{G}_\textrm{N}}}
\def\ketX{\ensuremath{\ket{\vec{x}}}}
\def\ve{\ensuremath{\vec{\epsilon}}}


\def\ABCDMatrix{\ensuremath{\begin{pmatrix} A &  B  \\ C  & D \end{pmatrix}}}
\def\xyprime{\ensuremath{\begin{pmatrix} x' \\ y' \end{pmatrix}}}
\def\xyprimeT{\ensuremath{\begin{pmatrix} x' &  y' \end{pmatrix}}}
\def\xy{\ensuremath{\begin{pmatrix} x \\ y \end{pmatrix}}}
\def\xyT{\ensuremath{\begin{pmatrix} x & y \end{pmatrix}}}

\def\IMatrix{\ensuremath{\begin{pmatrix} 0 &  1  \\ -1  & 0 \end{pmatrix}}}
\def\IBoostMatrix{\ensuremath{\begin{pmatrix} 0 &  1  \\ 1  & 0 \end{pmatrix}}}
\def\JThree{\ensuremath{\begin{pmatrix}    0 & -i & 0  \\ i & 0  & 0 \\ 0 & 0 & 0 \end{pmatrix}}} 
\def\JTwo{\ensuremath{\begin{bmatrix}    0 & 0 & -i  \\ 0 & 0  & 0 \\ i & 0 & 0 \end{bmatrix}}}
\def\JOne{\ensuremath{\begin{bmatrix}    0 & 0 & 0  \\ 0 & 0  & -i \\ 0 & i & 0 \end{bmatrix}}}
\def\etamn{\ensuremath{\eta_{\mu\nu}}}
\def\Lmn{\ensuremath{\Lambda^\mu_\nu}}
\def\dmn{\ensuremath{\delta^\mu_\nu}}
\def\wmn{\ensuremath{\omega^\mu_\nu}}
\def\be{\begin{equation*}}
\def\ee{\end{equation*}}
\def\bea{\begin{eqnarray*}}
\def\eea{\end{eqnarray*}}
\def\bi{\begin{itemize}}
\def\ei{\end{itemize}}
\def\fmn{\ensuremath{F_{\mu\nu}}}
\def\fMN{\ensuremath{F^{\mu\nu}}}
\def\bc{\begin{center}}
\def\ec{\end{center}}
\def\nus{$\nu$s}

\def\adagger{\ensuremath{a_{p\sigma}^\dagger}}
\def\lineacross{\noindent\rule{\textwidth}{1pt}}

\newcommand{\multiline}[1] {
\begin{tabular} {|l}
#1
\end{tabular}
}

\newcommand{\multilineNoLine}[1] {
\begin{tabular} {l}
#1
\end{tabular}
}



\newcommand{\lineTwo}[2] {
\begin{tabular} {|l}
#1 \\
#2
\end{tabular}
}

\newcommand{\rmt}[1] {
\textrm{#1}
}


%
% Units
%
\def\m{\ensuremath{\rmt{m}}}
\def\GeV{\ensuremath{\rmt{GeV}}}
\def\pt{\ensuremath{p_\rmt{T}}}


\def\parity{\ensuremath{\mathcal{P}}}

\usepackage{cancel}
\usepackage{ mathrsfs }
\def\bigL{\ensuremath{\mathscr{L}}}

\usepackage{ dsfont }



\usepackage{fancyhdr}
\fancyhf{}

%\documentclass[margin,line]{res}
\usepackage{braket}
\usepackage{bbm}
\usepackage{relsize}

\def\ketX{\ensuremath{\ket{\vec{x}}}}
%\def\ketY{\ensuremath{\ket {\Psi}}}
\def\iGeV{\ensuremath{\textrm{GeV}^{-1}}}


\def\ABCDMatrix{\ensuremath{\begin{pmatrix} A &  B  \\ C  & D \end{pmatrix}}}
\def\xyprime{\ensuremath{\begin{pmatrix} x' \\ y' \end{pmatrix}}}
\def\xyprimeT{\ensuremath{\begin{pmatrix} x' &  y' \end{pmatrix}}}
\def\xy{\ensuremath{\begin{pmatrix} x \\ y \end{pmatrix}}}
\def\xyT{\ensuremath{\begin{pmatrix} x & y \end{pmatrix}}}

\def\IMatrix{\ensuremath{\begin{pmatrix} 0 &  1  \\ -1  & 0 \end{pmatrix}}}
\def\IBoostMatrix{\ensuremath{\begin{pmatrix} 0 &  1  \\ 1  & 0 \end{pmatrix}}}
\def\JThree{\ensuremath{\begin{pmatrix}    0 & -i & 0  \\ i & 0  & 0 \\ 0 & 0 & 0 \end{pmatrix}}} 
\def\JTwo{\ensuremath{\begin{bmatrix}    0 & 0 & -i  \\ 0 & 0  & 0 \\ i & 0 & 0 \end{bmatrix}}}
\def\JOne{\ensuremath{\begin{bmatrix}    0 & 0 & 0  \\ 0 & 0  & -i \\ 0 & i & 0 \end{bmatrix}}}
\def\etamn{\ensuremath{\eta_{\mu\nu}}}
\def\Lmn{\ensuremath{\Lambda^\mu_\nu}}
\def\dmn{\ensuremath{\delta^\mu_\nu}}
\def\wmn{\ensuremath{\omega^\mu_\nu}}
\def\be{\begin{equation*}}
\def\ee{\end{equation*}}
\def\bea{\begin{eqnarray*}}
\def\eea{\end{eqnarray*}}

%\def\xMu{\ensuremath{x^\mu}

\usepackage{fancyhdr}

\fancyhf{}
\lhead{\Large 33-444} % \hfill Introduction to Particle Physics \hfill Spring 2019}
\chead{\Large Introduction to Particle Physics} % \hfill Spring 2019}
\rhead{\Large Spring 2019} % \hfill Introduction to Particle Physics \hfill Spring 2019}

\begin{document}
\thispagestyle{fancy}

\begin{center}
{\huge \textbf{Lecture 6}}
\end{center}

{\fontsize{14}{16}\selectfont

\textbf{\underline{Review Quantum Mechanics}} 

\underline{QM} Linear Algebra in a complex vector space.

State of a system is a vector (ray) in the Complex Vector Space. 

\begin{center}
$\ket \alpha$  - state vector
\end{center}


\textbf{\underline{Linear Superpositions} }

\be
\ket \psi = c_1 \ket{\alpha_1} + c_2 \ket{\alpha_2}
\ee

If the $\ket{\alpha's}$ are vectors and the c's are complex numbers, then $\ket \psi$ is another vector in the space.\\


\textbf{\underline{Dual Space:}}

For every vector $\ket \alpha$ (``ket'') there is another vector $\bra \alpha$ (``bra'') in a ``dual'' space.

Dual space is a mirro image  of the ket space.


If, $\ket \psi = c_1 \ket{\alpha_1} + c_2 \ket{\alpha_2}$, then $\bra \psi = c_1^* \bra{\alpha_1} + c_2^* \bra{\alpha_2}$
where the $c^*$ is the complex conjugate.

\noindent\rule{\textwidth}{1pt}

\textbf{\underline{Inner Product:}} Dual space allows us to define an inner product between to vectors. 

Given 2 vectors can get a ``c\#'' 

$\braket{\alpha|\beta}$

\underline{Properties}
\begin{itemize}
\item[1.] $\braket{\beta|\alpha} = \braket{\alpha| \beta}^*$
\item[2.] $\braket{\alpha|\alpha} \geq 0$
\item[3.] $\bra{\beta} (c_1 \ket{\alpha_1} + c_2 \ket{\alpha_2} ) =  c_1 \braket{\beta|\alpha_1} + c_2 \braket{\beta| \alpha_2}$
\end{itemize}

$\ket{\alpha}$ and $\ket{\beta}$ are \underline{orthogonal} if $\braket{\alpha|\beta} = 0$.

States can be \underline{normalized} $\braket{\alpha|\alpha} = 1$.

\noindent\rule{\textwidth}{1pt}

\textbf{\underline{Operators:}} Thigs that act on a given state, and return another state

$X\ket{\alpha} = \ket{\alpha'}$

Product:

$(YX)\ket{\alpha} = Y(X\ket{\alpha}$

In general, not commutative:  $XY \neq YX$, but they are associative.


$\bra{\alpha} = \bra{\alpha'}X^\dagger$

In general, $X^\dagger \neq X$,  if so $X$ is said to be ``Hermitian''.

\noindent\rule{\textwidth}{1pt}

System characheterized by single observable A

eg: position / Momentum / energy

Measuremnet of A gives possible values

\be
a_1, a_2, ...
\ee

$\ket{\alpha} = $  State for which A has a value of a

\be
\sum\limits_{a'} \ket{\alpha'}\bra{\alpha'} = 1 
\ee

\be
\braket{\alpha|\alpha'} = \delta_{aa'}
\ee

Any physical observable corrisponds to an operator like

\be
A = \sum\limits_{a'} a' \ket{\alpha'}\bra{\alpha'}  
\ee

eg: 


\be
A\ket{\alpha} = \left( \sum\limits_{a'} a' \ket{\alpha'}\bra{\alpha'} \right)\ket{\alpha} = a \ket{\alpha}
\ee

Physical observables are real numbers, therefore physical operators A are \underline{Hermitian}

proof:

\be
A^\dagger = \sum\limits_{a} a^* \ket{\alpha}\bra{\alpha} = \sum\limits_{a} a \ket{\alpha}\bra{\alpha} = A
\ee

\noindent\rule{\textwidth}{1pt}

\textbf{\underline{Probabilities:}}

Condsider a filter $M(a) = \ket{a}\bra{a}$ on a general state $\ket{s}$.

\bea
M(a)\ket{s} =& \ket{a}\braket{a|s}\\
            =& \braket{a|s}\ket{a}
\eea

where $\braket{a|s}$ is a c\# that tells you something about what fraction of the time you get through. 


$\braket{a|s}$ is related to the pass fraction

*But, a) not real,  b) not normalized

However, we know that $|\braket{a|s}|^2 = \braket{a|s}\braket{s|a} = \braket{s|a}\braket{a|s}$
is both real and normalized. 

\be
\sum\limits_{a} \braket{s|a}\braket{a|s} = \braket{s|s} = 1
\ee

\underline{Interpretation}

$|\braket{a|s}|^2$ = Probability that a system prepared in state $\ket{s}$ will be found in a state $\ket{a}$ with value a for observable A after measurement. 


Comments on the measurement problem....


\clearpage

\textbf{\underline{Position Operator}}

\be
\vec{X} \ket{\vec{x}} = \vec{x}\ket{\vec{x}}
\ee

where, $\vec{X}$ is position operator and $\vec{x}$ is position eigenvalue.

Position of course is a continous observable so ``sums go to intergals'' etc.

eg: (Completnees and Orthogonality)

\be
\sum\limits_{a'} \ket{\alpha'}\bra{\alpha'} = 1   \Rightarrow \int d^3x \ket{x}\bra{x} = 1
\ee

\be
\braket{\alpha'|\alpha'} = \delta_{\alpha,\alpha'}   \Rightarrow \braket{\vec{x}|\vec{x'}} = \delta^3(\vec{x} - \vec{x'})
\ee


\textbf{\underline{Wave function}}

\be
\ket{\psi} = \int d^3x \ket{x}\braket{x|\psi} = \int d^3x \psi(x)\ket{x}
\ee
where $\psi(x) = \braket{x|\psi}$ is called the Position-space wavefunction.

\be
\braket{\psi|\psi} = 1  \Rightarrow \int d^3x \psi(x)\braket{\psi|x} = \int d^3x \psi^*(x)\psi(x) = 1
\ee

\underline{Interpretation}

 $|\psi(x)|^2d^3x$ is the probability to find the particle in volume $d^3x$ around $\vec{x}$.


\clearpage

\textbf{\underline{Translation Operator}}
``The operator that moves you over''

\be
T(\vec{a}) \ketX = \ket{\vec{x} + \vec{a}}
\ee


What is $T^\dagger(\vec{a})$ ???

Well, 

\be
\bra{\vec{x'}}\left(T(\vec{a})\ketX \right) = \delta^3((\vec{x} + \vec{a}) - \vec{x'})
\ee

or 

\be
\left(\bra{\vec{x'}}T(\vec{a}) \right) \ketX  = \delta^3(\vec{x} - (\vec{x'} - \vec{a}))
\ee

\be
\Rightarrow \bra{x'}T(a) = \bra{x-a}
\ee
which says that $T^\dagger(a)\ketX = \ket{\vec{x}-\vec{a}}$

So, 
\be
T^\dagger(\vec{a}) = T(-\vec{a}) = T^{-1}(\vec{a})
\ee

Properties of T
\begin{itemize}
\item[1.] Unitary $T^\dagger T = 1$
\item[2.] $T(\vec{a})T(\vec{b}) = T(\vec{a} + \vec{b}) = T(\vec{b})T(\vec{a})$,  Translations commute ($[T(\vec{a}),T(\vec{b})] = 0$)
\item[3.] $T(0) = 1$
\end{itemize}


\noindent\rule{\textwidth}{1pt}

\textbf{\underline{Infintesimal Translations}}

consider $\vec{a} = N\vec{\epsilon}$



}
\end{document}


