\input{../latexSetup}

\lhead{\Large 33-444} % \hfill Introduction to Particle Physics \hfill Spring 2020}
\chead{\Large Introduction to Particle Physics} % \hfill Spring 2020}
\rhead{\Large Spring 2020} % \hfill Introduction to Particle Physics \hfill Spring 2020}

\begin{document}
\thispagestyle{fancy}

\begin{center}
{\huge \textbf{Lecture 7}}
\end{center}

{\fontsize{14}{16}\selectfont

\textbf{\underline{Review Quantum Mechanics (Dynamics)}}

\be
\ket{\alpha,t_0} \rightarrow \ket{\alpha, t}
\ee

This is what we mean by time evolution.  

In QM, then there has to be an operator associated with taking the first state to the second. 

Time Evolution Operator:  $U(t,t_0)$


\be
\ket{\alpha,t} = U(t,t_0) \ket{\alpha, t_0}
\ee

$U(t,t_0)$ Properties
\begin{itemize}
\item[1.] $U^\dagger(t,t_0)U(t,t_0) = 1$ Unitary  See this from $\braket{\alpha t_0|\alpha t_0} = \braket{\alpha t|\alpha t}$
\item[2.] $U(t_2, t_0) = U(t_2,t_1)U(t_1,t_0)$ Composition Rule
\item[3.] $U(t_0, t_0) = 1$
\end{itemize}


How can we possibly determine what the time operator is ????
(Should be getting old by now...)


\underline{Start with infinitesimal time evolution}

\be
U(t+\epsilon, t)  = 1 - i\Omega \epsilon
\ee
where $\Omega$ is a Hermitian operator (b/c) U is unitary


So, 

\be
\ket{\alpha, t+\epsilon} = (1 - i \epsilon \Omega ) \ket{\alpha, t}
\ee

OR,

\be
\Omega \ket{\alpha, t} = i \frac{\ket{\alpha, t+\epsilon} - \ket{\alpha, t}}{\epsilon} = i \frac{\partial}{\partial t} \ket{\alpha, t}
\ee
in limit $\epsilon \rightarrow 0$


\underline{Physical Meaning of $\Omega$:}

As before to get the generate for the finite movement you have to exponetiate

\be
U(t) = e^{-i \Omega t}
\ee

Note that $\Omega$ has units 1/[time].  
Just like energy.

Identify $\Omega = \frac{1}{\hbar}H$  where H is the Hamiltonian operator.


\be
i \frac{\partial}{\partial t}\ket{\psi} = \Omega \ket{\psi} = \frac{E}{\hbar} \ket{\psi}
\ee


\underline{Schrodinger Equation}

\be
i \frac{\partial}{\partial t}\ket{\psi} = H(t) \ket{\psi}
\ee

Non-relativistically:  $H \sim \frac{p^2}{2m} = -\frac{1}{2m} \frac{\partial^2}{\partial^2 x}$

\noindent\rule{\textwidth}{1pt}

\underline{Time Evolution}

\be
\ket{\alpha,t} = U(t,t_0) \ket{\alpha, t_0}
\ee


\textbf{\underline{Case I:}} H is time independent

\be
U(t,t_0) = \lim_{N\rightarrow\infty} \prod\limits_{i}^{N} e^{-\frac{i}{\hbar}H\Delta t}  = e^{\frac{-iH(t-t_0)}{\hbar}}
\ee


\textbf{\underline{Case II:}} H is time dependent, but $[H(t), H(t')] = 0$

\be
U(t,t_0) =  e^{\frac{-i}{\hbar} \int_{t_0}^{t} H(t') dt'}
\ee


\textbf{\underline{Case III:}} H is time dependent, and $[H(t), H(t')] \ne 0$

(Note: in general $e^A e^B \ne e^{A+B}$ if $[A,B] \ne 0$)

\be
U(t,t_0) =  T \left[ e^{\frac{-i}{\hbar} \int_{t_0}^{t} H(t') dt'} \right]
\ee

``Time-ordered product'' - Power series expansion with earlier terms on the right
(Will come back to this later)


\noindent\rule{\textwidth}{1pt}

\textbf{\underline{Example}}

For a time \underline{independent} Hamiltonian.

\be
U(t,0) = e^{-iHt}
\ee

Choose a basis of eigenstates of H

\be
H\ket{n} = E_n\ket{n}
\ee

and 

\be
H = \sum\limits_{n} E_n \ket{n}\bra{n}
\ee

Then 

\be
U(t,0) = \sum\limits_{n} e^{-i E_n t} \ket{n}\bra{n}
\ee


Now some arbitrary state:

\bea
\ket{\psi(t)} = U(t,0)\ket{\psi(0)} =&  \sum\limits_{n} U(t,0) \ket{n}\braket{n|\psi} \\
                                    =&  \sum\limits_{n} \ket{n} e^{-i E_n t} \underbrace{\braket{n|\psi}}_{\textrm{time independent}} \\
\eea


\noindent\rule{\textwidth}{1pt}

Now lets talk about the expectation values of an observable and how they change with time....


\bea
\braket{A}(t) = \braket{\psi(t)|A|\psi(t)} =& \braket{\psi(0)|U^\dagger(t,0)A U(t,0)|\psi(0)}\\
              =& \underbrace{\left(\bra{\psi(0)} U^\dagger(t,0)\right) A \left( U(t,0) \ket{\psi(0)} \right)}_{\textrm{``Schrodinger Picture''}} \\
              =& \underbrace{\bra{\psi(0)} \left(U^\dagger(t,0) A U(t,0) \right)\ket{\psi(0)} }_{\textrm{``Heisenberg Picture''}} 
\eea


\underline{Schrodinger Picture}
\begin{itemize}
\item[-] $\ket{\psi(t)}$s move through Hilbert Space guided by U(t)
\item[-] Operators are independent of time
\item[-] Basis kets (eigenstates of observables) eg: $\ket{x}$ and $\ket{p}$ are \underline{independent} of time.
\end{itemize}



\underline{Heisenberg Picture}
\begin{itemize}
\item[-] $\ket{\psi(t)} = \ket{\psi}_H$ is fixed and independent of time
\item[-] {Operators in Heisenberg picture \underline{are} time dependent 
\be
A_H(t) = U^\dagger(t) A_S U(t) = e^{iHt}A_s e^{-iHt}
\ee
}
\end{itemize}

\noindent\rule{\textwidth}{1pt}

\underline{Time dependent perturbation theory}

\be
H(t) = H_0 + V(t)
\ee
where V(t) is small (this will always be the case for us)

\be
H_0 \ket{n} = E_n\ket{n}
\ee

Now, a general state at t=0

\be
\ket{\psi(0)} = \sum\limits_{n} c_n \ket{n}
\ee

\underline{For V = 0}

\be
\ket{\psi(t)} = \sum\limits_{n} c_n e^{-iE_n t} \ket{n}
\ee

\underline{For V $\ne$ 0}

\be
\ket{\psi(t)} = \sum\limits_{n} c_n(t) e^{-iE_n t} \ket{n}
\ee

where time dependence in $c_n(t)$ due to only V.


\textbf{\underline{Interaction Picture}}

Define...

\be
\ket{\psi(t)}_I \equiv e^{+iH_0 t} \ket{\psi(t)}_S
\ee

\be
A_I \equiv e^{+iH_0 t}  A_S e^{-iH_0 t}
\ee


From these definitions, its clear that 

\be
_I\braket{\psi(t) | A_I | \psi(t) }_I = _S\braket{\psi(t) | A_S | \psi(t) }_S
\ee

When V=0, Heisenberg and Interaction Picture Coincide.


Ok, here's why we care about the interaction picture....


\bea
i \frac{d}{dt} \ket{\psi(t)}_I =& i \frac{d}{dt} e^{iH_0 t} \ket{\psi(t)}_S = e^{iH_0 t}\left( -H_0 \ket{\psi(t)}_S + i \frac{d}{dt} \ket{\psi(t)}_S\right)\\
                               =& e^{iH_0 t}\left( -H_0 + (H_0 + V_S) \right)  \ket{\psi(t)}_S\\
                               =& e^{iH_0 t} V_S e^{-iH_0 t}e^{iH_0 t}  \ket{\psi(t)}_S\\
                               =& V_I(t) \ket{\psi(t)}_I
\eea

The interaction picture is a hybrid of the Schrodinger and Heisenberg pictures.

Time evolution of state kets and operators depend on different parts of H.


\be
\ket{\psi(t)}_S = \sum\limits_{n} c_n(t) e^{-i E_n t} \ket{n} = e^{-i H_0 t} \sum\limits_{n} c_n(t) \ket{n}
\ee

\be
\ket{\psi(t)}_I = e^{iH_0 t} \ket{\psi(t)}_S = \sum\limits_{n} c_n(t)  \ket{n} 
\ee

where,
\be
c_n(t) = \braket{n|\psi(t)}_I 
\ee
so once we have $\braket{\psi(t)}_I$ we are done.

\noindent\rule{\textwidth}{1pt}

Solve the ``Schrodinger Eq'' iteratively

\be
i \frac{d}{dt} \ket{\psi(t)}_I = V_I(t) \ket{\psi(t)}_I
\ee

integrate and get


\be
\ket{\psi(t)}_I = \ket{\psi(t_0)} + \int_{t_0}^t dt'   \left[ -i V_I(t') \ket{\psi(t')}\right]    
\ee

where the second term is of order $V_0$ which is small.

Now we keep iterating

\be
 = \ket{\psi(t_0)} - i \int_{t_0}^t dt' V_I(t') \left[  \ket{\psi(t_0)} - i \int_{t_0}^{t'} dt'' V_I(t'') \ket{\psi(t'')}   \right]
\ee


at 3rd order (iterate again...)


\bea
  =  \ket{\psi(t_0)} -& i \int_{t_0}^t dt' V_I(t') \ket{\psi(t_0)} + (-i)^2 \int_{t_0}^t dt' \int_{t_0}^{t'} dt''V(t')V(t'') \ket{\psi(t_0)} \\
                     +& (-i)^3 \int \int \int V(t')V(t'') V(t''') \ket{\psi(t_0)}
\eea


\be
\ket{\psi(t)} = U_I(t,t_0) \ket{\psi(t_0)}
\ee

where 
\be
U_I(t,t_0) = 1 + (-i)\int V(t')dt' + (-i)^2 \int \int V(t')V(t'') + ....
\ee

``Dyson Series'', can be written in slick form (by doing the sum)

\be
U_I(t,t_0) = T \left[ e^{-i \int_{t_0}^{t} V_I(t') dt'} \right]
\ee

Where the ``T'' stands for a ``time ordered product'' 



} \end{document}


