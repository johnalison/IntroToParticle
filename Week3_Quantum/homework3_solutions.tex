\documentclass[paper=letter,11pt]{scrartcl}

\KOMAoptions{headinclude=true, footinclude=false}
\KOMAoptions{DIV=14, BCOR=5mm}
\KOMAoptions{numbers=noendperiod}
\KOMAoptions{parskip=half}
\addtokomafont{disposition}{\rmfamily}
\addtokomafont{part}{\LARGE}
\addtokomafont{descriptionlabel}{\rmfamily}
%\setkomafont{pageheadfoot}{\normalsize\sffamily}
\setkomafont{pagehead}{\normalsize\rmfamily}
%\setkomafont{publishers}{\normalsize\rmfamily}
\setkomafont{caption}{\normalfont\small}
\setcapindent{0pt}
\deffootnote[1em]{1em}{1em}{\textsuperscript{\thefootnotemark}\ }


\usepackage{amsmath}
\usepackage[varg]{txfonts}
\usepackage[T1]{fontenc}
\usepackage{graphicx}
\usepackage{xcolor}
\usepackage[american]{babel}
% hyperref is needed in many places, so include it here
\usepackage{hyperref}

\usepackage{xspace}
\usepackage{multirow}
\usepackage{float}


\usepackage{braket}
\usepackage{bbm}
\usepackage{relsize}
\usepackage{tcolorbox}

\def\ketY{\ensuremath{\ket {\Psi}}}
\def\iGeV{\ensuremath{\textrm{GeV}^{-1}}}
%\def\mp{\ensuremath{m_{\textrm{proton}}}}
\def\rp{\ensuremath{r_{\textrm{proton}}}}
\def\me{\ensuremath{m_{\textrm{electron}}}}
\def\aG{\ensuremath{\alpha_G}}
\def\rAtom{\ensuremath{r_{\textrm{atom}}}}
\def\rNucl{\ensuremath{r_{\textrm{nucleus}}}}
\def\GN{\ensuremath{\textrm{G}_\textrm{N}}}
\def\ketX{\ensuremath{\ket{\vec{x}}}}
\def\ve{\ensuremath{\vec{\epsilon}}}


\def\ABCDMatrix{\ensuremath{\begin{pmatrix} A &  B  \\ C  & D \end{pmatrix}}}
\def\xyprime{\ensuremath{\begin{pmatrix} x' \\ y' \end{pmatrix}}}
\def\xyprimeT{\ensuremath{\begin{pmatrix} x' &  y' \end{pmatrix}}}
\def\xy{\ensuremath{\begin{pmatrix} x \\ y \end{pmatrix}}}
\def\xyT{\ensuremath{\begin{pmatrix} x & y \end{pmatrix}}}

\def\IMatrix{\ensuremath{\begin{pmatrix} 0 &  1  \\ -1  & 0 \end{pmatrix}}}
\def\IBoostMatrix{\ensuremath{\begin{pmatrix} 0 &  1  \\ 1  & 0 \end{pmatrix}}}
\def\JThree{\ensuremath{\begin{pmatrix}    0 & -i & 0  \\ i & 0  & 0 \\ 0 & 0 & 0 \end{pmatrix}}} 
\def\JTwo{\ensuremath{\begin{bmatrix}    0 & 0 & -i  \\ 0 & 0  & 0 \\ i & 0 & 0 \end{bmatrix}}}
\def\JOne{\ensuremath{\begin{bmatrix}    0 & 0 & 0  \\ 0 & 0  & -i \\ 0 & i & 0 \end{bmatrix}}}
\def\etamn{\ensuremath{\eta_{\mu\nu}}}
\def\Lmn{\ensuremath{\Lambda^\mu_\nu}}
\def\dmn{\ensuremath{\delta^\mu_\nu}}
\def\wmn{\ensuremath{\omega^\mu_\nu}}
\def\be{\begin{equation*}}
\def\ee{\end{equation*}}
\def\bea{\begin{eqnarray*}}
\def\eea{\end{eqnarray*}}
\def\bi{\begin{itemize}}
\def\ei{\end{itemize}}
\def\fmn{\ensuremath{F_{\mu\nu}}}
\def\fMN{\ensuremath{F^{\mu\nu}}}
\def\bc{\begin{center}}
\def\ec{\end{center}}
\def\nus{$\nu$s}

\def\adagger{\ensuremath{a_{p\sigma}^\dagger}}
\def\lineacross{\noindent\rule{\textwidth}{1pt}}

\newcommand{\multiline}[1] {
\begin{tabular} {|l}
#1
\end{tabular}
}

\newcommand{\multilineNoLine}[1] {
\begin{tabular} {l}
#1
\end{tabular}
}



\newcommand{\lineTwo}[2] {
\begin{tabular} {|l}
#1 \\
#2
\end{tabular}
}

\newcommand{\rmt}[1] {
\textrm{#1}
}


%
% Units
%
\def\m{\ensuremath{\rmt{m}}}
\def\GeV{\ensuremath{\rmt{GeV}}}
\def\pt{\ensuremath{p_\rmt{T}}}


\def\parity{\ensuremath{\mathcal{P}}}

\usepackage{cancel}
\usepackage{ mathrsfs }
\def\bigL{\ensuremath{\mathscr{L}}}

\usepackage{ dsfont }



\usepackage{fancyhdr}
\fancyhf{}


\lhead{\Large 33-444} % \hfill Introduction to Particle Physics \hfill Spring 2020}
\chead{\Large Introduction to Particle Physics} % \hfill Spring 2020}
\rhead{\Large Spring 2020} % \hfill Introduction to Particle Physics \hfill Spring 2020}
\begin{document}
\thispagestyle{fancy}





%\begin{tabular}{c}
%{\large 33-444 \hfill Intro To Particle \hfill Spring 2020\\}
%\hline 
%\end{tabular}

\begin{center}
{\huge \textbf{Homework Set \#3}}
\large

{\textbf{ Solutions} }
\end{center}

{\large
\textbf{1) Work out the commutation relation among the $\vec{X}$ and $\vec{P}$ operators: \\ ie: $[\vec{X},\vec{X}]$,  $[\vec{P},\vec{P}]$, and $[\vec{X},\vec{P}]$ } \hfill \textit{(5 points)}
\begin{itemize}
\item[] {
On one hand,
\be
\vec{X}T(\vec{a}) \ket{\vec{x}} = (\vec{x} + \vec{a}) \ket{\vec{x} + \vec{a}}
\ee

on the other,
\be
T(\vec{a})\vec{X} \ket{\vec{x}} = \vec{x}  \ket{\vec{x} + \vec{a}}
\ee

So

\be
[\vec{X},T(\vec{a})] = \vec{a} T(\vec{a})
\ee

Consider the infinitesimal translation ($\vec{a} \rightarrow \vec{\epsilon}$).  Then,

\begin{eqnarray*}
[\vec{X},1- i \vec{\epsilon} \cdot \vec{P}] = \vec{\epsilon} \cdot (1- i \vec{\epsilon} \cdot \vec{P})
\end{eqnarray*}
Keeping terms to first order in $\epsilon$. 

\begin{eqnarray*}
[X_i,1- i \epsilon_j P_j] = \epsilon_i\\
-i \epsilon_j [X_i,P_j] = \epsilon_i\\
\end{eqnarray*}
or
\begin{eqnarray*}
[X_i,P_j] = i \delta_{ij}\\
\end{eqnarray*}


}
\end{itemize}




\vspace*{0.25in}

{\large
\textbf{2) Harmonic Oscillator } \hfill \textit{(10 points)}

The 1D Harmonic oscillator has Hamiltonian:
\be
H = \frac{P^2}{2m} + \frac{1}{2}mw^2X^2
\ee
where P and X are position and momentum operators

\begin{itemize}
\item[a]{ Define ``raising'' and ``lowering'' operators as 
\be
a= \sqrt{\frac{mw}{2}}\left(X + i \frac{P}{mw}\right) \hspace{0.5in} a^\dagger = \sqrt{\frac{mw}{2}}\left(X - i \frac{P}{mw}\right)
\ee
What are the position and momentum operators in terms of the raising and lowering operators?

\textit{Solution:}
\be
x =  \frac{x_0}{\sqrt{2}}(a + a^\dagger) \hspace{0.5in}  p =  \frac{-i}{x_0 \sqrt{2}}(a - a^\dagger)
\ee 
where $x_0 = \sqrt{\frac{1}{mw}}$.

}
\item[b]{ 

\begin{eqnarray*}
[a,a^\dagger] =& \frac{mw}{2} [ x + \frac{i p}{mw }, x -  \frac{ip}{mw}]\\
              =& \frac{mw}{2}\frac{-i}{mw} [ x , p] + \frac{mw}{2}\frac{i}{mw} [p,x]\\
              =& \frac{1}{2}(-i [ x , p] + i [p,x]) = 1
\end{eqnarray*}

}
\item[c]{ 
What is the Hamiltonian in terms of $a$ and $a^\dagger$?

\be
H = \omega(a^\dagger a + \frac{1}{2})
\ee
}
\item[d]{ 
Define the ``Number'' operator $N$ as $a^\dagger a$.  What is the Hamiltonian in terms of the number operator?

\be
H = \omega(N + \frac{1}{2})
\ee


}
\item[e]{ 
Work out the commutation relations: $[N,a^\dagger]$ and $[N,a]$.

\be
[N,a^\dagger] = a^\dagger  \hspace{0.5in} [N,a] = -a
\ee

}
\item[f]{ 
Show that the eigenvalues of N (n) are real and satisfy $n \geq 0$.

\be
n = \bra{n}N\ket{n} = \bra{n}a^\dagger a \ket{n} = \bra{an}\underbrace{\ket{an}}_{\ket{n'}} = \braket{n'|n'} \geq 0
\ee


}
\item[g]{ 
Show that $a \ket n$  is an eigenstate of N, with eigenvalue (n-1). This implies $a \ket{n} \propto \ket{n-1}$ and justifies calling $a$ the lower operator.

\be
\ket{n'} = a\ket{n} 
\ee

\be
N\ket{n'} = Na\ket{n} = (aN + [N,a])\ket{n} = (aN - a)\ket{n} = (an -a)\ket{n} = (n-1)\ket{n'}
\ee


}
\item[h]{ 
Show that $a^\dagger \ket n$  is an eigenstate of N, with eigenvalue (n+1). This implies $a^\dagger \ket{n} \propto \ket{n+1}$ and justifies calling $a^\dagger$ the raising operator.


\be
\ket{n'} = a^\dagger\ket{n} 
\ee
\be
N\ket{n'} = (a^\dagger N + [N,a^\dagger])\ket{n} = (n+1)\ket{n'}
\ee

}
\item[i]{ 
Find $c_n$ such that $\ket{n+1} = c_n a\dagger \ket{n}$ is normalized.

\be
\ket{n+1} = c_n a^\dagger \ket{n}
\ee

\be
1 = \braket{n+1|n+1} = |c_n|^2 \bra{n} a a^\dagger \ket{n} = |c_n|^2 (n+1)
\ee

\be
c_n = \frac{1}{\sqrt{n+1}}
\ee

}
\item[j]{ 
Since $n \geq 0$, there must be a state $\ket{0}$ which satisfies $a \ket{0} = 0$ and n must be an integer.
What is the general state $\ket{n}$ in terms of $\ket{0}$ and $a^\dagger$ ?
What is the energy associated to this state ?


\be
\ket{n} = \frac{1}{\sqrt{n!}}(a^\dagger)^n\ket{0}
\ee

The energy associated to $\ket{n}$ is $\omega(n+1/2)$.
}

\end{itemize}
}

\end{document}
