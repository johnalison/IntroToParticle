\input{../latexSetup}

\lhead{\Large 33-444} % \hfill Introduction to Particle Physics \hfill Spring 2022}
\chead{\Large Introduction to Particle Physics} % \hfill Spring 2022}
\rhead{\Large Spring 2022} % \hfill Introduction to Particle Physics \hfill Spring 2022}
\begin{document}
\thispagestyle{fancy}





%\begin{tabular}{c}
%{\large 33-444 \hfill Intro To Particle \hfill Spring 2022\\}
%\hline 
%\end{tabular}

\begin{center}
{\huge \textbf{Homework Set \#2}}
\large

{\textbf{ Due Date:} Friday February 11th  }
\end{center}

{\large
\textbf{1) Work out the commutation relation among the $\vec{X}$ and $\vec{P}$ operators: \\ ie: $[\vec{X},\vec{X}]$,  $[\vec{P},\vec{P}]$, and $[\vec{X},\vec{P}]$ } \hfill \textit{(5 points)}
\begin{itemize}
\item[] {
Hint: for $[\vec{X},\vec{P}]$ work out how the commutator $[\vec{X},T(\vec{a})]$ acts on position eigenstate for generic translation. Then apply the result to the infinitesimal case.
}
\end{itemize}




\vspace*{0.25in}

{\large
\textbf{2) Harmonic Oscillator } \hfill \textit{(10 points)}

The 1D Harmonic oscillator has Hamiltonian:
\be
H = \frac{P^2}{2m} + \frac{1}{2}mw^2X^2
\ee
where P and X are position and momentum operators

\begin{itemize}
\item[a]{ Define ``raising'' and ``lowering'' operators as 
\be
a= \sqrt{\frac{mw}{2}}\left(X + i \frac{P}{mw}\right) \hspace{0.5in} a^\dagger = \sqrt{\frac{mw}{2}}\left(X - i \frac{P}{mw}\right)
\ee
What are the position and momentum operators in terms of the raising and lowering operators?
}
\item[b]{ 
Find $[a,a^\dagger]$
}
\item[c]{ 
What is the Hamiltonian in terms of $a$ and $a^\dagger$?
}
\item[d]{ 
Define the ``Number'' operator $N$ as $a^\dagger a$.  What is the Hamiltonian in terms of the number operator?
}
\item[e]{ 
Work out the commutation relations: $[N,a^\dagger]$ and $[N,a]$.
}
\item[f]{ 
Show that the eigenvalues of N (n) are real and satisfy $n \geq 0$.
(Hint: consider $\bra n N \ket n = \bra n a^\dagger a \ket n $, where $\ket n $ are eigenkets of N  )
}
\item[g]{ 
Show that $a \ket n$  is an eigenstate of N, with eigenvalue (n-1). This implies $a \ket{n} \propto \ket{n-1}$ and justifies calling $a$ the lower operator.
}
\item[h]{ 
Show that $a^\dagger \ket n$  is an eigenstate of N, with eigenvalue (n+1). This implies $a^\dagger \ket{n} \propto \ket{n+1}$ and justifies calling $a^\dagger$ the raising operator.
}
\item[i]{ 
Find $c_n$ such that $\ket{n+1} = c_n a\dagger \ket{n}$ is normalized.
}
\item[j]{ 
Since $n \geq 0$, there must be a state $\ket{0}$ which satisfies $a \ket{0} = 0$ and n must be an integer.
What is the general state $\ket{n}$ in terms of $\ket{0}$ and $a^\dagger$ ?
What is the energy associated to this state ?
}

\end{itemize}
}

\end{document}
