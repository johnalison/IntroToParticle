\input{latexSetup}
\usepackage{braket}

\def\ketY{\ensuremath{\ket {\Psi}}}
\def\iGeV{\ensuremath{\textrm{GeV}^{-1}}}
\def\mp{\ensuremath{m_{\textrm{proton}}}}
\def\rp{\ensuremath{r_{\textrm{proton}}}}
\def\me{\ensuremath{m_{\textrm{electron}}}}
\def\aG{\ensuremath{\alpha_G}}
\def\rAtom{\ensuremath{r_{\textrm{atom}}}}
\def\rNucl{\ensuremath{r_{\textrm{nucleus}}}}
\def\GN{\ensuremath{\textrm{G}_\textrm{N}}}
\def\nue{\ensuremath{\nu_e}}
\def\numu{\ensuremath{\nu_\mu}}
\def\nutau{\ensuremath{\nu_\tau}}


\def\be{\begin{equation*}}
\def\ee{\end{equation*}}


\usepackage{fancyhdr}
\usepackage{cancel}
\usepackage{ mathrsfs }





\fancyhf{}
\lhead{\Large 33-444} % \hfill Introduction to Particle Physics \hfill Spring 2019}
\chead{\Large Introduction to Particle Physics} % \hfill Spring 2019}
\rhead{\Large Spring 2019} % \hfill Introduction to Particle Physics \hfill Spring 2019}
\begin{document}
\thispagestyle{fancy}





%\begin{tabular}{c}
%{\large 33-444 \hfill Intro To Particle \hfill Spring 2019\\}
%\hline 
%\end{tabular}

\begin{center}
{\huge \textbf{Homework Set \#12 }}
\large

{\textbf{ Due Date:} Before class Friday May 4th  } 
\end{center}

{\large

\textbf{1) Neutrino Oscillations } \hfill \textit{(5 points)}

In the two flavour approximation, work out the probability for a \nue\ to be detected as a \numu\ as a function of mixing angle, mass difference, distance traveled and Energy. 

Sketch the probability as a function of L/E. 

\vspace*{0.25in}



\textbf{2) Cosmic Rays } \hfill \textit{(5 points)}

What ratio of \nue\ and \numu\ do you expect in cosmic rays at low energies ?\\
\textit{(Treat cosmic rays as protons which produce pions.) }

What can cause this ratio to change at higher energies ?

\vspace*{0.25in}

\textbf{3) SNO } \hfill \textit{(5 points)}

SNO measured the $\nu$ flux in three different ways.
Draw the corresponding Feynman diagrams and indicate if you would expect a difference in cross section between the different $\nu$ flavours. 

\vspace*{0.25in}

\textbf{4) $\nu$ beams } \hfill \textit{(5 points)}

\begin{itemize}
\item[a)]How could you make a beam of $\nu$s ? 
\item[b)]What $\nu$ flavours would be produced ? 
\item[c)]Would you expect this to make more $\nu$ or anti-$\nu$ ? 
\item[d)]How could you enhance the $\nu$ fraction ? 
\item[e)]How about the  anti-$\nu$ fraction ?
\end{itemize}

}
\end{document}
