\input{../latexSetup}

\lhead{\Large 33-444} % \hfill Introduction to Particle Physics \hfill Spring 2019}
\chead{\Large Introduction to Particle Physics} % \hfill Spring 2019}
\rhead{\Large Spring 2019} % \hfill Introduction to Particle Physics \hfill Spring 2019}
\begin{document}
\thispagestyle{fancy}





%\begin{tabular}{c}
%{\large 33-444 \hfill Intro To Particle \hfill Spring 2019\\}
%\hline 
%\end{tabular}

\begin{center}
{\huge \textbf{Final}}
\large

\end{center}

{\large



\textbf{1) What are three major consequences of combining QM and Relativity?}\hfill \textit{(3 points)}\\

\vspace{0.5in}


\textbf{2) Lorentz Transforms } \hfill \textit{(4 points)}\\
\begin{itemize}
\item[a)] How does a \textbf{massive} particle $\ket{p^\mu,\sigma}$ transform under a general Lorentz transformation ($\Lambda_\mu^\nu$) ?
\vspace*{0.2in}
\item[b)] How does a \textbf{mass-less} particle $\ket{p^\mu,\sigma}$ transform under a general Lorentz transformation ($\Lambda_\mu^\nu$)?
\vspace*{0.2in}
\end{itemize}



\textbf{3) What are two restrictions to the interactions of mass-less particles that are a consequence gauge invariance ? }\hfill \textit{(2 points)}\\

\vspace{0.5in}


\textbf{4) Why do the weak and strong interactions look so different from the electromagnetic interaction despite the fact that at short distances they are all described by similar Feynman diagrams with similar coupling constants?} \hfill \textit{(2 points)}\\

\vspace{0.5in}

\clearpage

\textbf{5) Muon decays: } \hfill \textit{(8 points)}\\
\begin{itemize}
  \item[a)]{ The muon decays via the weak interaction.  At low energy ($E << m_W$), this can be approximated as a point-like interaction. 
  Draw the diagram describing muon decay to an electron assuming a point-like weak interaction. 
  \vspace*{0.5in}
}
  \item[b)]{What are the dimensions of the coupling constant, associated to this diagram  ?
\vspace*{0.5in}
  }
  \item[c)] How does the decay rate $\Gamma$ (decays/unit time)  depend on the muon mass ? 
\vspace*{0.5in}
  \item[d)]{ The muon has a mass of $\sim$0.1 GeV and a lifetime of $\sim 1 \mu s$. The tau lepton has a mass of {$\sim$1 GeV}. Estimate the lifetime of the tau lepton in $\mu s$.
\vspace*{0.5in}
}
\end{itemize}


\textbf{6) Electron-positron Collisions } \hfill \textit{(8 points)}\\
\begin{itemize}
\item[a)]{Consider electron-positron collisions with a center-of-mass  energy of 40 GeV.
Estimate the ratio of jet production to di-muon production. 
\vspace*{0.5in}
}
\item[b)]{Sketch a graph of the total cross section of $ee\rightarrow\mu\mu$ as a function of $E_{CM}$ from 40 GeV to 200. 
Also sketch the component of the cross section due to the electro-magnetic interaction.
\vspace*{0.5in}
}
\end{itemize}

\clearpage

\textbf{7) Branching ratios:  } \hfill \textit{(6 points)}\\
\begin{itemize}
\item[a)]{How often does a $\tau$ decay to a charged lepton ? \\ \textit{(HINT: you can neglect $\tau$ decays to charm and strange quarks) }  
\vspace*{0.5in}
}
\item[b)]{How often does a $Z$-boson decay to charged leptons ?
\vspace*{0.5in}
}
\item[c)]{How often does a $W$-boson decay to a charged lepton ?
\vspace*{0.5in}
}
\end{itemize}


\textbf{8) If you want to improve momentum resolution, is it better to have a tracking detector that is twice as big or that has twice the magnetic field ?  Justify your answer.} \hfill \textit{(3 points)}\\
\vspace*{0.5in}


\textbf{9) Spontaneous Symmetry Breaking  } \hfill \textit{(6 points)}\\
\begin{itemize}
\item[a)]{ What is the particle spectra (ie: for each particle, is it massive or mass-less and what is the spin) from the Lagrangian: \be\mathcal{L} = (\partial_\mu \phi^*) (\partial^\mu \phi) - V(\phi), \textrm{ where } \phi = \phi_1 + i \phi_2,\  V(\phi) = \mu^2\phi^*\phi + \lambda (\phi^*\phi)^2\textrm { and } \lambda, \mu^2 > 0   \ee
\vspace{0.5in}
}
\item[b)]{ What is the particle spectra from the setup in a) but with $\mu^2 < 0$ ?
\vspace{0.5in}
}
\end{itemize}


\textbf{10) What is experimental evidence against a fourth generation of leptons ? } \hfill \textit{(4 points)}\\
Other than the fact that they have not be directly observed.
\vspace{0.5in}

\textbf{11) Neutrino Physics } \hfill \textit{(10 points)}\\
\begin{itemize}
\item[a)]{How was the distinction between \numu\ and \nue\ discovered ?
\vspace*{0.5in}
}

\item[b)]{Why was the distinction between \numu\ and \nue\  expected ?
\vspace*{0.5in}
}
\item[c)]{What was the main reason to study \nus\ in the 60s and 70s, before we knew they had mass?
\vspace*{0.5in}
}
\item[d)]{What are dominant kind(s) (indicate particle or anti-particle and flavor) of \nus\ that are produced (ignore oscillations) from:
\begin{itemize}
\item[i)]{ The sun ? 
\vspace*{0.25in}
} 
\item[ii)]{Nuclear reactors ? 
\vspace*{0.25in}
}
\item[ii)]{Cosmic-rays ? 
\vspace*{0.25in}
}
\item[iv)]{$\nu$-beams ?
\vspace*{0.25in}
}
\end{itemize}
}
\end{itemize}


\textbf{12) In a two $\nu$ model, what combination of $\Delta m^2$, E, and L do the transition probabilities depend on?  } \hfill \textit{(3 points)}\\
\vspace*{0.5in}

\textbf{13) What was the solar neutrino puzzle ? How was it resolved?  } \hfill \textit{(3 points)}\\
\vspace{0.5in}


\textbf{14) What are the three fundamental length scales in nature and their associated problems.  } \hfill \textit{(3 points)}\\
\vspace{0.5in}


} % Begning Large
\end{document}
