
\input{latexSetup}
\usepackage{braket}

\def\ketY{\ensuremath{\ket {\Psi}}}
\def\iGeV{\ensuremath{\textrm{GeV}^{-1}}}
\def\mp{\ensuremath{m_{\textrm{proton}}}}
\def\rp{\ensuremath{r_{\textrm{proton}}}}
\def\me{\ensuremath{m_{\textrm{electron}}}}
\def\aG{\ensuremath{\alpha_G}}
\def\rAtom{\ensuremath{r_{\textrm{atom}}}}
\def\rNucl{\ensuremath{r_{\textrm{nucleus}}}}
\def\GN{\ensuremath{\textrm{G}_\textrm{N}}}
\def\nus{$\nu$s}
\def\nue{\ensuremath{\nu_e}}
\def\numu{\ensuremath{\nu_\mu}}

\def\be{\begin{equation*}}
\def\ee{\end{equation*}}


\usepackage{fancyhdr}
\usepackage{cancel}
\usepackage{ mathrsfs }





\fancyhf{}
\lhead{\Large 33-444} % \hfill Introduction to Particle Physics \hfill Spring 2019}
\chead{\Large Introduction to Particle Physics} % \hfill Spring 2019}
\rhead{\Large Spring 2019} % \hfill Introduction to Particle Physics \hfill Spring 2019}
\begin{document}
\thispagestyle{fancy}





%\begin{tabular}{c}
%{\large 33-444 \hfill Intro To Particle \hfill Spring 2019\\}
%\hline 
%\end{tabular}

\begin{center}
{\huge \textbf{Final}}
\large

\end{center}

{\large


\textbf{1) What are three major consequences of combining QM and Relativity?}\hfill \textit{(3 points)}\\

\vspace{3in}


\textbf{2) Lorentz Transforms } \hfill \textit{(5 points)}\\
\begin{itemize}
\item[a)] How does a massive particle $\ket{p^\mu,\sigma}$ transform under a little group transformation ($W_\mu^\nu$)  ?
\vspace*{1in}
\item[d)] How does a mass-less particle $\ket{p^\mu,\sigma}$ transform under a little group transformation ($W_\mu^\nu$)  ?
\vspace*{1in}
\item[c)] How does a massive particle $\ket{p^\mu,\sigma}$ transform under a general Lorentz transformation ($\Lambda_\mu^\nu$)  ?
\vspace*{1in}
\item[d)] How does a mass-less particle $\ket{p^\mu,\sigma}$ transform under a general Lorentz transformation ($\Lambda_\mu^\nu$)  ?
\vspace*{1in}

\end{itemize}


\textbf{3) List or draw a diagram of the particles in the Standard model. } \hfill \textit{(3 points)}\\
What is the spin of each particle ?

\vspace*{3in}


\textbf{4) Why is the weak interaction so much weaker than the electromagnetic interaction at low energies?}\hfill \textit{(2 points)}\\

\vspace{2in}



%scalar QED. treat electron as scalar, photon as massless spin 1 particle
%  ee->ee scattering: Draw diagrmas, what are the associated matrix elements.


%what about ee->tau tau? again at high energy where $E_{CM}$ >> tau tau.

% work out the differential cross section ee->tautau in the massless limit.

%\clearpage

\textbf{5) Muon decays: } \hfill \textit{(10 points)}\\
\begin{itemize}
  \item[a)]{ The muon decays via the weak interaction,  At low energy ($E << m_W$), this can be approximated as a point-like interaction. 
  Draw the diagram describing muon decay to an electron assuming a point-like weak interaction. (Indicate which is the XXX
\vspace*{1.5in}
}
  \item[b)]{ What are the dimensions of the coupling constant, associated to this diagram  ?
\vspace*{1.0in}
  }
  \item[c)] How does the decay rate $\Gamma$ (decays/unit time)  depend on the muon mass ? 
\vspace*{1.0in}
  \item[d)]{ The muon has a mass of $\sim$0.1 GeV and a lifetime of $\sim 1 \mu s$. The tau lepton has a mass of {$\sim$1 GeV}. Estimate the lifetime of the tau lepton in $\mu s$.
\vspace*{2.4in}
}
  \item[e)] {Suppose that the photon could couple at the same vertex to the muon and the electron. Then the muon could decay as $\mu\rightarrow e \gamma$. 
  Estimate the ratio of the $\mu$ lifetime in this world to that in our world without this interaction.
  \vspace*{3.0in}
  }
\end{itemize}


\textbf{6) What are some experimantal constraints on a fourth generation of leptons ? } \hfill \textit{(3 points)}\\

\textbf{7) What problems might SuperSymmetry solve ? } \hfill \textit{(3 points)}\\


\textbf{8) Muon Neutrinos  } \hfill \textit{(3 points)}
\begin{itemize}
\item[a)]How was the distinction between \numu\ and \nue\ discovered ?
\item[b)]Why was this expected ?
\end{itemize}

\textbf{9) Accelerators: } \hfill \textit{(3 points)}
\begin{itemize}
\item[a)]What limits the energy of circular proton accelerators ?
\item[b)]What limits the energy of circular electron accelerators ?
\end{itemize}



\textbf{10) Electron-positron Collisions } \hfill \textit{(3 points)}\\
\begin{itemize}
\item[a)]{Consider electron-positron collisions with a center-of-mass  energy of 40 GeV.
Estimate the ratio of hadron production to di-muon production. }
\item[b)]{Sketch a graph of the total cross section of $ee\rightarrow\mu\mu$ as a function of $E_{CM}$ from 40 GeV to 200. 
Also sketch the compent of the cross section due to the electro-magnetic interaction.}
\end{itemize}


\textbf{11) Spontaneous Symmetry Breaking  } \hfill \textit{(3 points)}\\
What particle spectra would you expect from  $L = (dP)^2 + mu^2 + $
V w  mu2 > 0
V w  mu2 < 0 
U(1) + V   mu2 < 0

\textbf{12) Leptons  } \hfill \textit{(3 points)}\\
Branching ratios:
$\tau \rightarrow e or \mu $
%  / Z->bb 

\textbf{13) Collider Detectors  } \hfill \textit{(3 points)}\\
In a collider experiment at the LHC,
\begin{itemize}
\item[a)] In what ways do the detector signatures of electrons and muons look a-like, in what ways are they different ?
\item[b)] In what ways do the detector signatures of electrons and photons look a-like, in what ways are they different ?
\item[c)] In what ways do the detector signatures of electrons and quarks look a-like, in what ways are they different ?
\end{itemize}

\textbf{14) Why are hadronic showers more challenging to measure than electro-magnetic showers ?  } \hfill \textit{(3 points)}\\



\textbf{15) Other things being equal is it better to have a tracking detector that is twice as big or that has double the magnetic field ?  Justify your answer.} \hfill \textit{(5 points)}\\

\textbf{16) How are \nus\ detected at the LHC ?} \hfill \textit{(3 points)}\\

\textbf{16) Higgs Physics } \textit{(3 points)}\\
\begin{itemize}
\item[a)]{Estimate the ratio of Br($H\rightarrow\mu\mu$)/Br($H\rightarrow\tau\tau$) }
\item[b)]{Draw one possible feynman diagram for production and decay fo the Higgs boson at the LHC.}
\item[c)]{Draw one possible feyman diagram for the possible and decay of the Higgs boson at an eletron -positron collider.}
\end{itemize}


\textbf{16) Neutrino Physics } \hfill \textit{(10 points)}\\
\begin{itemize}
\item[a)]{What was the main reason to study \nus\ in the 60s and 70s, before we knew they had mass?}
\item[b)]{What are dominant kind (indicate particle or anti-particle and flavor) of \nus that come from:
\begin{itemize}
\item[i)]{ The sun ? } 
\item[ii)]{nuclear reactors ? }
\item[ii)]{cosmic rays ? }
\item[iv)]{nu beams ?}
\end{itemize}
}
\end{itemize}





} % Begning Large
\end{document}
