
\documentclass[paper=letter,11pt]{scrartcl}

\KOMAoptions{headinclude=true, footinclude=false}
\KOMAoptions{DIV=14, BCOR=5mm}
\KOMAoptions{numbers=noendperiod}
\KOMAoptions{parskip=half}
\addtokomafont{disposition}{\rmfamily}
\addtokomafont{part}{\LARGE}
\addtokomafont{descriptionlabel}{\rmfamily}
%\setkomafont{pageheadfoot}{\normalsize\sffamily}
\setkomafont{pagehead}{\normalsize\rmfamily}
%\setkomafont{publishers}{\normalsize\rmfamily}
\setkomafont{caption}{\normalfont\small}
\setcapindent{0pt}
\deffootnote[1em]{1em}{1em}{\textsuperscript{\thefootnotemark}\ }


\usepackage{amsmath}
\usepackage[varg]{txfonts}
\usepackage[T1]{fontenc}
\usepackage{graphicx}
\usepackage{xcolor}
\usepackage[american]{babel}
% hyperref is needed in many places, so include it here
\usepackage{hyperref}

\usepackage{xspace}
\usepackage{multirow}
\usepackage{float}


\usepackage{braket}
\usepackage{bbm}
\usepackage{relsize}
\usepackage{tcolorbox}

\def\ketY{\ensuremath{\ket {\Psi}}}
\def\iGeV{\ensuremath{\textrm{GeV}^{-1}}}
%\def\mp{\ensuremath{m_{\textrm{proton}}}}
\def\rp{\ensuremath{r_{\textrm{proton}}}}
\def\me{\ensuremath{m_{\textrm{electron}}}}
\def\aG{\ensuremath{\alpha_G}}
\def\rAtom{\ensuremath{r_{\textrm{atom}}}}
\def\rNucl{\ensuremath{r_{\textrm{nucleus}}}}
\def\GN{\ensuremath{\textrm{G}_\textrm{N}}}
\def\ketX{\ensuremath{\ket{\vec{x}}}}
\def\ve{\ensuremath{\vec{\epsilon}}}


\def\ABCDMatrix{\ensuremath{\begin{pmatrix} A &  B  \\ C  & D \end{pmatrix}}}
\def\xyprime{\ensuremath{\begin{pmatrix} x' \\ y' \end{pmatrix}}}
\def\xyprimeT{\ensuremath{\begin{pmatrix} x' &  y' \end{pmatrix}}}
\def\xy{\ensuremath{\begin{pmatrix} x \\ y \end{pmatrix}}}
\def\xyT{\ensuremath{\begin{pmatrix} x & y \end{pmatrix}}}

\def\IMatrix{\ensuremath{\begin{pmatrix} 0 &  1  \\ -1  & 0 \end{pmatrix}}}
\def\IBoostMatrix{\ensuremath{\begin{pmatrix} 0 &  1  \\ 1  & 0 \end{pmatrix}}}
\def\JThree{\ensuremath{\begin{pmatrix}    0 & -i & 0  \\ i & 0  & 0 \\ 0 & 0 & 0 \end{pmatrix}}} 
\def\JTwo{\ensuremath{\begin{bmatrix}    0 & 0 & -i  \\ 0 & 0  & 0 \\ i & 0 & 0 \end{bmatrix}}}
\def\JOne{\ensuremath{\begin{bmatrix}    0 & 0 & 0  \\ 0 & 0  & -i \\ 0 & i & 0 \end{bmatrix}}}
\def\etamn{\ensuremath{\eta_{\mu\nu}}}
\def\Lmn{\ensuremath{\Lambda^\mu_\nu}}
\def\dmn{\ensuremath{\delta^\mu_\nu}}
\def\wmn{\ensuremath{\omega^\mu_\nu}}
\def\be{\begin{equation*}}
\def\ee{\end{equation*}}
\def\bea{\begin{eqnarray*}}
\def\eea{\end{eqnarray*}}
\def\bi{\begin{itemize}}
\def\ei{\end{itemize}}
\def\fmn{\ensuremath{F_{\mu\nu}}}
\def\fMN{\ensuremath{F^{\mu\nu}}}
\def\bc{\begin{center}}
\def\ec{\end{center}}
\def\nus{$\nu$s}

\def\adagger{\ensuremath{a_{p\sigma}^\dagger}}
\def\lineacross{\noindent\rule{\textwidth}{1pt}}

\newcommand{\multiline}[1] {
\begin{tabular} {|l}
#1
\end{tabular}
}

\newcommand{\multilineNoLine}[1] {
\begin{tabular} {l}
#1
\end{tabular}
}



\newcommand{\lineTwo}[2] {
\begin{tabular} {|l}
#1 \\
#2
\end{tabular}
}

\newcommand{\rmt}[1] {
\textrm{#1}
}


%
% Units
%
\def\m{\ensuremath{\rmt{m}}}
\def\GeV{\ensuremath{\rmt{GeV}}}
\def\pt{\ensuremath{p_\rmt{T}}}


\def\parity{\ensuremath{\mathcal{P}}}

\usepackage{cancel}
\usepackage{ mathrsfs }
\def\bigL{\ensuremath{\mathscr{L}}}

\usepackage{ dsfont }



\usepackage{fancyhdr}
\fancyhf{}

\usepackage{braket}

\def\ketY{\ensuremath{\ket {\Psi}}}
\def\iGeV{\ensuremath{\textrm{GeV}^{-1}}}
\def\mp{\ensuremath{m_{\textrm{proton}}}}
\def\rp{\ensuremath{r_{\textrm{proton}}}}
\def\me{\ensuremath{m_{\textrm{electron}}}}
\def\aG{\ensuremath{\alpha_G}}
\def\rAtom{\ensuremath{r_{\textrm{atom}}}}
\def\rNucl{\ensuremath{r_{\textrm{nucleus}}}}
\def\GN{\ensuremath{\textrm{G}_\textrm{N}}}
\def\nus{$\nu$s}
\def\nue{\ensuremath{\nu_e}}
\def\numu{\ensuremath{\nu_\mu}}

\def\be{\begin{equation*}}
\def\ee{\end{equation*}}


\usepackage{fancyhdr}
\usepackage{cancel}
\usepackage{ mathrsfs }





\fancyhf{}
\lhead{\Large 33-444} % \hfill Introduction to Particle Physics \hfill Spring 2019}
\chead{\Large Introduction to Particle Physics} % \hfill Spring 2019}
\rhead{\Large Spring 2019} % \hfill Introduction to Particle Physics \hfill Spring 2019}
\begin{document}
\thispagestyle{fancy}





%\begin{tabular}{c}
%{\large 33-444 \hfill Intro To Particle \hfill Spring 2019\\}
%\hline 
%\end{tabular}

\begin{center}
{\huge \textbf{Final}}
\large

\end{center}

{\large


\textbf{1) What are three major consequences of combining QM and Relativity?}\hfill \textit{(3 points)}\\

\vspace{2.5in}


\textbf{2) Lorentz Transforms } \hfill \textit{(5 points)}\\
\begin{itemize}
\item[a)] How does a massive particle $\ket{p^\mu,\sigma}$ transform under a little group transformation ($W_\mu^\nu$)  ?
\vspace*{0.75in}
\item[d)] How does a mass-less particle $\ket{p^\mu,\sigma}$ transform under a little group transformation ($W_\mu^\nu$)  ?
\vspace*{0.75in}
\item[c)] How does a massive particle $\ket{p^\mu,\sigma}$ transform under a general Lorentz transformation ($\Lambda_\mu^\nu$)  ?
\vspace*{0.75in}
\item[d)] How does a mass-less particle $\ket{p^\mu,\sigma}$ transform under a general Lorentz transformation ($\Lambda_\mu^\nu$)  ?

\end{itemize}


\textbf{3) List or draw a diagram of the particles in the Standard model. } \hfill \textit{(3 points)}\\
What is the spin of each particle ?

\vspace*{3in}


\textbf{4) Why is the weak interaction so much weaker than the electromagnetic interaction at low energies?}\hfill \textit{(2 points)}\\

\vspace{1in}

\textbf{5) Accelerators: } \hfill \textit{(4 points)}
\begin{itemize}
\item[a)]{What limits the energy of circular proton accelerators ?
\vspace*{1.0in}
}
\item[b)]{What limits the energy of circular electron accelerators ?
\vspace*{1.0in}
}
\end{itemize}


\clearpage

\textbf{6) Muon decays: } \hfill \textit{(10 points)}\\
\begin{itemize}
  \item[a)]{ The muon decays via the weak interaction.  At low energy ($E << m_W$), this can be approximated as a point-like interaction. 
  Draw the diagram describing muon decay to an electron assuming a point-like weak interaction. 
  \vspace*{1.5in}
}
  \item[b)]{ What are the dimensions of the coupling constant, associated to this diagram  ?
\vspace*{1.0in}
  }
  \item[c)] How does the decay rate $\Gamma$ (decays/unit time)  depend on the muon mass ? 
\vspace*{1.0in}
  \item[d)]{ The muon has a mass of $\sim$0.1 GeV and a lifetime of $\sim 1 \mu s$. The tau lepton has a mass of {$\sim$1 GeV}. Estimate the lifetime of the tau lepton in $\mu s$.
\vspace*{1.0in}
}
\clearpage
  \item[e)] {Suppose that the photon could couple at the same vertex to the muon and the electron. Then the muon could decay as $\mu\rightarrow e \gamma$. 
  Estimate the ratio of the $\mu$ lifetime in this world to that in our world without this interaction.
  \vspace*{3.0in}
  }
\end{itemize}


\textbf{7) Electron-positron Collisions } \hfill \textit{(10 points)}\\
\begin{itemize}
\item[a)]{Consider electron-positron collisions with a center-of-mass  energy of 40 GeV.
Estimate the ratio of hadron production to di-muon production. 
\vspace*{2in}
}
\item[b)]{Sketch a graph of the total cross section of $ee\rightarrow\mu\mu$ as a function of $E_{CM}$ from 40 GeV to 200. 
Also sketch the component of the cross section due to the electro-magnetic interaction.
\vspace*{1in}
}
\end{itemize}


\textbf{8) Branching ratios:  } \hfill \textit{(9 points)}\\
\begin{itemize}
\item[a)]{How often does a $\tau$ decay to a charged lepton ?
\vspace*{1in}
}
\item[b)]{How often does a $Z$-boson decay to charged leptons ?
\vspace*{1in}
}
\item[c)]{How often does a $W$-boson decay to a charged lepton ?
\vspace*{1in}
}
\end{itemize}

\textbf{9) Collider Detectors  } \hfill \textit{(6 points)}\\
\begin{itemize}
\item[a)]{ In what ways do the detector signatures of electrons and muons look a-like, in what ways are they different ?
\vspace*{1in}
}
\item[b)]{ In what ways do the detector signatures of electrons and photons look a-like, in what ways are they different ?
\vspace*{1in}
}
\end{itemize}



\textbf{10) What are some reasons that hadronic showers are more challenging to measure than electro-magnetic showers ?  } \hfill \textit{(5 points)}\\
\vspace*{1.5in}

\textbf{11) If you want to improve momentum resolution, is it better to have a tracking detector that is twice as big or that has twice the magnetic field ?  Justify your answer.} \hfill \textit{(5 points)}\\
\vspace*{2.0in}

\textbf{12) For a new particle X with mass $\sim$ 2 TeV,  would you expect to measure the X mass more precisely from $X\rightarrow ee$ or $X \rightarrow \mu\mu$ ? Justify your answer.} \hfill \textit{(5 points)}\\
\vspace*{1.0in}

\clearpage
\textbf{13) How are \nus\ detected at the LHC ?} \hfill \textit{(3 points)}\\
\vspace*{1.0in}


\textbf{14) Spontaneous Symmetry Breaking  } \hfill \textit{(9 points)}\\
\begin{itemize}
\item[a)]{ What is the particle spectra (ie: for each particle, is it massive or mass-less and what is the spin) from the Lagrangian: \be\mathcal{L} = (\partial_\mu \phi^*) (\partial^\mu \phi) - V(\phi), \textrm{ where } \phi = \phi_1 + i \phi_2,\  V(\phi) = \mu^2\phi^*\phi + \lambda (\phi^*\phi)^2\textrm { and } \lambda, \mu^2 > 0   \ee
\vspace{1in}
}
\item[b)]{ What is the particle spectra from the setup in a) but with $\mu^2 < 0$ ?
\vspace{1.5in}
}
\item[c)]{What is the particle spectra from the Lagrangian: 
\be 
\mathcal{L} = (D_\mu \phi^*) (D^\mu \phi) - V(\phi) - \frac{1}{4}F_{\mu\nu}F^{\mu\nu} 
\ee 
where  $\phi$ and $V(\phi)$ are as before, $D_\mu = \partial_\mu - ieA_\mu$ , $\lambda > 0 $, and $\mu^2 < 0$ ?
\vspace{1in}
}
\end{itemize}

\clearpage

\textbf{15) Higgs Physics } \textit{(7 points)}\\
\begin{itemize}
\item[a)]{Estimate the ratio of Br($H\rightarrow\mu\mu$)/Br($H\rightarrow\tau\tau$) 
\vspace{2.0in}
}
\item[b)]{Draw one possible Feynman diagram for production and decay of the Higgs boson at the LHC.
\vspace{1.0in}
}
\item[c)]{Draw one possible Feynman diagram for the possible and decay of the Higgs boson at an electron -positron Collider.
\vspace{1.0in}
}
\end{itemize}

\textbf{16) What are some experimental constraints on a fourth generation of leptons ? } \hfill \textit{(3 points)}\\
\vspace{1.0in}


\clearpage

\textbf{17) Neutrino Physics } \hfill \textit{(17 points)}\\
\begin{itemize}
\item[a)]{How was the distinction between \numu\ and \nue\ discovered ?
\vspace*{1in}
}

\item[b)]{Why was this expected ?
\vspace*{1in}
}
\item[c)]{What was the main reason to study \nus\ in the 60s and 70s, before we knew they had mass?
\vspace*{1in}
}
\item[d)]{What are dominant kind(s) (indicate particle or anti-particle and flavor) of \nus\ that are produced (ignore oscillations) from:
\begin{itemize}
\item[i)]{ The sun ? 
\vspace*{0.5in}
} 
\item[ii)]{Nuclear reactors ? 
\vspace*{0.5in}
}
\item[ii)]{Cosmic-rays ? 
\vspace*{0.5in}
}
\item[iv)]{$\nu$-beams ?
\vspace*{0.5in}
}
\end{itemize}
}
\end{itemize}


\textbf{18) In a two $\nu$ model, what combination of $\Delta m^2$, E, and L do the transition probabilities depend on?  } \hfill \textit{(3 points)}\\
\vspace*{2in}

\textbf{19) What problems might Super-Symmetry solve ? } \hfill \textit{(3 points)}\\
\vspace{1in}

} % Begning Large
\end{document}
