\input{../latexSetup}

\lhead{\Large 33-444} % \hfill Introduction to Particle Physics \hfill Spring 2019}
\chead{\Large Introduction to Particle Physics} % \hfill Spring 2019}
\rhead{\Large Spring 2022} % \hfill Introduction to Particle Physics \hfill Spring 2019}
\begin{document}
\thispagestyle{fancy}





%\begin{tabular}{c}
%{\large 33-444 \hfill Intro To Particle \hfill Spring 2019\\}
%\hline 
%\end{tabular}

\begin{center}
{\huge \textbf{Midterm}}
\large

\end{center}

{\large




\textbf{1) Units/Dimensional Analysis:  Muonic Atoms}\hfill \textit{(6 points)}\\  

\begin{itemize}
\item[a)] The muon was the first elementary particle discovered that does not appear in ordinary atoms.  Negative muons can, however, form muonic atoms by replacing an electron in an ordinary atom. 
Estimate the size and binding energy of muonic hydrogen in GeV. A muon is 200 times heavier than an electron. 

$m_\mu << m_p \Rightarrow $ to first order all motion in muon. 


\be
E\sim \frac{p^2}{m} - \frac{\alpha}{r} 
\ee


QM $\Rightarrow  p \sim \frac{1}{r}$ 


\be
E\sim \frac{1}{m_\mu r^2} - \frac{\alpha}{r} 
\ee

Equilibrium when terms are $\sim$ equal.

\be
\frac{1}{m_\mu r^2} = \frac{\alpha}{r} \Rightarrow r \sim \frac{1}{\alpha m_\mu} = 10^2\ 10^1\ GeV^{-1} = 10^3\ GeV^{-1}
\ee

Can get E from plugging this back into either of the two terms.

\be
E \sim \frac{\alpha}{r} \sim \alpha^2 m_\mu \sim 10^{-4}\ 10^{-1}\ GeV = 10^{-5}\ GeV
\ee

For regular hydrogen, same arguments as above with $m_\mu \rightarrow m_e$

$\Rightarrow$

\be
\frac{E^\mu}{E^e} = \frac{\alpha^2 m_\mu}{\alpha^2 m_e} =  \frac{m_\mu}{m_e}
\ee


\item[b)] How much bigger/smaller would matter be in a world made out of muonic atoms ?

From above,

\be
\frac{r_{muonic}}{r_{normal}} \sim \frac{1}{\alpha m_\mu} \times \frac{\alpha m_e}{1} \sim \frac{m_e}{m_\mu} \sim \frac{1}{200}
\ee

Distances would be 200 times smaller.

\end{itemize}

\clearpage 
\textbf{2)  Relativity: How many generators does the Lorentz group have ? What transformations do they correspond to ?}\hfill \textit{(3 points)}

\bc
6:  3-rotations about x,y,z  and 3 boosts along x,y,and z
\ec


\textbf{3)  Quantum Mechanics:  }\hfill \textit{(4 points)}

How do position eigenstates transform under  Translations ?

 $T(\vec{a}) \ket{\vec{x}} = \ket{\vec{x} + \vec{a}} $


How do momentum eigenstates transform under Translations ?

 $T(\vec{a}) \ket{\vec{p}} = e^{ip\cdot a}\ket{\vec{p}} $

\vspace*{0.3in}

\textbf{4) Why does combining QM and Relativity require the existence of anti-particles?}\hfill \textit{(5 points)}


Causality or , more precisely,  the need for the Hamiltonitan to commute outside the light cone

\vspace*{0.3in}
\textbf{5) What is the little group? Why is it useful ?}\hfill \textit{(2 points)}\\

Subgroup of Lorentz transformations that leave the momentum invariant.
They are useful b/c they simplify the problem of finding the other labels (spin and helicity) that the lorentz group acts on.

\vspace*{0.3in}

\textbf{6) Lorentz Transforms } \hfill \textit{(6 points)}\\
\begin{itemize}
\item[a)] How does a \textbf{massive} particle transform under a general Lorentz transformation ($\Lambda_\mu^\nu$) ?

$\Lambda_\mu^\nu \ket{p^\mu,\sigma} = \sum_{\sigma'} R_{\sigma \sigma'} \ket{\Lambda p,\sigma'}$, where R is a rotation matrix.

\item[b)] How does a \textbf{mass-less} particle $\ket{p^\mu,h}$ transform under a general Lorentz transformation ($\Lambda_\mu^\nu$)?

$\Lambda_\mu^\nu \ket{p^\mu,\sigma} = e^{ih\theta} \ket{\Lambda p,\sigma'}$, where h is the particles helicity.xo
  
\end{itemize}


%\textbf{3)  GZK cutoff energy} \hfill \textit{(5 points)}\\
%High-energy cosmic rays (high energy protons), lose energy by interacting with CMB photons by producing a neutral pion:
%\be
%p+\gamma_{\mathrm{CMB}} \rightarrow p+\pi_0.
%\ee
%The proton energy at which this process can occur is called the GZK cutoff. 
%Estimate the GZK cutoff energy.
%The energy of CMB photons is $10^{-13}$ GeV (which follows from the measurement of CMB temperature of about 2.7 K).
%Use 0.1 GeV for the mass of the pion and 1 GeV for the mass of the proton

\vspace*{0.3in}
\textbf{7) Relativistic Wave Equations } \hfill \textit{(4 points)}\\

\begin{itemize}
\item[a)] What modifications where required to the Schrodinger equation to make it relativistic?

Need to use the same number of space and time derivatives; eg: klien gordon equation or the dirac equation.

\item[b)] What modifications to Maxwell's equations where required to make them relativistic?

None. The Maxwell's equations are already relativistic.

\end{itemize}

\vspace*{0.3in}
\textbf{8) Fields } \hfill \textit{(3 points)}
In QFT why do we write interactions in terms of fields (Fourier transforms of creation and annihilation operators)  instead  of the creation and annihilation operators directly?

In order to make the interactions local in space-time. 

\vspace*{0.3in}

\textbf{9) Lagrangians } \hfill \textit{(4 points)}\\
Consider the following Lagrangian:
\be
L = \frac{1}{2} (\partial_\mu\phi)(\partial^\mu\phi) - \frac{m^2}{2}\phi^2 + \bar{\psi}[i\gamma_\mu\partial^\mu]\psi + g_1 \phi \psi \psi + g_2 \psi \psi \psi \psi + g_3 \phi \phi \phi \phi
\ee

\bi
\item[a)] What is the dimension of $g_1$ ? 

dimension-less

\item[b)] What is the dimension of $g_2$ ? 

[GeV]$^{-2}$

\item[c)] What is the dimension of $g_3$ ? 

dimension-less

\ei

\vspace*{0.3in}

\textbf{10) Feynman Diagrams }\hfill \textit{(6 points)}
\bi
\item[a)]What is the S-matrix ?

Operator that takes asymptotic states in the far past to asymptotic states in the far future. 

\item[b)]What is a Feynman diagram ? and what is its relationship to the s-matrix?

A feynman diagram is a term in the expansion of the s-matrix.  It (roughly) describes one possible way an interaction could have occured.

\ei

\vspace*{0.3in}
\textbf{11) What are \underline{two} constraints on the interactions of mass-less Spin-1 particles, apart form coming from a Lorentz-invariant Lagrangian?}\hfill \textit{(6 points)}

- they have to couple to particles of the same mass

- Charge must be conserved

- Interaction must obey a lie group


\vspace*{0.3in}
\textbf{12) Can the SM have interactions between fermions and mass-less particles with Spin 2 ? If not, why not.  What about interactions between mass-less Spin 3 particles and fermions ? If not, why not. }\hfill \textit{(4 points)}

- Spin-2: yes, althought the couplingsa are unique.

- Spin-3 No, would violate $p\cdot M$ = 0 as required by Lorentz invariance. 


%
%\textbf{9) $ee\rightarrow\tau\tau$ scattering } \hfill \textit{(5 points)}\\
%\begin{itemize}
%\item[a)]{
%At high energy ($E_{CM} >> m_\tau$), what is the dependence of the $ee\rightarrow\tau\tau$ cross section on $E_{CM}$ ?
%\vspace*{1in}
%}
%
%%\item[b)]{
%%At high energy ($E_{CM} >> m_\tau$), derive the dependence of the cross section on scattering angle $\theta$.
%%\vspace*{4in}
%%}
%
%\item[b)]{
%Sketch a graph of $\frac{\sigma(ee\rightarrow\tau\tau)}{\sigma(ee\rightarrow\mu\mu)}$ as a function of $E_{CM}$ from $2\times m_{\mu}$ to 1 TeV (= 1000 GeV).
%\vspace*{1in}
%}
%\end{itemize}

%scalar QED. treat electron as scalar, photon as massless spin 1 particle
%  ee->ee scattering: Draw diagrmas, what are the associated matrix elements.


%what about ee->tau tau? again at high energy where $E_{CM}$ >> tau tau.

% work out the differential cross section ee->tautau in the massless limit.
%\clearpage
%
%\textbf{10) Muon decays: } \hfill \textit{(10 points)}\\
%\begin{itemize}
%  \item[a)]{ The muon decays via the weak interaction,  At low energy ($E << m_W$), this can be approximated as a point-like interaction. 
%  Draw the diagram describing muon decay to an electron assuming a point-like weak interaction.
%\vspace*{1.5in}
%}
%  \item[b)]{ What are the dimensions of the coupling constant, associated to this diagram  ?
%\vspace*{1.0in}
%  }
%  \item[c)] How does the decay rate $\Gamma$ (decays/unit time)  depend on the muon mass ? 
%\vspace*{1.0in}
%  \item[d)]{ The muon has a mass of $\sim$0.1 GeV and a lifetime of $\sim 1 \mu s$. The tau lepton has a mass of {$\sim$1 GeV}. Estimate the lifetime of the tau lepton in $\mu s$.
%\vspace*{2.4in}
%}
%  \item[e)] {Suppose that the photon could couple at the same vertex to the muon and the electron. Then the muon could decay as $\mu\rightarrow e \gamma$. 
%  Estimate the ratio of the $\mu$ lifetime in this world to that in our world without this interaction.
%  \vspace*{3.0in}
%  }
%\end{itemize}


%\textbf{11) Why is the weak interaction so much weaker than then electromagnetic interactions at low energies? } \hfill \textit{(2 points)}\\





%Assuming lepton universality, would you expect a difference between the way $\nu_e$ and $\nu_\mu$ interact with matter ?
%Justify your answer by drawing the leading order diagrams for $\nu_e + e \rightarrow \nu_e + e$  and $\nu_\mu + e \rightarrow \nu_\mu + e$.
%
%\vspace*{0.25in}


%\clearpage
%
%\textbf{16) A new force. } \hfill \textit{(5 points)}\\
%Assume there is another force of nature felt by electrons associated with the exchange of a new X boson of mass 1 TeV (= 1000 GeV).
%\begin{itemize}
%\item[a)]{ Estimate an upper limit on the range of this new force in meters.
%\vspace{2in}
%}
%%\item[b)]{ Assume muons also interact with this new boson. How could you determine the spin of X ?}
%\item[b)]{ Assume that this new X boson could also decay to spin-1/2 dark matter particles $\psi_{DM}$. At low energies (<< 1 TeV) the X interaction can be described by a point-like interaction. Estimate the coupling constant associated to dark matter scattering $e \psi_{DM} \rightarrow e \psi_{DM}$.   (Assume the X coupling at high energies is the same as for EM)  }
%\vspace*{2in}
%\item[c)]{ Assume there was a direct $X\rightarrow e\mu$ interaction. This would allow the muon to decay via $\mu^- \rightarrow e^-e^+e^-$. Draw the corresponding diagram and estimate the impact of the muon lifetime from this process.  How does it compare to the lifetime in the standard model?     }
%\end{itemize}






} % Begning Large
\end{document}
