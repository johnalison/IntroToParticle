
\input{../latexSetup}

\lhead{\Large 33-444} % \hfill Introduction to Particle Physics \hfill Spring 2019}
\chead{\Large Introduction to Particle Physics} % \hfill Spring 2019}
\rhead{\Large Spring 2019} % \hfill Introduction to Particle Physics \hfill Spring 2019}
\begin{document}
\thispagestyle{fancy}





%\begin{tabular}{c}
%{\large 33-444 \hfill Intro To Particle \hfill Spring 2019\\}
%\hline 
%\end{tabular}

\begin{center}
{\huge \textbf{Midterm}}
\large

\end{center}

{\large


\textbf{1) Why are energy and momentum conserved in the Standard Model ?  What would it mean if we found evidence for non-conservation of Energy ? What about non-conservation of Momentum? }\hfill \textit{(3 points)}\\

Physics is invariant to Space and time translations. 
These continuous symmetries lead to conserved currents via Noether's theorem.
The energy/momentum tensor is the associated conserved current. 

Non-conservation of energy would imply a violation of time translation invariance (eg: like in cosmology) 
Non-conservation of momentum would imply a violation of space translation invariance (eg: like near the surface of the earth)  

\vspace*{0.3in}

\textbf{2) What are three major consequences of combining QM and Relativity?}\hfill \textit{(3 points)}\\

Many possible answers, some I had in mind...\\
- Anti-particles\\
- Connection between spin and statistics\\
- Only contain certain types of leading interactions

\vspace*{0.3in}

\textbf{3)  GZK cutoff energy} \hfill \textit{(5 points)}\\
High-energy cosmic rays (high energy protons), lose energy by interacting with CMB photons by producing a neutral pion:
\be
p+\gamma_{\mathrm{CMB}} \rightarrow p+\pi_0.
\ee
The proton energy at which this process can occur is called the GZK cutoff. 
Estimate the GZK cutoff energy.
The energy of CMB photons is $10^{-13}$ GeV (which follows from the measurement of CMB temperature of about 2.7 K).
Use 0.1 GeV for the mass of the pion and 1 GeV for the mass of the proton.\\

$(P+\gamma) \rightarrow (p + \pi_0)$

The final momentum in the center of mass frame is:
\be
P_{F}^\mu = (m_p + m_\pi, 0, 0, 0 )
\ee

\be
P_{F}^2 = (m_p + m_\pi)^2 = m_p^2 + 2 m_p  m_\pi + m_\pi^2
\ee

The initial four vector is given by the sum of the proton and CMB photon four vectors.
\be
P_{I} = (P_p^\mu + P_\gamma^\mu)
\ee

\be
P_{I}^2 = \underbrace{P_p^2}_{= m_p^2} + 2 P_p \cdot P_\gamma + \underbrace{P_\gamma^2}_{= 0}
\ee
To evaluate $P_p \cdot P_\gamma$, can use reference frame where the photon and proton are colliding head on.

(Assume for the moment that we can neglect the proton mass compared to it momentum ...
\be
P_{p} = (p_p, 0 , 0, p_p)  \hspace{0.4in} P_{\gamma} = (E_{CMB}, 0 , 0, -E_{CMB})
\ee
Where $E_{CMB} = 3 \cdot 10^{-13} $GeV.

\be
P_p\cdot P_\gamma = 2 p_p E_{CMB} 
\ee

Imposing $P_I^2 = P_F^2$, allows us to solve for $p_p$.

\be
m_p^2 + 4 p_p E_{CMB}  = m_p^2 + 2 m_p  m_\pi + m_\pi^2
\ee

or

\be
p_p = \frac{ 2 m_p  m_\pi + m_\pi^2}{4 E_{CMB}} = \frac{2\cdot1\cdot0.1 + 0.1^2}{10\cdot10^{-13}} \textrm{GeV} \sim 3\times 10^{11} \textrm{GeV} \sim 10^{20} \textrm{eV}
\ee

\vspace*{0.3in}

\textbf{4) Neutrino masses. The seesaw mechanism is a proposed source for neutrino masses. In this scenario the neutrino would get a mass from the following diagram.} \hfill \textit{(2 points)}\\
Describing the interaction of a neutrino($\nu$) with a Higgs boson ($h$).
What is the dimension of the coupling constant associated with this diagram ?

\begin{figure}[h!]
\centering
\includegraphics[width=0.3\textwidth]{./NuMassDiagram.pdf}
\end{figure}

The mass dimension of the Higgs boson is 1 (b/c its a boson).\\
The mass dimension of the neutrino is 3/2 (b/c its a fermion).\\
So the mass dimension of the diagram without the coupling constant is 1+1+3/2+3/2 = 5. 
The overall diagram (including the coupling) needs to have mass dimension 4.  So the coupling must have mass dimension -1.
Or GeV$^{-1}$

\vspace*{0.3in}

\textbf{5) Why do we label particle states by momentum? } \hfill \textit{(2 points)}\\

B/c translations are a good symmetry and momentum states transform nicely under translations.

\vspace*{0.3in}

\textbf{6) Why is the weak interaction so much weaker than the electromagnetic interaction at low energies?}\hfill \textit{(2 points)}\\

B/c of the large mass of the W and Z-bosons.

\vspace*{0.3in}


\textbf{7) What are three restrictions to particle interactions that are a consequence gauge invariance ? }\hfill \textit{(3 points)}\\

Again there are many, here are a few that I have in mind.\\
-) Charge conservation\\
-) Principle of equivalence\\
-) Massless Spin-1 particles have to couple to particles of the same mass\\
-) Cannot have massless interacting particles with spin > 2.

\vspace*{0.3in}

\textbf{8) Lorentz Transforms } \hfill \textit{(5 points)}\\
\begin{itemize}
\item[a)] How does a massive particle $\ket{p^\mu,\sigma}$ transform under a general Lorentz transformation ($\Lambda_\mu^\nu$)  ?

$\Lambda_\mu^\nu \ket{p^\mu,\sigma} = \sum_{\sigma'} R_{\sigma \sigma'} \ket{\Lambda p,\sigma'}$, where R is a rotation matrix.

\item[b)] How does a massive particle $\ket{p^\mu,\sigma}$ transform under a little group transformation ($W_\mu^\nu$)  ?

$W_\mu^\nu \ket{p^\mu,\sigma} = \sum_{\sigma'} R_{\sigma \sigma'} \ket{p^\mu,\sigma'}$, where R is a rotation matrix.

\item[c)] How does a mass-less particle $\ket{p^\mu,\sigma}$ transform under a general Lorentz transformation ($\Lambda_\mu^\nu$)  ?

$\Lambda_\mu^\nu \ket{p^\mu,\sigma} = e^{ih\theta} \ket{\Lambda p,\sigma'}$, where h is the particles helicity.

\item[d)] How does a mass-less particle $\ket{p^\mu,\sigma}$ transform under a little group transformation ($W_\mu^\nu$)  ?

$W_\mu^\nu \ket{p^\mu,\sigma} = e^{ih\theta} \ket{p^\mu,\sigma'}$, where h is the particles helicity.

\end{itemize}

\vspace*{0.3in}

\textbf{9) $ee\rightarrow\tau\tau$ scattering } \hfill \textit{(5 points)}\\
\begin{itemize}
\item[a)]{
At high energy ($E_{CM} >> m_\tau$), what is the dependence of the $ee\rightarrow\tau\tau$ cross section on $E_{CM}$ ?

At high energy the only scale in the problem is $E_{CM}$.  By dimensional analysis:
$\sigma \sim \frac{1}{E_{CM}}$

}

%\item[b)]{
%At high energy ($E_{CM} >> m_\tau$), derive the dependence of the cross section on scattering angle $\theta$.
%\vspace*{4in}
%}

\item[b)]{
Sketch a graph of $\frac{\sigma(ee\rightarrow\tau\tau)}{\sigma(ee\rightarrow\mu\mu)}$ as a function of $E_{CM}$ from $2\times m_{\mu}$ to 1 TeV (= 1000 GeV).

The cross section ratio R would be 0 from $2\times m_{\mu}$ to $2\times m_{\tau}$, because there is not enough energy to produce the mass in the final state.
Above $2\times m_{\tau}$ the ratio would be 1 from lepton universality.

}
\end{itemize}

%scalar QED. treat electron as scalar, photon as massless spin 1 particle
%  ee->ee scattering: Draw diagrmas, what are the associated matrix elements.


%what about ee->tau tau? again at high energy where $E_{CM}$ >> tau tau.

% work out the differential cross section ee->tautau in the massless limit.

\vspace*{0.3in}

\textbf{10) Muon decays: } This is a homework problem will post solution after due date. \hfill \textit{(10 points)}\\
\begin{itemize}
  \item[a)]{ The muon decays via the weak interaction,  At low energy ($E << m_W$), this can be approximated as a point-like interaction. 
  Draw the diagram describing muon decay to an electron assuming a point-like weak interaction.
\vspace*{1.5in}
}
  \item[b)]{ What are the dimensions of the coupling constant, associated to this diagram  ?
\vspace*{1.0in}
  }
  \item[c)] How does the decay rate $\Gamma$ (decays/unit time)  depend on the muon mass ? 
\vspace*{1.0in}
  \item[d)]{ The muon has a mass of $\sim$0.1 GeV and a lifetime of $\sim 1 \mu s$. The tau lepton has a mass of {$\sim$1 GeV}. Estimate the lifetime of the tau lepton in $\mu s$.
\vspace*{2.4in}
}
  \item[e)] {Suppose that the photon could couple at the same vertex to the muon and the electron. Then the muon could decay as $\mu\rightarrow e \gamma$. 
  Estimate the ratio of the $\mu$ lifetime in this world to that in our world without this interaction.
  \vspace*{3.0in}
  }
\end{itemize}

\textbf{11) Can the SM have interactions between fermions and massive particles with Spin 3 ? If not, why not.  What about interactions between mass-less Spin 3 particles and fermions ? If not, why not. }\hfill \textit{(2 points)}\\

No constraint on massive spin-3 particles\\
The interactions of massless spin-3 particles would violate Lorentz invariance. 

\vspace*{0.3in}

\textbf{12) Yukawa’s Theory.}\hfill \textit{(2 points)}
In the 1930s, Hideki Yukawa predicted the existence of a new particle, now called the pion, which is responsible for binding protons and neutrons together in atomic nuclei. 
Estimate the mass of the pion from the assumed range of the force. ($10^{-15}$ meters). 
Express the mass in GeV.

$10$ GeV$^{-1} \sim 10^{-15} m  \Rightarrow m_\pi \sim \frac{1}{10} $ GeV


\vspace*{0.3in}

\textbf{13) List or draw a diagram of the particles in the Standard model. } \hfill \textit{(3 points)}\\
What is the spin of each particle ?

Fermions (Spin 1/2): \\

Leptons:  electron, muon, taus + 3 neutrinos\\
Quarks: up, down, charm, strange, top, bottom x 3 colors\\

Bosons (Spin1)\\
  W+/-,  Z,  photon,  gluon x 8

Higgs boson (spin0)

\vspace*{0.3in}

\textbf{14)  How many generators does the Lorentz group have ? }\hfill \textit{(2 points)}\\
What transformations do they correspond to ?

6:  3-rotations about x,y,z  and 3 boosts along x,y,and z

\vspace*{0.3in}

\textbf{15) Neutrino interactions } \hfill \textit{(2 points)}\\
Assuming lepton universality, would you expect a difference between the way $\nu_e$ and $\nu_\mu$ interact with matter ?
Justify your answer by drawing the leading order diagrams for $\nu_e + e \rightarrow \nu_e + e$  and $\nu_\mu + e \rightarrow \nu_\mu + e$.

Yes there is an electron/muon asymmetry in the material the neutrinos are interacting with.
So eg and electron neutrino can scatter off an electron in the atom whereas a muon neutrino cannot.

\clearpage

\textbf{16) A new force. } This is a homework problem will post solution after due date.\hfill \textit{(5 points)}\\
Assume there is another force of nature felt by electrons associated with the exchange of a new X boson of mass 1 TeV (= 1000 GeV).
\begin{itemize}
\item[a)]{ Estimate an upper limit on the range of this new force in meters.
\vspace{2in}
}
%\item[b)]{ Assume muons also interact with this new boson. How could you determine the spin of X ?}
\item[b)]{ Assume that this new X boson could also decay to spin-1/2 dark matter particles $\psi_{DM}$. At low energies (<< 1 TeV) the X interaction can be described by a point-like interaction. Estimate the coupling constant associated to dark matter scattering $e \psi_{DM} \rightarrow e \psi_{DM}$.   (Assume the X coupling at high energies is the same as for EM)  }
\vspace*{2in}
\item[c)]{ Assume there was a direct $X\rightarrow e\mu$ interaction. This would allow the muon to decay via $\mu^- \rightarrow e^-e^+e^-$. Draw the corresponding diagram and estimate the impact of the muon lifetime from this process.  How does it compare to the lifetime in the standard model?     }
\end{itemize}






} % Begning Large
\end{document}
