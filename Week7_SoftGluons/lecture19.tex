\input{latexSetup}
%\documentclass[margin,line]{res}
\usepackage{braket}
\usepackage{bbm}
\usepackage{relsize}
\usepackage{tcolorbox}


\def\ketY{\ensuremath{\ket {\Psi}}}
\def\iGeV{\ensuremath{\textrm{GeV}^{-1}}}
%\def\mp{\ensuremath{m_{\textrm{proton}}}}
\def\rp{\ensuremath{r_{\textrm{proton}}}}
\def\me{\ensuremath{m_{\textrm{electron}}}}
\def\aG{\ensuremath{\alpha_G}}
\def\rAtom{\ensuremath{r_{\textrm{atom}}}}
\def\rNucl{\ensuremath{r_{\textrm{nucleus}}}}
\def\GN{\ensuremath{\textrm{G}_\textrm{N}}}


\def\ABCDMatrix{\ensuremath{\begin{pmatrix} A &  B  \\ C  & D \end{pmatrix}}}
\def\xyprime{\ensuremath{\begin{pmatrix} x' \\ y' \end{pmatrix}}}
\def\xyprimeT{\ensuremath{\begin{pmatrix} x' &  y' \end{pmatrix}}}
\def\xy{\ensuremath{\begin{pmatrix} x \\ y \end{pmatrix}}}
\def\xyT{\ensuremath{\begin{pmatrix} x & y \end{pmatrix}}}

\def\IMatrix{\ensuremath{\begin{pmatrix} 0 &  1  \\ -1  & 0 \end{pmatrix}}}
\def\IBoostMatrix{\ensuremath{\begin{pmatrix} 0 &  1  \\ 1  & 0 \end{pmatrix}}}
\def\JThree{\ensuremath{\begin{pmatrix}    0 & -i & 0  \\ i & 0  & 0 \\ 0 & 0 & 0 \end{pmatrix}}} 
\def\JTwo{\ensuremath{\begin{bmatrix}    0 & 0 & -i  \\ 0 & 0  & 0 \\ i & 0 & 0 \end{bmatrix}}}
\def\JOne{\ensuremath{\begin{bmatrix}    0 & 0 & 0  \\ 0 & 0  & -i \\ 0 & i & 0 \end{bmatrix}}}
\def\etamn{\ensuremath{\eta_{\mu\nu}}}
\def\Lmn{\ensuremath{\Lambda^\mu_\nu}}
\def\dmn{\ensuremath{\delta^\mu_\nu}}
\def\wmn{\ensuremath{\omega^\mu_\nu}}
\def\be{\begin{equation*}}
\def\ee{\end{equation*}}
\def\bea{\begin{eqnarray*}}
\def\eea{\end{eqnarray*}}
\def\bi{\begin{itemize}}
\def\ei{\end{itemize}}
\def\fmn{\ensuremath{F_{\mu\nu}}}
\def\fMN{\ensuremath{F^{\mu\nu}}}
\def\bc{\begin{center}}
\def\ec{\end{center}}

\def\adagger{\ensuremath{a_{p\sigma}^\dagger}}
\def\lineacross{\noindent\rule{\textwidth}{1pt}}

\newcommand{\multiline}[1] {
\begin{tabular} {|l}
#1
\end{tabular}
}

\newcommand{\multilineNoLine}[1] {
\begin{tabular} {l}
#1
\end{tabular}
}



\newcommand{\lineTwo}[2] {
\begin{tabular} {|l}
#1 \\
#2
\end{tabular}
}

\newcommand{\rmt}[1] {
\textrm{#1}
}


%
% Units
%
\def\m{\ensuremath{\rmt{m}}}
\def\GeV{\ensuremath{\rmt{GeV}}}
\def\pt{\ensuremath{p_\rmt{T}}}


\def\parity{\ensuremath{\mathcal{P}}}


\usepackage{cancel}

\usepackage{fancyhdr}

\fancyhf{}
\lhead{\Large 33-444} % \hfill Introduction to Particle Physics \hfill Spring 2020}
\chead{\Large Introduction to Particle Physics} % \hfill Spring 2020}
\rhead{\Large Spring 2020} % \hfill Introduction to Particle Physics \hfill Spring 2020}

\begin{document}
\thispagestyle{fancy}

\begin{center}
{\huge \textbf{Lecture 19}}
\end{center}

{\fontsize{14}{16}\selectfont

\underline{Standard Model}

The version of QFT that our universe is characterized by.

As weve seen much of the world is fixed by the basic principles of Quantum Mechanics and Lorentz Invariance.

\begin{center}
0, 1/2, 1, 3/2, 2  
\end{center}

\begin{figure}[h]
\centering
\includegraphics[width=0.9\textwidth]{../Week5_Lagrangians/Interactions.pdf}
\end{figure}

Spin-1 must be Yang-Mills (ect)

SM attempts to explain all phenomena of particle physics in terms of small number of particles. 

Actually 4 different types

\underline{Leptons} Spin-1/2 (Left and Right handend fields) Experience Weak and EM interactions
\be
\begin{pmatrix} \nu_e \\ e \end{pmatrix} \hspace*{0.1in} \begin{pmatrix} \nu_\mu \\ \mu \end{pmatrix} \hspace*{0.1in}  \begin{pmatrix} \nu_\tau \\ \tau \end{pmatrix}  
\ee

\underline{Quarks} Spin-1/2 Strong, Weak and EM interactions
\be
 \begin{pmatrix} u \\ d \end{pmatrix} \hspace*{0.1in}   \begin{pmatrix} c \\ s \end{pmatrix} \hspace*{0.1in}   \begin{pmatrix} t \\ b \end{pmatrix}
\ee

\underline{Gauge Bosons} Spin-1  Force Carriers
\be
 \gamma \hspace*{0.2in}   g (\times 8) \hspace*{0.2in} W^{\pm}  \hspace*{0.2in} Z
\ee

\underline{Higgs Boson} Spin-0  
\be
  H
\ee

All are assumed to be elementary.  No internal structure.

\underline{Leptons}

Quantum Number associated with each generation. 

eg: 

\be
L_e \equiv N(e^-) - N(e^+) + N(\nu_e) - N(\bar{\nu_e})
\ee

All other particles has $L_e = 0$.

Electron number is a conserved quantity.

$\Rightarrow$ electrons ($\nu_e$) must be created/destroyed in pairs.


Corrisponding conserved lepton numbers for $\mu$s and $\tau$s.

\be
L_\mu  \hspace*{1in} L_\tau
\ee


Leptons masses (GeV):
  
\be
\begin{pmatrix} m_\nu \\ 10^{-3} \end{pmatrix} \hspace*{0.1in} \begin{pmatrix} m_\nu \\ 10^{-1} \end{pmatrix} \hspace*{0.1in}  \begin{pmatrix} m_\nu \\ 1.7 \end{pmatrix}  
\ee

Note: $\sum m_\nu < 10^{-9} GeV$

Once thought to be 0.  Now known that at least 2 (maybe all) have $m_\nu > 0$.

\clearpage

\underline{W/Z Exchange}

EM mediated by the exchange of a photon.

\begin{figure}[h]
\includegraphics[width=0.5\textwidth]{./PhotonExchange.pdf}
\end{figure}


Similarly for the weak interaction, force mediated by $W^{\pm}$ or Z.

When drawing diagrams, must remember to conserve Lepton numbers and EM charge.

eg: 

\includegraphics[width=0.3\textwidth]{./ZVertex.pdf}   
\includegraphics[width=0.6\textwidth]{./WVertex.pdf}   

\includegraphics[width=0.6\textwidth]{./NuScattering.pdf}   

\underline{Lepton Universality}
All data consistent with hypothesis that interaction of all generations are the same.
(Modulo mass differences)

\be
m_z = 90 GeV \hspace*{0.5in} m_W = 80 GeV  \rmt{which} \ne 0 !
\ee

This has a major implication for the range, or effective strength of the interaction.


\includegraphics[width=0.6\textwidth]{./GFermi.pdf}   

}
\end{document}

