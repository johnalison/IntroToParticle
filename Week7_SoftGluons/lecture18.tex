\input{../latexSetup}

\lhead{\Large 33-444} % \hfill Introduction to Particle Physics \hfill Spring 2020}
\chead{\Large Introduction to Particle Physics} % \hfill Spring 2020}
\rhead{\Large Spring 2020} % \hfill Introduction to Particle Physics \hfill Spring 2020}

\begin{document}
\thispagestyle{fancy}

\begin{center}
{\huge \textbf{Lecture 18}}
\end{center}

{\fontsize{14}{16}\selectfont

\underline{Now do the same thing to a more general interaction}


\begin{figure}[h]
\includegraphics[width=0.99\textwidth]{./generalScattering.pdf}
\end{figure}

Consider what happens if we attach a ``photon'' to an incoming leg

\begin{figure}[h]
\includegraphics[width=0.99\textwidth]{./incomingPhoton.pdf}
\end{figure}

\clearpage

Can also attach photon to outgoing leg

\begin{figure}[h]
\includegraphics[width=0.99\textwidth]{./outgoingPhoton.pdf}
\end{figure}

Total Amplitude is then given by

\be
M  = \sum_{\rmt{incoming}} Q_i \frac{(p\cdot\epsilon)}{(p\cdot q)}\ i M_0(p-q) + \sum_{\rmt{outgoing}} - Q_i \frac{(p\cdot\epsilon)}{(p\cdot q)}\ i M_0(p+q)
\ee

Take soft limit: $M_0(p\pm q) \rightarrow M_0(p)$

\be
M  = iM_0 \left( \sum_{\rmt{incoming}} Q_i \frac{(p\cdot\epsilon)}{(p\cdot q)}\  + \sum_{\rmt{outgoing}} - Q_i \frac{(p\cdot\epsilon)}{(p\cdot q)}\   \right)
\ee

Now as before $\epsilon_\mu \rightarrow \epsilon'_\mu + q_\mu$ means that M must vanish when $\epsilon_\mu \rightarrow q_\mu$.

OR under a Lorentz Transform

\be
\epsilon_\mu \cdot M \rightarrow \epsilon'_\mu \cdot M' + i M_0 \underbrace{\left( \sum_{\rmt{incoming}} Q_i  + \sum_{\rmt{outgoing}} - Q_i    \right)}_{\substack{= 0\ \rmt{only if} \\ \sum_{\rmt{incoming}} Q_i = \sum_{\rmt{outgoing}} Q_i }}
\ee

\underline{\underline{Charge has to be conserved! }}

\lineacross

Now same logic for Spin-2  (describes interaction w/Gravitons)

Same as above except 2-component polarization vector.


\be
\epsilon_{\mu\nu} \underbrace{\rightarrow}_{\rmt{under little group}} \epsilon_{\mu\nu} + \underbrace{A_\mu q_\nu + B_\mu q_\mu + C q_\mu q_\nu}_{\rmt{effect from all of these need to be 0 as before}}
\ee

where A, B C's are non-zero and depend on the particular little group transformation done.


\begin{minipage}{0.4\textwidth}
\includegraphics[width=0.9\textwidth]{./gravitonVertex.pdf}
\end{minipage} %\hfill
\begin{minipage}{0.45\textwidth}
\be
= i (iK_i) \epsilon_{\mu\nu} \frac{(p^\mu p^\nu)}{-p\cdot q}
\ee
\end{minipage} %\hfill

(Same idea with the outgoing leg)


Now, (lets focus on piece that goes like $\epsilon_{\mu\nu} \rightarrow  \epsilon_{\mu\nu}  + q_\mu B_\nu$

\bea
\epsilon_{\mu\nu} \rightarrow  \epsilon'_{\mu\nu} M'^{\mu\nu} &+& M \left( \sum_{\rmt{incoming}} K_i B_\nu p^\nu - \sum_{\rmt{outgoing}} K_i B_\nu p^\nu \right)\\
&+& M B_\nu \left( \sum_{\rmt{incoming}} K_i  p^\nu - \sum_{\rmt{outgoing}} K_i  p^\nu \right)
\eea

$\Rightarrow  K_i p_i^\nu$  is \underline{\underline{conserved}}


We know that $p_i^\nu$ is conserved by E and momentum conservation. 

Only way can have nontrivial solutions is if $k_i = k$ for all i

All particles interact with gravity with the same strength. 

\underline{Gravitational interaction is Universal !}

Discovered the ``Principle of Equivalence'' that is the starting point of General Relativity!

\lineacross

Can keep going...

For a massless spin-3 particle we would do the same exercise. 

We would find we need

\be
 \sum_{\rmt{incoming}} \beta_i  p_i^\mu p_i^\nu = \sum_{\rmt{outgoing}} \beta_i  p_i^\mu  p_i^\nu 
\ee

eg: $\mu\nu = 0$

\be
 \sum_{\rmt{incoming}} \beta_i  E_i^2 = \sum_{\rmt{outgoing}} \beta_i  E_i^2
\ee


Way too constraining.  

Only way if $\beta_i = 0$

\underline{\underline{There can be no interacting theories of massless particles of Spin greater than 2 !}}


}
\end{document}

