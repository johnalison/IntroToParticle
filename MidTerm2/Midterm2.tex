
\documentclass[paper=letter,11pt]{scrartcl}

\KOMAoptions{headinclude=true, footinclude=false}
\KOMAoptions{DIV=14, BCOR=5mm}
\KOMAoptions{numbers=noendperiod}
\KOMAoptions{parskip=half}
\addtokomafont{disposition}{\rmfamily}
\addtokomafont{part}{\LARGE}
\addtokomafont{descriptionlabel}{\rmfamily}
%\setkomafont{pageheadfoot}{\normalsize\sffamily}
\setkomafont{pagehead}{\normalsize\rmfamily}
%\setkomafont{publishers}{\normalsize\rmfamily}
\setkomafont{caption}{\normalfont\small}
\setcapindent{0pt}
\deffootnote[1em]{1em}{1em}{\textsuperscript{\thefootnotemark}\ }


\usepackage{amsmath}
\usepackage[varg]{txfonts}
\usepackage[T1]{fontenc}
\usepackage{graphicx}
\usepackage{xcolor}
\usepackage[american]{babel}
% hyperref is needed in many places, so include it here
\usepackage{hyperref}

\usepackage{xspace}
\usepackage{multirow}
\usepackage{float}


\usepackage{braket}
\usepackage{bbm}
\usepackage{relsize}
\usepackage{tcolorbox}

\def\ketY{\ensuremath{\ket {\Psi}}}
\def\iGeV{\ensuremath{\textrm{GeV}^{-1}}}
%\def\mp{\ensuremath{m_{\textrm{proton}}}}
\def\rp{\ensuremath{r_{\textrm{proton}}}}
\def\me{\ensuremath{m_{\textrm{electron}}}}
\def\aG{\ensuremath{\alpha_G}}
\def\rAtom{\ensuremath{r_{\textrm{atom}}}}
\def\rNucl{\ensuremath{r_{\textrm{nucleus}}}}
\def\GN{\ensuremath{\textrm{G}_\textrm{N}}}
\def\ketX{\ensuremath{\ket{\vec{x}}}}
\def\ve{\ensuremath{\vec{\epsilon}}}


\def\ABCDMatrix{\ensuremath{\begin{pmatrix} A &  B  \\ C  & D \end{pmatrix}}}
\def\xyprime{\ensuremath{\begin{pmatrix} x' \\ y' \end{pmatrix}}}
\def\xyprimeT{\ensuremath{\begin{pmatrix} x' &  y' \end{pmatrix}}}
\def\xy{\ensuremath{\begin{pmatrix} x \\ y \end{pmatrix}}}
\def\xyT{\ensuremath{\begin{pmatrix} x & y \end{pmatrix}}}

\def\IMatrix{\ensuremath{\begin{pmatrix} 0 &  1  \\ -1  & 0 \end{pmatrix}}}
\def\IBoostMatrix{\ensuremath{\begin{pmatrix} 0 &  1  \\ 1  & 0 \end{pmatrix}}}
\def\JThree{\ensuremath{\begin{pmatrix}    0 & -i & 0  \\ i & 0  & 0 \\ 0 & 0 & 0 \end{pmatrix}}} 
\def\JTwo{\ensuremath{\begin{bmatrix}    0 & 0 & -i  \\ 0 & 0  & 0 \\ i & 0 & 0 \end{bmatrix}}}
\def\JOne{\ensuremath{\begin{bmatrix}    0 & 0 & 0  \\ 0 & 0  & -i \\ 0 & i & 0 \end{bmatrix}}}
\def\etamn{\ensuremath{\eta_{\mu\nu}}}
\def\Lmn{\ensuremath{\Lambda^\mu_\nu}}
\def\dmn{\ensuremath{\delta^\mu_\nu}}
\def\wmn{\ensuremath{\omega^\mu_\nu}}
\def\be{\begin{equation*}}
\def\ee{\end{equation*}}
\def\bea{\begin{eqnarray*}}
\def\eea{\end{eqnarray*}}
\def\bi{\begin{itemize}}
\def\ei{\end{itemize}}
\def\fmn{\ensuremath{F_{\mu\nu}}}
\def\fMN{\ensuremath{F^{\mu\nu}}}
\def\bc{\begin{center}}
\def\ec{\end{center}}
\def\nus{$\nu$s}

\def\adagger{\ensuremath{a_{p\sigma}^\dagger}}
\def\lineacross{\noindent\rule{\textwidth}{1pt}}

\newcommand{\multiline}[1] {
\begin{tabular} {|l}
#1
\end{tabular}
}

\newcommand{\multilineNoLine}[1] {
\begin{tabular} {l}
#1
\end{tabular}
}



\newcommand{\lineTwo}[2] {
\begin{tabular} {|l}
#1 \\
#2
\end{tabular}
}

\newcommand{\rmt}[1] {
\textrm{#1}
}


%
% Units
%
\def\m{\ensuremath{\rmt{m}}}
\def\GeV{\ensuremath{\rmt{GeV}}}
\def\pt{\ensuremath{p_\rmt{T}}}


\def\parity{\ensuremath{\mathcal{P}}}

\usepackage{cancel}
\usepackage{ mathrsfs }
\def\bigL{\ensuremath{\mathscr{L}}}

\usepackage{ dsfont }



\usepackage{fancyhdr}
\fancyhf{}

\usepackage{braket}

\def\ketY{\ensuremath{\ket {\Psi}}}
\def\iGeV{\ensuremath{\textrm{GeV}^{-1}}}
%\def\mp{\ensuremath{m_{\textrm{proton}}}}
\def\rp{\ensuremath{r_{\textrm{proton}}}}
\def\me{\ensuremath{m_{\textrm{electron}}}}
\def\aG{\ensuremath{\alpha_G}}
\def\rAtom{\ensuremath{r_{\textrm{atom}}}}
\def\rNucl{\ensuremath{r_{\textrm{nucleus}}}}
\def\GN{\ensuremath{\textrm{G}_\textrm{N}}}
\def\ketX{\ensuremath{\ket{\vec{x}}}}
\def\ve{\ensuremath{\vec{\epsilon}}}


\def\ABCDMatrix{\ensuremath{\begin{pmatrix} A &  B  \\ C  & D \end{pmatrix}}}
\def\xyprime{\ensuremath{\begin{pmatrix} x' \\ y' \end{pmatrix}}}
\def\xyprimeT{\ensuremath{\begin{pmatrix} x' &  y' \end{pmatrix}}}
\def\xy{\ensuremath{\begin{pmatrix} x \\ y \end{pmatrix}}}
\def\xyT{\ensuremath{\begin{pmatrix} x & y \end{pmatrix}}}

\def\IMatrix{\ensuremath{\begin{pmatrix} 0 &  1  \\ -1  & 0 \end{pmatrix}}}
\def\IBoostMatrix{\ensuremath{\begin{pmatrix} 0 &  1  \\ 1  & 0 \end{pmatrix}}}
\def\JThree{\ensuremath{\begin{pmatrix}    0 & -i & 0  \\ i & 0  & 0 \\ 0 & 0 & 0 \end{pmatrix}}} 
\def\JTwo{\ensuremath{\begin{bmatrix}    0 & 0 & -i  \\ 0 & 0  & 0 \\ i & 0 & 0 \end{bmatrix}}}
\def\JOne{\ensuremath{\begin{bmatrix}    0 & 0 & 0  \\ 0 & 0  & -i \\ 0 & i & 0 \end{bmatrix}}}
\def\etamn{\ensuremath{\eta_{\mu\nu}}}
\def\Lmn{\ensuremath{\Lambda^\mu_\nu}}
\def\dmn{\ensuremath{\delta^\mu_\nu}}
\def\wmn{\ensuremath{\omega^\mu_\nu}}
\def\be{\begin{equation*}}
\def\ee{\end{equation*}}
\def\bea{\begin{eqnarray*}}
\def\eea{\end{eqnarray*}}
\def\bi{\begin{itemize}}
\def\ei{\end{itemize}}
\def\fmn{\ensuremath{F_{\mu\nu}}}
\def\fMN{\ensuremath{F^{\mu\nu}}}
\def\bc{\begin{center}}
\def\ec{\end{center}}
\def\nus{$\nu$s}

\def\adagger{\ensuremath{a_{p\sigma}^\dagger}}
\def\lineacross{\noindent\rule{\textwidth}{1pt}}

\newcommand{\multiline}[1] {
\begin{tabular} {|l}
#1
\end{tabular}
}

\newcommand{\multilineNoLine}[1] {
\begin{tabular} {l}
#1
\end{tabular}
}



\newcommand{\lineTwo}[2] {
\begin{tabular} {|l}
#1 \\
#2
\end{tabular}
}

\newcommand{\rmt}[1] {
\textrm{#1}
}


%
% Units
%
\def\m{\ensuremath{\rmt{m}}}
\def\GeV{\ensuremath{\rmt{GeV}}}
\def\pt{\ensuremath{p_\rmt{T}}}


\def\parity{\ensuremath{\mathcal{P}}}

\usepackage{cancel}
\usepackage{ mathrsfs }
\def\bigL{\ensuremath{\mathscr{L}}}

\usepackage{ dsfont }


%\def\ketY{\ensuremath{\ket {\Psi}}}
%\def\iGeV{\ensuremath{\textrm{GeV}^{-1}}}
%\def\mp{\ensuremath{m_{\textrm{proton}}}}
%\def\rp{\ensuremath{r_{\textrm{proton}}}}
%\def\me{\ensuremath{m_{\textrm{electron}}}}
%\def\aG{\ensuremath{\alpha_G}}
%\def\rAtom{\ensuremath{r_{\textrm{atom}}}}
%\def\rNucl{\ensuremath{r_{\textrm{nucleus}}}}
%\def\GN{\ensuremath{\textrm{G}_\textrm{N}}}
%
%\def\be{\begin{equation*}}
%\def\ee{\end{equation*}}


\usepackage{fancyhdr}
\usepackage{cancel}
\usepackage{ mathrsfs }





\fancyhf{}
\lhead{\Large 33-444} % \hfill Introduction to Particle Physics \hfill Spring 2019}
\chead{\Large Introduction to Particle Physics} % \hfill Spring 2019}
\rhead{\Large Spring 2020} % \hfill Introduction to Particle Physics \hfill Spring 2019}
\begin{document}
\thispagestyle{fancy}





%\begin{tabular}{c}
%{\large 33-444 \hfill Intro To Particle \hfill Spring 2019\\}
%\hline 
%\end{tabular}

\begin{center}
{\huge \textbf{Midterm 2}}
\large

\end{center}

{\large

\textbf{1) List or draw a diagram of the particles in the Standard model. } \hfill \textit{(4 points)}\\
What is the spin of each particle ?

\vspace*{3in}



\textbf{2) Why do the electromagnetic, weak and strong interactions look so different despite the fact that at short distances they are all described by similar Feynman diagrams with similar coupling constants?} \hfill \textit{(2 points)}\\

\vspace{1in}



\textbf{2) Di-boson physics:  } \hfill \textit{(X points)}\\
Processes in which pairs of gauge bosons are produced are a sensitive probe of the electro-weak theory. These are typically studied at the LHC by looking for signatures involving electrons or muons.  
Estimate how often a $WZ$ event decays into an electron or muon and three neutrinos.\\ ie: $e\nu\nu\nu$ or $\mu\nu\nu\nu$


%\vspace*{2in}


\textbf{3) Higgs Boson discovery: } \hfill \textit{(X points)}\\
The Higgs boson was discovered in its decays to $WW (Br\sim20\%)$, $ZZ\ (Br\sim3\%)$ and $\gamma\gamma\ (Br\sim0.2\%)$.
However for the $WW$ and $ZZ$ decays, only the ``fully-leptonic'' channels -- where each boson decays leptonicically to $e$ or $\mu$ -- were used. \\

Estimate how often $WW \rightarrow \ell\nu\ell'\nu'$ and $ZZ\rightarrow\ell\ell\ell'\ell'$, where $\ell$ is e or $\mu$. (ignore decays through taus)\\

Including these factors, rank the Higgs boson discovery channels by how many signal events they are expected to have.

\vspace*{2in}

\textbf{4) Accelerators: } \hfill \textit{(4 points)}\\
\begin{itemize}
\item[a)]{What limits the energy of circular proton accelerators ?
\vspace*{1.0in}
}
\item[b)]{What limits the energy of circular electron accelerators ?
\vspace*{1.0in}
}
\end{itemize}

\textbf{5) Electron-positron Collisions } \hfill \textit{(10 points)}\\
\begin{itemize}
\item[a)]{
How does the value of $R(E_{CM}) \equiv \sigma(ee\rightarrow \rmt{jets})/\sigma(ee\rightarrow \mu\mu)$ change as $E_{CM}$ is increased beyond twice the mass of the charm quark? 
What values of $R(< 2 m_{\rmt{charm}})$ and $R(> 2 m_{\rmt{charm}})$ do you expect?
\vspace*{1in}
}

\item[b)]{Sketch a graph of the total cross section of $ee\rightarrow\mu\mu$ as a function of $E_{CM}$ from 40 GeV to 200. 
Also sketch the component of the cross section due to the electro-magnetic interaction.
\vspace*{1in}
}
\end{itemize}


\textbf{6) Collider Detectors  } \hfill \textit{(6 points)}\\
\begin{itemize}
\item[a)]{ In what ways do the detector signatures of electrons and muons look a-like, in what ways are they different ?
\vspace*{1in}
}
\item[b)]{ In what ways do the detector signatures of electrons and photons look a-like, in what ways are they different ?
\vspace*{1in}
}
\end{itemize}
        

\textbf{7) 
Which are more challenging to accurately measure and why : Hadronic showers or electro-magnetic showers?  } \hfill \textit{(5 points)}\\
\vspace*{1.5in}

\textbf{8) For a new particle X with mass $\sim$ 2 TeV,  would you expect to measure the X mass more precisely from $X\rightarrow ee$ or $X \rightarrow \mu\mu$ ? Justify your answer.} \hfill \textit{(5 points)}\\
\vspace*{1.0in}


\textbf{9) How are \nus\ detected at the LHC ?} \hfill \textit{(3 points)}\\
\vspace*{1.0in}


\textbf{10) Higgs Boson Production : } \hfill \textit{(XX points)}\\
\begin{itemize}
\item[a)]{ Draw one possible Feynman diagram for production and decay of the Higgs boson at the LHC.}
\vspace*{1in}
\item[b)]{ Why is Higgs boson production so much rarer then W or Z production despite the fact that their masses are similar?    }
\vspace*{1in}
\end{itemize}


\textbf{11) Higgs-Lepton interactions } \hfill \textit{(XX points)}\\
The coupling of the Higgs field to leptons can be studied by looking for detector signals where the Higgs boson decays to pairs leptons.
Which of the possible decay modes is the would be the best option.  Justify your answer.

\vspace*{1in}

\clearpage

\textbf{12) Interaction Symmetries } \hfill \textit{(XX points)} \\ 
In the first part of the course we learned that the interactions of mass-less spin-1 particles must be described by group symmetries. 
\begin{itemize}
\item[a)] {How is the group symmetry of an underlying interaction related to the particle content ?}
\vspace*{1in}
\item[b)]{ What are the symmetry groups of the electro-weak interaction in the SM ? }
\vspace*{1in}
\item[c)]{How does the observed physical particle content reflect this ? (Qualitatively, no formulas required!) }
\vspace*{1in}
\end{itemize}

\textbf{13) Spontaneous Symmetry Breaking} \hfill \textit{(XX points)} \\ 
\begin{itemize}
\item[a)]{What is Spontaneous Symmetry Breaking  ?  }
\vspace*{1in}
\item[b)]{What properties of the fundamental particles is it responsible for describing ?}
\vspace*{1in}
\item[c)]{What experimental evidence do we have that spontaneous symmetry breaking is actually responsible for these properties?}
\end{itemize}



} % Begning Large
\end{document}
